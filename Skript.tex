
\documentclass[a4paper]{book}%[a4paper,oneside]
\usepackage[pass]{geometry}%[hmarginratio=1:1]{geometry}
%\usepackage[utf8]{inputenc}

\usepackage{polyglossia}
\setdefaultlanguage[spelling=new]{german}
\setotherlanguage{english}
%\setotherlanguage[numerals=western]{farsi}
\usepackage{fontspec}
\usepackage{xeCJK}
\usepackage{csquotes}

\usepackage[
    backend=biber,
    style=numeric,
    sortlocale=de_DE,
    natbib=true,
    url=false,
    doi=true,
    eprint=false
]{biblatex}
\addbibresource{itteerde.bib}
\bibliography{itteerde}	

\usepackage{makeidx}
\usepackage{mdframed}
\usepackage{graphics}
\usepackage{graphicx}
\usepackage{pictex}
\usepackage{subfig}
\usepackage{float}
\usepackage{array}
\usepackage{xspace}
\usepackage{xcolor}
\usepackage{pdfpages}

\usepackage{longtable}

\usepackage{tikz}
\usetikzlibrary{shapes,backgrounds,arrows,positioning,calc,decorations.markings,matrix}


\usepackage{verse}

\usepackage{amsmath}
\usepackage{amssymb}
\usepackage{amstext}
\usepackage{amsfonts}
\usepackage{mathrsfs}
\usepackage{amsthm}

\usepackage{chemfig}
\usepackage{listings}
\usepackage{soul}
\usepackage{calc}
\usepackage{metalogo}
\usepackage{hologo}

%\usepackage{stmaryrd}
%\usepackage{marvosym}

\usepackage{url}
\usepackage[position=top]{caption}

\usepackage{etoolbox}
\usepackage{etaremune}

\usepackage[citecolor=black,urlcolor=black,linkcolor=black]{hyperref}
\hypersetup{
    colorlinks=true,
}
\usepackage[xindy={language=german,codepage=duden-utf8},
    nonumberlist,
    toc,
    nopostdot,
    style=altlist,
    nogroupskip
    ]{glossaries}
    \GlsSetXdyCodePage{duden-utf8}
\apptocmd{\thebibliography}{\raggedright}{}{}

\newfontfamily\arabicfont{Amiri}



\def\UrlBreaks{\do\A\do\B\do\C\do\D\do\E\do\F\do\G\do\H\do\I\do\J\do\K
\do\L%
\do\M\do\N\do\O\do\P\do\Q\do\R\do\S\do\T\do\U\do\V\do\W\do\X\do\Y\do\Z
\do\0%
\do\a\do\b\do\c\do\d\do\e\do\f\do\g \do\h\do\i\do\j\do\k\do\l%
\do\m\do\n\do\o\do\p\do\q\do\r\do\s\do\t\do\u\do\v \do\w\do\x\do\y\do\z
%
\do\1\do\2\do\3\do\4\do\5\do\6\do\7\do\8\do\9\do\-}%
\def\UrlBigBreaks{\do\_}


\newcommand{\uu}[1]{\underline{#1}}
\newcommand{\ii}[1]{\textit{#1}}
\newcommand{\tm}{\textsuperscript{\tiny{TM}}}
\newcommand{\rtm}{\textsuperscript{\tiny{\circledR}}}
\newcommand{\whatsapp}{\textit{WhatsApp}\xspace}

\newcommand{\youtube}{\url{http://www.youtube.com}\xspace}
\newcommand{\tagesschau}{\url{http://www.tagesschau.de}\xspace}
\newcommand{\wikipedia}{\url{http://en.wikipedia.org}\xspace}
\newcommand{\cosmos}{\textit{Cosmos - A Space Time Odyssee}\index{Unterhaltung!Cosmos - A Space Time Odyssee}\xspace}

\newcommand{\vcenteredincludeIcon}[1]{\begingroup
    \setbox0=\hbox{\includegraphics[height=1em]{#1}}%
    \parbox{\wd0}{\box0}\endgroup
}
\newcommand\crule[3][black]{\textcolor{#1}{\rule{#2}{#3}}}

\newcommand{\TikZ}{Ti\textit{k}z\xspace}
\newcommand*\circled[1]{\tikz[baseline=(char.base)]{
            \node[shape=circle,draw,inner sep=2pt] (char) {#1};}}

\newcommand{\emojiSmile}{\vcenteredincludeIcon{graphics/smile.jpg}\xspace}
\newcommand{\emojiSmileT}{\vcenteredincludeIcon{graphics/smile_st.jpg}\xspace}
\newcommand{\emojiSet}{\vcenteredincludeIcon{graphics/smile_set.jpg}\xspace}
\newcommand{\emojiSaint}{\vcenteredincludeIcon{graphics/smile_saint.jpg}\xspace}
\newcommand{\emojiTears}{\vcenteredincludeIcon{graphics/smile_tears.jpg}\xspace}
\newcommand{\emojiGrin}{\vcenteredincludeIcon{graphics/smile_grin.jpg}\xspace}
\newcommand{\emojiWink}{\vcenteredincludeIcon{graphics/smile_wink.jpg}\xspace}
\newcommand{\emojiTOE}{\vcenteredincludeIcon{graphics/smile_toe.jpg}\xspace}
\newcommand{\emojiTES}{\vcenteredincludeIcon{graphics/smile_tes.jpg}\xspace}
\newcommand{\emojiSunglasses}{\vcenteredincludeIcon{graphics/smile_sunglasses.jpg}\xspace}
\newcommand{\emojiBlushLips}{\vcenteredincludeIcon{graphics/smile_blushlips.jpg}\xspace}
\newcommand{\emojiBlushSmile}{\vcenteredincludeIcon{graphics/smile_blushsmile.jpg}\xspace}
\newcommand{\emojiDevil}{\vcenteredincludeIcon{graphics/smile_devil.jpg}\xspace}
\newcommand{\emojiBlusBigEyes}{\vcenteredincludeIcon{graphics/smile_blushbigeyes.jpg}\xspace}
\newcommand{\emojiSmileTeeth}{\vcenteredincludeIcon{graphics/smile_teeth.jpg}\xspace}
\newcommand{\emojiSmileTears}{\vcenteredincludeIcon{graphics/smile_tears.jpg}\xspace}
\newcommand{\emojiSmileKiss}{\vcenteredincludeIcon{graphics/smile_kiss.jpg}\xspace}
\newcommand{\emojiSeeNoEvil}{\vcenteredincludeIcon{graphics/emojiSeeNoEvil.jpg}\xspace}
\newcommand{\emojiSmileWink}{\vcenteredincludeIcon{graphics/smile_wink.jpg}\xspace}
\newcommand{\emojiSmileSad}{\vcenteredincludeIcon{graphics/smile_sad.jpg}\xspace}
\newcommand{\emojiHandPointUp}{\vcenteredincludeIcon{graphics/emoji_hand_point_up.jpg}\xspace}

\newcommand{\emojiBWSmile}{\vcenteredincludeIcon{graphics/smile_bw.jpg}\xspace}
\newcommand{\emojiBWSmileT}{\vcenteredincludeIcon{graphics/smile_bw_st.jpg}\xspace}
\newcommand{\emojiBWSet}{\vcenteredincludeIcon{graphics/smile_bw_set.jpg}\xspace}
\newcommand{\emojiBWSaint}{\vcenteredincludeIcon{graphics/smile_bw_saint.jpg}\xspace}
\newcommand{\emojiBWTears}{\vcenteredincludeIcon{graphics/smile_bw_tears.jpg}\xspace}
\newcommand{\emojiBWGrin}{\vcenteredincludeIcon{graphics/smile_bw_grin.jpg}\xspace}
\newcommand{\emojiBWWink}{\vcenteredincludeIcon{graphics/smile_bw_wink.jpg}\xspace}
\newcommand{\emojiBWTOE}{\vcenteredincludeIcon{graphics/smile_bw_toe.jpg}\xspace}
\newcommand{\emojiBWTES}{\vcenteredincludeIcon{graphics/smile_bw_tes.jpg}\xspace}
\newcommand{\emojiBWSunglasses}{\vcenteredincludeIcon{graphics/smile_bw_sunglasses.jpg}\xspace}
\newcommand{\emojiBWBlushLips}{\vcenteredincludeIcon{graphics/smile_bw_blushlips.jpg}\xspace}
\newcommand{\emojiBWBlushSmile}{\vcenteredincludeIcon{graphics/smile_bw_blushsmile.jpg}\xspace}
\newcommand{\emojiBWDevil}{\vcenteredincludeIcon{graphics/smile_bw_devil.jpg}\xspace}
\newcommand{\emojiBWBlusBigEyes}{\vcenteredincludeIcon{graphics/smile_bw_blushbigeyes.jpg}\xspace}
\newcommand{\emojiBWSmileTeeth}{\vcenteredincludeIcon{graphics/smile_bw_teeth.jpg}\xspace}
\newcommand{\emojiBWSmileBWTears}{\vcenteredincludeIcon{graphics/smile_bw_tears.jpg}\xspace}
\newcommand{\emojiBWSmileKiss}{\vcenteredincludeIcon{graphics/smile_bw_kiss.jpg}\xspace}
\newcommand{\emojiBWSeeNoEvil}{\vcenteredincludeIcon{graphics/emoji_bw_SeeNoEvil.jpg}\xspace}
\newcommand{\emojiBWSmileWink}{\vcenteredincludeIcon{graphics/smile_bw_wink.jpg}\xspace}
\newcommand{\emojiBWSmileSad}{\vcenteredincludeIcon{graphics/smile_bw_sad.jpg}\xspace}
\newcommand{\emojiBWHandPointUp}{\vcenteredincludeIcon{graphics/emoji_bw_hand_point_up.jpg}\xspace}

\newcommand{\proofsquare}{
    \begin{flushright}
      $\square$
    \end{flushright}\xspace
}
\newcommand{\done}{
    \begin{flushright}
      $\bullet$
    \end{flushright}\xspace
}
\newcommand{\donenot}{
    \begin{flushright}
      $\ldots$
    \end{flushright}\xspace
}
\newcommand{\noresult}{
    \begin{flushright}
      $\varnothing$
    \end{flushright}\xspace
}

\newcommand{\progress}{
    \begin{flushright}
      $\Delta$
    \end{flushright}\xspace
}

\newcommand{\topicend}{
      $\blacktriangleleft$
}

\newcommand{\meineChance}{\textit{Meine Chance}\xspace\index{Meine Chance}}

\newcommand{\anweisungArbeitsblatt}{Löse für $x$ auf kariertem Papier. Achte auf eine gut lesbare/ nachvollziehbare Notation und Aufteilung. Nummeriere die Lösungen wie auf dem Arbeitsblatt und gibt sie mit dem Arbeitsblatt ab. Wenn Du die Aufgaben in einer Gruppe bearbeitest notiere dennoch selbst die Lösungen und gibt sie ab, so dass die Dozenten Tipps zur Notation geben können. In diesem Fall gibt bitte zusätzlich zu Deinem Namen an wer in der Gruppe zusammen gearbeitet hat.}

\tikzstyle{every node}=[font=\small]
\tikzstyle{every node}=[inner sep=1pt]
\def\mcr{\pgfmatrixcurrentrow}\def\mcc{\pgfmatrixcurrentcolumn}
\def\width{12}
\def\hauteur{12}

\definecolor{rosso}{RGB}{220,57,18}
\definecolor{giallo}{RGB}{255,153,0}
\definecolor{blu}{RGB}{102,140,217}
\definecolor{verde}{RGB}{16,150,24}
\definecolor{viola}{RGB}{153,0,153}

\makeatletter

\pgfdeclarelayer{background}
\pgfdeclarelayer{foreground}
\pgfsetlayers{background,main,foreground}


\newcommand{\pie}[3][]{
    \begin{scope}[#1]
    \pgfmathsetmacro{\curA}{90}
    \pgfmathsetmacro{\r}{1}
    \def\c{(0,0)}
    \node[pie title] at (90:1.3) {#2};
    \foreach \v/\s in{#3}{
        \pgfmathsetmacro{\deltaA}{\v/100*360}
        \pgfmathsetmacro{\nextA}{\curA + \deltaA}
        \pgfmathsetmacro{\midA}{(\curA+\nextA)/2}

        \path[slice,\s] \c
            -- +(\curA:\r)
            arc (\curA:\nextA:\r)
            -- cycle;
        \pgfmathsetmacro{\d}{max((\deltaA * -(.5/50) + 1) , .5)}

        \begin{pgfonlayer}{foreground}
        \path \c -- node[pos=\d,pie values,values of \s]{$\v\%$} +(\midA:\r);
        \end{pgfonlayer}

        \global\let\curA\nextA
    }
    \end{scope}
}

\newcommand{\legend}[2][]{
    \begin{scope}[#1]
    \path
        \foreach \n/\s in {#2}
            {
                  ++(0,-10pt) node[\s,legend box] {} +(5pt,0) node[legend label] {\n}
            }
    ;
    \end{scope}
}


\theoremstyle{definition}
%\theorembodyfont{\small}
\newtheorem{definition}{Definition}
\newtheorem{uebung}{Übung}
\newtheorem{beispiel}{Beispiel}

\newglossary*{symbols}{Symbolverzeichnis}

\lstset{
    language={java},
    basicstyle=\ttfamily\footnotesize,
    breaklines=true,
    numbers=left,
    stepnumber=5,
    numberstyle=\tiny\color{gray},
    mathescape=true,
    showstringspaces=false
}

\newglossaryentry{CAS}{
    name={Computeralgebrasystem},
    description={\url{https://de.wikipedia.org/wiki/Computeralgebrasystem}: ''Ein Computeralgebrasystem (CAS) ist ein Computerprogramm, das der Bearbeitung algebraischer Ausdrücke dient. Es löst nicht nur mathematische Aufgaben mit Zahlen (wie ein einfacher Taschenrechner), sondern auch solche mit symbolischen Ausdrücken (wie Variablen, Funktionen, Polynomen und Matrizen).''
    }
}

\newglossaryentry{GGT}{
    name={ggT},
    description={Größter gemeinsamer Teiler.}
}

\newglossaryentry{KGV}{
    name={kgV},
    description={Das kleinste Gemeinsame Vielfache einer Zahl.}
}

\newglossaryentry{SISystem}{
    name={SI-System},
    description={\enquote{Das Internationale Einheitensystem oder SI (frz. Système international d’unités) ist das am weitesten verbreitete Einheitensystem für physikalische Größen.} (\url{https://de.wikipedia.org/wiki/Internationales_Einheitensystem}, abgerufen 2017-09-13 15:27)}
}

\newglossaryentry{TRS}{
    name={Term Replacement System (TRS)},
    description={
        \enquote{In mathematics, computer science, and logic, rewriting covers a wide range of (potentially non-\-de\-term\-inistic) methods of replacing subterms of a formula with other terms. What is considered are rewriting systems (also known as rewrite systems, rewrite engines or reduction systems). In their most basic form, they consist of a set of objects, plus relations on how to transform those objects. (\url{https://en.wikipedia.org/wiki/Rewriting}, abgerufen 2017-09-17 14:15)}
    }
}

\newglossaryentry{Uhrzeigersinn}{
    name={Uhrzeigersinn},
    description={
        \enquote{im Uhrzeigersinn} und \enquote{gegen den Uhrzeigersinn} bezeichnet die Richtung in der ein Rundweg in der Ebene betrachtet wird. Im Uhrzeigersinn ist die Richtung die die Zeiger einer Uhr beschreiben, gegen den Uhrzeigersinn die entgegengesetzte Richtung.
    }
} 
\newglossaryentry{symb:Abrunden}{
    type=symbols,
    name={$\left\lfloor x \right\rfloor$},
    description={Abrunden. $\left\lfloor 1,99 \right\rfloor = 1$.},
    sort={abrunden}
}

\newglossaryentry{symb:alpha}{
    type=symbols,
    name={$\alpha$},
    description={Der griechische Buchstabe $\alpha$, sprich \enquote{alpha} bezeichnet insbesondere den Winkel am Punkt $A$ in Polygonen.},
    sort={alpha}
}

\newglossaryentry{symb:Area}{
    type=symbols,
    name={$A_F$},
    description={Flächeninhalt (engl. area) einer Fläche $F$. Wir ersetzen $F$ für elementare Flächen durch den ersten Buchstaben des Namens der Fläche.},
    sort={A}
}

\newglossaryentry{symb:Aufrunden}{
    type=symbols,
    name={$\left\lceil x \rceil$},
    description={Abrunden. $\left\lfloor 1,99 \right\rfloor = 1$.},
    sort={abrunden}
}

\newglossaryentry{symb:beta}{
    type=symbols,
    name={$\beta$},
    description={Der griechische Buchstabe $\beta$, sprich \enquote{beta} bezeichnet insbesondere den Winkel am Punkt $B$ in Polygonen.},
    sort={beta}
}

\newglossaryentry{symb:delta}{
    type=symbols,
    name={$\delta$},
    description={Der griechische Buchstabe $\delta$, sprich \enquote{delta} bezeichnet insbesondere den Winkel am Punkt $D$ in Polygonen.},
    sort={delta}
}

\newglossaryentry{symb:Div}{
    type=symbols,
    name={$\div$},
    description={Symbol für die Division. Gelegentilich wird \enquote{/} benutzt, insbesondere in Texten in denen keine Brüche gesetzt werden können sowieo auf vielen Taschenrechnern.
    },
    sort={Division}
}

\newglossaryentry{symb:gamma}{
    type=symbols,
    name={$\gamma$},
    description={Der griechische Buchstabe $\gamma$, sprich \enquote{gamma} bezeichnet insbesondere den Winkel am Punkt $C$ in Polygonen.},
    sort={gamma}
}

\newglossaryentry{symb:Gleich}{
    type=symbols,
    name={$=$},
    description={
        Wir sagen \enquote{gleich} oder \enquote{ist gleich} und bezeichnen damit die Gleichheit der seiten einer Gleichung, insbesondere die Gleichheit eines zu berechnenden Ausdrucks und des Ergebnisses der Rechnung. $2-1=1$, \enquote{zwei minus eins (ist) gleich eins.}
        },
    sort={gleich}
}

\newglossaryentry{symb:Lambda}{
    name={$\lambda$},
    description={Eine beliebige Zahl, mit der der nachfolgende Ausdruck multipliziert wird.},
    type=symbols,
    sort={lambda}
}

\newglossaryentry{symb:Mal}{
    type=symbols,
    name={$\cdot$},
    description={Symbol für die Multiplikation. Gelegentlich wird, insbesondere für die Lesbarkeit $\times$ benutzt},
    sort={mal}
}

\newglossaryentry{symb:Minus}{
    type = symbols,
    name={$-$},
    description={Wir sagen \enquote{minus} und bezeichnen damit die Operation der Subtraktion. $2-1=1$, \enquote{zwei minus eins (ist) gleich eins.}},
    sort={minus}
}

\newglossaryentry{symb:N}{
    type=symbols,
    name={$\mathbb{N}$},
    description={Die Menge der Natürlichen Zahlen $\{0, 1, 2, 3, 4, ...\}$},
    sort={N}
}

\newglossaryentry{symb:Phi}{
    name={$\varphi$},
    description={Ein beliebiger Winkel.},
    type=symbols,
    sort={phi}
}

\newglossaryentry{symb:Pi}{
    name={$\pi$},
    description={Die Kreiszahl.},
    type=symbols,
    sort={pi}
}

\newglossaryentry{symb:Plus}{
    type=symbols,
    name={$+$},
    description={Wir sagen \enquote{plus} und bezeichnen damit die Operation der Addition. $1+2=3$, \enquote{eins plus zwei (ist) gleich drei}.},
    sort={plus}
}

\newglossaryentry{symb:Sum}{
    type=symbols,
    name={$\Sigma$},
    description={Summe},
    sort=Sigma
}

\newglossaryentry{symb:Vereinigung}{
    type=symbols,
    name={$\bigcup$},
    description={Vereinigung (von Mengen)},
    sort=Vereinigung
} 
\makeglossaries

\setlength{\parskip}{1ex}

\let\origitemize\itemize
\def\itemize{\origitemize\itemsep0pt}

%\setcounter{tocdepth}{1}
%\setcounter{secnumdepth}{0}

\makeindex

\begin{titlepage}
%\title{Meine Chance II - 2017\\Mathematik}
\centering
\title{Mathematik\\ \vspace{1cm} \normalsize \centering Konzept für den Mathematik-Unterricht in Vorbereitung für die Abschlussprüfung für den Hauptschulabschluss als Nichtschülerprüfung in Hessen, konkretisiert für das Projekt \enquote{Meine Chance} ($mc^2$)\\\vspace{3em}\footnotesize Die freie, insbesondere unentgeltliche, Nutzung im Rahmen von \enquote{Meine Chance} ist auf Dauer vom Urheber genehmigt. Sollte der Autor dauerhaft nicht erreichbar sein geht das komplette Material inklusive der \LaTeX-Sources (\url{https://github.com/itteerde/MathematikHS}) in public domain über.}

\centering
\author{Erik Itter}
\end{titlepage}


\begin{document}

\newgeometry{hmarginratio=1:1}
\maketitle
\restoregeometry
\tableofcontents

\printglossary[type=symb, style=long]

\part{Konzept}

\chapter{Heterogenität}

Ausgangspunkt für die Überlegungen zur Bedeutung von und Umgang mit Heterogenität der Teilnehmer, bezogen auf den Unterricht in Mathematik, ist zunächst \citep{Leiss2014}. Der Unterricht muss mit bewusster Binnendifferenzierung arbeiten, die über das typische Maß deutscher Schulklassen hinaus geht. Die Herausforderungen im Unterricht in einer Gruppe von Migranten mit unterschiedlichen sprachlichen Niveaus kommt dazu die besondere Problematik der Sprachbarriere hinzu, die aber für den Unterricht in den ersten Monaten keine unüberwindbare Hürde darstellt, da das Material gut visuell gestützt vermittelt werden kann. Entsprechend können Textaufgaben in den ersten Monaten nicht genutzt werden. Die Modellierung muss damit, zumindest als explizites Thema, Richtung Ende der Maßnahme rutschen - auch wenn das u.a. hinsichtlich der Motivation der Themen des Mathematik-Unterrichts nicht ideal ist.

Eine strukturelle Vereinfachung, die jedoch eher zu Lasten der Lernenden gehen wird, ist, dass die Prüfungen als Nichtschüler mit sich bringen, dass die Bewertung (und das Bestehen) alleine von Kompetenzerwerb und -nachweis abhängen und es kaum\footnote{Theoretisch keinen, praktisch durch Berücksichtigung des Hintergrundes in der mündlichen Prüfung.} Raum für eine Würdigung der Fortschritte gibt.

Der organisatorische Schwerpunkt zur Berücksichtigung der Besonderheiten wird auf Tutorien liegen, die weitgehend kooperative Lernumgebung bieten sollen, als Ergänzung zum klassischen vortragenden und diskutierenden Unterricht (Zu den Gründen vgl. \citep[S.7ff]{Leiss2014}, besonders betont sei hier die Eigenverantwortung der Lernenden, die im klassischen Setting leicht an die Lehrperson zurückgegeben werden kann.).

Gerade in einer Lern- und Lehrsituation hoher Heterogenität kann eine systematische Beobachtung und Dokumentation wertvoll sein um im Team der Lehrenden einander ergänzende Maßnahmen durchzuführen. Dokumentation erleichtert u.a. individuelle Schwächen aufzudecken, die im Laufe der Maßnahme zu korrigieren sind. Vorerst ist das Werkzeug dafür der Modulplan der für den Abschluss vorausgesetzten Kompetenzen, zusammenfassend jedem Teilnehmer als Lernprotokoll/ Fortschrittskontrolle ausgehändigt, sowie für die laufende Vorbereitung des Unterrichts als \textit{Microsoft Excel} spreadsheet für den gesamten Jahrgang fortwährend geführt.


\chapter{Tutorium, Lern- und Übungsgruppen}

Kaum jemand kann Mathematik gut alleine lernen. Mathematik so zu lernen, dass man sie operativ kompetent anwenden kann, erfordert i.d.R. dass man die Inhalte selbst erklären kann. So lange das nicht gelingt fehlen üblicherweise Aspekte des Sachverhalts. Schon deshalb rein egoistisch ist es für alle Teilnehmer sinnvoll sich gegenseitig die Inhalte beizubringen, sie gemeinsam einzuüben und zu vertiefen und sie für andere zu präsentieren/ referieren. Jeder Teilnehmer muss an einem Tutorium teilnehmen, das als betreute Lerngruppe verstanden wird. Verglichen mit dem universitären Lehren der Mathematik handelt es sich eher um Lerngruppen. Wir nennen es Tutorien, um es von überhaupt nicht betreutem Lernen in eigener Regie der Teilnehmer zu unterscheiden.


\section{Lehrerinterventionen}

Da die meisten Teilnehmer noch nie in Gruppen gemeinsam neues Wissen systematisch erschlossen haben ist von Seiten der Dozenten vor allem wichtig gegenzusteuern wenn die Gruppen fragmentieren, insbesondere wenn Einzelne die Gruppenarbeit aufgeben und alleine weiter arbeiten (und natürlich noch mehr wenn eine Mehrheit einer Gruppe eine Minderheit von der Arbeit ausschließt. Bei den Interventionen ist hilfreich die Zwecke von Lerngruppen zu erläutern.


\section{Mathematisches Modellieren}

Das Modellieren (einfacher) Probleme sollte von Anfang an geübt werden, so dass textuell gestellte Probleme mit den Inhalten des Unterrichts verbunden werden, die sonst häufig nur als Rechenvorschriften gelernt werden, so dass Textaufgaben in der Prüfung nicht bewältigt werden können, geschweige denn als Basis für wahrscheinliche weitergehende Kontakte mit der Mathematik, insbesondere in der Berufsausbildung, nicht nur aber vor allem in technischen Berufen.


\part{Lehrinhalte}

\chapter{Mathematik?}

\section{Was ist Mathematik?}

Die \textbf{Mathematik} im Hauptschulabschluss\footnote{Für den Realschulabschluss kommt schon einiges zu Funktionen, auch wenn man noch ohne das durch die Prüfung kommt.} verstehen wir als \textbf{Zählen}\index{zählen}, \textbf{Messen}\index{messen}, (elementares) \textbf{Rechnen}\index{rechnen} und die \textbf{Kommunikation} dessen.

Während die Schule in der Tat häufig auf das Rechnen konzentriert ist, ist die Aufgabe der Mathematik eine weiter gehende (für die allerdings rechnen zu können Voraussetzung ist). Die Mathematik erlaubt uns Probleme der Wirklichkeit zu lösen indem wir sie abstrahieren und in dem Modell eine Lösung finden, die wir zurück auf die Realität übertragen können und die in der Weise erfolgreich ist, dass sie einen Vorgang beschreiben kann inklusive Vorhersagen für die Zukunft. Eine frühe Erfolgsgeschichte der Mathematik ist zum Beispiel zukünftige Positionen der Planeten des Sonnensystems vorherzusagen, Sonnenfinsternisse, Mondphasen, Jahreszeiten und so weiter. Die selbe Mathematik, die für diese astronomischen Berechnungen benutzt wird erlaubte viel später Flugbahnen, zunächst von Artillerie-Geschossen, später auch der Apollo-Missionen, die 1969 Menschen ermöglicht hat auf dem Mond zu gehen, fahren, Mondgestein zur Erde zurück zu bringen usw. Auch die gesamte Informatik, also alles was mit Computern und ihren Berechnungen zu tun hat - inklusive der zugrundeliegenden Theorie - ist eine Abspaltung der Mathematik, die so relevant geworden ist, dass wir ihr einen eigenen Namen gegeben haben.

Wir konzentrieren uns stark auf die Arten von Aufgaben, die in den Prüfungen gestellt werden, so dass die Teilnehmer ihre Abschlüsse schaffen. Darüber hinaus bereiten wir diejenigen, die nicht voll ausgelastet sind mit der Mathematik des Hauptschulabschlusses, auch darauf vor in den Jahren nach dem Abschluss etwas schwierigere Probleme bewältigen zu können. Viele Teilnehmer an \meineChance werden nach dem Hauptschulabschluss eine technische Ausbildung machen, die sehr viel mehr Mathematik benötigt im Abschluss der Berufsausbildung und an der Berufsschule als der Hauptschulabschluss und werden in der Praxis auch hin und wieder selbst Modelle modellieren müssen und mit ihrer Berechnung Entscheidungen im Beruf treffen. Auch daraufhin wollen wir ein Stück weit vorbereiten durch Beispiele aus dem Alltag und möglicher beruflicher Zukunft.


\section{Wieso müssen wir Mathematik lernen?}

\begin{enumerate}
  \item Mathematik erlaubt es sicheres Wissen zu erlangen, z.B. sicher sein zu können dass ein Vertrag billiger ist als ein anderer.
  \item Mathematik erlaubt es zwingend zu argumentieren, z.B. nachzuweisen dass ein Angebot eines Konkurrenten günstiger ist (und der Verkäufer dem man das zeigt vielleicht auf diesen Preis herunter geht).
  \item \enquote{Mathematiker sind intelligent.} Mal davon abgesehen dass das stimmt\footnote{... aber ziemlich unklar ist was das bedeutet über das Abschneiden im Intelligenztest hinaus...} lässt sich immer wieder feststellen dass wer eine solide Basis mathematischer Kompetenzen zeigt für intelligent gehalten wird - mit entsprechend vorteilhaften Folgen z.B. hinsichtlich beruflicher Karriere. Eine Abwandlung ist die Annahme von vielen Personalverantwortlichen, dass wer sich durch die Mathematik durchgekämpft hat mit guten Ergebnissen auch alles lernen kann was man (in einer nicht handwerklichen) beruflichen Tätigkeit braucht.
  \item Mathematik macht Spaß. (WTF?) Leider können wir mit der schulischen Mathematik interessante Themen, die Spaß machen, nur gelegentlich streifen/ andeuten. Der ein oder andere wird aber vielleicht später einmal \enquote{richtige} Mathematik betreiben (können oder müssen..?), die in der Tat viel Freude machen kann.
  \item Mathematik ist die Sprache aller (Natur-) Wissenschaften und darüber hinaus auch in fast allen anderen Wissenschaften massiv im Vormarsch - ausgenommen vielleicht die Theologien.
  \item Ohne elementare Kompetenzen der Mathematik kann man fast keinen Beruf ausüben. Auch hier seien nochmals Verträge erwähnt und die zugrunde liegenden Angebote und Rechnungen, wobei letztere nicht umsonst \enquote{Rechnungen} genannt werden - sie berechnen die Kosten.
\end{enumerate}


\chapter{Mengen}

Wir machen Mengen nicht zu einem großen zentralen Thema, da sie meistens nur mittelbar für die Prüfung relevant sind. Wir benutzen sie aber bei Erklärungen und auch bei Definitionen und führen sie deshalb (nicht rigoros) ein.

\begin{definition}[Menge]\index{Menge}
    Wir nennen Eine \enquote{Ansammlung/ Sammlung} von \enquote{Dingen} eine Menge. Die Dinge in einer Menge nennen wir die \enquote{Elemente} der Menge. Wir schreiben $m \in \mathbb{M}$ falls das Element $m$ in der Menge $\mathbb{M}$ enthalten ist und $m \notin \mathbb{M}$ falls das Element $m$ nicht in der Menge $\mathbb{M}$ enthalten ist.
\end{definition}


\chapter{Zahlen}

\section{Natürliche Zahlen ($\mathbb{N}$)}\index{natürliche Zahlen}

\textbf{\glsdisp{symb:N}{Natürliche Zahlen}} benutzen wir zum Zählen. Ein Apfel, zwei Äpfel, drei Äpfel, vier Äpfel, ... $\mathbb{N} = \{ 0$\footnote{Viele Mathematiker betrachten die Null nicht als natürliche Zahl. Für uns spielt keine Rolle ob Null eine natürliche Zahl ist oder nicht.}\textsuperscript{,}\footnote{Wir verzichten auf eine Definition und gehen davon aus, dass Beispiele ausreichen und spätestens mit der Teilbarkeit Klarheit intuitiv (natürlich!) entsteht.}$, 1, 2, 3, 4, 5, 6, 7, 8, 9, ...\}$.


\section{Teilbarkeit}\index{Teilbarkeit}\label{def:teilbarkeit}

\begin{definition}[Teilbarkeit]
    Wir sagen dass eine natürliche Zahl $a$ ($a \neq 1, a \neq b$)\footnote{Wir nennen 1 und die Zahl selbst \textbf{triviale} Teiler und führen sie nicht auf und schließen sie aus der Menge der Teiler aus, weil sie bei \textbf{jeder} Zahl auftauchen und damit nur die Lesbarkeit verschlechtern ohne Information zu gewinnen.} eine natürliche Zahl $b$ \textbf{teilt} wenn $b \div a$ eine natürliche Zahl ist. Wir sagen \enquote{a teilt b} und schreiben \enquote{\glsdisp{symb:teilt}{$a | b$}} und \enquote{a teilt nicht b} und schreiben \glsdisp{symb:teiltnicht}{\enquote{$a \nmid b$}}.
\end{definition}

\begin{beispiel}[Teilbarkeit]
    $2 | 4$, $1 | 2$, $3 \nmid 4$, $3 | 9$, $3 \nmid 7$, $3 | 27$, $15 | 60$, $3 \nmid 10$, $10 | 10$, $ 5 \nmid 7$, $16 | 32$.
\end{beispiel}


\subsection{Größter gemeinsamer Teiler (ggT)}\index{ggT}

\begin{definition}
    Wir nennen die größte Zahl, die alle Zahlen einer Menge \textbf{teilt}, den \enquote{\textbf{\glsdisp{GGT}{größten gemeinsamen Teiler} (ggT)}} der Menge.
\end{definition}

\begin{beispiel}[ggT]
    Die Teiler von 60 sind $\text{T}_{60} = \{2,3,4,5,6,10,12,15,20,30\}$ und die Teiler von 40 sind $\text{T}_{40}\{2,4,5,8,10,20\}$. Die \textbf{gemeinsamen Teiler} von 60 und 40 sind $\text{gT}_{\{60,40\}} = \text{T}_{60} \bigcup \text{T}_{40} = \{2,4,5,10,20\}$. Der \textbf{\glsdisp{GGT}{größte gemeinsame Teiler (ggT)}} von 60 und 40 ist das größte \textbf{Element} von $\text{gT}_{\{60,40\}}$: $\text{ggT}_{\{60,40\}}=20$.
\end{beispiel}


\subsection{kgV}\index{kgV}



\section{Primzahlen ($\mathbb{P}$)}\index{Primzahlen}

Eine natürliche Zahl $n > 1$\footnote{Wieso wir die Eins nicht \textbf{Primzahl} nennen braucht uns hier nicht zu interessieren, weder für den Hauptschulabschluss noch für den Realschulabschluss.}, die nur durch 1 und sich selbst \textbf{teilbar} ist, nennen wir eine \textbf{Primzahl}. Die ersten paar Primzahlen lernen wir auswendig um die \textbf{Primfaktoren} einer Zahl schneller zu finden, die wir in der Bruchrechnung brauchen: $\mathbb{P}=\left\{2, 3, 5, 7, 11, 13, 17, 19, 23, 29, ...\right\}$.

\begin{beispiel}
    Die \textbf{Primfaktorzerlegung}\index{Primfaktorzerlegung} von $6$ ist $\{2,3\}$, die von $120$ ist $\{2,2,2,3,5\}$. Wir schreiben $\{2,2,2,3,5\}$ auch als $\{2^3,3^1,5^1\}$ oder $\{2^3,3,5\}$.\topicend
\end{beispiel}


\section{Primfaktorzerlegung}\index{Primfaktorzerlegung}

Jede natürliche Zahl $n$ kann (in einer eindeutigen Weise) in \textbf{Primfaktoren}\index{Primfaktor} zerlegt werden (\textbf{Faktorisierung}). 6 ist \textbf{teilbar} durch 2 und 3. 1 und die Zahl selbst zählen hier nicht.

\begin{equation}
    6 = 2 \cdot 3
\end{equation}

Die Reihenfolge der \textbf{Faktoren} ist egal. $3 \cdot 2$ und $2 \cdot 3$ werden als die selbe \textbf{Zerlegung} (\textbf{Primfaktorzerlegung}) betrachtet.

\begin{equation}
    60 = 2 \cdot 2 \cdot 3 \cdot 5
\end{equation}


\section{Ganze Zahlen ($\mathbb{Z}$)}\index{ganze Zahlen}

Die Menge der ganzen Zahlen $\mathbb{Z}$ besteht aus den natürlichen Zahlen und die negativen Zahlen gleichen Betrages: $\mathbb{Z}=\{..., -2, -1, 0, 1, 2, 3, ...\}$. Die Ordnungsrelation \enquote{$<$}\glsadd{symb:Kleiner}\index{$<$} sprechen wir \enquote{kleiner als}. \enquote{$1<2$} ist eine wahre Aussage; \enquote{$2<1$} ist eine falsche Aussage. Die Ordnungsrelation \enquote{$>$}\glsadd{symb:Groeszer}\index{$>$} sprechen wir \enquote{größer als}. \enquote{$2>1$} ist eine wahre Aussage; \enquote{$1>2$} ist eine falsche Aussage.


\section{Rationale Zahlen ($\mathbb{Q}$)}\index{rationale Zahlen}

Die Menge der rationalen Zahlen $\mathbb{Q}$ besteht aus den Zahlen der Form $\frac{a}{b}, a, b \in \mathbb{Z}, b \neq 0$. Wir nennen diese Zahlen (insbesondere in dieser Form aufgeschrieben) Brüche.


\section{Reelle Zahlen ($\mathbb{R}$)}\index{reelle Zahlen}

Die Menge der reellen Zahlen $\mathbb{R}$ \enquote{besteht aus allen Stellen auf dem Zahlenstrahl\index{Zahlenstrahl}}\footnote{Eine rigorose Einführung/ Definition ist für alle Schulabschlüsse überflüssig und wird in naturwissenschaftlichen Studiengängen im ersten Semester an der Universität unterrichtet.}. Wenn wir nicht ausdrücklich eine andere Domäne angeben betrachten wir alle Zahlen mit denen wir arbeiten als reelle Zahlen.


\section{Rundungen}\index{Rundung}

Wir können Zahlen \textbf{runden}\index{runden}. Wir sprechen von \enquote{runden auf $n$ Stellen (nach dem Komma)} oder \enquote{runden auf ganze Tausend, Millionen, ...}. Zu runden bedeutet die nächste Stelle nach der gewünschten Genauigkeit zu betrachten und bei Ziffern kleiner als 5 \textbf{abzurunden} und ab 5 \textbf{aufzurunden}. Oft legt die Bedeutung der Zahl (i.d.R. mit Maßeinheit) eine bestimmte Rundung nahe. Z.B. wird man €-Werte fast immer auf zwei Nachkommastellen gerundet erwarten (auf Cent genau). Wollen wir kennzeichnen dass eine Zahl nicht mehr genau/ exakt ist, sondern (mit bestimmter Präzession) gerundet, schreiben wir statt $=$, \enquote{(ist) gleich}, \glsdisp{symb:Ungefaehr}{$\approx$}, \enquote{(ist) ungefähr (gleich)} oder \enquote{(ist) rund}.


\section{Numeralia (Zahlwörter)}\label{numeralia}

\enquote{eins, zwei, drei, vier, fünf, sechs, sieben, ...} ist der Beginn der Zahlwörter (als Kardinalzahlen). \citep[Randzahl 509]{DudenGrammatik2016} gibt eine unvollständige Einleitung. Demnach sind sowohl Schreibweisen als auch gesprochene Zahlen nicht (mehr) streng festgelegt. Eine veraltete Regel lautet dass natürliche Zahlen bis zwölf als Zahlwort geschrieben werden, darüber hinaus als Ziffernfolge. Es bleibt das Problem Zahlwörter für größere Zahlen festzulegen, so dass größere Zahlen gesprochen werden können. Die Numeralia von 0 bis 12 sind \enquote{eins, zwei, drei, vier, fünf, sechs, sieben, acht, neun, zehn, elf, zwölf}. Für Zahlen mit Zehnerstellen\footnote{Mit \enquote{echten} Zehnerstellen, also keine Null in der vorletzten Ziffer der Zahl.} werden die Zehner-Suffixe benötigt, diese sind \enquote{-zehn, -zwanzig, -dreißig, -vierzig, -fünfzig, -sechzig, -siebzig, -achtzig, -neunzig}. Für die Hunderter-Stellen werden die Numerale von 1 bis 9 vor das Numeral \enquote{hundert} gesetzt. Statt \enquote{eins} ist das Numeral hier \enquote{ein}. Sind Hunderter vorhanden wird zwischen ihnen und den Zehnern \enquote{und} eingefügt: \enquote{eins, zwei, drei, vier, fünf, sechs, sieben, acht, neun, zehn, elf, zwölf, dreizehn, vierzehn, ..., neunzehn, zwanzig, einundzwanzig, zweiundzwanzig, ..., neunundzwanzig, dreißig, einunddreißig, ..., neunundneunzig, einhundert, einhundertundeins, ..., einhundertundneunundneunzig\footnote{\cite{DudenGrammatik2016} erkennt ausdrücklich auch hundertneunundneunzig als korrekt an.}, zweihundert, zweihundertundeins, ... neunhundertundneunundneunzig}.

Bei Zahlen ab 1.000 (\enquote{eintausend}) werden die Numeralia in 3er-Paketen (Tripeln) gebildet. Das kleinste Tripel wird mit \enquote{und} verknüpft angehängt. 123.456 wird \enquote{einhundertunddreiundzwanzigtausendundvierhundertundsechsundfünfzig} gelesen\footnote{\citep{DudenGrammatik2016} erkennt hier gleich mehrere unterschiedliche Varianten als korrekt an.}. Es wird zunächst das höchstwertige Tripel gebildet, so dass alle noch folgenden Teile der Zahl echte (dreistellige) Tripel sind. Also wird für 23456 456 abgetrennt und das (unvollständige) höchstwertige Tripel 23 gelesen: \enquote{dreiundzwanzig}. Angehängt wird der Name der Größenordnung (für das vorletzte Tripel 1.000 ($10^3$), \enquote{eintausend}). Danach wird das nächstkleinere Tripel (hier das letzte, 456 gelesen, \enquote{vierhundertundsechsundfünfzig}. Zahlen unter einer Million werden zusammen geschrieben, so dass 23.456 \enquote{dreiundzwanzigtausendundvierhundertundsechsundfünfzig}\footnote{Korrekt wäre u.a. auch \enquote{dreiundzwanzigtausendvierhundertsechsundfünfzig}, aber die Varianten mit weniger \enquote{und} brauchen mehr Bildungsregeln.} heist. Die Namen der ersten Größenordnungen sind (Singular/Plural) $10^3$: tausend/tausend, $10^6$: Million/Millionen, $10^9$: Milliarde/Milliarden, $10^{12}$: Billion/Billionen, $10^{15}$: Billiarde/Billiarden. Für dem Alltag sollten die Namen bis zu den Milliarden bekannt sein (z.B. Bundeshaushalt).

Es ergeben sich z.B. $1.123.456.789.123$: \enquote{eine Billion einhundertunddreiundzwanzig Milliarden vierhundertundsechsundfünfzig Millionen siebenhundertundneunundachtzigtausendundeinhundertunddreiundzwanzig} und $89.001.456.901$: \enquote{neunundachtzig Milliarden eine Million vierhundertundsechsundfünfzigtausendundneunhunderundeins}. Solche Numeralia werden nie geschrieben, die in Ziffern geschrieben Zahl muss aber zumindest bis in den Bereich der Milliarden vorgelesen/ ausgesprochen werden können. Die hier angegebene Variante (mit maximalen \enquote{und}-Verbindungen) scheint die Variante mit den wenigsten Bildungsregeln aus den korrekten Varianten zu sein. Scheinen Stellen nach der größten Größenordnung zu fehlen (Nullen) werden diese nicht gesprochen: $999.999.999$, $1.000.000.000$, $1.000.000.001$ wird \enquote{neunhundertundneunundneunzig Millionen neunhunderundneunungneunzigtausendundneunhundertundneunundneunzig, eine Milliarde, eine Milliarde und eins} gelesen.

\begin{uebung}[Numeralia (Zahlwörter)]
    Schreibe die Numeralia auf für 1 bis 21, 31, 42, 53, 64, 72, 84, 93, 99, 100, 101, 199, 200, 201, 999, 1.000, 1.001, 9.000, 9.999, 123.456, 200.000, 999.999, 1.000.000, 10.000.001, 999.999.999, 1.000.000.000, 1.000.000.001 und 123.456.789.012.345.\topicend
\end{uebung}

\section{Variablen (Unbekannte/ Veränderliche)}\index{Variablen}\index{Unbekannte|see{Variable}}\index{Veränderliche|see{Variable}}

In der Mathematik wollen wir (insbesondere in \textbf{Formeln}) möglichst \textbf{allgemeine} Aussagen treffen um möglichst viel mit diesen anfangen zu können. Oftmals benutzen wir dazu keine Zahlen, sondern Namen wie \enquote{a}, \enquote{b} und \enquote{c} oder \enquote{x} und \enquote{y}. Diese Namen stehen immer für Zahlen. Man kann sie durch beliebige Zahlen ersetzen (aber in einer Rechnung immer durch die Selbe). Die Fläche eines Rechtecks nennen wir z.B. allgemein $a \cdot b$, oder \enquote{Länge mal Breite}. Jeder (positive) Wert ergibt ein Rechteck, dessen Fläche wir so berechnen können: $A_{\text{Rechteck}} = a \cdot b = A_R = a b$\footnote{\enquote{A} steht für \enquote{area}, englisch für \textbf{Fläche/ Flächeninhalt}}.

\begin{beispiel}
    Gegeben sei ein Rechteck mit der Länge $a = 12 \text{cm}$ und Breite $b = 5 \text{cm}$. Zu berechnen sei der Flächeninhalt des Rechtecks $A_R = a b$.
    \begin{align}
      A_R & =  a b && a \rightarrow 12 cm, b \rightarrow 6 cm\\
       & =  12 cm \cdot 6 cm &&\\
       & =  60 cm^2 &&
    \end{align}

    Es ist weitgehend eine sehr gute Weise an die Schulmathematik heran zu gehen indem man \enquote{Rechnen} als \glsdisp{TRS}{Term Replacement System} (System von Ersetzungsregeln/ systematisches Einsetzen) versteht.\topicend
\end{beispiel}

Genau genommen ersetzen wir wie im Beispiel zu sehen, in diesem Fall mit einer reellen Zahl und einer Maßeinheit. Diese Feinheiten können wir jedoch zunächst ignorieren und in den kommenden Monaten intuitiv durch Üben lernen.


\chapter{Elementare Rechenoperationen}

\enquote{$1+2=3$}\glsadd{symb:Plus}\glsadd{symb:Minus}\glsadd{symb:Gleich} sprechen wir \enquote{eins plus zwei (ist) gleich drei}. \textbf{Plus} zu rechnen nennen wir \textbf{addieren} oder \textbf{die Addition}. \enquote{$3-2=1$} sprechen wir \enquote{drei minus zwei (ist) gleich 1}. \textbf{Minus} zu rechnen nennen wir \textbf{subtrahieren} oder \textbf{die Subtraktion}. \textbf{Das Ergebnis} einer \textbf{Subtraktion} nennen wir \textbf{die Differenz}. Wir sagen auch \enquote{die Differenz von drei und zwei ist eins}\footnote{und umgangssprachlich/ alltagssprachlich auch \enquote{die Differenz von zwei und drei ist eins (obwohl $2-3=-1$). Hierbei kommt die Vorstellung der \textbf{Differenz} als \textbf{Abstand} am \textbf{Zahlenstrahl} $\mathbb{R}^1$ zum intuitiven Ausdruck}}.

\section{Addition und Subtraktion am Zahlenstrahl}

Wir verstehen das Rechnen mit \glsdisp{symb:Plus}{$+$} und \glsdisp{symb:Minus}{$-$} als Bewegung auf dem Zahlenstrahl der reellen Zahlen $\mathbb{R}$\footnote{Der Zahlenstrahl bringt einen natürlichen Weg zu $\mathbb{R}$ mit, da eine kontinuierliche \enquote{Strecke} schlecht in diskrete Abschnitte aufgeteilt sein kann elementar, da man jedes Stück wieder teilen kann ins Unendliche. (Dass vielleicht die Realität gar keine kontinuierlichen Räume beinhaltet wird die Lernenden sicher nicht stören und sei mit einem Hinweis auf die breite Anwendbarkeit der Analysis beiseite gelegt.)}. Von den Lehrenden aus wird das Berechnen von Summen als bekannt vorausgesetzt.

\begin{equation}\label{eqn:00001}
    1 + 2 + 3
\end{equation}

\begin{figure}[H]
  \centering
\begin{tikzpicture}[>=triangle 60]

    \draw (1,1) -- +(10,0);

    \foreach \x in {1,...,11}{
        \pgfmathtruncatemacro{\label}{\x-1}
        \draw (\x,0.8) -- +(0,0.4);
        \node[below] (xa) at (\x,0.7) {\label};
    }

    \coordinate[label={[label distance=10pt]90:1}] (1) at (2,2);
    \coordinate[label={[label distance=10pt]90:3}] (3) at (4,2);
    \coordinate[label={[label distance=10pt]90:6}] (6) at (7,2);
    \fill (1) circle (1pt);
    \fill (3) circle (1pt);
    \draw[->,shorten >=4pt] (1) -- (3) node[midway, below] {+2};
    \fill (6) circle (1pt);
    \draw[->,shorten >=4pt] (3) -- (6) node[midway, below] {+3};

\end{tikzpicture}
  \caption{$+/-$ am Zahlenstrahl $\mathbb{R}$}\label{fig:zahlenstrahlAddition}
\end{figure}

Problematisch wird das Subtrahieren von negativen Zahlen, für das man sich am Zahlenstrahl zwar mit \textquote{Drehungen} um 180° und damit $-(-x) = x$ aus der Affäre ziehen kann, das aber mathematisch so nicht vertretbar ist. Sauber wäre hier schlicht die Regel für die Äquivalenz anzugeben und die Lernenden als \gls{TRS} arbeiten zu lassen. Die Repräsentation kontinuierlicher Strecken vermeidet bei Lernenden, die bereits verfestigt zählen\index{verfestigtes Zählen} (vgl. \citep[S. 112]{Hasemann2014}) statt zu rechnen dies weiter zu bedienen, wie z.B. durch eine Darstellung $\clubsuit + \clubsuit = \clubsuit \clubsuit$ oder $\clubsuit + \clubsuit = 2 \clubsuit$\footnote{Die Darstellung $\clubsuit + \clubsuit = 2 \clubsuit$ hat ihre Berechtigung in der Algebra, wenn die Lernenden dort den abstrakten Umgang mit einer Unbekannten, $x$, erlernen.}. D.h. beim Sprechen über Arithmetik in $\mathbb{R}$ ist wennimmer möglich in Strecken zu reden und Zählen zu vermeiden.

\begin{equation}
    -1 + 5 - 3
\end{equation}

\begin{figure}[H]
  \centering
\begin{tikzpicture}[>=triangle 60]

    \draw (1,1) -- +(10,0);

    \foreach \x in {-2,...,8}{
        \pgfmathtruncatemacro{\label}{\x}
        \pgfmathtruncatemacro{\posx}{\x+3}
        \draw (\posx,0.8) -- +(0,0.4);
        \node[below] (xa) at (\posx,0.7) {\label};
    }

    \coordinate[label={[label distance=10pt]90:-1}] (-1) at (2,3);
    \coordinate[label={[label distance=10pt]90:4}] (4) at (7,3);
    \fill (-1) circle (1pt);
    \fill (4) circle (1pt);
    \draw[->,shorten >=4pt] (-1) -- (4) node[midway, below] {+5};
    \fill (6) circle (1pt);

    \coordinate[label={[label distance=10pt]90:4}] (4l) at (7,2);
    \coordinate[label={[label distance=10pt]90:1}] (1l) at (4,2);
    \fill (1l) circle (1pt);
    \fill (4l) circle (1pt);
    \draw[->,shorten >=4pt] (4l) -- (1l) node[midway, below] {-3};

\end{tikzpicture}
  \caption{$+/-$ am Zahlenstrahl $\mathbb{R}$}\label{fig:zahlenstrahlSubtraktion}
\end{figure}

\begin{definition}[Betrag]\index{Betrag}
    Wir nennen $|x|$ den Betrag (engl. abs) von $x$. Es sei $|-x| = x$, $|x|=x$ und damit $|-x|=|x|$.
\end{definition}


\begin{beispiel}[Summe mit gemischten Vorzeichen]
    \begin{equation}
        \Sigma = 12345 - 43512 + 31987 - 47614
    \end{equation}

    \begin{figure}[H]
      \centering
    \begin{tikzpicture}[>=triangle 60]

        \draw (1,1) -- +(10,0);

        \foreach \x in {-2,...,8}{
            \pgfmathtruncatemacro{\label}{\x}
            \pgfmathtruncatemacro{\posx}{\x+3}
            \draw (\posx,0.8) -- +(0,0.4);
            %\node[below] (xa) at (\posx*10000,0.7) {\label};
        }

        \node[rotate=90,anchor=center] at (6,0.6) {\small 0};
        \node[rotate=90,anchor=center] at (1,0.1) {\small -50.000};
        \node[rotate=90,anchor=center] at (2,0.1) {\small -40.000};
        \node[rotate=90,anchor=center] at (3,0.1) {\small -30.000};
        \node[rotate=90,anchor=center] at (4,0.1) {\small -20.000};
        \node[rotate=90,anchor=center] at (5,0.1) {\small -10.000};
        \node[rotate=90,anchor=center] at (7,0.1) {\small 10.000};
        \node[rotate=90,anchor=center] at (8,0.1) {\small 20.000};
        \node[rotate=90,anchor=center] at (9,0.1) {\small 30.000};
        \node[rotate=90,anchor=center] at (10,0.1) {\small 40.000};
        \node[rotate=90,anchor=center] at (11,0.1) {\small 50.000};

        \coordinate[label={[label distance=10pt]90:0}] (null) at (6,3);
        \coordinate[label={[label distance=10pt]90:12345}] (a) at (7.2345,3);
        \fill (null) circle (1pt);
        \fill (a) circle (1pt);
        \draw[->,shorten >=4pt] (null) -- (a) node[midway, below] {\tiny +12345};
        %\fill (6) circle (1pt);

        \coordinate[label={[label distance=10pt]90:-31167}] (m31167) at (2.89,2.5);
        \coordinate[label={[label distance=10pt]90:}] (123456a25) at (7.2345,2.5);
        \fill (m31167) circle (1pt);
        \fill (123456a25) circle (1pt);
        \draw[->,shorten >=4pt] (123456a25) -- (m31167) node[midway, below] {\tiny -43512};

        \coordinate[label={[label distance=10pt]90:}] (m31167a2) at (2.89,2);
        \coordinate[label={[label distance=10pt]90:820}] (820) at (6.082,2);
        \fill (m31167a2) circle (1pt);
        \fill (820) circle (1pt);
        \draw[->,shorten >=4pt] (m31167a2) -- (820) node[midway, below] {\tiny +31987};

        \coordinate[label={[label distance=10pt]90:}] (820a15) at (6.082,1.5);
        \coordinate[label={[label distance=10pt]90:-46794}] (m46794) at (1.32,1.5);
        \fill (820a15) circle (1pt);
        \fill (m46794) circle (1pt);
        \draw[->,shorten >=4pt] (820a15) -- (m46794) node[midway, below] {\tiny -47614};

    \end{tikzpicture}
      \caption{$+/-$ am Zahlenstrahl $\mathbb{R}$}\label{fig:zahlenstrahlSumme01}
    \end{figure}
\end{beispiel}



\section{Multiplikation}

Wir schreiben die \textbf{Multiplikation} von \textbf{Faktoren} $f_1 \cdot f_2$ und das Ergebnis einer \textbf{Multiplikation} das \textbf{Produkt}.\glsadd{symb:Mal}


\section{Division}

Wir schreiben die \textbf{Division} von \textbf{Faktoren}\footnote{Die Mathematik nennt das in der Tat so (und unterscheidet Multiplikation und Division kaum). Wem das zu unklar ist kann die Begriffe Dividend und Divisor benutzen. Dann bezeichnet der Dividend die zu teilende Zahl und Divisor die Zahl durch die der Dividend geteilt wird.} $f_1 \div f_2$ und bezeichnen das Ergebnis einer \textbf{Division} der \textbf{Quotient}.\glsadd{symb:Div}


\chapter{Maßeinheiten}\index{Maßeinheiten}

\section{SI-System}\index{SI-System}

Das \glsdisp{SISystem}{SI-System} wird weltweit für im technischen und wissenschaftlichen Bereich genutzt und fast überall (leider gehören die USA zu den drei Ausnahmen) alltäglich. Für den Hauptschulabschluss brauchen wir \enquote{kms}, das Kilogramm (kg) für Massen und \enquote{Gewichte}, den Meter (m) für Strecken und die Sekunde (s) für Zeitspannen. Wer das Problem der Maßeinheiten versteht findet vielleicht interessant was man bei \youtube unter \enquote{si einheiten} findet und von da aus auch das komplette System erschließen kann. Manches was durch Umrechnungen passiert ist, ist von heute aus zurück schauend auch witzig - wenn z.B. \textit{NASA} 200 Millionen Dollars in den Mars rammt weil ein Lieferant (USA...) in amerikanischen Pfund statt Kilogramm gerechnet hat (\url{https://de.wikipedia.org/wiki/Mars_Climate_Orbiter})...

\section{Präfixe für Größenordnungen}\index{Größenordnungen}\index{Präfixe|see{Größenordnungen}}

\enquote{Kilo} bedeutet 1000, \enquote{Dezi-} bedeutet $\frac{1}{10}$, \enquote{Zenti-} bedeutet $\frac{1}{100}$ (beachte, dass die Abkürzungen \enquote{c} statt \enquote{z} verwenden, da wir die englische Schreibweise übernommen haben), \enquote{Milli} bedeutet $\frac{1}{1000}$. Ein Millimeter ist also ein tausendstel Meter und ein Zentimeter ein hundertstel Meter.

\section{Der Meter, Maßeinheit für Strecken}

Mit dem Meter messen wir Entfernungen, Längen, Strecken. Alle anderen Maßeinheiten für Strecken werden durch den Meter definiert. Ein Millimeter ist ein tausendstel Meter. \enquote{Milli} ist das Präfix für $\frac{1}{1000}$. Prüfungsrelevant sind Meter, Kilometer, Dezimeter, Zentimeter und Millimeter. Mikrometer ($10^{-6}m$) und Nanometer ($10^{-9}m$) kommen gelegentlich in Nachrichten vor, z.B. als Größe von (gesundheitsschädlichen) Partikeln. Für die Prüfung relevant sind:

\begin{eqnarray*}
% \nonumber to remove numbering (before each equation)
  1mm &=& \frac{1}{1000}m \\
  1cm &=& \frac{1}{100}m \\
  1dm &=& \frac{1}{10}m \\
  1km &=& 1000m
\end{eqnarray*}

Für weniger alltäglichen Gebrauch gibt es in beide Richtungen weiter Namen für Maßeinheiten für Strecken, die sich ebenfalls jeweils auf den Meter beziehen, z.B. ist $1m = \frac{1}{9460730472580800}\text{ly}$ (Lichtjahre), d.h. ein Lichtjahr (kommt gelegentlich in den Nachrichten vor) ist somit gleich $9460730472580800\text{m}$. In der anderen Richtung sind die nächsten Präfixe \enquote{Piko} und \enquote{Femto}, so dass $1\text{m}=1000000000000000\text{fm}$.

\begin{uebung}
    Rechne um/ löse für $x$:

    \begin{equation}
        2km + 23m = x m
    \end{equation}
    \begin{equation}
        17km + 791m = x cm
    \end{equation}
    \begin{equation}
        3km + 17m + 5dm + 23cm + 2mm = x mm
    \end{equation}
\end{uebung}

\tikzset{
%Define standard arrow tip
>=stealth',
%Define style for boxes
punkt/.style={
       %rectangle,
       %rounded corners,
       %draw=black, very thick,
       text width=3em,
       minimum height=2em,
       text centered},
% Define arrow style
pil/.style={
       ->,
       thick,
       shorten <=2pt,
       shorten >=2pt,}
       }

\begin{figure}
  \centering
    \begin{tikzpicture}[node distance=1cm, auto,]
        \node[punkt] (meter) {m};
        \node[punkt, above=of meter] (dezimeter) {dm};
        \node[punkt, above=of dezimeter] (zentimeter) {cm};
        \node[punkt, above=of zentimeter] (millimeter) {mm};
        \node[punkt, above=of millimeter] (mikrometer) {$\mu$m};
        \node[punkt, above=of mikrometer] (nanometer) {nm};
        \node[punkt, below=of meter] (kilometer) {km};
        \node[punkt, below=of meter] (kilometerdummy) {}
            edge[pil,bend right=70] node[text width= 15em, anchor=west] {$\cdot 1000$} (meter.east);
        \node[punkt, above=of kilometer] (meterdummy) {}
            edge[pil,bend right=70] node[text width= 11em, anchor=west] {$\cdot 10$} (dezimeter.east)
            edge[pil,bend right=90] node[text width= 15em, anchor=west] {$\cdot 1000$} (millimeter.east)
            edge[pil,bend right=70] node[text width= 3em, anchor=east] {$\div 1000$} (kilometer.west);
        \node[punkt, above=of meter] (dezimeterdummy) {}
            edge[pil,bend right=70] node[text width= 11em, anchor=west] {$\cdot 10$} (zentimeter.east)
            edge[pil,bend right=70] node[text width= 2em, anchor=east] {$\div 10$} (meter.west);
        \node[punkt, above=of dezimeter] (zentimeterdummy) {}
            edge[pil,bend right=70] node[text width= 2em, anchor=east] {$\div 10$} (dezimeter.west)
            edge[pil,bend right=70] node[text width= 11em, anchor=west] {$\cdot 10$} (millimeter.east);
        \node[punkt, above=of zentimeter] (millimeterdummy) {}
            edge[pil,bend right=70] node[text width= 2em, anchor=east] {$\div 10$} (zentimeter.west)
            edge[pil,bend right=90] node[text width= 4em, anchor=east] {$\div 1000$} (meter.west)
            edge[pil,bend right=70] node[text width= 11em, anchor=west] {$\cdot 1000$} (mikrometer.east);
        \node[punkt, above=of millimeter] (mikrometerdummy) {}
            edge[pil,bend right=70] node[text width= 11em, anchor=west] {$\cdot 1000$} (nanometer.east)
            edge[pil,bend right=70] node[text width= 3em, anchor=east] {$\div 1000$} (millimeter.west);
        \node[punkt, above=of mikrometer] (nanometerdummy) {}
            edge[pil,bend right=70] node[text width= 3em, anchor=east] {$\div 1000$} (mikrometer.west);

        %\node[punkt, below=of kilometer] (lowerpaddingdummy) {};
        \node[punkt, left=of kilometer] (leftpaddingdummy) {};
        \node[punkt, left=of leftpaddingdummy] (leftpaddingdummy2) {};
    \end{tikzpicture}
  \caption{Umwandlungen von Strecken (m)}\label{fig:sisystemMUmwandlungenM}
\end{figure}


\section{Das Kilogramm, Maßeinheit für Massen}

\textbf{Das Kilogramm (kg)}\index{Kilogramm}\index{kg|see{Kilogramm}} ist die \textbf{Maßeinheit} des \glsdisp{SISystem}{SI-Systems} für \textbf{Massen}. In der Prüfung werden damit auch Gewichte gemessen.\footnote{Tatsächlich ist das Kilogramm die Einheit für Massen. Da alle Massen im Hauptschulabschluss auf der Erde, mehr oder minder in gleich bleibendem Abstand vom Mittelpunkt der Erde, gemessen werden können wir die Unterscheidung vernachlässigen, auch wenn das dem Physiker weh tut...}

\begin{figure}
  \centering
    \begin{tikzpicture}[node distance=1cm, auto,]
        \node[punkt] (kilogramm) {kg};
        \node[punkt, above=of kilogramm] (gramm) {g};
        \node[punkt, above=of gramm] (milligramm) {mg};
        \node[punkt, above=of milligramm] (mikrogramm) {$\mu$g};
        \node[punkt, below=of kilogramm] (tonne) {t};
        \node[punkt, below=of gramm] (kilogrammdummy) {}
            edge[pil,bend right=70] node[text width= 11em, anchor=west] {$\cdot 1000$} (gramm.east)
            edge[pil,bend right=70] node[text width= 3em, anchor=east] {$\div 1000$} (tonne.west);
        \node[punkt, above=of kilogramm] (grammdummy) {}
            edge[pil,bend right=70] node[text width= 11em, anchor=west] {$\cdot 1000$} (milligramm.east)
            edge[pil,bend right=70] node[text width= 3em, anchor=east] {$\div 1000$} (kilogramm.west);
        \node[punkt, above=of gramm] (milligrammdummy) {}
            edge[pil,bend right=70] node[text width= 3em, anchor=east] {$\div 1000$} (gramm.west)
            edge[pil,bend right=70] node[text width= 11em, anchor=west] {$\cdot 1000$} (mikrogramm.east);
        \node[punkt, above=of milligramm] (mikrogrammdummy) {}
            edge[pil,bend right=70] node[text width= 3em, anchor=east] {$\div 1000$} (milligramm.west);
        \node[punkt, below=of kilogramm] (tonnedummy) {}
            edge[pil,bend right=70] node[text width= 15em, anchor=west] {$\cdot 1000$} (kilogramm.east);

        \node[punkt, left=of kilogramm] (leftpaddingdummy) {};
        \node[punkt, left=of leftpaddingdummy] (leftpaddingdummy2) {};
    \end{tikzpicture}

  \caption{Umwandlungen von Massen (kg)}\label{fig:sisystemMUmwandlungenKG}
\end{figure}


\section{Die Sekunde, Maßeinheit für Zeitspannen}

\textbf{Die Sekunde}\index{Sekunde}\index{s|see{Sekunde}} ist die \textbf{Maßeinheit} des \glsdisp{SISystem}{SI-Systems} für Zeitspannen. Bei den Zeiteinheiten haben sich die Schritte des 60er-Stellensystem der Babylonier gehalten. In der Prüfung müssen wir mit den 60er-Schritten zwischen Sekunden, Minuten und Stunden zurechtkommen. Technisch und Wissenschaftlich rechnet man in Sekunden und den im \gls{SISystem} regelmäßigen Präfixen.

\begin{figure}
  \centering
    \begin{tikzpicture}[node distance=1cm, auto,]
        \node[punkt] (sekunde) {s};
        \node[punkt, above=of sekunde] (millisekunde) {ms};
        \node[punkt, above=of millisekunde] (mikrosekunde) {$\mu$s};
        \node[punkt, above=of mikrosekunde] (nanosekunde) {ns};
        \node[punkt, below=of sekunde] (minute) {min};
        \node[punkt, below=of minute] (stunde) {h};
        \node[punkt, below=of stunde] (tag) {d};
        \node[punkt, below=of tag] (jahr) {y};

        \node[punkt, below=of millisekunde] (sekundedummy) {}
            edge[pil,bend right=70] node[text width= 11em, anchor=west] {$\cdot 1000$} (millisekunde.east)
            edge[pil,bend right=70] node[text width= 3em, anchor=east] {$\div 60$} (minute.west);
        \node[punkt, below=of sekunde] (minutedummy) {}
            edge[pil,bend right=70] node[text width= 11em, anchor=west] {$\cdot 60$} (sekunde.east)
            edge[pil,bend right=70] node[text width= 3em, anchor=east] {$\div 60$} (stunde.west);
        \node[punkt, below=of minute] (stundedummy) {}
            edge[pil,bend right=70] node[text width= 11em, anchor=west] {$\cdot 60$} (minute.east)
            edge[pil,bend right=70] node[text width= 3em, anchor=east] {$\div 24$} (tag.west);
        \node[punkt, below=of stunde] (tagdummy) {}
            edge[pil,bend right=70] node[text width= 11em, anchor=west] {$\cdot 24$} (stunde.east)
            edge[pil,bend right=70] node[text width= 3em, anchor=east] {$\div 365$} (jahr.west);
        \node[punkt, below=of tag] (jahrdummy) {}
            edge[pil,bend right=70] node[text width= 11em, anchor=west] {$\cdot 365$} (tag.east);
        \node[punkt, above=of sekunde] (millisekundedummy) {}
            edge[pil,bend right=70] node[text width= 11em, anchor=west] {$\cdot 1000$} (mikrosekunde.east)
            edge[pil,bend right=70] node[text width= 3em, anchor=east] {$\div 1000$} (sekunde.west);
        \node[punkt, above=of millisekunde] (mikroskundedummy) {}
            edge[pil,bend right=70] node[text width= 11em, anchor=west] {$\cdot 1000$} (nanosekunde.east)
            edge[pil,bend right=70] node[text width= 3em, anchor=east] {$\div 1000$} (millisekunde.west);
        \node[punkt, above=of mikrosekunde] (nanoskundedummy) {}
            edge[pil,bend right=70] node[text width= 3em, anchor=east] {$\div 1000$} (mikrosekunde.west);

        \node[punkt, left=of sekunde] (leftpaddingdummy) {};
        \node[punkt, left=of leftpaddingdummy] (leftpaddingdummy2) {};
    \end{tikzpicture}
  \caption{Umwandlungen von Zeitspannen (s)}\label{fig:sisystemMUmwandlungenS}
\end{figure}

Monate sind nicht eindeutig. Kalendarische Monate haben 28 bis 31 Tage, betriebswirtschaftliche häufig 30. Wichtig für die Prüfung sollte höchstens sein dass ein Jahr zwölf Monate hat (Zinsen, Gehalt, ...).

In Wissenschaft und Technik wird durchgehend mit der Sekunde gerechnet und die Größenordnung wissenschaftlich notiert. So ist ein Tag $1 \text{d} = 60 \cdot 60 \cdot 24 \text{s} = 86.400\text{s} = 8,64 \times 10^4 \text{s}$ und das Universum ist ca. 14 Milliarden Jahre alt, was wissenschaftlich notiert $4,3 \times 10^{17}\text{s}$ sind.

Der Vollständigkeit halber einmal alle Einheiten des \glsdisp{SISystem}{SI-Systems}: Wir kennen bereits die Maßeinheit für Längen, den Meter (m), die Einheit für Massen, das Kilogramm (kg) und die Einheit für Zeit, die Sekunde (s). Die weiteren Einheiten, die nicht relevant sind für die Prüfung, da wir nur diese drei (kms) in der Mathematik brauchen und Erdkunde haben statt Physik, sind das Ampere (A) für Stromstärken, Kelvin (K) für Temperaturen, Mol (mol) für Stoffmengen und Candela (cd) für Lichtstärke.

Aus den elementaren Maßeinheiten können wir neue gewinnen durch Kombinationen. So wird Geschwindigkeit in Metern pro Sekunde ($\frac{m}{s}$) gemessen, Beschleunigung in Metern pro Quadratsekunden ($\frac{m}{s^2}$\footnote{Das ist die Veränderung der Geschwindigkeit pro Zeit}), Dichte in Kilogramm pro Kubikmeter ($\frac{kg}{m^3}$) und so weiter...


\chapter{1 mal 1}

Das \enquote{$1 \times 1$}, sprich \enquote{ein mal eins}, wird perfekt auswendig gelernt.

\begin{tikzpicture}[border style/.style={
    draw,fill=#1,minimum size=0.8cm,anchor=center,outer sep=0,
    name=\tikzmatrixname-\the\mcr-\the\mcc
}]
\matrix[row sep=-.5*\pgflinewidth,column sep=-.5*\pgflinewidth,
   execute at empty cell={%
       \ifnum1=\mcr\relax%
           \ifnum1=\mcc\relax\node[border style=black!10!white]{$\cdot$};%
           \else\node[border style=black!10!white]{$\number\numexpr\the\mcc-1\relax$};\fi
       \else%
         \ifnum1=\mcc\relax\node[border style=black!10!white]{$\cdot$};%
           \node[border style=black!10!white]{$\number\numexpr\the\mcr-1\relax$};%
         \else%
           \pgfmathparse{int(abs(\mcr-\mcc))}%
           \ifnum5=\pgfmathresult\relax\def\temp{white}\else\def\temp{none}\fi%
           \node[border style=\temp]{\number\numexpr\numexpr\the\mcc-1\relax*\numexpr\the\mcr-1\relax\relax};%
         \fi%
       \fi}
] (a) {
&&&&&&&&&&&\\
&&&&&&&&&&&\\
&&&&&&&&&&&\\
&&&&&&&&&&&\\
&&&&&&&&&&&\\
&&&&&&&&&&&\\
&&&&&&&&&&&\\
&&&&&&&&&&&\\
&&&&&&&&&&&\\
&&&&&&&&&&&\\
&&&&&&&&&&&\\
&&&&&&&&&&&\\
};
\end{tikzpicture}

Die perfekte Beherrschung des \enquote{$1 \times 1$} ist \textbf{notwendige Voraussetzung} für sicheres und schnelles schriftliches Rechnen.

\chapter{Schriftlich Rechnen}\index{Schriftlich Rechnen}

Nochmal: Die perfekte Beherrschung des \enquote{$1 \times 1$} ist \textbf{notwendige Voraussetzung} für sicheres und schnelles schriftliches Rechnen! Weitere Beispiele und Erklärung in der Weise wie hier schriftlich festgehalten findet Ihr auch unter \url{https://www.youtube.com/watch?v=SQvYo9ZZ4qg&list=PLVBzQZPmcRQL6vbGv1uDj7styo4I0nzHr} als \textit{Youtube}-Videos.


\section{Schriftliche Addition}\index{Addition!schriftliche}

Wir addieren einen Term von \textbf{Summanden} indem wir alle \textbf{Summanden} korrekt nach 10er-\textbf{Stellenwertsystem} untereinander schreiben und von rechts nach links, also von der 1er-\textbf{Stelle} beginnend, alle gleichen Stellen addieren. Für Ergebnisse ab 10 ist die 1er-\textbf{Stelle} des Ergebnisses die 1er-\textbf{Stelle} der \textbf{Summe} der 1er-\textbf{Stellen} aller \textbf{Summanden}. Weitere \textbf{Stellen} werden an den \textbf{Stellen ihres Stellenwertes} vermerkt und beim nach links gehenden \textbf{Addieren} mit \textbf{addiert} (kleine Beispiele übertragen i.d.R. nur einen Zehner. Das ist die  \enquote{1}, die häufig auf dem Strich über dem Ergebnis erscheint. Bei \textbf{Summen} mit vielen \textbf{Summanden} kommen aber auch größere \textbf{Überträge} zustande.).

\begin{beispiel}[ausführliche Beschreibung schriftlich addieren]
    Wir addieren $12345$ und $6789$, $12345 + 6789$ = x (s. Abb. \ref{fig:schriftlichAddieren}, S. \pageref{fig:schriftlichAddieren}). Wir schreiben die beiden Summanden sauber ausgerichtet (1er unter 1er, 10er unter 10er, ...) untereinander.  Danach addieren wir Stellenweise, zuerst die 1er-Stellen: $9+5=14$. Wir schreiben die $4$ als 1er-Stelle des Ergebnisses auf und notieren die $1$ als Übertrag auf die 10er-Stelle. Für die 10er-Stelle des Ergebnisses addieren wir den Übertrag und die 10er-Stellen der Summanden: $1+8+4=13$. Die letzte Stelle der Summe der 10er-Stellen wird als 10er-Stelle des Ergebnisses notiert.\topicend
\end{beispiel}

\begin{beispiel}[ausführliche Beschreibung schriftlich addieren]
    Wir addieren eine ganze Reihe von Zahlen. Insbesondere beim Addieren (und Subtrahieren) von vielen Zahlen sagen wir auch dass wir \textbf{die Summe -n}\index{Summe}\index{summieren} bilden. Die Zahlen bezeichnen wir dann als \textbf{der Summand -en}\index{Summand}. Wir sagen: \enquote{Wir summieren die Summanden zur Summe.} oder \enquote{Wir bilden die Summe.}

    Wir bilden die Summe von \{83337, 47466, 43895, 18792, 68914, 36092, 644, 17669, 849, 825, 23559, 29410, 423, 29403, 69104, 44024, 220, 82771, 23239\} (s. Abb. \ref{fig:schriftlichAddieren}, S. \pageref{fig:schriftlichAddieren}):

    Wir schreiben alle Summanden sauber der Stellenwertigkeiten nach an den 1er-Stellen ausgerichtet untereinander. Wir summieren die Stellen jeweils einzeln von rechts nach links (und damit von den 1er-Stellen aufwärts). Die 1er-Stelle des Ergebnisses der Summe aller 1er-Stellen schreiben wir als 1er-Stelle des Ergebnisses auf. Die größeren Stellen der Summe der 1er-Stellen schreiben wir (klein als Merkhilfe) unter die entsprechenden Stellen der 10er-Stellen (oder bei sehr langen Listen von Summanden auch darüber hinaus).

    Die Summe der 1er-Stellen ist 85. Wir notieren die 5 als 1er-Stelle des Ergebnisses und die 8 als \textbf{Übertrag} zu den 10er-Stellen. Die Summe der 10er-Stellen und des \textbf{Übertrags} ist 83. Wir notieren die 3 als 10er-Stelle des Ergebnisses und die 8 als \textbf{Übertrag} zu den 100er-Stellen. Die Summe der 100er-Stellen und des Übertrags ist 96. Wir notieren die 6 als 100er-Stelle des Ergebnisses und die 9 als Übertrag zu den 1000er-Stellen. Die Summe der 1000er-Stellen und des Übertrags ist 90. Wir notieren die 0 als 1000er-Stelle des Übertrags und die 9 als Übertrag zu den 10000er-Stellen. Die Summe der 10000er-Stellen und des Übertrags ist 62. Wir notieren die 2 als 10000er-Stellen des Ergebnisses und die 6 als Übertrag zu den 100000er-Stellen. Die Summe der 100000er-Stellen und des Übertrags ist 6. Wir notieren die 6 als 100000er-Stelle des Ergebnisses und lesen das Ergebnis ab: 620635.\topicend
\end{beispiel}

Die Beispiele zeigen, dass schriftlich zu addieren bei Nutzung des karierten Papiers, ordentlicher Notation und etwas Geduld, mit 20 Summanden nicht schwieriger ist als mit zwei, nur längere Konzentration erfordert.

\begin{figure}
  \centering
  \begin{tikzpicture}
    %\draw[step=1mm, line width=0.1mm, black!30!white] (0,0) grid (\width,\hauteur);
    \draw[step=5mm, line width=0.1mm, black!40!white] (0,0) grid (\width,\hauteur);
    %\draw[step=5cm, line width=0.5mm, black!50!white] (0,0) grid (\width,\hauteur);
    %\draw[step=1cm, line width=0.3mm, black!90!white] (0,0) grid (\width,\hauteur);

    \node at (0.75cm, 11.75cm) {1};
    \node at (1.25cm, 11.75cm) {2};
    \node at (1.75cm, 11.75cm) {3};
    \node at (2.25cm, 11.75cm) {4};
    \node at (2.75cm, 11.75cm) {5};
    \node at (0.25cm, 11.25cm) {+};
    \node at (1.25cm, 11.25cm) {6};
    \node at (1.75cm, 11.25cm) {7};
    \node at (2.25cm, 11.25cm) {8};
    \node at (2.75cm, 11.25cm) {9};

    \draw (0.2cm,10.7cm) -- (3cm,10.7cm);

    \node at (2.75cm, 10.25cm) {4};
    \node at (2.35cm, 10.8cm) {\tiny 1};
    \node at (2.25cm, 10.25cm) {3};
    \node at (1.85cm, 10.8cm) {\tiny 1};
    \node at (1.75cm, 10.25cm) {1};
    \node at (1.35cm, 10.8cm) {\tiny 1};
    \node at (1.25cm, 10.25cm) {9};
    \node at (0.75cm, 10.25cm) {1};

    \node at (5.75cm, 11.75cm) {};
    \node at (6.25cm, 11.75cm) {8};
    \node at (6.75cm, 11.75cm) {3};
    \node at (7.25cm, 11.75cm) {3};
    \node at (7.75cm, 11.75cm) {3};
    \node at (8.25cm, 11.75cm) {7};

    \node at (5.25cm, 11.25cm) {+};
    \node at (5.75cm, 11.25cm) {};
    \node at (6.25cm, 11.25cm) {4};
    \node at (6.75cm, 11.25cm) {7};
    \node at (7.25cm, 11.25cm) {4};
    \node at (7.75cm, 11.25cm) {6};
    \node at (8.25cm, 11.25cm) {6};

    \node at (5.25cm, 10.75cm) {+};
    \node at (5.75cm, 10.75cm) {};
    \node at (6.25cm, 10.75cm) {4};
    \node at (6.75cm, 10.75cm) {3};
    \node at (7.25cm, 10.75cm) {8};
    \node at (7.75cm, 10.75cm) {9};
    \node at (8.25cm, 10.75cm) {5};

    \node at (5.25cm, 10.25cm) {+};
    \node at (5.75cm, 10.25cm) {};
    \node at (6.25cm, 10.25cm) {1};
    \node at (6.75cm, 10.25cm) {8};
    \node at (7.25cm, 10.25cm) {7};
    \node at (7.75cm, 10.25cm) {9};
    \node at (8.25cm, 10.25cm) {1};

    \node at (5.25cm, 9.75cm) {+};
    \node at (5.75cm, 9.75cm) {};
    \node at (6.25cm, 9.75cm) {6};
    \node at (6.75cm, 9.75cm) {8};
    \node at (7.25cm, 9.75cm) {9};
    \node at (7.75cm, 9.75cm) {1};
    \node at (8.25cm, 9.75cm) {4};

    \node at (5.25cm, 9.25cm) {+};
    \node at (5.75cm, 9.25cm) {};
    \node at (6.25cm, 9.25cm) {3};
    \node at (6.75cm, 9.25cm) {6};
    \node at (7.25cm, 9.25cm) {0};
    \node at (7.75cm, 9.25cm) {9};
    \node at (8.25cm, 9.25cm) {2};

    \node at (5.25cm, 8.75cm) {+};
    \node at (5.75cm, 8.75cm) {};
    \node at (6.25cm, 8.75cm) {};
    \node at (6.75cm, 8.75cm) {};
    \node at (7.25cm, 8.75cm) {6};
    \node at (7.75cm, 8.75cm) {4};
    \node at (8.25cm, 8.75cm) {4};

    \node at (5.25cm, 8.25cm) {+};
    \node at (5.75cm, 8.25cm) {};
    \node at (6.25cm, 8.25cm) {1};
    \node at (6.75cm, 8.25cm) {7};
    \node at (7.25cm, 8.25cm) {6};
    \node at (7.75cm, 8.25cm) {6};
    \node at (8.25cm, 8.25cm) {9};

    \node at (5.25cm, 7.75cm) {+};
    \node at (5.75cm, 7.75cm) {};
    \node at (6.25cm, 7.75cm) {};
    \node at (6.75cm, 7.75cm) {};
    \node at (7.25cm, 7.75cm) {8};
    \node at (7.75cm, 7.75cm) {4};
    \node at (8.25cm, 7.75cm) {9};

    \node at (5.25cm, 7.25cm) {+};
    \node at (5.75cm, 7.25cm) {};
    \node at (6.25cm, 7.25cm) {};
    \node at (6.75cm, 7.25cm) {};
    \node at (7.25cm, 7.25cm) {8};
    \node at (7.75cm, 7.25cm) {2};
    \node at (8.25cm, 7.25cm) {5};

    \node at (5.25cm, 6.75cm) {+};
    \node at (5.75cm, 6.75cm) {};
    \node at (6.25cm, 6.75cm) {2};
    \node at (6.75cm, 6.75cm) {3};
    \node at (7.25cm, 6.75cm) {5};
    \node at (7.75cm, 6.75cm) {5};
    \node at (8.25cm, 6.75cm) {9};

    \node at (5.25cm, 6.25cm) {+};
    \node at (5.75cm, 6.25cm) {};
    \node at (6.25cm, 6.25cm) {2};
    \node at (6.75cm, 6.25cm) {9};
    \node at (7.25cm, 6.25cm) {4};
    \node at (7.75cm, 6.25cm) {1};
    \node at (8.25cm, 6.25cm) {0};

    \node at (5.25cm, 5.75cm) {+};
    \node at (5.75cm, 5.75cm) {};
    \node at (6.25cm, 5.75cm) {};
    \node at (6.75cm, 5.75cm) {};
    \node at (7.25cm, 5.75cm) {4};
    \node at (7.75cm, 5.75cm) {2};
    \node at (8.25cm, 5.75cm) {3};

    \node at (5.25cm, 5.25cm) {+};
    \node at (5.75cm, 5.25cm) {};
    \node at (6.25cm, 5.25cm) {2};
    \node at (6.75cm, 5.25cm) {9};
    \node at (7.25cm, 5.25cm) {4};
    \node at (7.75cm, 5.25cm) {0};
    \node at (8.25cm, 5.25cm) {3};

    \node at (5.25cm, 4.75cm) {+};
    \node at (5.75cm, 4.75cm) {};
    \node at (6.25cm, 4.75cm) {6};
    \node at (6.75cm, 4.75cm) {9};
    \node at (7.25cm, 4.75cm) {1};
    \node at (7.75cm, 4.75cm) {0};
    \node at (8.25cm, 4.75cm) {4};

    \node at (5.25cm, 4.25cm) {+};
    \node at (5.75cm, 4.25cm) {};
    \node at (6.25cm, 4.25cm) {4};
    \node at (6.75cm, 4.25cm) {4};
    \node at (7.25cm, 4.25cm) {0};
    \node at (7.75cm, 4.25cm) {2};
    \node at (8.25cm, 4.25cm) {4};

    \node at (5.25cm, 3.75cm) {+};
    \node at (5.75cm, 3.75cm) {};
    \node at (6.25cm, 3.75cm) {};
    \node at (6.75cm, 3.75cm) {};
    \node at (7.25cm, 3.75cm) {2};
    \node at (7.75cm, 3.75cm) {2};
    \node at (8.25cm, 3.75cm) {0};

    \node at (5.25cm, 3.25cm) {+};
    \node at (5.75cm, 3.25cm) {};
    \node at (6.25cm, 3.25cm) {8};
    \node at (6.75cm, 3.25cm) {2};
    \node at (7.25cm, 3.25cm) {7};
    \node at (7.75cm, 3.25cm) {7};
    \node at (8.25cm, 3.25cm) {1};

    \node at (5.25cm, 2.75cm) {+};
    \node at (5.75cm, 2.75cm) {};
    \node at (6.25cm, 2.75cm) {2};
    \node at (6.75cm, 2.75cm) {3};
    \node at (7.25cm, 2.75cm) {2};
    \node at (7.75cm, 2.75cm) {3};
    \node at (8.25cm, 2.75cm) {9};

    \draw (5.2cm,2.2cm) -- (9cm,2.2cm);

    \node at (5.75cm, 1.75cm) {6};
    \node at (5.8cm, 2.3cm) {\tiny 6};
    \node at (6.25cm, 1.75cm) {2};
    \node at (6.3cm, 2.3cm) {\tiny 9};
    \node at (6.75cm, 1.75cm) {0};
    \node at (6.8cm, 2.3cm) {\tiny 9};
    \node at (7.25cm, 1.75cm) {6};
    \node at (7.3cm, 2.3cm) {\tiny 8};
    \node at (7.75cm, 1.75cm) {3};
    \node at (7.8cm, 2.3cm) {\tiny 8};
    \node at (8.25cm, 1.75cm) {5};//85

  \end{tikzpicture}
  \caption{schriftlich addieren}\label{fig:schriftlichAddieren}
\end{figure}


\section{Schriftliche Subtraktion}\index{Subtraktion!schriftliche}

Wir schreiben $3-2=1$ und lesen \enquote{drei minus zwei (ist) gleich 1}. Eine negative Zahl $-x$ zu subtrahieren bedeutet ihren Betrag zu addieren: $3 - (-2) = 5$\footnote{Diese Schreibweise (und noch mehr $3--2$) sollte vermieden werden.} \enquote{Minus minus ist plus.} Wenn man in der Vorstellung des Zahlenstrahls bleiben möchte, dann kann man das Minus als Richtungswechsel sehen und zwei mal 180 Grad ergibt wieder die ursprüngliche Richtung. In \glsdisp{symb:Sum}{Summen} können positive und negative Summanden gemischt vorkommen. Um eine Summe aus Summanden mit unterschiedlichen Vorzeichen\index{Vorzeichen} zu bilden fassen wir zunächst die positiven und negativen Summanden getrennt zusammen. Wollen wir die Summe $\Sigma = -123 +456 -789 +321 -124 +34 -123 +127 -2009$ berechnen bilden wir zunächst die Summen $\Sigma_{\text{+}}$ der positiven Summanden und $\Sigma_{\text{-}}$ der negativen Summanden und bilden anschließend die Differenz der beiden Teilsummen. $\Sigma_{\text{+}} = 456 + 321 + 34 + 123 = 934$. $\Sigma_{\text{-}} = -(123 + 789 + 124 + 127) = -1163$.


\begin{figure}
  \centering
  \begin{tikzpicture}
    %\draw[step=1mm, line width=0.1mm, black!30!white] (0,0) grid (\width,\hauteur);
    \draw[step=5mm, line width=0.1mm, black!40!white] (0,0) grid (\width,\hauteur);
    %\draw[step=5cm, line width=0.5mm, black!50!white] (0,0) grid (\width,\hauteur);
    %\draw[step=1cm, line width=0.3mm, black!90!white] (0,0) grid (\width,\hauteur);

    \node at (0.75cm, 11.75cm) {1};
    \node at (1.25cm, 11.75cm) {2};
    \node at (1.75cm, 11.75cm) {3};
    \node at (2.25cm, 11.75cm) {4};
    \node at (2.75cm, 11.75cm) {5};
    \node at (0.25cm, 11.25cm) {-};
    \node at (1.25cm, 11.25cm) {6};
    \node at (1.75cm, 11.25cm) {7};
    \node at (2.25cm, 11.25cm) {8};
    \node at (2.75cm, 11.25cm) {9};

    \draw (0.2cm,10.7cm) -- (3cm,10.7cm);

    \node at (2.75cm, 10.25cm) {6};
    \node at (2.35cm, 10.8cm) {\tiny 1};
    \node at (2.25cm, 10.25cm) {5};
    \node at (1.85cm, 10.8cm) {\tiny 1};
    \node at (1.75cm, 10.25cm) {5};
    \node at (1.35cm, 10.8cm) {\tiny 1};
    \node at (1.25cm, 10.25cm) {5};
    \node at (0.85cm, 10.8cm) {\tiny 1};
    \node at (0.75cm, 10.25cm) {};

    \node at (4.75cm, 11.75cm) {};
    \node at (5.25cm, 11.75cm) {8};
    \node at (5.75cm, 11.75cm) {3};
    \node at (6.25cm, 11.75cm) {3};
    \node at (6.75cm, 11.75cm) {3};
    \node at (7.25cm, 11.75cm) {7};

    \node at (4.25cm, 11.25cm) {-};
    \node at (4.75cm, 11.25cm) {};
    \node at (5.25cm, 11.25cm) {4};
    \node at (5.75cm, 11.25cm) {7};
    \node at (6.25cm, 11.25cm) {4};
    \node at (6.75cm, 11.25cm) {6};
    \node at (7.25cm, 11.25cm) {6};

    \node at (4.25cm, 10.75cm) {-};
    \node at (4.75cm, 10.75cm) {};
    \node at (5.25cm, 10.75cm) {4};
    \node at (5.75cm, 10.75cm) {3};
    \node at (6.25cm, 10.75cm) {8};
    \node at (6.75cm, 10.75cm) {9};
    \node at (7.25cm, 10.75cm) {5};

    \node at (4.25cm, 10.25cm) {-};
    \node at (4.75cm, 10.25cm) {};
    \node at (5.25cm, 10.25cm) {1};
    \node at (5.75cm, 10.25cm) {8};
    \node at (6.25cm, 10.25cm) {7};
    \node at (6.75cm, 10.25cm) {9};
    \node at (7.25cm, 10.25cm) {1};

    \node at (4.25cm, 9.75cm) {-};
    \node at (4.75cm, 9.75cm) {};
    \node at (5.25cm, 9.75cm) {6};
    \node at (5.75cm, 9.75cm) {8};
    \node at (6.25cm, 9.75cm) {9};
    \node at (6.75cm, 9.75cm) {1};
    \node at (7.25cm, 9.75cm) {4};

    \node at (4.25cm, 9.25cm) {-};
    \node at (4.75cm, 9.25cm) {};
    \node at (5.25cm, 9.25cm) {3};
    \node at (5.75cm, 9.25cm) {6};
    \node at (6.25cm, 9.25cm) {0};
    \node at (6.75cm, 9.25cm) {9};
    \node at (7.25cm, 9.25cm) {2};

    \node at (4.25cm, 8.75cm) {-};
    \node at (4.75cm, 8.75cm) {};
    \node at (5.25cm, 8.75cm) {};
    \node at (5.75cm, 8.75cm) {};
    \node at (6.25cm, 8.75cm) {6};
    \node at (6.75cm, 8.75cm) {4};
    \node at (7.25cm, 8.75cm) {4};

    \node at (4.25cm, 8.25cm) {-};
    \node at (4.75cm, 8.25cm) {};
    \node at (5.25cm, 8.25cm) {1};
    \node at (5.75cm, 8.25cm) {7};
    \node at (6.25cm, 8.25cm) {6};
    \node at (6.75cm, 8.25cm) {6};
    \node at (7.25cm, 8.25cm) {9};

    \node at (4.25cm, 7.75cm) {-};
    \node at (4.75cm, 7.75cm) {};
    \node at (5.25cm, 7.75cm) {};
    \node at (5.75cm, 7.75cm) {};
    \node at (6.25cm, 7.75cm) {8};
    \node at (6.75cm, 7.75cm) {4};
    \node at (7.25cm, 7.75cm) {9};

    \node at (4.25cm, 7.25cm) {-};
    \node at (4.75cm, 7.25cm) {};
    \node at (5.25cm, 7.25cm) {};
    \node at (5.75cm, 7.25cm) {};
    \node at (6.25cm, 7.25cm) {8};
    \node at (6.75cm, 7.25cm) {2};
    \node at (7.25cm, 7.25cm) {5};

    \node at (4.25cm, 6.75cm) {-};
    \node at (4.75cm, 6.75cm) {};
    \node at (5.25cm, 6.75cm) {2};
    \node at (5.75cm, 6.75cm) {3};
    \node at (6.25cm, 6.75cm) {5};
    \node at (6.75cm, 6.75cm) {5};
    \node at (7.25cm, 6.75cm) {9};

    \node at (4.25cm, 6.25cm) {-};
    \node at (4.75cm, 6.25cm) {};
    \node at (5.25cm, 6.25cm) {2};
    \node at (5.75cm, 6.25cm) {9};
    \node at (6.25cm, 6.25cm) {4};
    \node at (6.75cm, 6.25cm) {1};
    \node at (7.25cm, 6.25cm) {0};

    \node at (4.25cm, 5.75cm) {-};
    \node at (4.75cm, 5.75cm) {};
    \node at (5.25cm, 5.75cm) {};
    \node at (5.75cm, 5.75cm) {};
    \node at (6.25cm, 5.75cm) {4};
    \node at (6.75cm, 5.75cm) {2};
    \node at (7.25cm, 5.75cm) {3};

    \node at (4.25cm, 5.25cm) {-};
    \node at (4.75cm, 5.25cm) {};
    \node at (5.25cm, 5.25cm) {2};
    \node at (5.75cm, 5.25cm) {9};
    \node at (6.25cm, 5.25cm) {4};
    \node at (6.75cm, 5.25cm) {0};
    \node at (7.25cm, 5.25cm) {3};

    \node at (4.25cm, 4.75cm) {-};
    \node at (4.75cm, 4.75cm) {};
    \node at (5.25cm, 4.75cm) {6};
    \node at (5.75cm, 4.75cm) {9};
    \node at (6.25cm, 4.75cm) {1};
    \node at (6.75cm, 4.75cm) {0};
    \node at (7.25cm, 4.75cm) {4};

    \node at (4.25cm, 4.25cm) {-};
    \node at (4.75cm, 4.25cm) {};
    \node at (5.25cm, 4.25cm) {4};
    \node at (5.75cm, 4.25cm) {4};
    \node at (6.25cm, 4.25cm) {0};
    \node at (6.75cm, 4.25cm) {2};
    \node at (7.25cm, 4.25cm) {4};

    \node at (4.25cm, 3.75cm) {-};
    \node at (4.75cm, 3.75cm) {};
    \node at (5.25cm, 3.75cm) {};
    \node at (5.75cm, 3.75cm) {};
    \node at (6.25cm, 3.75cm) {2};
    \node at (6.75cm, 3.75cm) {2};
    \node at (7.25cm, 3.75cm) {0};

    \node at (4.25cm, 3.25cm) {-};
    \node at (4.75cm, 3.25cm) {};
    \node at (5.25cm, 3.25cm) {8};
    \node at (5.75cm, 3.25cm) {2};
    \node at (6.25cm, 3.25cm) {7};
    \node at (6.75cm, 3.25cm) {7};
    \node at (7.25cm, 3.25cm) {1};

    \node at (4.25cm, 2.75cm) {-};
    \node at (4.75cm, 2.75cm) {};
    \node at (5.25cm, 2.75cm) {2};
    \node at (5.75cm, 2.75cm) {3};
    \node at (6.25cm, 2.75cm) {2};
    \node at (6.75cm, 2.75cm) {3};
    \node at (7.25cm, 2.75cm) {9};

    \draw (4.2cm,2.2cm) -- (8cm,2.2cm);

    \node at (4.75cm, 1.75cm) {};
    \node at (4.8cm, 2.3cm) {\tiny };
    \node at (5.25cm, 1.75cm) {};
    \node at (5.3cm, 2.3cm) {\tiny };
    \node at (5.75cm, 1.75cm) {};
    \node at (5.8cm, 2.3cm) {\tiny };
    \node at (6.25cm, 1.75cm) {};
    \node at (6.3cm, 2.3cm) {\tiny };
    \node at (6.75cm, 1.75cm) {};
    \node at (6.8cm, 2.3cm) {\tiny };
    \node at (7.25cm, 1.75cm) {?};

    \node at (8.75cm, 11.75cm) {};
    \node at (9.25cm, 11.75cm) {8};
    \node at (9.75cm, 11.75cm) {3};
    \node at (10.25cm, 11.75cm) {3};
    \node at (10.75cm, 11.75cm) {3};
    \node at (11.25cm, 11.75cm) {7};

    \node at (8.25cm, 11.25cm) {-};
    \node at (8.75cm, 11.25cm) {(};
    \node at (8.75cm, 11.25cm) {};
    \node at (9.25cm, 11.25cm) {4};
    \node at (9.75cm, 11.25cm) {7};
    \node at (10.25cm, 11.25cm) {4};
    \node at (10.75cm, 11.25cm) {6};
    \node at (11.25cm, 11.25cm) {6};

    \node at (8.25cm, 10.75cm) {+};
    \node at (8.75cm, 10.75cm) {};
    \node at (9.25cm, 10.75cm) {4};
    \node at (9.75cm, 10.75cm) {3};
    \node at (10.25cm, 10.75cm) {8};
    \node at (10.75cm, 10.75cm) {9};
    \node at (11.25cm, 10.75cm) {5};

    \node at (8.25cm, 10.25cm) {+};
    \node at (8.75cm, 10.25cm) {};
    \node at (9.25cm, 10.25cm) {1};
    \node at (9.75cm, 10.25cm) {8};
    \node at (10.25cm, 10.25cm) {7};
    \node at (10.75cm, 10.25cm) {9};
    \node at (11.25cm, 10.25cm) {1};

    \node at (8.25cm, 9.75cm) {+};
    \node at (8.75cm, 9.75cm) {};
    \node at (9.25cm, 9.75cm) {6};
    \node at (9.75cm, 9.75cm) {8};
    \node at (10.25cm, 9.75cm) {9};
    \node at (10.75cm, 9.75cm) {1};
    \node at (11.25cm, 9.75cm) {4};

    \node at (8.25cm, 9.25cm) {+};
    \node at (8.75cm, 9.25cm) {};
    \node at (9.25cm, 9.25cm) {3};
    \node at (9.75cm, 9.25cm) {6};
    \node at (10.25cm, 9.25cm) {0};
    \node at (10.75cm, 9.25cm) {9};
    \node at (11.25cm, 9.25cm) {2};

    \node at (8.25cm, 8.75cm) {+};
    \node at (8.75cm, 8.75cm) {};
    \node at (9.25cm, 8.75cm) {};
    \node at (9.75cm, 8.75cm) {};
    \node at (10.25cm, 8.75cm) {6};
    \node at (10.75cm, 8.75cm) {4};
    \node at (11.25cm, 8.75cm) {4};

    \node at (8.25cm, 8.25cm) {+};
    \node at (8.75cm, 8.25cm) {};
    \node at (9.25cm, 8.25cm) {1};
    \node at (9.75cm, 8.25cm) {7};
    \node at (10.25cm, 8.25cm) {6};
    \node at (10.75cm, 8.25cm) {6};
    \node at (11.25cm, 8.25cm) {9};

    \node at (8.25cm, 7.75cm) {+};
    \node at (8.75cm, 7.75cm) {};
    \node at (9.25cm, 7.75cm) {};
    \node at (9.75cm, 7.75cm) {};
    \node at (10.25cm, 7.75cm) {8};
    \node at (10.75cm, 7.75cm) {4};
    \node at (11.25cm, 7.75cm) {9};

    \node at (8.25cm, 7.25cm) {+};
    \node at (8.75cm, 7.25cm) {};
    \node at (9.25cm, 7.25cm) {};
    \node at (9.75cm, 7.25cm) {};
    \node at (10.25cm, 7.25cm) {8};
    \node at (10.75cm, 7.25cm) {2};
    \node at (11.25cm, 7.25cm) {5};

    \node at (8.25cm, 6.75cm) {+};
    \node at (8.75cm, 6.75cm) {};
    \node at (9.25cm, 6.75cm) {2};
    \node at (9.75cm, 6.75cm) {3};
    \node at (10.25cm, 6.75cm) {5};
    \node at (10.75cm, 6.75cm) {5};
    \node at (11.25cm, 6.75cm) {9};

    \node at (8.25cm, 6.25cm) {+};
    \node at (8.75cm, 6.25cm) {};
    \node at (9.25cm, 6.25cm) {2};
    \node at (9.75cm, 6.25cm) {9};
    \node at (10.25cm, 6.25cm) {4};
    \node at (10.75cm, 6.25cm) {1};
    \node at (11.25cm, 6.25cm) {0};

    \node at (8.25cm, 5.75cm) {+};
    \node at (8.75cm, 5.75cm) {};
    \node at (9.25cm, 5.75cm) {};
    \node at (9.75cm, 5.75cm) {};
    \node at (10.25cm, 5.75cm) {4};
    \node at (10.75cm, 5.75cm) {2};
    \node at (11.25cm, 5.75cm) {3};

    \node at (8.25cm, 5.25cm) {+};
    \node at (8.75cm, 5.25cm) {};
    \node at (9.25cm, 5.25cm) {2};
    \node at (9.75cm, 5.25cm) {9};
    \node at (10.25cm, 5.25cm) {4};
    \node at (10.75cm, 5.25cm) {0};
    \node at (11.25cm, 5.25cm) {3};

    \node at (8.25cm, 4.75cm) {+};
    \node at (8.75cm, 4.75cm) {};
    \node at (9.25cm, 4.75cm) {6};
    \node at (9.75cm, 4.75cm) {9};
    \node at (10.25cm, 4.75cm) {1};
    \node at (10.75cm, 4.75cm) {0};
    \node at (11.25cm, 4.75cm) {4};

    \node at (8.25cm, 4.25cm) {+};
    \node at (8.75cm, 4.25cm) {};
    \node at (9.25cm, 4.25cm) {4};
    \node at (9.75cm, 4.25cm) {4};
    \node at (10.25cm, 4.25cm) {0};
    \node at (10.75cm, 4.25cm) {2};
    \node at (11.25cm, 4.25cm) {4};

    \node at (8.25cm, 3.75cm) {+};
    \node at (8.75cm, 3.75cm) {};
    \node at (9.25cm, 3.75cm) {};
    \node at (9.75cm, 3.75cm) {};
    \node at (10.25cm, 3.75cm) {2};
    \node at (10.75cm, 3.75cm) {2};
    \node at (11.25cm, 3.75cm) {0};

    \node at (8.25cm, 3.25cm) {+};
    \node at (8.75cm, 3.25cm) {};
    \node at (9.25cm, 3.25cm) {8};
    \node at (9.75cm, 3.25cm) {2};
    \node at (10.25cm, 3.25cm) {7};
    \node at (10.75cm, 3.25cm) {7};
    \node at (11.25cm, 3.25cm) {1};

    \node at (8.25cm, 2.75cm) {+};
    \node at (8.75cm, 2.75cm) {};
    \node at (9.25cm, 2.75cm) {2};
    \node at (9.75cm, 2.75cm) {3};
    \node at (10.25cm, 2.75cm) {2};
    \node at (10.75cm, 2.75cm) {3};
    \node at (11.25cm, 2.75cm) {9};
    \node at (11.75cm, 2.75cm) {)};

    \draw (8.2cm,2.2cm) -- (11.9cm,2.2cm);

    \node at (8.75cm, 1.75cm) {5};
    \node at (8.8cm, 2.3cm) {\tiny 5};
    \node at (9.25cm, 1.75cm) {3};
    \node at (9.3cm, 2.3cm) {\tiny 8};
    \node at (9.75cm, 1.75cm) {7};
    \node at (9.8cm, 2.3cm) {\tiny 9};
    \node at (10.25cm, 1.75cm) {2};
    \node at (10.3cm, 2.3cm) {\tiny 7};
    \node at (10.75cm, 1.75cm) {9};
    \node at (10.8cm, 2.3cm) {\tiny 7};
    \node at (11.25cm, 1.75cm) {8};

    \node at (8.25cm, 1.25cm) {-};
    \node at (8.75cm, 1.25cm) {};
    \node at (9.25cm, 1.25cm) {8};
    \node at (9.75cm, 1.25cm) {3};
    \node at (10.25cm, 1.25cm) {3};
    \node at (10.75cm, 1.25cm) {3};
    \node at (11.25cm, 1.25cm) {7};

    \draw (8.2cm,0.7cm) -- (11.9cm,0.7cm);

    \node at (8.25cm, 0.25cm) {-};
    \node at (8.75cm, 0.25cm) {4};
    \node at (8.8cm, 0.8cm) {\tiny 1};
    \node at (9.25cm, 0.25cm) {5};
    \node at (9.3cm, 0.8cm) {\tiny };
    \node at (9.75cm, 0.25cm) {3};
    \node at (9.8cm, 0.8cm) {\tiny 1};
    \node at (10.25cm, 0.25cm) {9};
    \node at (10.3cm, 0.8cm) {\tiny };
    \node at (10.75cm, 0.25cm) {6};
    \node at (10.8cm, 0.8cm) {\tiny };
    \node at (11.25cm, 0.25cm) {1};

  \end{tikzpicture}
  \caption{schriftlich subtrahieren}\label{fig:schriftlichSubtrahieren}
\end{figure}

\section{Schriftliche Multiplikation}\index{Multiplikation!schriftliche}

\begin{figure}
  \centering
  \begin{tikzpicture}
    %\draw[step=1mm, line width=0.1mm, black!30!white] (0,0) grid (\width,\hauteur);
    \draw[step=5mm, line width=0.1mm, black!40!white] (0,0) grid (\width,\hauteur);
    %\draw[step=5cm, line width=0.5mm, black!50!white] (0,0) grid (\width,\hauteur);
    %\draw[step=1cm, line width=0.3mm, black!90!white] (0,0) grid (\width,\hauteur);

    \node at (1.25cm, 11.25cm) {1};
    \node at (1.75cm, 11.25cm) {2};
    \node at (2.25cm, 11.25cm) {3};
    \node at (2.75cm, 11.25cm) {4};
    \node at (3.25cm, 11.25cm) {5};
    \node at (3.75cm, 11.25cm) {$\cdot$};
    \node at (4.25cm, 11.25cm) {6};
    \node at (4.75cm, 11.25cm) {7};
    \node at (5.25cm, 11.25cm) {8};
    \node at (5.75cm, 11.25cm) {9};
    \node at (6.25cm, 11.25cm) {=};
    \node at (6.75cm, 11.25cm) {8};
    \node at (7.25cm, 11.25cm) {3};
    \node at (7.75cm, 11.25cm) {8};
    \node at (8.25cm, 11.25cm) {1};
    \node at (8.75cm, 11.25cm) {0};
    \node at (9.25cm, 11.25cm) {2};
    \node at (9.75cm, 11.25cm) {0};
    \node at (10.25cm, 11.25cm) {5};

    \draw (0.2cm,10.75cm) -- (6.2cm,10.75cm);
    \node at (1.25cm, 9.75cm) {+};
    \node at (1.25cm, 9.25cm) {+};
    \node at (1.25cm, 8.75cm) {+};

    \node at (2.25cm, 10.25cm) {7};
    \node at (2.75cm, 10.25cm) {4};
    \node at (3.25cm, 10.25cm) {0};
    \node at (3.75cm, 10.25cm) {7};
    \node at (4.25cm, 10.25cm) {0};

    \node at (2.75cm, 9.75cm) {8};
    \node at (3.25cm, 9.75cm) {6};
    \node at (3.75cm, 9.75cm) {4};
    \node at (4.25cm, 9.75cm) {1};
    \node at (4.75cm, 9.75cm) {5};

    \node at (3.25cm, 9.25cm) {9};
    \node at (3.75cm, 9.25cm) {8};
    \node at (4.25cm, 9.25cm) {7};
    \node at (4.75cm, 9.25cm) {6};
    \node at (5.25cm, 9.25cm) {0};

    \node at (3.25cm, 8.75cm) {1};
    \node at (3.75cm, 8.75cm) {1};
    \node at (4.25cm, 8.75cm) {1};
    \node at (4.75cm, 8.75cm) {1};
    \node at (5.25cm, 8.75cm) {0};
    \node at (5.75cm, 8.75cm) {5};

    \draw (0.2cm,8.2cm) -- (6.2cm,8.2cm);

    \node at (2.35cm, 8.3cm) {\tiny 1};
    \node at (2.25cm, 7.75cm) {8};
    \node at (2.85cm, 8.3cm) {\tiny 1};
    \node at (2.75cm, 7.75cm) {3};
    \node at (3.35cm, 8.3cm) {\tiny 2};
    \node at (3.25cm, 7.75cm) {8};
    \node at (3.85cm, 8.3cm) {\tiny 1};
    \node at (3.75cm, 7.75cm) {1};
    \node at (4.35cm, 8.3cm) {\tiny 1};
    \node at (4.25cm, 7.75cm) {0};
    \node at (4.75cm, 7.75cm) {2};
    \node at (5.25cm, 7.75cm) {0};
    \node at (5.75cm, 7.75cm) {5};

  \end{tikzpicture}
  \caption{schriftlich multiplizieren}\label{fig:schriftlichMultiplizieren}
\end{figure}


\section{Schriftliche Division}\index{Division!schriftliche}

\begin{figure}
  \centering
  \begin{tikzpicture}
    %\draw[step=1mm, line width=0.1mm, black!30!white] (0,0) grid (\width,\hauteur);
    \draw[step=5mm, line width=0.1mm, black!40!white] (0,0) grid (\width,\hauteur);
    %\draw[step=5cm, line width=0.5mm, black!50!white] (0,0) grid (\width,\hauteur);
    %\draw[step=1cm, line width=0.3mm, black!90!white] (0,0) grid (\width,\hauteur);

    \node at (1.25cm, 11.25cm) {1};
    \node at (1.75cm, 11.25cm) {:};
    \node at (2.25cm, 11.25cm) {2};
    \node at (2.75cm, 11.25cm) {=};
    \node at (3.25cm, 11.25cm) {0};
    \node at (3.75cm, 11.10cm) {,};
    \node at (4.25cm, 11.25cm) {5};

    \node at (0.75cm, 10.75cm) {-};
    \node at (1.25cm, 10.75cm) {0};

    \draw (0.7cm,10.25cm) -- (1.8cm,10.25cm);
    \node at (1.25cm, 9.75cm) {1};
    \node at (1.75cm, 9.75cm) {0};

    \node at (0.75cm, 9.25cm) {-};
    \node at (1.25cm, 9.25cm) {1};
    \node at (1.75cm, 9.25cm) {0};

    \draw (0.7cm,8.75cm) -- (2.3cm,8.75cm);
    \node at (1.75cm, 8.25cm) {0};


    \node at (6.25cm, 11.25cm) {1};
    \node at (6.75cm, 11.25cm) {:};
    \node at (7.25cm, 11.25cm) {1};
    \node at (7.75cm, 11.25cm) {6};
    \node at (8.25cm, 11.25cm) {=};
    \node at (8.75cm, 11.25cm) {0};
    \node at (9.25cm, 11.10cm) {,};
    \node at (9.75cm, 11.25cm) {0};
    \node at (10.25cm, 11.25cm) {6};
    \node at (10.75cm, 11.25cm) {2};
    \node at (11.25cm, 11.25cm) {5};

    \node at (5.75cm, 10.75cm) {-};
    \node at (6.25cm, 10.75cm) {0};
    \draw (5.75cm,10.25cm) -- (6.75cm,10.25cm);

    \node at (6.25cm, 9.75cm) {1};
    \node at (6.75cm, 9.75cm) {0};
    \node at (5.75cm, 9.25cm) {-};
    \node at (6.75cm, 9.25cm) {0};
    \draw (5.75cm,8.75cm) -- (7.25cm,8.75cm);
    \node at (6.25cm, 8.25cm) {1};
    \node at (6.75cm, 8.25cm) {0};
    \node at (7.25cm, 8.25cm) {0};

    \node at (5.75cm, 7.75cm) {-};
    \node at (6.75cm, 7.75cm) {9};
    \node at (7.25cm, 7.75cm) {6};
    \draw (5.75cm,7.25cm) -- (7.75cm,7.25cm);
    \node at (7.25cm, 6.75cm) {4};
    \node at (7.75cm, 6.75cm) {0};

    \node at (6.25cm, 6.25cm) {-};
    \node at (7.25cm, 6.25cm) {3};
    \node at (7.75cm, 6.25cm) {2};
    \draw (6.75cm,5.75cm) -- (8.25cm,5.75cm);
    \node at (7.75cm, 5.25cm) {8};
    \node at (8.25cm, 5.25cm) {0};

    \node at (7.25cm, 4.75cm) {-};
    \node at (7.75cm, 4.75cm) {8};
    \node at (8.25cm, 4.75cm) {0};
    \draw (6.75cm,4.25cm) -- (8.75cm,4.25cm);
    \node at (8.25cm, 3.75cm) {0};

  \end{tikzpicture}
  \caption{schriftlich dividieren}\label{fig:schriftlichDividieren}
\end{figure}



\chapter{Bruchrechnung}\index{Bruchrechnung}

\section{Motivation}

Mit Brüchen zu rechnen vereinfacht das Leben mit Mathematik sehr. Natürliche Zahlen sind uns eben natürlich. Ein Ganzes, ein Halbes oder ein Drittel von etwas, einer Pizza z.B., ist uns unmittelbar verständlich. $33,\overline{3}\%$ sind es nicht und machen erst durch den Gedanken \enquote{$33,\overline{3}$\% \textbf{ist} ein Drittel} gedanklich Sinn. Brüche sind einfach zu begreifen, Dezimalentwicklungen nicht. Unter $0,\overline{142857}$ kann sich Niemand etwas vorstellen, der nicht weiß, dass das $\frac{1}{7}$ ist.


\section{Grundlagen}\index{Bruchrechnung!Grundlagen}\label{Bruchrechnung}

Wir sprechen jeweils über die grauen Teile/ Flächen relativ zur gesamten Fläche, dem ganzen Kreis.

\begin{longtable}{|m{0.3\linewidth}|m{0.6\linewidth}|}
\hline
$
    \begin{tikzpicture}
        \filldraw[fill=gray!20] circle(0.75cm);
    \end{tikzpicture}
$
& Das ist eine Pizza. Das ist eine ganze Pizza. Das ist ein Ganzes. Das ist ein Stück einer Pizza, die aus einem Stück besteht. Es ist 1. Es ist $\frac{1}{1}$.\\
\hline
$
    \begin{tikzpicture}[scale=0.5, .style={fontsize=\footnotesize}]
            \filldraw[fill=gray!20] (0,0) circle(0.75cm);
            \draw (1.1,0) node{+};
            \filldraw[fill=gray!20] (2.2,0) circle(0.75cm);
            \draw (3.3,0) node{=};
            \filldraw[fill=gray!20] (4.4,0) circle(0.75cm);
            \filldraw[fill=gray!20] (5.9,0) circle(0.75cm);
    \end{tikzpicture}
$
& Das sind zwei Ganze (Kreise, Pizzen, Dinge, ...). Ein Kreis war $\frac{1}{1}$. Zwei sind $\frac{1}{1}+\frac{1}{1}=\frac{2}{1} = 2$ oder $2 \cdot \frac{1}{1} = \frac{2}{1} = 2$.\\
\hline
$
    \begin{tikzpicture}
            \filldraw[fill=gray!20] (0,0) circle(0.75cm);
            \draw (0, -0.75) -- (0,0.75);
    \end{tikzpicture}
$
& Das ist immer noch ein Kreis. Dass er durchgeschnitten ist ändert das nicht. Es sind zwei (gleich große) Teile eines Kreises aus zwei Teilen. Das sind $\frac{2}{2}$. Also sind $\frac{2}{2} = \frac{1}{1} = 1$.\\
\hline
$
    \begin{tikzpicture}[radius=7.5mm, delta angle=120]
        \filldraw[fill=black!10!white, draw=black!70!white, rotate=90]
            (0,0) -- (7.5mm, 0) arc (0:180:7.5mm) -- cycle;
        \filldraw[fill=white, draw=black!70!white, rotate=270]
            (0,0) -- (7.5mm, 0) arc (0:180:7.5mm) -- cycle;
    \end{tikzpicture}
$
& Das ist immer noch ein Kreis. Dass er durchgeschnitten ist ändert das nicht. Es ist ein Teil von zwei (gleich großen) Teilen eines Kreises aus zwei Teilen. Das sind $\frac{1}{2}$.\\
\hline
$
    \begin{tikzpicture}[radius=7.5mm, delta angle=120]
        \filldraw[fill=black!10!white, draw=black!70!white]
            (0,0) -- (7.5mm, 0) arc (0:120:7.5mm) -- cycle;
        \filldraw[fill=white, draw=black!70!white, rotate=120]
            (0,0) -- (7.5mm, 0) arc (0:120:7.5mm) -- cycle;
        \filldraw[fill=white, draw=black!70!white, rotate=240]
            (0,0) -- (7.5mm, 0) arc (0:120:7.5mm) -- cycle;
    \end{tikzpicture}
$
& Das ist immer noch ein Kreis. Dass er durchgeschnitten ist ändert das nicht. Es ist ein Teil von drei (gleich großen) Teile eines Kreises aus drei Teilen. Das sind $\frac{1}{3}$.\\\hline
$
    \begin{tikzpicture}[radius=7.5mm, delta angle=120]
        \filldraw[fill=black!10!white, draw=black!70!white]
            (0,0) -- (7.5mm, 0) arc (0:120:7.5mm) -- cycle;
        \filldraw[fill=black!10!white, draw=black!70!white, rotate=120]
            (0,0) -- (7.5mm, 0) arc (0:120:7.5mm) -- cycle;
        \filldraw[fill=white, draw=black!70!white, rotate=240]
            (0,0) -- (7.5mm, 0) arc (0:120:7.5mm) -- cycle;
    \end{tikzpicture}
$
& Das ist immer noch ein Kreis. Dass er durchgeschnitten ist ändert das nicht. Es sind zwei (gleich große) Teile von drei (gleich großen) Teilen eines Kreises aus drei Teilen. Das sind $\frac{2}{3}$.\\\hline
$
    \begin{tikzpicture}[baseline=-1mm,scale=0.5, .style={fontsize=\footnotesize}]
        \filldraw[fill=black!10!white, draw=black!70!white]
            (0,0) -- (7.5mm, 0) arc (0:120:7.5mm) -- cycle;
        \filldraw[fill=white, draw=black!70!white, rotate=120]
            (0,0) -- (7.5mm, 0) arc (0:120:7.5mm) -- cycle;
        \filldraw[fill=white, draw=black!70!white, rotate=240]
            (0,0) -- (7.5mm, 0) arc (0:120:7.5mm) -- cycle;
    \end{tikzpicture}
    +
    \begin{tikzpicture}[baseline=-1mm,scale=0.5, .style={fontsize=\footnotesize}]
        \filldraw[fill=white, draw=black!70!white]
            (0,0) -- (7.5mm, 0) arc (0:120:7.5mm) -- cycle;
        \filldraw[fill=black!10!white, draw=black!70!white, rotate=120]
            (0,0) -- (7.5mm, 0) arc (0:120:7.5mm) -- cycle;
        \filldraw[fill=black!10!white, draw=black!70!white, rotate=240]
            (0,0) -- (7.5mm, 0) arc (0:120:7.5mm) -- cycle;
    \end{tikzpicture}
    =
    \begin{tikzpicture}[baseline=-1mm,scale=0.5, .style={fontsize=\footnotesize}]
        \filldraw[fill=black!10!white, draw=black!70!white]
            (0,0) -- (7.5mm, 0) arc (0:120:7.5mm) -- cycle;
        \filldraw[fill=black!10!white, draw=black!70!white, rotate=120]
            (0,0) -- (7.5mm, 0) arc (0:120:7.5mm) -- cycle;
        \filldraw[fill=black!10!white, draw=black!70!white, rotate=240]
            (0,0) -- (7.5mm, 0) arc (0:120:7.5mm) -- cycle;
    \end{tikzpicture}
$
& S.o. $\frac{1}{3} + \frac{2}{3} = \frac{3}{3} = 1$. Gesprochen wir dies als \enquote{ein Drittel plus zwei Drittel gleich drei Drittel gleich eins}.\\\hline
$
    \begin{tikzpicture}[baseline=-1mm,scale=0.5, .style={fontsize=\footnotesize}]
        \filldraw[fill=black!10!white, draw=black!70!white]
            (0,0) -- (7.5mm, 0) arc (0:120:7.5mm) -- cycle;
        \filldraw[fill=white, draw=black!70!white, rotate=120]
            (0,0) -- (7.5mm, 0) arc (0:120:7.5mm) -- cycle;
        \filldraw[fill=white, draw=black!70!white, rotate=240]
            (0,0) -- (7.5mm, 0) arc (0:120:7.5mm) -- cycle;
    \end{tikzpicture}
    +
    \begin{tikzpicture}[baseline=-1mm,scale=0.5, .style={fontsize=\footnotesize}]
        \filldraw[fill=black!10!white, draw=black!70!white, rotate=120]
            (0,0) -- (7.5mm, 0) arc (0:60:7.5mm) -- cycle;
        \filldraw[fill=black!10!white, draw=black!70!white, rotate=180]
            (0,0) -- (7.5mm, 0) arc (0:60:7.5mm) -- cycle;
        \filldraw[fill=white, draw=black!70!white]
            (0,0) -- (7.5mm, 0) arc (0:120:7.5mm) -- cycle;
        \filldraw[fill=white, draw=black!70!white, rotate=240]
            (0,0) -- (7.5mm, 0) arc (0:60:7.5mm) -- cycle;
        \filldraw[fill=white, draw=black!70!white, rotate=60]
            (0,0) -- (7.5mm, 0) arc (0:60:7.5mm) -- cycle;
        \filldraw[fill=white, draw=black!70!white, rotate=300]
            (0,0) -- (7.5mm, 0) arc (0:60:7.5mm) -- cycle;
    \end{tikzpicture}
    =
    \begin{tikzpicture}[baseline=-1mm,scale=0.5, .style={fontsize=\footnotesize}]
        \filldraw[fill=black!10!white, draw=black!70!white, rotate=120]
            (0,0) -- (7.5mm, 0) arc (0:60:7.5mm) -- cycle;
        \filldraw[fill=black!10!white, draw=black!70!white, rotate=180]
            (0,0) -- (7.5mm, 0) arc (0:60:7.5mm) -- cycle;
        \filldraw[fill=black!10!white, draw=black!70!white]
            (0,0) -- (7.5mm, 0) arc (0:120:7.5mm) -- cycle;
        \filldraw[fill=white, draw=black!70!white, rotate=240]
            (0,0) -- (7.5mm, 0) arc (0:120:7.5mm) -- cycle;
    \end{tikzpicture}
$ & $\frac{1}{3} + \frac{2}{6} = \frac{2}{3}$ und $\frac{2}{6} = \frac{1}{3}$\\\hline
\end{longtable}

Die Zahl über dem Bruchstrich nennen wir \enquote{\textbf{Zähler}}, die Zahl unter dem Bruchstrich \enquote{\textbf{Nenner}}. Wir lesen Brüche als $\frac{1}{2}$\enquote{ein Halb} (oder \enquote{ein Halbes}), $\frac{1}{3}$ als \enquote{ein Drittel}, $\frac{1}{4}$ als \enquote{ein Viertel}, $\frac{1}{12}$ als \enquote{ein Zwölftel}, $\frac{13}{1000}$ als \enquote{dreizehn Tausendstel}. Wird der Bruch nicht als Nomen (der Zahl) verwendet, sondern als adverbiale Bestimmung der Anzahl eines benannten Etwas, wie in \enquote{dreiviertel Pizza}, dann wird klein und mit den unter Numeralia (s. \ref{numeralia}) Regeln zusammen oder getrennt geschrieben. Da wir solche Zahlen stets mit Ziffern schreiben, und beim Lesen/ Sprechen Groß- und Kleinschreibung und Trennungen wenig interessant sind, ist das nicht besonders wichtig. Wir machen einen großen Sprung und geben die Rechenregeln der Bruchrechnung an. Keine Angst, wir kehren nach Angabe einer Regel jeweils zu Beispielen zurück und nehmen uns alle Zeit der Welt um an diesen wichtigen Stellen niemanden zu verlieren.

\section{Rechenregeln}

Wir \textbf{addieren} Brüche indem wir sie auf gleiche \textbf{Nenner} bringen und dann die angepassten \textbf{Zähler} \textbf{addieren}. Den kleinsten gemeinsamen \textbf{Nenner} finden wir indem wir das \textbf{kleinste gemeinsame Vielfache (KGV)} der \textbf{Nenner} bestimmen. Das \textbf{KGV} finden wir indem wir die \textbf{Primfaktorzerlegungen} der \textbf{Nenner} \glsdisp{symb:Vereinigung}{vereinigen}\footnote{Wir vereinfachen hier etwas sprachlich, denn die Vereinigung von Mengen ist es nur dann genau wenn wir die Faktoren $m^n$ für verschiedene $n$ aufführen, also z.B. $Pfz(4)=\{2^1, 2^2\}$}.

Wir dürfen nur Brüche mit gleichem \textbf{Nenner} (direkt) \textbf{addieren} oder \textbf{subtrahieren}. Ansonsten müssen wir zuerst alle \textbf{Summanden} auf den gleichen \textbf{Nenner} bringen (\enquote{erweitern}).

\begin{equation}
    \frac{a}{c} + \frac{b}{c} = \frac{a+b}{c}
\end{equation}

\begin{figure}
  \centering
  \begin{tikzpicture}
    %\draw[step=1mm, line width=0.1mm, black!30!white] (0,0) grid (\width,\hauteur);
    \draw[step=5mm, line width=0.1mm, black!40!white] (0,0) grid (\width,\hauteur);
    %\draw[step=5cm, line width=0.5mm, black!50!white] (0,0) grid (\width,\hauteur);
    %\draw[step=1cm, line width=0.3mm, black!90!white] (0,0) grid (\width,\hauteur);

    \node at (1.25cm, 11.25cm) {1};
    \draw (0.75cm, 10.75) -- (1.75cm, 10.75);
    \node at (1.25cm, 10.25cm) {2};

    \node at (2.25cm, 10.75cm) {+};

    \node at (3.25cm, 11.25cm) {1};
    \draw (2.75cm, 10.75) -- (3.75cm, 10.75);
    \node at (3.25cm, 10.25cm) {3};

    \node at (4.25cm, 10.75cm) {=};

    \node at (4.75cm, 10.75cm) {?};


    \node at (5.75cm, 10.75cm) {K};
    \node at (6.25cm, 10.75cm) {G};
    \node at (6.75cm, 10.75cm) {V};
    \node at (7.25cm, 10.75cm) {(};
    \node at (7.75cm, 10.75cm) {2};
    \node at (8.25cm, 10.75cm) {;};
    \node at (8.75cm, 10.75cm) {3};
    \node at (9.25cm, 10.75cm) {)};
    \node at (9.75cm, 10.75cm) {=};
    \node at (10.25cm, 10.75cm) {?};

    \node at (5.75cm, 9.75cm) {P};
    \node at (6.25cm, 9.75cm) {F};
    \node at (6.75cm, 9.75cm) {Z};
    \node at (7.25cm, 9.75cm) {(};
    \node at (7.75cm, 9.75cm) {2};
    \node at (8.25cm, 9.75cm) {)};
    \node at (8.75cm, 9.75cm) {=};
    \node at (9.25cm, 9.75cm) {\{};
    \node at (9.75cm, 9.75cm) {2};
    \node at (10.25cm, 9.75cm) {\}};

    \node at (5.75cm, 9.25cm) {P};
    \node at (6.25cm, 9.25cm) {F};
    \node at (6.75cm, 9.25cm) {Z};
    \node at (7.25cm, 9.25cm) {(};
    \node at (7.75cm, 9.25cm) {3};
    \node at (8.25cm, 9.25cm) {)};
    \node at (8.75cm, 9.25cm) {=};
    \node at (9.25cm, 9.25cm) {\{};
    \node at (9.75cm, 9.25cm) {3};
    \node at (10.25cm, 9.25cm) {\}};

    \node at (1.25cm, 8.25cm) {P};
    \node at (1.75cm, 8.25cm) {F};
    \node at (2.25cm, 8.25cm) {Z};
    \node at (2.75cm, 8.25cm) {(};
    \node at (3.25cm, 8.25cm) {2};
    \node at (3.75cm, 8.25cm) {)};
    \node at (4.25cm, 8.25cm) {$\bigcup$};
    \node at (4.75cm, 8.25cm) {P};
    \node at (5.25cm, 8.25cm) {F};
    \node at (5.75cm, 8.25cm) {Z};
    \node at (6.25cm, 8.25cm) {(};
    \node at (6.75cm, 8.25cm) {3};
    \node at (7.25cm, 8.25cm) {)};
    \node at (7.75cm, 8.25cm) {=};
    \node at (8.25cm, 8.25cm) {\{};
    \node at (8.75cm, 8.25cm) {2};
    \node at (9.25cm, 8.25cm) {;};
    \node at (9.75cm, 8.25cm) {3};
    \node at (10.25cm, 8.25cm) {\}};

    \node at (1.25cm, 7.25cm) {$\Rightarrow$};
    \node at (1.75cm, 7.25cm) {K};
    \node at (2.25cm, 7.25cm) {G};
    \node at (2.75cm, 7.25cm) {V};
    \node at (3.25cm, 7.25cm) {(};
    \node at (3.75cm, 7.25cm) {2};
    \node at (4.25cm, 7.25cm) {;};
    \node at (4.75cm, 7.25cm) {3};
    \node at (5.25cm, 7.25cm) {)};
    \node at (5.75cm, 7.25cm) {=};
    \node at (6.25cm, 7.25cm) {2};
    \node at (6.75cm, 7.25cm) {$\cdot$};
    \node at (7.25cm, 7.25cm) {3};
    \node at (7.75cm, 7.25cm) {=};
    \node at (8.25cm, 7.25cm) {6};

    \node at (1.25cm, 6.25cm) {1};
    \draw (0.75cm, 5.75) -- (1.75cm, 5.75);
    \node at (1.25cm, 5.25cm) {2};

    \node at (2.25cm, 5.75cm) {+};

    \node at (3.25cm, 6.25cm) {1};
    \draw (2.75cm, 5.75) -- (3.75cm, 5.75);
    \node at (3.25cm, 5.25cm) {3};

    \node at (4.25cm, 5.75cm) {=};

    \node at (5.25cm, 6.25cm) {3};
    \draw (4.75cm, 5.75) -- (5.75cm, 5.75);
    \node at (5.25cm, 5.25cm) {6};

    \node at (6.25cm, 5.75cm) {+};

    \node at (7.25cm, 6.25cm) {2};
    \draw (6.75cm, 5.75) -- (7.75cm, 5.75);
    \node at (7.25cm, 5.25cm) {6};

    \node at (8.25cm, 5.75cm) {=};

    \node at (9.25cm, 6.25cm) {5};
    \draw (8.75cm, 5.75) -- (9.75cm, 5.75);
    \node at (9.25cm, 5.25cm) {6};
  \end{tikzpicture}
  \caption{Brüche addieren}\label{fig:BruecheAddieren1}
\end{figure}

\begin{figure}
  \centering
  \begin{tikzpicture}
    %\draw[step=1mm, line width=0.1mm, black!30!white] (0,0) grid (\width,\hauteur);
    \draw[step=5mm, line width=0.1mm, black!40!white] (0,0) grid (\width,\hauteur);
    %\draw[step=5cm, line width=0.5mm, black!50!white] (0,0) grid (\width,\hauteur);
    %\draw[step=1cm, line width=0.3mm, black!90!white] (0,0) grid (\width,\hauteur);

    \node at (0.75cm, 11.25cm) {};
    \node at (1.25cm, 11.25cm) {3};
    \draw (0.25cm, 10.75) -- (1.75cm, 10.75);
    \node at (0.75cm, 10.25cm) {1};
    \node at (1.25cm, 10.25cm) {4};

    \node at (2.25cm, 10.75cm) {+};

    \node at (2.75cm, 11.25cm) {};
    \node at (3.75cm, 11.25cm) {5};
    \draw (2.75cm, 10.75) -- (4.25cm, 10.75);
    \node at (3.25cm, 10.25cm) {1};
    \node at (3.75cm, 10.25cm) {2};

    \node at (4.75cm, 10.75cm) {=};

    \node at (5.25cm, 10.75cm) {?};


    \node at (9.75cm, 10.75cm) {K};
    \node at (10.25cm, 10.75cm) {G};
    \node at (10.75cm, 10.75cm) {V};
    \node at (11.25cm, 10.75cm) {=};
    \node at (11.75cm, 10.75cm) {?};

    \node at (0.25cm, 9.25cm) {P};
    \node at (0.75cm, 9.25cm) {F};
    \node at (1.25cm, 9.25cm) {Z};
    \node at (1.75cm, 9.25cm) {(};
    \node at (2.25cm, 9.25cm) {1};
    \node at (2.75cm, 9.25cm) {4};
    \node at (3.25cm, 9.25cm) {)};
    \node at (3.75cm, 9.25cm) {=};
    \node at (4.25cm, 9.25cm) {\{};
    \node at (4.75cm, 9.25cm) {2};
    \node at (5.25cm, 9.25cm) {;};
    \node at (5.75cm, 9.25cm) {7};
    \node at (6.25cm, 9.25cm) {\}};

    \node at (0.25cm, 8.75cm) {P};
    \node at (0.75cm, 8.75cm) {F};
    \node at (1.25cm, 8.75cm) {Z};
    \node at (1.75cm, 8.75cm) {(};
    \node at (2.25cm, 8.75cm) {1};
    \node at (2.75cm, 8.75cm) {2};
    \node at (3.25cm, 8.75cm) {)};
    \node at (3.75cm, 8.75cm) {=};
    \node at (4.25cm, 8.75cm) {\{};
    \node at (4.75cm, 8.75cm) {2};
    \node at (5.25cm, 8.75cm) {;};
    \node at (5.75cm, 8.75cm) {2};
    \node at (6.25cm, 8.75cm) {;};
    \node at (6.75cm, 8.75cm) {3};
    \node at (7.25cm, 8.75cm) {\}};


    \node at (0.25cm, 8.25cm) {P};
    \node at (0.75cm, 8.25cm) {F};
    \node at (1.25cm, 8.25cm) {Z};
    \node at (1.75cm, 8.25cm) {(};
    \node at (2.25cm, 8.25cm) {1};
    \node at (2.75cm, 8.25cm) {4};
    \node at (3.25cm, 8.25cm) {)};
    \node at (3.75cm, 8.25cm) {$\bigcup$};
    \node at (4.25cm, 8.25cm) {P};
    \node at (4.75cm, 8.25cm) {F};
    \node at (5.25cm, 8.25cm) {Z};
    \node at (5.75cm, 8.25cm) {(};
    \node at (6.25cm, 8.25cm) {1};
    \node at (6.75cm, 8.25cm) {2};
    \node at (7.25cm, 8.25cm) {)};
    \node at (0.75cm, 7.75cm) {=};
    \node at (1.25cm, 7.75cm) {\{};
    \node at (1.75cm, 7.75cm) {2};
    \node at (2.25cm, 7.75cm) {;};
    \node at (2.75cm, 7.75cm) {2};
    \node at (3.25cm, 7.75cm) {;};
    \node at (3.75cm, 7.75cm) {3};
    \node at (4.25cm, 7.75cm) {;};
    \node at (4.75cm, 7.75cm) {7};
    \node at (5.25cm, 7.75cm) {\}};

    \node at (0.25cm, 7.25cm) {K};
    \node at (0.75cm, 7.25cm) {G};
    \node at (1.25cm, 7.25cm) {V};
    \node at (1.75cm, 7.25cm) {=};
    \node at (2.25cm, 7.25cm) {2};
    \node at (2.75cm, 7.25cm) {$\cdot$};
    \node at (3.25cm, 7.25cm) {2};
    \node at (3.75cm, 7.25cm) {$\cdot$};
    \node at (4.25cm, 7.25cm) {3};
    \node at (4.75cm, 7.25cm) {$\cdot$};
    \node at (5.25cm, 7.25cm) {7};
    \node at (5.75cm, 7.25cm) {=};
    \node at (6.25cm, 7.25cm) {8};
    \node at (6.75cm, 7.25cm) {4};

    \node at (0.75cm, 6.25cm) {1};
    \node at (1.25cm, 6.25cm) {8};
    \draw (0.25cm, 5.75) -- (1.75cm, 5.75);
    \node at (0.75cm, 5.25cm) {8};
    \node at (1.25cm, 5.25cm) {4};

    \node at (2.25cm, 5.75cm) {+};

    \node at (3.25cm, 6.25cm) {3};
    \node at (3.75cm, 6.25cm) {5};
    \draw (2.75cm, 5.75) -- (4.25cm, 5.75);
    \node at (3.25cm, 5.25cm) {8};
    \node at (3.75cm, 5.25cm) {4};

    \node at (4.75cm, 5.75cm) {=};

    \node at (5.75cm, 6.25cm) {5};
    \node at (6.25cm, 6.25cm) {3};
    \draw (5.25cm, 5.75) -- (6.75cm, 5.75);
    \node at (5.75cm, 5.25cm) {8};
    \node at (6.25cm, 5.25cm) {4};
  \end{tikzpicture}
  \caption{Brüche addieren}\label{fig:BruecheAddieren2}
\end{figure}


\begin{equation}
    \frac{a}{c} - \frac{b}{c} = \frac{a-b}{c}
\end{equation}

Wir multiplizieren Brüche indem wir die Zähler multiplizieren und das Ergebnis der Zähler des Produktes ist und die Nenner multiplizieren und das Ergebnis der Nenner des Produktes ist.

\begin{beispiel}[Brüche multiplizieren]
    \begin{align}\label{eqn:BruecheMultiplikation01}
      \frac{2}{3} \cdot \frac{5}{7} & = \frac{10}{21}\\
      3 \cdot \frac{5}{7} = \frac{3}{1} \cdot \frac{5}{7} & = \frac{15}{7}\\
      \frac{8}{21} \cdot \frac{7}{16} & = \frac{1}{3} \cdot \frac{1}{2} = \frac{1}{6}\\
      \frac{5}{3} \cdot \frac{3}{5} & = 1
    \end{align}
\end{beispiel}

\begin{equation}
    \frac{a}{c} \cdot \frac{b}{d} = \frac{a b}{c d}
\end{equation}

Wir dividieren durch Brüche wir indem wir mit dem \textbf{Kehrwert} multiplizieren.

\begin{equation}\label{dividieren durch multiplizieren mit Kehrwert}
    \frac{a}{c} \div \frac{b}{d} = \frac{a}{c} \cdot \frac{d}{b} = \frac{a d}{b c}
\end{equation}

Brüche deren \textbf{Zähler} und \textbf{Nenner} einen gemeinsamen \textbf{Teiler} haben kann man \textbf{kürzen}.

\begin{equation}
    \frac{n a}{n b} = \frac{a}{b}
\end{equation}

Wir sagen \enquote{$\frac{n a}{n b}$ lässt sich mit $n$ \textbf{kürzen} und ist als \textbf{gekürzter Bruch} $\frac{a}{b}$}.

\begin{beispiel}
    Der Bruch $r=\frac{2}{6}$ lässt sich \textbf{kürzen}, denn $2$ und $6$ haben den \textbf{gemeinsamen Teiler} $2$. $\frac{2}{6} = \frac{1}{3}$.\topicend
\end{beispiel}

Brüche deren \textbf{Zähler} größer als ihr \textbf{Nenner} ist können als \textbf{gemischter Bruch} geschrieben werden.

\begin{equation}\glsadd{symb:Abrunden}
    \frac{a}{b} = \left\lfloor\frac{a}{b}\right\rfloor \frac{a-b \left\lfloor\frac{a}{b}\right\rfloor}{b}\footnote{$\left\lfloor \frac{a}{b} \right\rfloor$ bedeutet das auf die nächst kleinere ganze Zahl \textbf{abgerundete} Ergebnis von  $a \div b$. $\frac{5}{2} = 2,5$. $\left\lfloor 2,5 \right\rfloor = 2$. Also $\left\lfloor\frac{5}{2} \right\rfloor = 2$.}
\end{equation}

\begin{beispiel}
    $\frac{13}{2} = 6 \frac{1}{2}$
\end{beispiel}


\section{Prozentrechnung}\label{Prozentrechnung}\index{Prozentrechnung}

Wir betrachten \enquote{Prozentrechnung} nicht als eigenes (Haupt-)Thema. \enquote{Prozentrechnung} ist einfach Bruchrechnung mit Hundertsteln.

\begin{eqnarray}
% \nonumber to remove numbering (before each equation)
  \text{\textbf{Grundwert}} (GW) \cdot \text{\textbf{Prozentsatz}} (PS) &=& \text{\textbf{Prozentwert}} (PW)\\
  PS &=& \frac{PW}{GW} \\
  GW &=& \frac{PW}{PS}
\end{eqnarray}

Beim Rechnen mit Geld sind \textbf{Kapital} und \textbf{Guthaben}\footnote{sowie \textbf{Schulden} als negatives \textbf{Guthaben}} Synonyme für \textbf{Grundwert}, \textbf{Zinssatz} für \textbf{Prozentsatz} und \textbf{Zinsen} für \textbf{Prozentwert}. Außerdem ist \textbf{Ist} synonym mit positivem \textbf{Guthaben} und \textbf{Soll} mit negativem \textbf{Guthaben} (\textbf{Schulden}).

\begin{beispiel}
    Zu bestimmen sei wie viel Gramm die 23 Prozent Zucker eines 35 Gramm Schokolandenriegels sind.
    \begin{eqnarray}
    % \nonumber to remove numbering (before each equation)
      Prozentwert &=& Grundwert \cdot Prozentsatz \\
       &=& 35g \cdot \frac{7}{100} \\
       &=& \frac{35}{100}g\\
       &=& 0,35g
    \end{eqnarray}\topicend
\end{beispiel}


\chapter{Geometrie}\index{Geometrie}

In der prüfungsrelevanten Geometrie konstruieren und berechnen wir Dreiecke, andere elementare 2-dimensionale Flächen und 3-dimensionale Körper, \textbf{Prismen}\index{Prisma} und Zylinder.


\section{Flächen}

\subsection{Flächen (-inhalt)}

Eine rechteckige (ebne/ 2-dimensionale) Fläche beschreiben wir durch Länge und Breite. Wir schreiben für ihre Fläche/ ihren Flächeninhalt $A_R = a b$ und lesen \enquote{Die Fläche eines Rechtecks ist gleich Länge mal Breite.} Gemäß \gls{SISystem} ist unsere Standard-Einheit für Strecken der Meter. Eine quadratische (und damit auch rechteckige) Fläche mit einer Seitenlänge von einem Meter nennen wir einen Quadratmeter und schreiben $A_Q = a^2 = (1\text{m})^2 = 1\text{m}^2$ und lesen \enquote{Die Fläche eines Quadrats mit Seitenlängen von je einem Meter ist gleich ein Quadratmeter} oder \enquote{Der Flächeneinheit eines Quadrates mit Seitenlängen gleich ein Meter ist ein Quadratmeter}. Entsprechend hat ein Quadrat mit Seitenlängen von einem Millimeter eine Fläche oder einen Flächeninhalt von einem Quadratmillimeter. Mit solchen Kacheln können wir offensichtlich die Flächeninhalte aller planen Flächen/ Figuren ausdrücken (auch wenn nicht bei allen klar ist wie wir sie messen können).

\begin{beispiel}
    \begin{figure}
      \centering
        \begin{tikzpicture}
            \draw (1cm, 1cm) -- (2cm, 1cm) -- (2cm, 2cm) -- (1cm, 2cm) -- (1cm, 1cm);%Quadrat 1

            \draw (1cm, 0.9cm) -- (1cm, 0.7cm);%Bemaßung horizontal
            \draw (2cm, 0.9cm) -- (2cm, 0.7cm);
            \draw (1cm, 0.8cm) -- (1.2cm, 0.8cm);
            \draw (1.8cm, 0.8cm) -- (2cm, 0.8cm);
            \node at (1.5cm,0.8cm) {\tiny 1cm};

            \draw (0.6cm, 1cm) -- (0.8cm, 1cm);%Bemaßung vertikal
            \draw (0.6cm, 2cm) -- (0.8cm, 2cm);
            \draw (0.7cm, 1cm) -- (0.7cm, 1.2cm);
            \draw (0.7cm, 2cm) -- (0.7cm, 1.8cm);
            \node[rotate=90] at (0.7cm,1.5cm) {\tiny 1cm};

            \node at (1.5cm, 1.5cm) {$\scriptscriptstyle\text{1cm}^2$};


            \draw (3cm, 1cm) -- (5cm, 1cm) -- (5cm, 3cm) -- (3cm, 3cm) -- (3cm, 1cm);%Quadrat 2

            \draw (3cm, 0.9cm) -- (3cm, 0.7cm);%Bemaßung horizontal
            \draw (5cm, 0.9cm) -- (5cm, 0.7cm);
            \draw (3cm, 0.8cm) -- (3.5cm, 0.8cm);
            \draw (4.5cm, 0.8cm) -- (5cm, 0.8cm);
            \node at (4cm,0.8cm) {\tiny 2cm};

            \draw (2.6cm, 1cm) -- (2.8cm, 1cm);%Bemaßung vertikal
            \draw (2.6cm, 3cm) -- (2.8cm, 3cm);
            \draw (2.7cm, 1cm) -- (2.7cm, 1.5cm);
            \draw (2.7cm, 2.5cm) -- (2.7cm, 3cm);
            \node[rotate=90] at (2.7cm,2cm) {\tiny 2cm};

            \node at (4cm, 2cm) {$\scriptscriptstyle\text{4cm}^2$};


            \draw (6cm, 1cm) -- (9cm, 1cm) -- (9cm, 4cm) -- (6cm, 4cm) -- (6cm, 1cm);%Quadrat 3

            \draw[gray] (7cm, 1cm) -- (7cm, 4cm);
            \draw[gray] (8cm, 1cm) -- (8cm, 4cm);
            \draw[gray] (6cm, 2cm) -- (9cm, 2cm);
            \draw[gray] (6cm, 3cm) -- (9cm, 3cm);

            \draw (6cm, 0.9cm) -- (6cm, 0.7cm);%Bemaßung horizontal
            \draw (9cm, 0.9cm) -- (9cm, 0.7cm);
            \draw (6cm, 0.8cm) -- (7cm, 0.8cm);
            \draw (8cm, 0.8cm) -- (9cm, 0.8cm);
            \node at (7.5cm,0.8cm) {\tiny 3cm};

            \draw (5.6cm, 1cm) -- (5.8cm, 1cm);%Bemaßung vertikal
            \draw (5.6cm, 4cm) -- (5.8cm, 4cm);
            \draw (5.7cm, 1cm) -- (5.7cm, 2cm);
            \draw (5.7cm, 3cm) -- (5.7cm, 4cm);
            \node[rotate=90] at (5.7cm,2.5cm) {\tiny 3cm};

        \end{tikzpicture}
      \caption{Fläche/ Flächeninhalt}\label{fig:geometrieArea1}
    \end{figure}
\end{beispiel}


\subsection{Umfang}\index{Umfang}

\begin{definition}[Umfang]
    Als Umfang einer (ebenen zusammenhängenden, von Löchern freien) Fläche/ Figur bezeichnen wir die Strecke, die ein Faden/ Maßband, ... misst, der/das genau benötigt würde um die Außenlinie einmal einzuschließen. Man kann z.B. auch an die Länge denken, die bei Zeichnen (ohne abzusetzen) die Spitze des Stiftes zurücklegt. Wir schreiben \glsdisp{symb:Umfang}{$U_F$} für den Umfang der Fläche $F$ und ersetzen $F$ durch den ersten Buchstaben der Fläche, z.B. $U_Q$ für den Umfang des Quadrats.
\end{definition}

\begin{beispiel}[Umfang]
    \begin{figure}
      \centering
        \begin{tikzpicture}
            \draw (1cm, 1cm) -- (8cm, 1cm) -- (8cm, 5cm) -- (1cm, 5cm) -- (1cm, 1cm);

            \draw (1cm, 0.7cm) -- (1cm, 0.9cm);
            \draw[thick,->,shorten >=2pt,shorten <=2pt,>=stealth] (1cm, 0.8cm) -- (4cm, 0.8cm) (5cm, 0.8cm) -> (8cm, 0.8cm);
            \node at (4.5cm, 0.8cm) {\tiny 7cm};
            \draw[thick,->,shorten >=2pt,shorten <=2pt,>=stealth] (0.8cm, 5cm) -- (0.8cm, 3.5cm) (0.8cm, 2.5cm) -> (0.8cm, 1cm);
            \node[rotate=90] at (0.8cm, 3cm) {\tiny 4cm};
            \draw[thick,->,shorten >=2pt,shorten <=2pt,>=stealth] (8.2cm, 1cm) -- (8.2cm, 2.5cm) (8.2cm, 3.5cm) -> (8.2cm, 5cm);
            \node[rotate=90] at (8.2cm, 3cm) {\tiny 4cm};
            \draw[thick,->,shorten >=2pt,shorten <=2pt,>=stealth] (8cm, 5.2cm) -- (5cm, 5.2cm) (4cm, 5.2cm) -> (1cm, 5.2cm);
            \node at (4.5cm, 5.2cm) {\tiny 7cm};
        \end{tikzpicture}
      \caption{Umfang: $U_R=7\text{cm} + 4\text{cm} + 7\text{cm} + 4\text{cm} = 22\text{cm}$}\label{fig:geometrieUmfang1}
    \end{figure}

    Der Umfang des Rechtecks mit Länge $a=7\text{cm}$ und Breite $c=4\text{cm}$ (s. Abb. \ref{fig:geometrieUmfang1}) errechnet sich als $U_R=a+b+a+b=2a+2b=14\text{cm}+8\text{cm}=22\text{cm}$.
\end{beispiel}



\subsection{Rechteck}\label{Rechteck}\index{Rechteck}

\begin{figure}
  \centering
  \begin{longtable}{rcl}
      \begin{tikzpicture}
            \draw (1cm,1cm) -- (5cm,1cm) -- (5cm, 3cm) -- (1cm, 3cm) -- (1cm, 1cm);
            \node at (0.75cm, 0.75cm) {A};
            \node at (5.25cm, 0.75cm) {B};
            \node at (5.25cm, 3.25cm) {C};
            \node at (0.75cm, 3.25cm) {D};
            \node at (3cm, 0.75cm) {a};
            \node at (5.25cm, 2cm) {b};
            \node at (3cm, 3.25cm) {c};
            \node at (0.75cm, 2cm) {d};
            \draw (1cm,1cm) ++(0:0.5) arc (0:90:0.5cm);
            \node at (1.2,1.2) {$\scriptstyle \alpha$};
            \draw (4.5cm,1cm) ++(-0.5:0.0) arc (180:90:0.5cm);
            \node at (4.8,1.2) {$\scriptstyle \beta$};
            \draw (4.5cm,3cm) arc (180:270:0.5cm);
            \node at (4.8,2.8) {$\scriptstyle \gamma$};
            \draw (1cm,2.5cm) arc (-90:0:0.5cm);
            \node at (1.2,2.8) {$\scriptstyle \delta$};
      \end{tikzpicture}
      &
        \begin{tikzpicture}
            \node at (0cm, 0cm) {};
            \node at (0.5cm, 1.5cm) {=};
            \node at (1cm,4.5cm) {};
        \end{tikzpicture}
      &
      \begin{tikzpicture}
            \draw (1cm,1cm) -- (5cm,1cm) -- (5cm, 3cm) -- (1cm, 3cm) -- (1cm, 1cm);
            \node at (0.75cm, 0.75cm) {A};
            \node at (5.25cm, 0.75cm) {B};
            \node at (5.25cm, 3.25cm) {C};
            \node at (0.75cm, 3.25cm) {D};
            \node at (3cm, 0.75cm) {a};
            \node at (5.25cm, 2cm) {b};
            \node at (3cm, 3.25cm) {a};
            \node at (0.75cm, 2cm) {b};
            \draw (1cm,1cm) ++(0:0.5) arc (0:90:0.5cm);
            \node at (1.2,1.2) {$\scriptstyle \alpha$};
            \draw (4.5cm,1cm) ++(-0.5:0.0) arc (180:90:0.5cm);
            \node at (4.8,1.2) {$\scriptstyle \alpha$};
            \draw (4.5cm,3cm) arc (180:270:0.5cm);
            \node at (4.8,2.8) {$\scriptstyle \alpha$};
            \draw (1cm,2.5cm) arc (-90:0:0.5cm);
            \node at (1.2,2.8) {$\scriptstyle \alpha$};
      \end{tikzpicture}
  \end{longtable}
  \caption{Planskizze Rechteck}\label{fig:RechteckPlanskizze}
\end{figure}

\begin{definition}[Rechteck 1]
    Wir nennen ein Viereck mit vier rechten Winkeln ein Rechteck.
\end{definition}

\begin{definition}[Rechteck 2]
    Wir nennen ein Viereck mit zwei Paaren paralleler Seiten mit einem rechten Winkeln ein Rechteck.
\end{definition}

\begin{definition}[Rechteck 3]
    Wir nennen ein Viereck mit zwei diagonal gegenüber liegenden rechten Winkeln ein Rechteck.
\end{definition}

Wir nennen die Punkte eines Rechtecks (s. Abb. \ref{fig:RechteckPlanskizze}) gegen den \glsdisp{Uhrzeigersinn}{\textbf{Uhrzeigersinn}} $A$, $B$, $C$ und $D$. Die gerade Strecke von $A$ nach $B$, $\overline{AB}$, nennen wir \enquote{die Seite $a$}. Die gerade Strecke von $B$ nach $C$, $\overline{BC}$, nennen wir \enquote{die Seite $b$}. Die gerade Strecke von $C$ nach $D$, $\overline{CD}$, nennen wir \enquote{die Seite $c$}. Die gerade Strecke von $D$ nach $C$, $\overline{DC}$, nennen wir \enquote{die Seite $c$}. Die gerade Strecke von $D$ nach $A$, $\overline{DA}$, nennen wir \enquote{die Seite $d$}. Den Winkel am Punkt $A$ nennen wir \glsdisp{symb:alpha}{$\alpha$}, sprich \enquote{alpha}. Den Winkel am Punkt $B$ nennen wir \glsdisp{symb:beta}{$\beta$}, sprich \enquote{beta}. Den Winkel am Punkt $C$ nennen wir \glsdisp{symb:gamma}{$\gamma$}, sprich \enquote{gamma}. Den Winkel am Punkt $D$ nennen wir \glsdisp{symb:delta}{$\delta$}, sprich \enquote{delta}.

\begin{eqnarray}\label{eqn:RechteckFormeln}
    A_R &=& a b\\
    U_R &=& 2a + 2b
\end{eqnarray}


\subsection{Quadrat}\index{Quadrat}

\begin{figure}
    \centering
    \begin{longtable}{rcl}
        \begin{tikzpicture}
            \draw (1cm,1cm) -- (4cm,1cm) -- (4cm, 4cm) -- (1cm, 4cm) -- (1cm, 1cm);
            \node at (0.75cm, 0.75cm) {A};
            \node at (4.25cm, 0.75cm) {B};
            \node at (4.25cm, 4.25cm) {C};
            \node at (0.75cm, 4.25cm) {D};
            \node at (2.5cm, 0.75cm) {a};
            \node at (4.25cm, 2.5cm) {b};
            \node at (2.5cm, 4.25cm) {c};
            \node at (0.75cm, 2.5cm) {d};
            \draw (1cm,1cm) ++(0:0.5) arc (0:90:0.5cm);
            \node at (1.2,1.2) {$\scriptstyle \alpha$};
            \draw (3.5cm,1cm) ++(-0.5:0.0) arc (180:90:0.5cm);
            \node at (3.8,1.2) {$\scriptstyle \beta$};
            \draw (3.5cm,4cm) arc (180:270:0.5cm);
            \node at (3.8,3.8) {$\scriptstyle \gamma$};
            \draw (1cm,3.5cm) arc (-90:0:0.5cm);
            \node at (1.2,3.8) {$\scriptstyle \delta$};
        \end{tikzpicture}
      &
        \begin{tikzpicture}
            \node at (0cm, 0cm) {};
            \node at (0.5cm, 1.5cm) {=};
            \node at (1cm,4.5cm) {};
        \end{tikzpicture}
      &
        \begin{tikzpicture}
            \draw (1cm,1cm) -- (4cm,1cm) -- (4cm, 4cm) -- (1cm, 4cm) -- (1cm, 1cm);
            \node at (0.75cm, 0.75cm) {A};
            \node at (4.25cm, 0.75cm) {B};
            \node at (4.25cm, 4.25cm) {C};
            \node at (0.75cm, 4.25cm) {D};
            \node at (2.5cm, 0.75cm) {a};
            \node at (4.25cm, 2.5cm) {a};
            \node at (2.5cm, 4.25cm) {a};
            \node at (0.75cm, 2.5cm) {a};
            \draw (1cm,1cm) ++(0:0.5) arc (0:90:0.5cm);
            \node at (1.2,1.2) {$\scriptstyle \alpha$};
            \draw (3.5cm,1cm) ++(-0.5:0.0) arc (180:90:0.5cm);
            \node at (3.8,1.2) {$\scriptstyle \alpha$};
            \draw (3.5cm,4cm) arc (180:270:0.5cm);
            \node at (3.8,3.8) {$\scriptstyle \alpha$};
            \draw (1cm,3.5cm) arc (-90:0:0.5cm);
            \node at (1.2,3.8) {$\scriptstyle \alpha$};
        \end{tikzpicture}
    \end{longtable}
  \caption{Planskizze Quadrat}\label{fig:QuadratPlanskizze}
\end{figure}

\begin{definition}[Quadrat]
    Wir nennen ein Rechteck mit vier gleich langen Seiten \enquote{Quadrat}.
\end{definition}


\subsection{Dreieck}\index{Dreieck}\label{Dreieck}

\begin{definition}
    Wir nennen ein Polygon mit drei Ecken \enquote{Dreieck}.
\end{definition}

\begin{figure}
  \centering
  \begin{tikzpicture}
        \draw (1cm, 1cm) -- (4cm, 1cm) -- (3cm, 4cm) -- (1cm, 1cm);

        \node at (0.75cm, 0.75cm) {A};
        \node at (4.25cm, 0.75cm) {B};
        \node at (3cm, 4.25cm) {C};

        \node at (2.5cm, 0.75cm) {c};
        \node at (3.75cm, 2.75cm) {a};
        \node at (1.75cm, 2.5cm) {b};

        \draw[dashed] (3cm, 1cm) -- (3cm, 4cm);
        \node at (2.75, 2.5) {$h_c$};
        \draw (2.5cm, 1cm) arc (180:90:0.5cm);
        \node at (2.85cm, 1.15cm) {$\cdot$};

        \draw (1cm,1cm) ++(0:0.5) arc (0:57:0.5cm);
        \node at (1.3,1.2) {$\scriptstyle \alpha$};
        \draw (3.5cm,1cm) ++(-0.5:0.0) arc (180:110:0.5cm);
        \node at (3.7,1.2) {$\scriptstyle \beta$};
        \draw (2.65cm,3.5cm) arc (231:295:0.5cm);
        \node at (2.9,3.6) {$\scriptstyle \gamma$};
  \end{tikzpicture}
  \caption{Planskizze Dreieck}\label{fig:DreieckPlanskizze}
\end{figure}


\subsection{Parallelogramm}\index{Parallelogramm}\label{Parallelogramm}

\begin{definition}[Parallelogramm]
    Wir nennen ein Viereck mit zwei mal zwei \textbf{parallelen} Seiten \enquote{Parallelogramm}.
\end{definition}

Alle Parallelogramme sind Trapeze.


\subsection{Trapez}\index{Trapez}\label{Trapez}

\begin{definition}
    Wir nennen ein Viereck mit zwei Parallelen Seiten \enquote{Trapez}.
\end{definition}


\subsection{Kreis}\index{Kreis}\label{Kreis}

Der \textbf{Kreis} ist die einzige Fläche, die für den Hauptschulabschluss relevant ist, die kein Polygon ist. Sie hat keine Ecke. Sie wird beschrieben durch \textbf{Mittelpunkt} und \textbf{Radius}.

\begin{definition}
    Wir nennen eine Fläche, die von einer kreisförmigen Linie begrenzt ist, d.h. jeder Punkt auf dem Rand der Fläche ist gleich weit vom \textbf{Mittelpunkt} der Fläche entfernt, \enquote{Kreis}.
\end{definition}

\begin{align}\label{eqn:Kreis}
  A_K &= \pi r^2 \\
  U_K &= 2 \pi r
\end{align}



\section{Körper}

\subsection{Volumen}\index{Volumen}\label{Volumen}

Wir nennen den Raum/ Rauminhalt eines Körpers (\enquote{Wie viel Masse hat/ enthält der Körper}) \enquote{Volumen}.


\subsection{Oberfläche}\index{Oberfläche}\label{Oberfläche}

\begin{definition}[Oberfläche]
    Wir nennen die Summe der Flächeninhalte der Flächen, die einen Körper begrenzen, \enquote{Oberfläche}.    
\end{definition}




\subsection{Quader}\index{Quader}\label{Quader}

\subsection{Würfel}\index{Würfel}\label{Würfel}

\subsection{Prisma}\index{Prisma}\label{Prisma}

\subsection{Zylinder}\index{Zylinder}\label{Zylinder}

\subsection{Maßstab}\index{Maßstab}\label{Maßstab}

Zeichnungen geben Objekte der Realität häufig nicht in ihrer natürlichen Größe wieder. Große Objekte werden oft verkleinert dargestellt, sehr kleine vergrößert. Um aus solch einer nicht maßstabsgetreuen Zeichnung die wirklichen Strecken zu ermitteln müssen wir die Messungen an der Zeichnung mit dem Maßstab umrechnen. Eine Entfernung von 7cm auf einer Fahrradkarte im Maßstab 1:100000 sind z.B. in Wirklichkeit 7cm mal 100000, also 700000cm = 7km. Gebräuchlich sind Maßstäbe 1:n, sprich \enquote{eins zu n} für verkleinerte Darstellungen wie auf Landkarten sowie n:1 für vergrößerte Darstellungen, z.B. mikroskopische Lebewesen in einem Biologie-Buch. Häufig wird der Maßstab auch gezeichnet und eine Strecke am Rand der Abbildung, die z.B. 1cm lang ist mit 1mm beschriftet. Das bedeutet, dass die Abbildung im Maßstab 10:1 vergrößert dargestellt ist, z.B. ein Käfer wiederum im Biologie-Buch.


\chapter{Proportionale und Antiproportionale Zuordnungen}\index{proportionale Zuordnung}\index{antiproportionale Zuordnung|see{proportionale Zuordnung}}


\chapter{Graphische Darstellungen}

Viele Sachverhalte der (schulischen) Mathematik lassen sich sehr \enquote{anschaulich} grafisch darstellen. Wir skizzieren die groben Zusammenhänge in Skizzen, zeichnen Diagramme und Graphen in denen die Lösungen sogar in der grafischen Darstellung (in etwa/ näherungsweise) abgelesen werden können.


\section{Skizze}\index{Skizze}

Eine Skizze nennen wir eine grafische Darstellung, die die Struktur eines Objekts erfasst, die uns jeweils interessierenden Aspekte sichtbar macht, jedoch nicht präzise ist. In der Mathematik im Hauptschulabschluss verstehen wir darunter insbesondere die Planskizzen\index{Planskizze} in der Geometrie mit denen wir uns daran erinnern welche Punkte, Seiten, Strecken, Winkel, ... wo sind relativ zu einander, um die passenden Formeln zu finden und korrekt Werte einzusetzen, die im Text gegeben sind. Beachte dass die Autoren der Prüfungen manchmal die Begriffe Skizze und Zeichnung nicht unterscheiden.


\section{Zeichnung}\index{Zeichnung}

Als \enquote{Zeichnung} bezeichnen wir eine graphische Darstellung, der alle (für das Problem) relevanten Informationen zu entnehmen sind. Insbesondere nennen wir z:B. in einer Planskizze eines Dreiecks die Seiten $a$, $b$ und $c$, in einer Zeichnung haben in der Regel die Seiten Längenangaben wie $3 \text{cm}$, $4 \text{cm}$ und $5 \text{cm}$. Vorsicht: die Begriffe \enquote{Skizze} und \enquote{Zeichnung} werden nicht immer sauber getrennt.

\subsection{Schrägbild}\index{Schrägbild}\label{Schrägbild}

\subsection{Netz}\index{Netz}\label{Netz}


\section{Diagramm}\index{Diagramm}

Wir bezeichnen eine graphische Darstellung, die einen numerischen Sachverhalt \enquote{auf einen Blick} erkennbar darstellt, andere Aspekte jedoch gar nicht, ein \enquote{Diagramm}.

\subsection{Kreis- oder Tortendiagramm}\index{Kreisdiagramm}\index{Tortendiagramm|see Kreisdiagramm}

\subsection{Balken- oder Säulendiagramm}\index{Balkendiagramm}\index{Säulendiagramm|see Balkendiagramm}

\subsection{Liniendiagramm}\index{Liniendiagramm}

\subsection{Flächendiagramm}\index{Flächendiagramm}


\subsection{Venn-Diagramm}\index{Venn-Diagramm}

Mit einem \textbf{Venn-Diagramm} können wir Mengen hinsichtlich ihrer Schnitte darstellen. Venn-Diagramme können für (rigorose) Beweise benutzt werden. Die Mengen\footnote{Es können beliebig viele Mengen sein, wirklich übersichtlich sind i.d.R. nur bis zu drei Mengen (ggf. plus Komplementärmenge in der alle dargestellten Mengen liegt).} Werden als Kreise dargestellt, die sich teilweise überschneiden.

\begin{figure}
  \centering
    % Definition of circles
    \def\firstcircle{(0,0) circle (1.5cm)}
    \def\secondcircle{(0:2cm) circle (1.5cm)}

    \colorlet{circle edge}{blue!50}
    \colorlet{circle area}{blue!20}

    \tikzset{filled/.style={fill=circle area, draw=circle edge, thick},
        outline/.style={draw=circle edge, thick}}

    \setlength{\parskip}{5mm}
    % Set A and B
    \begin{tikzpicture}
        \begin{scope}
            \clip \firstcircle;
            \fill[filled] \secondcircle;
        \end{scope}
        \draw[outline] \firstcircle node {$A$};
        \draw[outline] \secondcircle node {$B$};
        \node[anchor=south] at (current bounding box.north) {$A \cap B$};
    \end{tikzpicture}

    %Set A or B but not (A and B) also known a A xor B
    \begin{tikzpicture}
        \draw[filled, even odd rule] \firstcircle node {$A$}
                                     \secondcircle node{$B$};
        \node[anchor=south] at (current bounding box.north) {$\overline{A \cap B}$};
    \end{tikzpicture}

    % Set A or B
    \begin{tikzpicture}
        \draw[filled] \firstcircle node {$A$}
                      \secondcircle node {$B$};
        \node[anchor=south] at (current bounding box.north) {$A \cup B$};
    \end{tikzpicture}

    % Set A but not B
    \begin{tikzpicture}
        \begin{scope}
            \clip \firstcircle;
            \draw[filled, even odd rule] \firstcircle node {$A$}
                                         \secondcircle;
        \end{scope}
        \draw[outline] \firstcircle
                       \secondcircle node {$B$};
        \node[anchor=south] at (current bounding box.north) {$A - B$};
    \end{tikzpicture}

    % Set B but not A
    \begin{tikzpicture}
        \begin{scope}
            \clip \secondcircle;
            \draw[filled, even odd rule] \firstcircle
                                         \secondcircle node {$B$};
        \end{scope}
        \draw[outline] \firstcircle node {$A$}
                       \secondcircle;
        \node[anchor=south] at (current bounding box.north) {$B - A$};
    \end{tikzpicture}
  \caption{Venn-Diagramm}\label{fig:VennDiagramm}
\end{figure}


\section{Funktionsgraph}\index{Funktionsgraph}\index{Graph|see{Funktionsgraph}}



\chapter{Term Replacement Systems (TRS)}\label{TRS}\index{TRS}

Um effizient Mathematik zu betreiben ist eine nützliche Sichtweise das Rechnen als System zu verstehen in dem Ausdrücke durch anders geformte Ausdrücke ersetzt werden können, ohne den ursprünglichen Ausdruck (wertmäßig) zu verändern. Insbesondere benutzen wir \textbf{Formeln} und ersetzen die in ihnen enthaltenen allgemeinen \textbf{Variablen} durch konkrete Werte (\textbf{einsetzen}), wodurch wir konkrete Ergebnisse \textbf{ausrechnen} können.


\section{Operatorrangfolge}\index{Operatorpriorität}

Die Regel, die in deutschen (Grundschulen) gelehrt wird lautet \enquote{Punkt- vor Strichrechnung}. D.h. \textbf{Multiplikation} und \textbf{Division} werden vor \textbf{Addition} und \textbf{Subtraktion} ausgeführt. Dieser Regel fehlen unter anderem \textbf{Potenz} und \textbf{Wurzel}.

Eine vollständige Regel lautet: \enquote{Arbeite die List von oben nach unten ab und Operationen mit der selben Priorität/ dem selben Rang von links nach rechts in der Reihenfolge ihres Auftretens im Ausdruck (nicht in der Liste):}.

\begin{enumerate}
  \item \enquote{$(t)$}
  \item \enquote{${t_1}^{t_2}$}, \enquote{$\sqrt[t_2]{t_1}$}
  \item \enquote{$\times$}, \enquote{$\div$}
  \item \enquote{$+$}, \enquote{$-$}
\end{enumerate}

\begin{beispiel}
    \begin{equation}
        2 + 3 \cdot 4 = 2 + (3 \cdot 4) = 2 + 12 = 14
    \end{equation}\topicend
\end{beispiel}


\section{Äquivalenzumformungen}\index{Äquivalenzumformungen}


Wir können \enquote{Rechnen} als Vereinfachung von Ausdrücken verstehen. Wir sagen \enquote{Zwei plus drei ist fünf.} $2+3=5$ Ist egal ob in moderner mathematischer Notation oder als Satz in natürlicher deutscher Sprache eine Aussage. Diese konkrete Aussage ist wahr. $2+3=7$ ist eine falsche Aussage. Wahr hingegen ist $2+3 \neq 7$. Eine andere Sicht auf die Mathematik ist, dass es Ziel der Mathematik ist wahre Aussagen zu finden, zu zeigen/ beweisen dass sie wahr sind und für Aussagen, für die man bisher nicht weiß ob sie wahr oder unwahr/ falsch sind, ein gültiges Argument/ Beweis zu finden dass die Aussage wahr ist oder dass sie falsch ist.


\chapter{Lineare Gleichungen einer Variablen}

\enquote{Lineare Gleichung einer Variablen} ist eine Bezeichnung deren Bedeutung man in der Regel erst (lange) nachdem man sie \enquote{zu lösen} gelernt hat etwas abfangen kann. Deshalb beginnen wir mit einem Beispiel.

\begin{beispiel}
    Zwei Anbieter stehen für die Stromversorgung Deiner Wohnung zur Verfügung. Du möchtest überprüfen welcher Anbieter für Dich der günstigere ist. Anbieter $A$ bietet einen Vertrag an mit einem festen Grundbetrag von 20 Euro und einem Preis pro Kilowattstunde von 25 Cents. Anbieter $B$ bietet einen Vertrag an ohne Grundbetrag mit einem Preis von 35 Cents pro Kilowattstunde. Sicher kannst Du das Problem lösen indem Du ausrechnest welcher Vertrag günstiger für Dich gewesen wäre im letzten Monat. Das Verfahren funktioniert sogar ganz ordentlich. Aber was machst Du wenn Dein Verbrauch im letzten Jahr von Monat zu Monat angestiegen ist und die beiden Verträge bei den aktuellen Zahlen nur einen geringen Unterschied ergeben?

    Wir lösen rechnerisch indem wir den \textbf{Schnittpunkt} der Geraden $y_a = 0,25x + 20$ und $y_b = 0,35x$ berechnen. Am Schnittpunkt ist ihr $y$-Wert gleich. Wir setzen also die Formeln der Verträge gleich:
    \begin{align}\label{eqn:lineareGleichungen1}
      0,25x + 20 &=& 0,35x && | -0,25x \\
      20 &=& 0,1x && | \cdot 10 \\
      200 &=& x &&
    \end{align}
    Das bedeutet, dass Vertrag $B$ bei einem Verbrauch von weniger als 200kWh günstiger ist und Vertrag $A$ bei einem größeren Verbrauch. Aus der Vergangenheit extrapolierend ermöglicht dieses Wissen eine wesentlich bessere Planung.\proofsquare
\end{beispiel}

Dieses Beispiel können wir auch grafisch interpretieren (und damit ungefähr lösen) (s. Abb. \ref{fig:lineareGleichungen1}). Der Schnittpunkt der beiden Geraden hat die $x$-Koordinate $x=200$, den Punkt an dem beide Verträge gleich teuer sind. Hier wir im wahren Sinn \enquote{offensichtlich}, dass $B$ links und damit bei kleinerem Verbrauch günstiger ist, und $A$ rechts und damit bei größerem Verbrauch als am Schnittpunkt.
\begin{figure}
  \centering
  \begin{tikzpicture}[domain=0:300, scale=0.04]
        \draw[very thin,color=gray, step=50] (-5,-5) grid (295,145);

        \draw[->] (-5,0) -- (300,0) node[right] {kWh};
        \draw[->] (0,-5) -- (0,150) node[above] {€};

        \draw[color=red] plot (\x,{0.25*\x +20});
        \draw[color=blue] plot (\x,{0.35*\x});

        \node at (50,-10) {\small50};
        \node at (100,-10) {\small100};
        \node at (150,-10) {\small150};
        \node at (200,-10) {\small200};
        \node at (250,-10) {\small250};
        \node at (-10,50) {\small50};
        \node at (-10,100) {\small100};

        \draw[text=red] node at (300,90) {A};
        \draw[text=blue] node at (300,115) {B};
  \end{tikzpicture}
  \caption{Grafische Lösung/Interpretation}\label{fig:lineareGleichungen1}
\end{figure}


\chapter{Stochastik}\index{Stochastik}


\chapter{Prüfungstaktik}

\section{Vorbereitung}

Der wichtigste Test ob die Vorbereitung für die schriftliche Prüfung in Mathematik erfolgreich abgeschlossen ist, ist das \textit{Stark-Heft} für Mathematik ohne Lösungsheft unter realistischen Bedingungen zu bearbeiten. Wir werden in den Monaten vor der Prüfung alle darin enthaltenen Prüfungen (die tatsächlichen Prüfungen der letzten fünf Jahre) bearbeiten. Um dadurch eine realistische Einschätzung zu bekommen bearbeitet die alten Prüfungen bitte nur wenn die Dozenten Euch das aufgeben - und dann i.d.R. für realistische Bedingungen sorgen. Braucht Ihr Material zum Üben sprecht die Dozenten an. Wir geben Euch dann passendes Material, das aber nicht in eine komplette Prüfung eingebunden ist, die wir später noch für die Prüfungsvorbereitung brauchen.

Noch vor diesen Tests in der eigentlichen Prüfungsvorbereitung ist die beste Vorbereitung (in Mathematik) jede Stunde mitzuschreiben, die Hausaufgaben zu machen, gemeinsam mit anderen Teilnehmern zu lernen und sich die Inhalte gegenseitig zu erklären, und jedes unbekannte Wort in die Vokabelliste (für die Mathematik) aufzunehmen und zu lernen.


\section{Priorisierung der Aufgaben}

In den Monaten vor der Prüfung besprechen wir welche Aufgaben einfach und welche schwierig sind und welche damit leicht zu bekommende Punkte für die Note sind und an welchen man sich besser nicht aufhält, sondern sie macht wenn man mit den einfachen Aufgaben, auf die es viele Punkte gibt (z.B. ein Dreieck zu konstruieren), fertig ist.


\section{Plausibilitätsprüfung}

Die Aufgaben in den Prüfungen sind mit dem Bemühen gestellt plausible Anwendungen der Realität zu modellieren und ihre Beherrschung zu prüfen. Insbesondere sind damit auch die Zahlen i.d.R. so gewählt, dass eine Plausibilitätsprüfung mit gesundem Menschenverstand möglich ist. Errechnet man als Prüfling z.B. für das Volumen eines Stausees einige Liter (nicht einmal tausende), dann kann das nicht stimmen. Mit griffigen Beispielen für Größenordnungen, u.a. dem Liter als Flasche Cola, Packung Milch o.ä., und dem Kubikmeter als immer wieder benutztes (gerne mit Armen visualisierend) konkretes Beispiel (1.000 Liter) und dem Lehren von fortwährender skeptischer Plausibilitätsprüfung des (rechnenden) Tuns kann ein solcher Fehler sicher aufgedeckt werden und Zeit und Einsicht vorausgesetzt vom Prüfling korrigiert werden. Realistische Beispiele von Größenordnungen sollten möglichst von allen Dozenten häufig angeboten werden, insbesondere bietet sich neben der Mathematik selbst der naturwissenschaftliche Unterricht, hier zur Zeit der Geographie, an.

\begin{uebung}
    Wiederhole falls unsicher die Maßeinheiten und finde zu jeder Maßeinheiten für die Bereiche von $\frac{1}{1000}$, $\frac{1}{100}$, $\frac{1}{10}$, eins, zehn, hundert und tausend mal der Einheit ein für Dich gut bildlich vorstellbares Beispiel. Z.B. ein Liter Cola oder Milch und ein tausendstel Meter als Millimeter auf dem Geodreieck.\topicend
\end{uebung}

\begin{uebung}
    Schätze ab: das Volumen des Raums in dem Du Dich gerade befindest, die Masse/ das Gewicht des Gebäudes in dem Du Dich gerade befindest, das Volumen Deines Körpers, die Entfernungen zum Rathaus, nach Berlin, nach New York, zum Mond, zur Sonne, die Länge Deiner Hand, deines Unterarms, der Spanne Deiner beiden ausgestreckten Arme, die Höhe eines Gebäudes, das Du aus dem Fenster sehen kannst, die Dichte von Wasser, Stahl, Holz, einem Menschen, wie viel Liter Wasser der Eder-Stausee fasst, wie viele Liter Wasser Du durchschnittlich pro Tag verbrauchst, die Masse der Erde, ...\topicend
\end{uebung}


\section{Proben rechnen}

Wenn man (in der Prüfung) Zeit hat kann man seine eigenen Ergebnisse überprüfen und oftmals Fehler selbst finden. Wir nennen eine zweite Rechnung (auf einem anderen Weg), die das selbe Problem löst, also das gleiche Ergebnis\footnote{Oder ein anderes Ergebnis, das sich leicht mit dem ersten vergleichen lässt, z.B. bei einer Rechnung mit verschiedenen Maßeinheiten in eine zweite Einheit ausrechnen und prüfen ob der Faktor zwischen den Ergebnissen passt.} haben muss, eine Proberechnung.

\begin{beispiel}[Proberechnung]
    \begin{align}
      12 \text{mm} + 34 \text{cm} + 56 \text{dm} && &&\\
      + 78 \text{m} + 90 \text{km} &=& x \text{m} && | \text{Aufgabe}\\
      0,012 \text{m} + 0,34 \text{m} + 5,6 \text{m} && &&\\
      + 78 \text{m} + 90.000 \text{m} &=& 90.083,952 \text{m} && | \text{Lösung}\\
      12 \text{mm} + 340 \text{mm} && &&\\
      + 5.600 \text{mm} + 78.000 \text{mm} && &&\\
      + 90.000.000 \text{mm} &=& 90.083.952 \text{mm} && | \text{Probe}
    \end{align}

    Die Proberechnung wird in diesem Fall aus der ersten Zeile errechnet und nicht aus der bereits gefundenen Lösung. Das unabhängig von der Lösung ermittelte zweite Ergebnis (Probe) ist 1.000 mal so hoch bei einer Einheit, die $\frac{1}{1000}$ der Einheit der Lösung ist. Die Probe und Lösung sind also gleich: $90.083,952 \text{m}=90.083.952 \text{mm}$. Wenn man wirklich das zweite Ergebnis ohne die Lösung zu benutzen errechnet hat ist sehr wahrscheinlich dass die Lösung richtig ist. Passt das Ergebnis von Probe und Lösung nicht zusammen ist sicher eine der beiden Rechnungen falsch und man sollte den Fehler suchen.
\end{beispiel}


\appendix

\chapter{Vokabeln}

Alle diese Vokabeln müssen gelernt werden! Zusätzlich werden noch viele gebraucht, die im Unterricht auftauchen. Diese hier haben perfekt \textbf{verstanden} zu werden (fehlerfreies Schreiben ist für Mathematik nicht notwendig).\section{Eingeführte Vokabeln}

Eingeführte Vokabeln werden als bekannt vorausgesetzt. Was hier unbekannt ist unverzüglich lernen!

abrunden,
abschätzen,
die Abschätzung (pl. die Abschätzungen, synonym mit Schätzung),
abziehen,
die Achse (pl. die Achsen),
addieren,
die Addition (pl. die Additionen),
die Antwort (pl. die Antworten),
der Antwortsatz (pl. die Antwortsätze),
antworten,
anwenden,
die Anwendung (pl. die Anwendungen),
der Ausdruck (pl. die Ausdrücke),
ausklammern,
ausmultiplizieren,
die Ausnahme (pl. die Ausnahmen),
ausnehmen,
ausdrücken,
aufrunden,
ausrechnen,
behaupten,
die Behauptung (pl. die Behauptungen),
berechnen,
der Beweis (pl. die Beweise),
beweisen,
die Breite (pl. die Breiten),
der Bruch (pl. die Brüche),
die Bruchrechnung (kein pl.),
dezi- (Präfix für Größenordnungen),
die Differenz (pl. die Differenzen),
die Division (pl. die Divisionen),
dividieren,
das Element (pl. die Elemente),
das Endergebnis (pl. die Endergebnisse),
das Ergebnis (pl. die Ergebnisse),
die Formelsammlung (pl. die Formelsammlungen),
die Geometrie (pl. die Geometrien),
die Höhe (pl. die Höhen),
der Inhalt (pl. die Inhalte),
das Komma (pl. die Kommata),
kubik-,
kürzen,
die Maßeinheit (pl. die Maßeinheiten),
die Menge (pl. die Mengen),
der Meter (pl. die Meter),
die Multiplikation (pl. die Multiplikationen),
multiplizieren,
der Nenner (pl. die Nenner),
prim,
die Primfaktorzerlegung (pl. die Primfaktorzerlegungen),
das Rechteck (pl. die Rechtecke),
rechtwinklig,
parallel,
die Parallele (pl. die Parallelen),
das Parallelogramm (pl. die Parallelogramme),
die Primzahl (pl. die Primzahlen),
quadrat-,
das Quadrat (pl. die Quadrate),
die Raute (pl. die Rauten),
der Raum (pl. die Räume),
senkrecht,
die Strecke (pl. die Strecken),
die Stunde (pl. die Stunden),
subtrahieren,
die Subtraktion (pl. die Subtraktionen),
der Summand (pl. die Summanden),
die Summe (pl. die Summen),
das Trapez (pl. die Trapeze),
der Übertrag (pl. die Überträge),
zählen,
das Volumen (pl. die Volumina),
der Zähler (pl. die Zähler),
die Zahl (pl. die Zahlen),
der Zahlenstrahl (pl. die Zahlenstrahlen),
die Ziffer (pl. die Ziffern),
das Zwischenergebnis (pl. die Zwischenergebnisse),

\section{Noch nicht eingeführte prüfungsrelevante Vokabeln}

abbuchen,
die Abbuchung (pl. die Abbuchungen),
abheben,
der Algorithmus (pl. die Algorithmen),
anheben,
die Anhebung (pl. die Anhebungen),
anpassen,
die Anpassung (pl. die Anpassungen),
antiproportional,
die Annahme (pl. die Annahmen),
annehmen,
auszahlen
die Auszahlung (pl. die Auszahlungen),
berühren,
buchen,
die Buchung (pl. die Buchungen),
deutlich,
das Diagramm (pl. die Diagramme),
drehen,
die Drehung (pl. die Drehungen),
der Durchschnitt (pl. die Durchschnitte),
die Ecke (pl. die Ecken),
die Einheit (pl. die Einheiten),
einzahlen,
die Einzahlung (pl. die Einzahlungen),
das Element (pl. die Elemente),
elementar,
ergeben,
erhöhen,
die Erhöhung (pl. die Erhöhungen),
erweitern,
die Erweiterung (pl. die Erweiterungen),
explizit,
der Faktor (pl. die Faktoren),
faktorisieren,
fehlen,
das Fehlen (kein pl.),
der Fehler (pl. die Fehler),
flach,
die Figur (pl. die Figuren),
die Fläche (pl. die Flächen),
die Folge (pl. die Folgen),
folgen,
folgern,
die Folgerung (pl. die Folgerungen),
die Formel (pl. die Formeln),
der Funktionsgraph (pl. die Funktionsgraphen),
die Ganze Zahl (pl. die Ganzen Zahlen),
das Gefäß (pl. die Gefäße),
das Gehalt (pl. die Gehälter),
das Geodreieck (pl. die Geodreiecke),
gerade,
die Gerade (pl. die Geraden),
das Gewicht (pl. die Gewichte),
gleich,
die Gleichheit (pl. die Gleichheiten),
die Gleichung (pl. die Gleichungen),
die Grafik (pl. die Grafiken),
grafisch,
das Gramm (kein pl.),
die Grundrechenart (pl. die Grundrechenarten),
der Grundsatz (pl. die Grundsätze [ungebräuchlich]),
grundsätzlich,
die Größenordnung (pl. die Größenordnungen),
das Guthaben (pl. die Guthaben),
hinreichend,
identisch,
die Identität (pl. die Identitäten),
das Kapital (kein pl.),
kariert,
der Kasten (pl. die Kästen),
das Kästchen (pl. die Kästchen),
der Kehrwert (pl. die Kehrwerte),
das Kilogramm (kein pl.),
die Klammer (pl. die Klammern),
die Kommunikation (pl. die Kommunikationen),
konstruieren,
die Konstruktion (pl. die Konstruktionen),
kommunizieren,
die Körperhöhe (pl. die Körperhöhen),
der Kreis (pl. die Kreise),
kurz,
die Implikation (pl. die Implikationen),
implizit,
lang,
die Länge (pl. die Längen),
der Lohn (pl. die Löhne),
kürzen,
irren,
der Irrtum (pl. die Irrtümer),
linear,
lösen,
die Lösung (pl. die Lösungen),
die Masse (pl. die Massen),
das Maß (pl. die Maße),
der Maßstab (pl. die Maßstäbe),
maßstabsgetreu,
das Maximum (pl. die Maxima),
messen,
die Messung (pl. die Messungen),
die Methode (pl. die Methoden),
milli- (Präfix für Größenordnungen),
das Minimum (pl. die Minima),
das Modell (pl. die Modelle),
modellieren,
die Nebenrechnung (pl. die Nebenrechnungen),
die Notation (pl. die Notationen),
notfalls,
der Notfall (pl. die Notfälle),
notieren,
notwendig,
ordnen,
die Ordnung\footnote{Eine Ordnung in der Mathematik ist etwas anderes als die Ordnung die man in seiner Wohnung schafft wenn man aufräumt.} (pl. die Ordnungen),
das Paar (pl. die Paare),
parallel,
die Parallele (pl. die Parallelen),
das Parallelogramm (pl. die Parallelogramme),
die Planskizze (pl. die Planskizzen),
plausibel,
die Plausibilität (kein Plural),
das Polygon (pl. die Polygone),
die Potenz (pl. die Potenzen),
potenzieren,
das Prisma (pl. die Prismen),
die Präfix (pl. die Präfixe),
die Probe (pl. die Proben),
die Proberechnung (pl. die Proberechnungen),
das Produkt (pl. die Produkte),
proportional,
das Prozent (pl. die Prozente),
die Prozentrechnung (kein pl.),
der Punkt (pl. die Punkte),
der Quader (pl. die Quader),
der Quotient (pl. die Quotienten),
der Raum (pl. die Räume),
die Raute (pl. die Rauten),
rechnen,
die Rechnung (pl. die Rechnungen),
das Recht (pl. die Rechte),
rechtwinklig,
die Regel (pl. die Regeln),
regeln,
die Relation (pl. die Relationen),
runden,
die Rundung (pl. die Rundungen),
schätzen,
die Schätzung (pl. die Schätzungen),
schneiden,
der Schnitt (pl. die Schnitte),
der Schnittpunkt (pl. die Schnittpunkte),
die Schulden (kein Singular),
schlussfolgern,
die Schlussfolgerung (pl. die Schlussfolgerungen),
die Sekunde (pl. die Sekunden),
senkrecht,
die Senkrechte (pl. die Senkrechten),
die Skizze (pl. die Skizzen),
skizzieren,
sortieren,
spiegeln,
spiegelsymmetrisch,
die Spiegelung (pl. die Spiegelungen),
steigen,
die Steigung (pl. die Steigungen),
die Stelle (pl. die Stellen),
der Stellenwert (pl. die Stellenwerte),
das Stellenwertsystem (pl. die Stellenwertsysteme),
der Stundenlohn (pl. die Stundenlöhne),
das Symbol (pl. die Symbole),
symmetrisch,
die Symmetrie (pl. die Symmetrien),
die Symmetrieachse (pl. die Symmetrieachsen),
der Tag (pl. die Tage),
der Term (pl. die Terme),
die Tonne (pl. die Tonnen),
das Tripel (pl. die Tripel),
das Tupel (pl. die Tupel),
der Uhrzeigersinn (kein pl.),
der Umfang (pl. die Umfänge),
umrechnen,
die Umrechnung (pl. die Umrechnungen),
ungerade,
das Verfahren (pl. die Verfahren),
das Volumen (pl. die Volumina),
die Voraussetzung (pl. die Voraussetzungen),
das Vorzeichen (pl. die Vorzeichen),
die Waage (pl. die Waagen),
der Wert (pl. die Werte),
wiegen,
der Winkel (pl. die Winkel),
der Würfel (pl. die Würfel),
würfeln,
die Wurzel (pl. die Wurzeln),
die Zahlung (pl. die Zahlungen),
die Zählung (pl. die Zählungen),
zeichnen,
die Zeichnung (pl. die Zeichnungen),
zeigen,
die Zeit (pl. die Zeiten),
die Zeitspanne (pl. die Zeitspannen),
zenti- (Präfix für Größenordnungen),
der Zentimeter (pl. die Zentimeter),
zerlegen,
die Zerlegung (pl. die Zerlegungen),
zerschneiden,
der Zins (pl. die Zinsen),
die Zinsrechnung (kein pl.),
der Zirkel (pl. die Zirkel),
zusammensetzen,
das Zwischenergebnis (pl. die Zwischenergebnisse),
der Zylinder (pl. die Zylinder) 


\chapter{Algorithmen}\index{Algorithmen}

\section{Was sind Algorithmen?}

\section{ggT - Euklidischer Algorithmus}


\chapter{Arbeitsblätter}

\newpage
\section{Maßeinheiten: Strecken}

\anweisungArbeitsblatt

\begin{eqnarray}
% \nonumber to remove numbering (before each equation)
  1,23456\text{dm} &=& x \text{cm}\\
  24,86\text{m} &=& x \text{km}\\
  0,0074\text{mm} &=& x \text{dm}\\
  654,456\text{km} &=& x \text{cm}\\
  75,3\text{mm} &=& x \text{km}\\
  54,1745\text{m} &=& x \text{cm}\\
  5678\text{cm} &=& x \text{dm}\\
  0,00054\text{dm} &=& x \text{mm}\\
  2,5468789\text{km} &=& x \text{dm}\\
  32458\text{dm} &=& x \text{km}\\
  12,6385\text{m} &=& x \text{dm}\\
  0,00578\text{km} &=& x \text{mm}\\
  54678534\text{cm} &=& x \text{km}\\
  567878,5\text{mm} &=& x \text{m}\\
  687798,24\text{cm} &=& x \text{m}\\
  534,32\text{dm} &=& x \text{m}\\
  0,005456\text{m} &=& x \text{mm}\\
  6546\text{mm} &=& x \text{cm}\\
  1,6574\text{cm} &=& x \text{mm}\\
  3,6578\text{km} &=& x \text{m}\\
  382\text{cm} &=& x \text{mm}\\
  1\text{km} + 2\text{m} +3\text{dm} +4\text{cm} +5\text{mm} &=& x \text{m} \\
  123.456.789\text{mm} &=& x \text{km} \\
  7 \cdot (13\text{m} + 17\text{cm}) &=& x\text{cm} \\
  27.364.782.634\text{mm} &=& x \text{km}\\
  42,5\text{km} &=& x \text{mm}\\
  12600\text{km} &=& x \text{m}
\end{eqnarray}



\newpage
\section{Maßeinheiten: Massen/ Gewichte}

\anweisungArbeitsblatt

\begin{eqnarray}
% \nonumber % Remove numbering (before each equation)
    758.432 \text{g} &=& x \text{t} \\
    0,0034 \text{t} &=& x \text{mg} \\
    73.534 \text{mg} &=& x \text{kg} \\
    0,378.4 \text{g} &=& x \text{mg} \\
    750 \text{g} &=& x \text{kg} \\
    0,3 \text{kg} &=& x \text{g} \\
    12,5 \text{mg} &=& x \text{g} \\
    123 \text{kg} &=& x \text{t} \\
    1,23 \text{t} &=& x \text{g} \\
    23.464.674 \text{mg} &=& x \text{t} \\
    2,034 \text{t} &=& x \text{kg} \\
    0,0125 \text{kg} &=& x \text{mg} \\
  832.747.982\text{mg} &=& x \text{t} \\
  1,236.78 \text{g} &=& x \text{t} \\
  98764\text{g} &=& x \text{kg} \\
  6,54768578654657\text{t} &=& x \text{mg} \\
  123 \text{mg} + 78 \text{g} + 2,34 \text{kg} + 0,75 \text{t} &=& x \text{kg} \\
  123.245 \text{mg} + 7,8 \text{g} + 12,634 \text{kg} + 0,00075 \text{t} &=& x \text{mg} \\
  6123 \text{mg} + 7,8 \text{g} + 2.123,4 \text{kg} + 0,0075 \text{t} &=& x \text{g}
\end{eqnarray}


\newpage
\section{Wortschatz 1}

Ordne die Vokabeln den Buchstaben in den Klammern zu. Die Liste enthält die benötigten Vokabeln, die alle in der Liste der eingeführten (und damit zu lernenden) Vokabeln vorkommen. Nicht jede Vokabel, die bereits zu lernen war kommt hingegen im Text vor:
addieren,
der Bruch (pl. die Brüche),
die Bruchrechnung (kein pl.),
dezi- (Präfix für Größenordnungen),
die Differenz (pl. die Differenzen),
die Division (pl. die Divisionen),
dividieren,
das Element (pl. die Elemente),
das Ergebnis (pl. die Ergebnisse),
die Formelsammlung (pl. die Formelsammlungen),
die Maßeinheit (pl. die Maßeinheiten),
die Menge (pl. die Mengen),
die Multiplikation (pl. die Multiplikationen),
multiplizieren,
der Nenner (pl. die Nenner),
prim,
die Primfaktorzerlegung (pl. die Primfaktorzerlegungen),
die Primzahl (pl. die Primzahlen),
zählen,
subtrahieren,
die Subtraktion (pl. die Subtraktionen),
der Summand (pl. die Summanden),
die Summe (pl. die Summen),
der Übertrag (pl. die Überträge),
der Zähler (pl. die Zähler),
die Zahl (pl. die Zahlen),
der Zahlenstrahl (pl. die Zahlenstrahlen)

\vspace{1em}

Um die \textbf{\texttt{[-a-]}} $\Sigma = -3 +2 -1 -4 -5 +7$ zu berechnen hilft es sich die \textbf{\texttt{[-b-]}} als Strecken/ Längen am \textbf{\texttt{[-c-]}} vorzustellen. So ist es klar ob das \textbf{\texttt{[-d-]}} eine positive oder negative \textbf{\texttt{[-e-]}} sein muss. Eine besonders gut nachvollziehbare (und damit einfache) Methode ist es zunächst die negativen und positiven Zahlen getrennt von einander zu \textbf{\texttt{[-f-]}}. Dann \textbf{\texttt{[-g-]}} wir die ohne Beachtung des Vorzeichens kleinere Zahl von der Größeren. So bilden wir die \textbf{\texttt{[-h-]}} zwischen den Beträgen der Zahlen und haben damit von der längeren Strecke von 0 bis zur Zahl die kürzere abgezogen.

Den \textbf{\texttt{[-i-]}} $q_1 = \frac{1}{2}$ können wir durch den \textbf{\texttt{[-i-]}} $q_2 = \frac{1}{2}$ \textbf{\texttt{[-j-]}} indem wir ihn mit dem Kehrwert ${(q_2)}^{-1}$ \textbf{\texttt{[-k-]}}. Die \textbf{\texttt{[-k-]}} der \textbf{\texttt{[-l-]}} führen wir durch indem wir die \textbf{\texttt{[-m-]}} miteinander \textbf{\texttt{[-k-]}} und die \textbf{\texttt{[-n-]}} miteinander \textbf{\texttt{[-k-]}} und das \textbf{\texttt{[-d-]}} kürzen.

\printglossary
\printglossary[type=symbols,style=long]

\chapter{Formelsammlung}

\section{Flächen}

\subsection{Viereck}\label{fs:Viereck}

Wir nennen ein Polygon mit vier Ecken ein \enquote{Viereck}.

Die Summe der Innenwinkel im Viereck beträgt 360°.


\subsection{Rechteck}\label{fs:Rechteck}

Wir nennen ein Viereck mit rechten Winkeln \enquote{Rechteck} (s. \ref{Rechteck}, S. \pageref{Rechteck}).
\begin{align}\label{eqn:fsRechteck}
    A_R &= a \cdot b\\
    U_R &= 2a + 2b
\end{align}


\subsection{Quadrat}\label{fs:Quadrat}

Wir nennen ein Rechteck mit gleichen Seitenlängen \enquote{Quadrat} ().
\begin{align}\label{eqn:fsQuadrat}
  A_Q &= 4a \\
  U_Q &= a^2
\end{align}


\subsection{Dreieck}\label{fs:Dreieck}

Wir nennen ein Polygon mit drei Ecken \enquote{Dreieck} ().
\begin{align}\label{eqn:fsDreieck}
  A_D &= \frac{a \cdot h_a}{2} \\
  U_D &= a+b+c
\end{align}


\subsection{Parallelogramm}

Wir nennen ein Viereck dessen gegenüber liegende Seiten zu einander parallel sind \enquote{Parallelogramm}.
\begin{align}\label{eqn:fsPrallelogramm}
  A_P &= a \cdot h_a \\
  U_P &= 2a + 2b
\end{align}

\subsection{Trapez}

Wir nennen ein Viereck mit zwei parallelen Seiten \enquote{Trapez}.
\begin{align}\label{eqn:fsTrapez}
  A_T &= \frac{a+c}{2} h_a \\
  U_T &= a+b+c+d
\end{align}


\subsection{Kreis}

Wir nennen die Fläche, deren Rand ein Zirkel um einen Mittelpunkt zieht, deren Rand damit einen festen Radius vom Mittelpunkt entfernt ist, \enquote{Kreis}.
\begin{align}\label{eqn:fsKreis}
  A_K &= \pi r^2 \\
  U_K &= 2 \pi r
\end{align}



\section{Körper}


\subsection{Würfel}

Wir nennen einen Körper dessen sechs Seiten (Seitenflächen) Quadrate sind \enquote{Würfel}.
\begin{align}\label{eqn:fsWuerfel}
  V &= a^3 \\
  O &= 6a^2
\end{align}


\subsection{Quader}

Wir nennen einen Körper dessen sechs Seiten (Seitenflächen) Rechtecke sind \enquote{Quader}.
\begin{align}\label{eqn:fsQuader}
  V &= a \cdot b \cdot c \\
  O &= 2ab + 2ac + 2bc
\end{align}


\subsection{Prisma}

Wir nennen einen Körper mit zwei gleichen Seiten (Seitenflächen) ein Polygon sind dessen gleiche Ecken mit geraden Strecken (senkrecht zum Polygon) verbunden sind \enquote{Prisma}.
\begin{align}\label{eqn:fsPrisma}
  V &= A_G \cdot h_K \\
  O &= 2 A_G + A_M
\end{align}


\subsection{Zylinder}

Wir nennen den Körper aus zwei Kreisen als Grundflächen, die durch einen (senkrechten, geraden) Mantel verbunden sind, \enquote{Zylinder}.

\printindex
\printbibliography

\end{document}


