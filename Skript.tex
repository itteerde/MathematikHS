
\documentclass[a4paper]{book}%[a4paper,oneside]
\usepackage[pass]{geometry}%[hmarginratio=1:1]{geometry}
%\usepackage[utf8]{inputenc}

\usepackage{polyglossia}
\setdefaultlanguage[spelling=new]{german}
\setotherlanguage{english}
%\setotherlanguage[numerals=western]{farsi}
\usepackage{fontspec}
\usepackage{xeCJK}
\usepackage{csquotes}

\usepackage[
    backend=biber,
    style=numeric,
    sortlocale=de_DE,
    natbib=true,
    url=false,
    doi=true,
    eprint=false
]{biblatex}
\addbibresource{itteerde.bib}
\bibliography{itteerde}	

\usepackage{makeidx}
\usepackage{longtable}
\usepackage{mdframed}
\usepackage{graphics}
\usepackage{graphicx}
\usepackage{pictex}
\usepackage{subfig}
\usepackage{float}
\usepackage{array}
\usepackage{xspace}
\usepackage{xcolor}
\usepackage{pdfpages}

\usepackage{tikz}
\usetikzlibrary{shapes,backgrounds,arrows,positioning}

\usepackage{verse}

\usepackage{amsmath}
\usepackage{amssymb}
\usepackage{amstext}
\usepackage{amsfonts}
\usepackage{mathrsfs}
\usepackage{amsthm}

\usepackage{chemfig}
\usepackage{listings}
\usepackage{soul}
\usepackage{calc}
\usepackage{metalogo}
\usepackage{hologo}

%\usepackage{stmaryrd}
%\usepackage{marvosym}

\usepackage{url}
\usepackage[position=top]{caption}

\usepackage{etoolbox}
\usepackage{etaremune}

\usepackage[citecolor=black,urlcolor=black,linkcolor=black]{hyperref}
\hypersetup{
    colorlinks=true,
}
\usepackage[xindy={language=german,codepage=duden-utf8},
    nonumberlist,
    toc,
    nopostdot,
    style=altlist,
    nogroupskip
    ]{glossaries}
    \GlsSetXdyCodePage{duden-utf8}
\apptocmd{\thebibliography}{\raggedright}{}{}

\newfontfamily\arabicfont{Amiri}



\def\UrlBreaks{\do\A\do\B\do\C\do\D\do\E\do\F\do\G\do\H\do\I\do\J\do\K
\do\L%
\do\M\do\N\do\O\do\P\do\Q\do\R\do\S\do\T\do\U\do\V\do\W\do\X\do\Y\do\Z
\do\0%
\do\a\do\b\do\c\do\d\do\e\do\f\do\g \do\h\do\i\do\j\do\k\do\l%
\do\m\do\n\do\o\do\p\do\q\do\r\do\s\do\t\do\u\do\v \do\w\do\x\do\y\do\z
%
\do\1\do\2\do\3\do\4\do\5\do\6\do\7\do\8\do\9\do\-}%
\def\UrlBigBreaks{\do\_}


\newcommand{\uu}[1]{\underline{#1}}
\newcommand{\ii}[1]{\textit{#1}}
\newcommand{\tm}{\textsuperscript{\tiny{TM}}}
\newcommand{\rtm}{\textsuperscript{\tiny{\circledR}}}
\newcommand{\whatsapp}{\textit{WhatsApp}\xspace}

\newcommand{\youtube}{\url{http://www.youtube.com}\xspace}
\newcommand{\tagesschau}{\url{http://www.tagesschau.de}\xspace}
\newcommand{\wikipedia}{\url{http://en.wikipedia.org}\xspace}
\newcommand{\cosmos}{\textit{Cosmos - A Space Time Odyssee}\index{Unterhaltung!Cosmos - A Space Time Odyssee}\xspace}

\newcommand{\vcenteredincludeIcon}[1]{\begingroup
    \setbox0=\hbox{\includegraphics[height=1em]{#1}}%
    \parbox{\wd0}{\box0}\endgroup
}
\newcommand\crule[3][black]{\textcolor{#1}{\rule{#2}{#3}}}

\newcommand{\TikZ}{Ti\textit{k}z\xspace}
\newcommand*\circled[1]{\tikz[baseline=(char.base)]{
            \node[shape=circle,draw,inner sep=2pt] (char) {#1};}}

\newcommand{\emojiSmile}{\vcenteredincludeIcon{graphics/smile.jpg}\xspace}
\newcommand{\emojiSmileT}{\vcenteredincludeIcon{graphics/smile_st.jpg}\xspace}
\newcommand{\emojiSet}{\vcenteredincludeIcon{graphics/smile_set.jpg}\xspace}
\newcommand{\emojiSaint}{\vcenteredincludeIcon{graphics/smile_saint.jpg}\xspace}
\newcommand{\emojiTears}{\vcenteredincludeIcon{graphics/smile_tears.jpg}\xspace}
\newcommand{\emojiGrin}{\vcenteredincludeIcon{graphics/smile_grin.jpg}\xspace}
\newcommand{\emojiWink}{\vcenteredincludeIcon{graphics/smile_wink.jpg}\xspace}
\newcommand{\emojiTOE}{\vcenteredincludeIcon{graphics/smile_toe.jpg}\xspace}
\newcommand{\emojiTES}{\vcenteredincludeIcon{graphics/smile_tes.jpg}\xspace}
\newcommand{\emojiSunglasses}{\vcenteredincludeIcon{graphics/smile_sunglasses.jpg}\xspace}
\newcommand{\emojiBlushLips}{\vcenteredincludeIcon{graphics/smile_blushlips.jpg}\xspace}
\newcommand{\emojiBlushSmile}{\vcenteredincludeIcon{graphics/smile_blushsmile.jpg}\xspace}
\newcommand{\emojiDevil}{\vcenteredincludeIcon{graphics/smile_devil.jpg}\xspace}
\newcommand{\emojiBlusBigEyes}{\vcenteredincludeIcon{graphics/smile_blushbigeyes.jpg}\xspace}
\newcommand{\emojiSmileTeeth}{\vcenteredincludeIcon{graphics/smile_teeth.jpg}\xspace}
\newcommand{\emojiSmileTears}{\vcenteredincludeIcon{graphics/smile_tears.jpg}\xspace}
\newcommand{\emojiSmileKiss}{\vcenteredincludeIcon{graphics/smile_kiss.jpg}\xspace}
\newcommand{\emojiSeeNoEvil}{\vcenteredincludeIcon{graphics/emojiSeeNoEvil.jpg}\xspace}
\newcommand{\emojiSmileWink}{\vcenteredincludeIcon{graphics/smile_wink.jpg}\xspace}
\newcommand{\emojiSmileSad}{\vcenteredincludeIcon{graphics/smile_sad.jpg}\xspace}
\newcommand{\emojiHandPointUp}{\vcenteredincludeIcon{graphics/emoji_hand_point_up.jpg}\xspace}

\newcommand{\emojiBWSmile}{\vcenteredincludeIcon{graphics/smile_bw.jpg}\xspace}
\newcommand{\emojiBWSmileT}{\vcenteredincludeIcon{graphics/smile_bw_st.jpg}\xspace}
\newcommand{\emojiBWSet}{\vcenteredincludeIcon{graphics/smile_bw_set.jpg}\xspace}
\newcommand{\emojiBWSaint}{\vcenteredincludeIcon{graphics/smile_bw_saint.jpg}\xspace}
\newcommand{\emojiBWTears}{\vcenteredincludeIcon{graphics/smile_bw_tears.jpg}\xspace}
\newcommand{\emojiBWGrin}{\vcenteredincludeIcon{graphics/smile_bw_grin.jpg}\xspace}
\newcommand{\emojiBWWink}{\vcenteredincludeIcon{graphics/smile_bw_wink.jpg}\xspace}
\newcommand{\emojiBWTOE}{\vcenteredincludeIcon{graphics/smile_bw_toe.jpg}\xspace}
\newcommand{\emojiBWTES}{\vcenteredincludeIcon{graphics/smile_bw_tes.jpg}\xspace}
\newcommand{\emojiBWSunglasses}{\vcenteredincludeIcon{graphics/smile_bw_sunglasses.jpg}\xspace}
\newcommand{\emojiBWBlushLips}{\vcenteredincludeIcon{graphics/smile_bw_blushlips.jpg}\xspace}
\newcommand{\emojiBWBlushSmile}{\vcenteredincludeIcon{graphics/smile_bw_blushsmile.jpg}\xspace}
\newcommand{\emojiBWDevil}{\vcenteredincludeIcon{graphics/smile_bw_devil.jpg}\xspace}
\newcommand{\emojiBWBlusBigEyes}{\vcenteredincludeIcon{graphics/smile_bw_blushbigeyes.jpg}\xspace}
\newcommand{\emojiBWSmileTeeth}{\vcenteredincludeIcon{graphics/smile_bw_teeth.jpg}\xspace}
\newcommand{\emojiBWSmileBWTears}{\vcenteredincludeIcon{graphics/smile_bw_tears.jpg}\xspace}
\newcommand{\emojiBWSmileKiss}{\vcenteredincludeIcon{graphics/smile_bw_kiss.jpg}\xspace}
\newcommand{\emojiBWSeeNoEvil}{\vcenteredincludeIcon{graphics/emoji_bw_SeeNoEvil.jpg}\xspace}
\newcommand{\emojiBWSmileWink}{\vcenteredincludeIcon{graphics/smile_bw_wink.jpg}\xspace}
\newcommand{\emojiBWSmileSad}{\vcenteredincludeIcon{graphics/smile_bw_sad.jpg}\xspace}
\newcommand{\emojiBWHandPointUp}{\vcenteredincludeIcon{graphics/emoji_bw_hand_point_up.jpg}\xspace}

\newcommand{\proofsquare}{
    \begin{flushright}
      $\square$
    \end{flushright}\xspace
}
\newcommand{\done}{
    \begin{flushright}
      $\bullet$
    \end{flushright}\xspace
}
\newcommand{\donenot}{
    \begin{flushright}
      $\ldots$
    \end{flushright}\xspace
}
\newcommand{\noresult}{
    \begin{flushright}
      $\varnothing$
    \end{flushright}\xspace
}

\newcommand{\progress}{
    \begin{flushright}
      $\Delta$
    \end{flushright}\xspace
}

\newcommand{\topicend}{
      $\blacktriangleleft$
}

\newcommand{\meineChance}{\textit{Meine Chance}\xspace\index{Meine Chance}}

\tikzstyle{every node}=[font=\small]
\tikzstyle{every node}=[inner sep=1pt]
\usetikzlibrary{calc}
\usetikzlibrary{matrix}
\def\mcr{\pgfmatrixcurrentrow}\def\mcc{\pgfmatrixcurrentcolumn}
\def\width{12}
\def\hauteur{12}

\definecolor{rosso}{RGB}{220,57,18}
\definecolor{giallo}{RGB}{255,153,0}
\definecolor{blu}{RGB}{102,140,217}
\definecolor{verde}{RGB}{16,150,24}
\definecolor{viola}{RGB}{153,0,153}

\makeatletter

\pgfdeclarelayer{background}
\pgfdeclarelayer{foreground}
\pgfsetlayers{background,main,foreground}


\newcommand{\pie}[3][]{
    \begin{scope}[#1]
    \pgfmathsetmacro{\curA}{90}
    \pgfmathsetmacro{\r}{1}
    \def\c{(0,0)}
    \node[pie title] at (90:1.3) {#2};
    \foreach \v/\s in{#3}{
        \pgfmathsetmacro{\deltaA}{\v/100*360}
        \pgfmathsetmacro{\nextA}{\curA + \deltaA}
        \pgfmathsetmacro{\midA}{(\curA+\nextA)/2}

        \path[slice,\s] \c
            -- +(\curA:\r)
            arc (\curA:\nextA:\r)
            -- cycle;
        \pgfmathsetmacro{\d}{max((\deltaA * -(.5/50) + 1) , .5)}

        \begin{pgfonlayer}{foreground}
        \path \c -- node[pos=\d,pie values,values of \s]{$\v\%$} +(\midA:\r);
        \end{pgfonlayer}

        \global\let\curA\nextA
    }
    \end{scope}
}

\newcommand{\legend}[2][]{
    \begin{scope}[#1]
    \path
        \foreach \n/\s in {#2}
            {
                  ++(0,-10pt) node[\s,legend box] {} +(5pt,0) node[legend label] {\n}
            }
    ;
    \end{scope}
}


\theoremstyle{definition}
%\theorembodyfont{\small}
\newtheorem{definition}{Definition}
\newtheorem{uebung}{Übung}
\newtheorem{beispiel}{Beispiel}

\newglossary*{symbols}{Symbolverzeichnis}

\lstset{
    language={java},
    basicstyle=\ttfamily\footnotesize,
    breaklines=true,
    numbers=left,
    stepnumber=5,
    numberstyle=\tiny\color{gray},
    mathescape=true,
    showstringspaces=false
}

\newglossaryentry{CAS}{
    name={Computeralgebrasystem},
    description={\url{https://de.wikipedia.org/wiki/Computeralgebrasystem}: ''Ein Computeralgebrasystem (CAS) ist ein Computerprogramm, das der Bearbeitung algebraischer Ausdrücke dient. Es löst nicht nur mathematische Aufgaben mit Zahlen (wie ein einfacher Taschenrechner), sondern auch solche mit symbolischen Ausdrücken (wie Variablen, Funktionen, Polynomen und Matrizen).''
    }
}

\newglossaryentry{GGT}{
    name={ggT},
    description={Größter gemeinsamer Teiler.}
}

\newglossaryentry{KGV}{
    name={kgV},
    description={Das kleinste Gemeinsame Vielfache einer Zahl.}
}

\newglossaryentry{SISystem}{
    name={SI-System},
    description={\enquote{Das Internationale Einheitensystem oder SI (frz. Système international d’unités) ist das am weitesten verbreitete Einheitensystem für physikalische Größen.} (\url{https://de.wikipedia.org/wiki/Internationales_Einheitensystem}, abgerufen 2017-09-13 15:27)}
}

\newglossaryentry{TRS}{
    name={Term Replacement System (TRS)},
    description={
        \enquote{In mathematics, computer science, and logic, rewriting covers a wide range of (potentially non-\-de\-term\-inistic) methods of replacing subterms of a formula with other terms. What is considered are rewriting systems (also known as rewrite systems, rewrite engines or reduction systems). In their most basic form, they consist of a set of objects, plus relations on how to transform those objects. (\url{https://en.wikipedia.org/wiki/Rewriting}, abgerufen 2017-09-17 14:15)}
    }
}

\newglossaryentry{Uhrzeigersinn}{
    name={Uhrzeigersinn},
    description={
        \enquote{im Uhrzeigersinn} und \enquote{gegen den Uhrzeigersinn} bezeichnet die Richtung in der ein Rundweg in der Ebene betrachtet wird. Im Uhrzeigersinn ist die Richtung die die Zeiger einer Uhr beschreiben, gegen den Uhrzeigersinn die entgegengesetzte Richtung.
    }
} 
\newglossaryentry{symb:Abrunden}{
    type=symbols,
    name={$\left\lfloor x \right\rfloor$},
    description={Abrunden. $\left\lfloor 1,99 \right\rfloor = 1$.},
    sort={abrunden}
}

\newglossaryentry{symb:alpha}{
    type=symbols,
    name={$\alpha$},
    description={Der griechische Buchstabe $\alpha$, sprich \enquote{alpha} bezeichnet insbesondere den Winkel am Punkt $A$ in Polygonen.},
    sort={alpha}
}

\newglossaryentry{symb:Area}{
    type=symbols,
    name={$A_F$},
    description={Flächeninhalt (engl. area) einer Fläche $F$. Wir ersetzen $F$ für elementare Flächen durch den ersten Buchstaben des Namens der Fläche.},
    sort={A}
}

\newglossaryentry{symb:Aufrunden}{
    type=symbols,
    name={$\left\lceil x \rceil$},
    description={Abrunden. $\left\lfloor 1,99 \right\rfloor = 1$.},
    sort={abrunden}
}

\newglossaryentry{symb:beta}{
    type=symbols,
    name={$\beta$},
    description={Der griechische Buchstabe $\beta$, sprich \enquote{beta} bezeichnet insbesondere den Winkel am Punkt $B$ in Polygonen.},
    sort={beta}
}

\newglossaryentry{symb:delta}{
    type=symbols,
    name={$\delta$},
    description={Der griechische Buchstabe $\delta$, sprich \enquote{delta} bezeichnet insbesondere den Winkel am Punkt $D$ in Polygonen.},
    sort={delta}
}

\newglossaryentry{symb:Div}{
    type=symbols,
    name={$\div$},
    description={Symbol für die Division. Gelegentilich wird \enquote{/} benutzt, insbesondere in Texten in denen keine Brüche gesetzt werden können sowieo auf vielen Taschenrechnern.
    },
    sort={Division}
}

\newglossaryentry{symb:gamma}{
    type=symbols,
    name={$\gamma$},
    description={Der griechische Buchstabe $\gamma$, sprich \enquote{gamma} bezeichnet insbesondere den Winkel am Punkt $C$ in Polygonen.},
    sort={gamma}
}

\newglossaryentry{symb:Gleich}{
    type=symbols,
    name={$=$},
    description={
        Wir sagen \enquote{gleich} oder \enquote{ist gleich} und bezeichnen damit die Gleichheit der seiten einer Gleichung, insbesondere die Gleichheit eines zu berechnenden Ausdrucks und des Ergebnisses der Rechnung. $2-1=1$, \enquote{zwei minus eins (ist) gleich eins.}
        },
    sort={gleich}
}

\newglossaryentry{symb:Lambda}{
    name={$\lambda$},
    description={Eine beliebige Zahl, mit der der nachfolgende Ausdruck multipliziert wird.},
    type=symbols,
    sort={lambda}
}

\newglossaryentry{symb:Mal}{
    type=symbols,
    name={$\cdot$},
    description={Symbol für die Multiplikation. Gelegentlich wird, insbesondere für die Lesbarkeit $\times$ benutzt},
    sort={mal}
}

\newglossaryentry{symb:Minus}{
    type = symbols,
    name={$-$},
    description={Wir sagen \enquote{minus} und bezeichnen damit die Operation der Subtraktion. $2-1=1$, \enquote{zwei minus eins (ist) gleich eins.}},
    sort={minus}
}

\newglossaryentry{symb:N}{
    type=symbols,
    name={$\mathbb{N}$},
    description={Die Menge der Natürlichen Zahlen $\{0, 1, 2, 3, 4, ...\}$},
    sort={N}
}

\newglossaryentry{symb:Phi}{
    name={$\varphi$},
    description={Ein beliebiger Winkel.},
    type=symbols,
    sort={phi}
}

\newglossaryentry{symb:Pi}{
    name={$\pi$},
    description={Die Kreiszahl.},
    type=symbols,
    sort={pi}
}

\newglossaryentry{symb:Plus}{
    type=symbols,
    name={$+$},
    description={Wir sagen \enquote{plus} und bezeichnen damit die Operation der Addition. $1+2=3$, \enquote{eins plus zwei (ist) gleich drei}.},
    sort={plus}
}

\newglossaryentry{symb:Sum}{
    type=symbols,
    name={$\Sigma$},
    description={Summe},
    sort=Sigma
}

\newglossaryentry{symb:Vereinigung}{
    type=symbols,
    name={$\bigcup$},
    description={Vereinigung (von Mengen)},
    sort=Vereinigung
} 
\makeglossaries

\setlength{\parskip}{1ex}

\let\origitemize\itemize
\def\itemize{\origitemize\itemsep0pt}

%\setcounter{tocdepth}{1}
%\setcounter{secnumdepth}{0}

\makeindex

\begin{titlepage}
%\title{Meine Chance II - 2017\\Mathematik}
\centering
\title{Mathematik\\ \vspace{1cm} \normalsize \centering Konzept für den Mathematik-Unterricht in Vorbereitung für die Abschlussprüfung für den Hauptschulabschluss als Nichtschülerprüfung in Hessen, konkretisiert für das Projekt \enquote{Meine Chance} ($mc^2$)\\\vspace{3em}\footnotesize Die freie, insbesondere unentgeltliche, Nutzung im Rahmen von \enquote{Meine Chance} ist auf Dauer vom Urheber genehmigt. Sollte der Autor dauerhaft nicht erreichbar sein geht das komplette Material inklusive der \LaTeX-Sources in public domain über.}

\centering
\author{Erik Itter}
\end{titlepage}


\begin{document}

\newgeometry{hmarginratio=1:1}
\maketitle
\restoregeometry
\tableofcontents

\printglossary[type=symb, style=long]

\part{Konzept}

\chapter{Heterogenität}

Ausgangspunkt für die Überlegungen zur Bedeutung von und Umgang mit Heterogenität der Teilnehmer, bezogen auf den Unterricht in Mathematik, ist zunächst \citep{Leiss2014}. Der Unterricht muss mit bewusster Binnendifferenzierung arbeiten, die über das typische Maß deutscher Schulklassen hinaus geht. Die Herausforderungen im Unterricht in einer Gruppe von Migranten mit unterschiedlichen sprachlichen Niveaus kommt dazu die besondere Problematik der Sprachbarriere hinzu, die aber für den Unterricht in den ersten Monaten keine unüberwindbare Hürde darstellt, da das Material gut visuell gestützt vermittelt werden kann. Entsprechend können Textaufgaben in den ersten Monaten nicht genutzt werden. Die Modellierung muss damit, zumindest als explizites Thema, Richtung Ende der Maßnahme rutschen - auch wenn das u.a. hinsichtlich der Motivation der Themen des Mathematik-Unterrichts nicht ideal ist.

Eine strukturelle Vereinfachung, die jedoch eher zu Lasten der Lernenden gehen wird, ist, dass die Prüfungen als Nichtschüler mit sich bringen, dass die Bewertung (und das Bestehen) alleine von Kompetenzerwerb und -nachweis abhängen und es kaum\footnote{Theoretisch keinen, praktisch durch Berücksichtigung des Hintergrundes in der mündlichen Prüfung.} Raum für eine Würdigung der Fortschritte gibt.

Der organisatorische Schwerpunkt zur Berücksichtigung der Besonderheiten wird auf Tutorien liegen, die weitgehend kooperative Lernumgebung bieten sollen, als Ergänzung zum klassischen vortragenden und diskutierenden Unterricht (Zu den Gründen vgl. \citep[S.7ff]{Leiss2014}, besonders betont sei hier die Eigenverantwortung der Lernenden, die im klassischen Setting leicht an die Lehrperson zurückgegeben werden kann.).

Gerade in einer Lern- und Lehrsituation hoher Heterogenität kann eine systematische Beobachtung und Dokumentation wertvoll sein um im Team der Lehrenden einander ergänzende Maßnahmen durchzuführen. Dokumentation erleichtert u.a. individuelle Schwächen aufzudecken, die im Laufe der Maßnahme zu korrigieren sind. Vorerst ist das Werkzeug dafür der Modulplan der für den Abschluss vorausgesetzten Kompetenzen, zusammenfassend jedem Teilnehmer als Lernprotokoll/ Fortschrittskontrolle ausgehändigt, sowie für die laufende Vorbereitung des Unterrichts als \textit{Microsoft Excel} spreadsheet für den gesamten Jahrgang fortwährend geführt.


\chapter{Tutorium, Lern- und Übungsgruppen}

Kaum jemand kann Mathematik gut alleine lernen. Mathematik so zu lernen, dass man sie operativ kompetent anwenden kann, erfordert i.d.R. dass man die Inhalte selbst erklären kann. So lange das nicht gelingt fehlen üblicherweise Aspekte des Sachverhalts. Schon deshalb rein egoistisch ist es für alle Teilnehmer sinnvoll sich gegenseitig die Inhalte beizubringen, sie gemeinsam einzuüben und zu vertiefen und sie für andere zu präsentieren/ referieren. Jeder Teilnehmer muss an einem Tutorium teilnehmen, das als betreute Lerngruppe verstanden wird. Verglichen mit dem universitären Lehren der Mathematik handelt es sich eher um Lerngruppen. Wir nennen es Tutorien, um es von überhaupt nicht betreutem Lernen in eigener Regie der Teilnehmer zu unterscheiden.


\section{Lehrerinterventionen}


\section{Mathematisches Modellieren}

Das Modellieren (einfacher) Probleme sollte von Anfang an geübt werden, so dass textuell gestellte Probleme mit den Inhalten des Unterrichts verbunden werden, die sonst häufig nur als Rechenvorschriften gelernt werden, so dass Textaufgaben in der Prüfung nicht bewältigt werden können, geschweige denn als Basis für wahrscheinliche weitergehende Kontakte mit der Mathematik, insbesondere in der Berufsausbildung, nicht nur aber vor allem in technischen Berufen.


\part{Lehrinhalte}

\chapter{Was ist Mathematik?}

Die \textbf{Mathematik} im Hauptschulabschluss\footnote{Für den Realschulabschluss kommt schon einiges zu Funktionen, auch wenn man noch ohne das durch die Prüfung kommt.} verstehen wir als \textbf{Zählen}, \textbf{Messen}, (elementares) \textbf{Rechnen} und die \textbf{Kommunikation} dessen.


\chapter{Zahlen}

\section{Natürliche Zahlen ($\mathbb{N}$)}

\textbf{\glsdisp{symb:N}{Natürliche Zahlen}} benutzen wir zum Zählen. Ein Apfel, zwei Äpfel, drei Äpfel, vier Äpfel, ... $\mathbb{N} = \{ 0$\footnote{Viele Mathematiker betrachten die Null nicht als natürliche Zahl. Für uns spielt keine Rolle ob Null eine natürliche Zahl ist oder nicht.}$, 1, 2, 3, 4, 5, 6, 7, 8, 9, ...\}$


\section{Primfaktorzerlegung}

Jede natürliche Zahl $n$ kann (in einer eindeutigen Weise) in \textbf{Primfaktoren}\index{Primfaktor} zerlegt werden (\textbf{Faktorisierung}). 6 ist \textbf{teilbar} durch 2 und 3. 1 und die Zahl selbst zählen hier nicht.

\begin{equation}
    6 = 2 \cdot 3
\end{equation}

Die Reihenfolge der \textbf{Faktoren} ist egal. $3 \cdot 2$ und $2 \cdot 3$ werden als die selbe \textbf{Zerlegung} (\textbf{Primfaktorzerlegung}) betrachtet.

\begin{equation}
    60 = 2 \cdot 2 \cdot 3 \cdot 5
\end{equation}


\section{Numeralia (Zahlwörter)}\label{numeralia}

\enquote{eins, zwei, drei, vier, fünf, sechs, sieben, ...} ist der Beginn der Zahlwörter (als Kardinalzahlen). \citep[Randzahl 509]{DudenGrammatik2016} gibt eine unvollständige Einleitung. Demnach sind sowohl Schreibweisen als auch gesprochene Zahlen nicht (mehr) streng festgelegt. Eine veraltete Regel lautet dass natürliche Zahlen bis zwölf als Zahlwort geschrieben werden, darüber hinaus als Ziffernfolge. Es bleibt das Problem Zahlwörter für größere Zahlen festzulegen, so dass größere Zahlen gesprochen werden können. Die Numeralia von 0 bis 12 sind \enquote{eins, zwei, drei, vier, fünf, sechs, sieben, acht, neun, zehn, elf, zwölf}. Für Zahlen mit Zehnerstellen\footnote{Mit \enquote{echten} Zehnerstellen, also keine Null in der vorletzten Ziffer der Zahl.} werden die Zehner-Suffixe benötigt, diese sind \enquote{-zehn, -zwanzig, -dreißig, -vierzig, -fünfzig, -sechzig, -siebzig, -achtzig, -neunzig}. Für die Hunderter-Stellen werden die Numerale von 1 bis 9 vor das Numeral \enquote{hundert} gesetzt. Statt \enquote{eins} ist das Numeral hier \enquote{ein}. Sind Hunderter vorhanden wird zwischen ihnen und den Zehnern \enquote{und} eingefügt: \enquote{eins, zwei, drei, vier, fünf, sechs, sieben, acht, neun, zehn, elf, zwölf, dreizehn, vierzehn, ..., neunzehn, zwanzig, einundzwanzig, zweiundzwanzig, ..., neunundzwanzig, dreißig, einunddreißig, ..., neunundneunzig, einhundert, einhundertundeins, ..., einhundertundneunundneunzig\footnote{\cite{DudenGrammatik2016} erkennt ausdrücklich auch hundertneunundneunzig als korrekt an.}, zweihundert, zweihunderundeins, ... neunhundertundneunundneunzig}.

Bei Zahlen ab 1.000 (\enquote{eintausend}) werden die Numeralia in 3er-Paketen (Tripeln) gebildet. Das kleinste Tripel wird mit \enquote{und} verknüpft angehängt. 123.456 wird \enquote{einhundertunddreiundzwanzigtausendundvierhundertundsechsundfünfzig} gelesen\footnote{\citep{DudenGrammatik2016} erkennt hier gleich mehrere unterschiedliche Varianten als korrekt an.}. Es wird zunächst das höchstwertige Tripel gebildet, so dass alle noch folgenden Teile der Zahl echte (dreistellige) Tripel sind. Also wird für 23456 456 abgetrennt und das (unvollständige) höchstwertige Tripel 23 gelesen: \enquote{dreiundzwanzig}. Angehängt wird der Name der Größenordnung (für das vorletzte Tripel 1.000 ($10^3$), \enquote{eintausend}). Danach wird das nächstkleinere Tripel (hier das letzte, 456 gelesen, \enquote{vierhundertundsechsundfünfzig}. Zahlen unter einer Million werden zusammen geschrieben, so dass 23.456 \enquote{dreiundzwanzigtausendundvierhundertundsechsundfünfzig}\footnote{Korrekt wäre u.a. auch \enquote{dreiundzwanzigtausendvierhundertsechsundfünfzig}, aber die Varianten mit weniger \enquote{und} brauchen mehr Bildungsregeln.} heist. Die Namen der ersten Größenordnungen sind (Singular/Plural) $10^3$: tausend/tausend, $10^6$: Million/Millionen, $10^9$: Milliarde/Milliarden, $10^{12}$: Billion/Billionen, $10^{15}$: Billiarde/Billiarden. Für dem Alltag sollten die Namen bis zu den Milliarden bekannt sein (z.B. Bundeshaushalt).

Es ergeben sich z.B. $1.123.456.789.123$: \enquote{eine Billion einhundertunddreiundzwanzig Milliarden vierhundertundsechsundfünfzig Millionen siebenhundertundneunundachtzigtausendundeinhundertunddreiundzwanzig} und $89.001.456.901$: \enquote{neunundachtzig Milliarden eine Million vierhundertundsechsundfünfzigtausendundneunhunderundeins}. Solche Numeralia werden nie geschrieben, die in Ziffern geschrieben Zahl muss aber zumindest bis in den Bereich der Milliarden vorgelesen/ ausgesprochen werden können. Die hier angegebene Variante (mit maximalen \enquote{und}-Verbindungen) scheint die Variante mit den wenigsten Bildungsregeln aus den korrekten Varianten zu sein. Scheinen Stellen nach der größten Größenordnung zu fehlen (Nullen) werden diese nicht gesprochen: $999.999.999$, $1.000.000.000$, $1.000.000.001$ wird \enquote{neunhundertundneunundneunzig Millionen neunhunderundneunungneunzigtausendundneunhundertundneunundneunzig, eine Milliarde, eine Milliarde und eins} gelesen.

\begin{uebung}[Numeralia (Zahlwörter)]
    Schreibe die Numeralia auf für 1 bis 21, 31, 42, 53, 64, 72, 84, 93, 99, 100, 101, 199, 200, 201, 999, 1.000, 1.001, 9.000, 9.999, 123.456, 200.000, 999.999, 1.000.000, 10.000.001, 999.999.999, 1.000.000.000, 1.000.000.001 und 123.456.789.012.345.\topicend
\end{uebung}

\section{Primzahlen ($\mathbb{P}$)}\index{Primzahlen}

Eine natürliche Zahl $n > 1$\footnote{Wieso wir die Eins nicht \textbf{Primzahl} nennen braucht uns hier nicht zu interessieren, weder für den Hauptschulabschluss noch für den Realschulabschluss.}, die nur durch 1 und sich selbst \textbf{teilbar} ist, nennen wir eine \textbf{Primzahl}. Die ersten paar Primzahlen lernen wir auswendig um die \textbf{Primfaktoren} einer Zahl schneller zu finden, die wir in der Bruchrechnung brauchen: $\mathbb{P}=\left\{2, 3, 5, 7, 11, 13, 17, 19, 23, 29, ...\right\}$.

\begin{beispiel}
    Die \textbf{Primfaktorzerlegung}\index{Primfaktorzerlegung} von $6$ ist $\{2,3\}$, die von $120$ ist $\{2,2,2,3,5\}$. Wir schreiben $\{2,2,2,3,5\}$ auch als $\{2^3,3^1,5^1\}$ oder $\{2^3,3,5\}$.\topicend
\end{beispiel}


\section{Variablen (Unbekannte/ Veränderliche)}\index{Variablen}\index{Unbekannte|see{Variable}}\index{Veränderliche|see{Variable}}

In der Mathematik wollen wir (insbesondere in \textbf{Formeln}) möglichst \textbf{allgemeine} Aussagen treffen um möglichst viel mit diesen anfangen zu können. Oftmals benutzen wir dazu keine Zahlen, sondern Namen wie \enquote{a}, \enquote{b} und \enquote{c} oder \enquote{x} und \enquote{y}. Diese Namen stehen immer für Zahlen. Man kann sie durch beliebige Zahlen ersetzen (aber in einer Rechnung immer durch die Selbe). Die Fläche eines Rechtecks nennen wir z.B. allgemein $a \cdot b$, oder \enquote{Länge mal Breite}. Jeder (positive) Wert ergibt ein Rechteck, dessen Fläche wir so berechnen können: $A_{\text{Rechteck}} = a \cdot b = A_R = a b$\footnote{\enquote{A} steht für \enquote{area}, englisch für \textbf{Fläche/ Flächeninhalt}}.

\begin{beispiel}
    Gegeben sei ein Rechteck mit der Länge $a = 12 \text{cm}$ und Breite $b = 5 \text{cm}$. Zu berechnen sei der Flächeninhalt des Rechtecks $A_R = a b$.
    \begin{align}
      A_R & =  a b && a \rightarrow 12 cm, b \rightarrow 6 cm\\
       & =  12 cm \cdot 6 cm &&\\
       & =  60 cm^2 &&
    \end{align}\topicend
\end{beispiel}

Genau genommen ersetzen wir wie im Beispiel zu sehen, in diesem Fall mit einer reellen Zahl und einer Maßeinheit. Diese Feinheiten können wir jedoch zunächst ignorieren und in den kommenden Monaten intuitiv durch Üben lernen.


\chapter{Elementare Rechenoperationen}

\enquote{$1+2=3$}\glsadd{symb:Plus}\glsadd{symb:Minus}\glsadd{symb:Gleich} sprechen wir \enquote{eins plus zwei (ist) gleich drei}. \textbf{Plus} zu rechnen nennen wir \textbf{addieren} oder \textbf{die Addition}. \enquote{$3-2=1$} sprechen wir \enquote{drei minus zwei (ist) gleich 1}. \textbf{Minus} zu rechnen nennen wir \textbf{subtrahieren} oder \textbf{die Subtraktion}. \textbf{Das Ergebnis} einer \textbf{Subtraktion} nennen wir \textbf{die Differenz}. Wir sagen auch \enquote{die Differenz von drei und zwei ist eins}\footnote{und umgangssprachlich/ alltagssprachlich auch \enquote{die Differenz von zwei und drei ist eins (obwohl $2-3=-1$). Hierbei kommt die Vorstellung der \textbf{Differenz} als \textbf{Abstand} am \textbf{Zahlenstrahl} $\mathbb{R}^1$ zum intuitiven Ausdruck}}.

\section{Addition und Subtraktion am Zahlenstrahl}

Das Rechnen mit \glsdisp{symb:Plus}{$+$} und \glsdisp{symb:Minus}{$-$} wird als Bewegung auf dem Zahlenstrahl der reellen Zahlen verstanden sofern es überhaupt problematisch ist. Von den Lehrenden aus wird das Berechnen von Summen als bekannt vorausgesetzt. Wird klar, dass das nicht gerechtfertigt ist, so wird der Zahlenstrahl\index{Zahlenstrahl} als Interpretation von $\mathbb{R}$ verwendet ohne $\mathbb{R}$ zu problematisieren, einzuführen oder zu definieren\footnote{Der Zahlenstrahl bringt einen natürlichen Weg zu $\mathbb{R}$ mit, da eine kontinuierliche \enquote{Strecke} schlecht in diskrete Abschnitte aufgeteilt sein kann elementar, da man jedes Stück wieder teilen kann ins Unendliche. (Dass vielleicht die Realität gar keine kontinuierlichen Räume beinhaltet wird die Lernenden sicher nicht stören und sei mit einem Hinweis auf die breite Anwendbarkeit der Analysis beiseite gelegt.)}.

\begin{equation}\label{eqn:00001}
    1 + 2 + 3
\end{equation}

\begin{figure}[H]
  \centering
\begin{tikzpicture}[>=triangle 60]

    \draw (1,1) -- +(10,0);

    \foreach \x in {1,...,11}{
        \pgfmathtruncatemacro{\label}{\x-1}
        \draw (\x,0.8) -- +(0,0.4);
        \node[below] (xa) at (\x,0.7) {\label};
    }

    \coordinate[label={[label distance=10pt]90:1}] (1) at (2,2);
    \coordinate[label={[label distance=10pt]90:3}] (3) at (4,2);
    \coordinate[label={[label distance=10pt]90:6}] (6) at (7,2);
    \fill (1) circle (1pt);
    \fill (3) circle (1pt);
    \draw[->,shorten >=4pt] (1) -- (3) node[midway, below] {+2};
    \fill (6) circle (1pt);
    \draw[->,shorten >=4pt] (3) -- (6) node[midway, below] {+3};

\end{tikzpicture}
  \caption{$+/-$ am Zahlenstrahl $\mathbb{R}$}\label{fig:zahlenstrahlAddition}
\end{figure}

Problematisch wird das Subtrahieren von negativen Zahlen, für das man sich am Zahlenstrahl zwar mit \textquote{Drehungen} um 180° und damit $-(-x) = x$ aus der Affäre ziehen kann, das aber mathematisch so nicht vertretbar ist. Sauber wäre hier schlicht die Regel für die Äquivalenz anzugeben und die Lernenden als \gls{TRS} arbeiten zu lassen. Die Repräsentation kontinuierlicher Strecken vermeidet bei Lernenden, die bereits verfestigt zählen\index{verfestigtes Zählen} (vgl. \citep[S. 112]{Hasemann2014}) statt zu rechnen dies weiter zu bedienen, wie z.B. durch eine Darstellung $\clubsuit + \clubsuit = \clubsuit \clubsuit$ oder $\clubsuit + \clubsuit = 2 \clubsuit$\footnote{Die Darstellung $\clubsuit + \clubsuit = 2 \clubsuit$ hat ihre Berechtigung in der Algebra, wenn die Lernenden dort den abstrakten Umgang mit einer Unbekannten, $x$, erlernen.}. D.h. beim Sprechen über Arithmetik in $\mathbb{R}$ ist wennimmer möglich in Strecken zu reden und Zählen zu vermeiden.

\begin{equation}
    -1 + 5 - 3
\end{equation}

\begin{figure}[H]
  \centering
\begin{tikzpicture}[>=triangle 60]

    \draw (1,1) -- +(10,0);

    \foreach \x in {-2,...,8}{
        \pgfmathtruncatemacro{\label}{\x}
        \pgfmathtruncatemacro{\posx}{\x+3}
        \draw (\posx,0.8) -- +(0,0.4);
        \node[below] (xa) at (\posx,0.7) {\label};
    }

    \coordinate[label={[label distance=10pt]90:-1}] (-1) at (2,3);
    \coordinate[label={[label distance=10pt]90:4}] (4) at (7,3);
    \fill (-1) circle (1pt);
    \fill (4) circle (1pt);
    \draw[->,shorten >=4pt] (-1) -- (4) node[midway, below] {+5};
    \fill (6) circle (1pt);

    \coordinate[label={[label distance=10pt]90:4}] (4l) at (7,2);
    \coordinate[label={[label distance=10pt]90:1}] (1l) at (4,2);
    \fill (1l) circle (1pt);
    \fill (4l) circle (1pt);
    \draw[->,shorten >=4pt] (4l) -- (1l) node[midway, below] {-3};

\end{tikzpicture}
  \caption{$+/-$ am Zahlenstrahl $\mathbb{R}$}\label{fig:zahlenstrahlSubtraktion}
\end{figure}

\begin{definition}[Betrag]\index{Betrag}
    Wir nennen $|x|$ den Betrag (engl. abs) von $x$. Es sei $|-x| = x$, $|x|=x$ und damit $|-x|=|x|$.
\end{definition}


\section{Multiplikation}

Wir schreiben die \textbf{Multiplikation} von \textbf{Faktoren} $f_1 \cdot f_2$ und das Ergebnis einer \textbf{Multiplikation} das \textbf{Produkt}.\glsadd{symb:Mal}


\section{Division}

Wir schreiben die \textbf{Division} von \textbf{Faktoren}\footnote{Die Mathematik nennt das in der Tat so (und unterscheidet Multiplikation und Division kaum). Wem das zu unklar ist kann die Begriffe Dividend und Divisor benutzen. Dann bezeichnet der Dividend die zu teilende Zahl und Divisor die Zahl durch die der Dividend geteilt wird.} $f_1 \div f_2$ und bezeichnen das Ergebnis einer \textbf{Division} der \textbf{Quotient}.\glsadd{symb:Div}


\chapter{Maßeinheiten}\index{Maßeinheiten}

\section{SI-System}\index{SI-System}

Das \glsdisp{SISystem}{SI-System} wird weltweit für im technischen und wissenschaftlichen Bereich genutzt und fast überall (leider gehören die USA zu den drei Ausnahmen) alltäglich. Für den Hauptschulabschluss brauchen wir \enquote{kms}, das Kilogramm (kg) für Massen und \enquote{Gewichte}, den Meter (m) für Strecken und die Sekunde (s) für Zeitspannen. Wer das Problem der Maßeinheiten versteht findet vielleicht interessant was man bei \youtube unter \enquote{si einheiten} findet und von da aus auch das komplette System erschließen kann. Manches was durch Umrechnungen passiert ist, ist von heute aus zurück schauend auch witzig - wenn z.B. \textit{NASA} 200 Millionen Dollars in den Jupiter rammt weil ein Lieferant (USA...) in amerikanischen Pfund statt Kilogramm gerechnet hat...


\section{Präfixe für Größenordnungen}\index{Größenordnungen}\index{Präfixe|see{Größenordnungen}}

\enquote{Kilo} bedeutet 1000, \enquote{Dezi-} bedeutet $\frac{1}{10}$, \enquote{Zenti-} bedeutet $\frac{1}{100}$ (beachte, dass die Abkürzungen \enquote{c} statt \enquote{z} verwenden, da wir die englische Schreibweise übernommen haben), \enquote{Milli} bedeutet $\frac{1}{1000}$. Ein Millimeter ist also ein tausendstel Meter und ein Zentimeter ein hundertstel Meter.

\section{Der Meter, Maßeinheit für Strecken}

Mit dem Meter messen wir Entfernungen, Längen, Strecken. Alle anderen Maßeinheiten für Strecken werden durch den Meter definiert. Ein Millimeter ist ein tausendstel Meter. \enquote{Milli} ist das Präfix für $\frac{1}{1000}$. Prüfungsrelevant sind Meter, Kilometer, Dezimeter, Zentimeter und Millimeter. Mikrometer ($10^{-6}m$) und Nanometer ($10^{-9}m$) kommen gelegentlich in Nachrichten vor, z.B. als Größe von (gesundheitsschädlichen) Partikeln. Für die Prüfung relevant sind:

\begin{eqnarray*}
% \nonumber to remove numbering (before each equation)
  1mm &=& \frac{1}{1000}m \\
  1cm &=& \frac{1}{100}m \\
  1dm &=& \frac{1}{10}m \\
  1km &=& 1000m
\end{eqnarray*}

Für weniger alltäglichen Gebrauch gibt es in beide Richtungen weiter Namen für Maßeinheiten für Strecken, die sich ebenfalls jeweils auf den Meter beziehen, z.B. ist $1m = \frac{1}{9460730472580800}\text{ly}$ (Lichtjahre), d.h. ein Lichtjahr (kommt gelegentlich in den Nachrichten vor) ist somit gleich $9460730472580800\text{m}$. In der anderen Richtung sind die nächsten Präfixe \enquote{Piko} und \enquote{Femto}, so dass $1\text{m}=1000000000000000\text{fm}$.

\begin{uebung}
    Rechne um/ löse für $x$:

    \begin{equation}
        2km + 23m = x m
    \end{equation}
    \begin{equation}
        17km + 791m = x cm
    \end{equation}
    \begin{equation}
        3km + 17m + 5dm + 23cm + 2mm = x mm
    \end{equation}
\end{uebung}

\tikzset{
%Define standard arrow tip
>=stealth',
%Define style for boxes
punkt/.style={
       %rectangle,
       %rounded corners,
       %draw=black, very thick,
       text width=3em,
       minimum height=2em,
       text centered},
% Define arrow style
pil/.style={
       ->,
       thick,
       shorten <=2pt,
       shorten >=2pt,}
       }

\begin{figure}
  \centering
    \begin{tikzpicture}[node distance=1cm, auto,]
        \node[punkt] (meter) {m};
        \node[punkt, above=of meter] (dezimeter) {dm};
        \node[punkt, above=of dezimeter] (zentimeter) {cm};
        \node[punkt, above=of zentimeter] (millimeter) {mm};
        \node[punkt, above=of millimeter] (mikrometer) {$\mu$m};
        \node[punkt, above=of mikrometer] (nanometer) {nm};
        \node[punkt, below=of meter] (kilometer) {km};
        \node[punkt, below=of meter] (kilometerdummy) {}
            edge[pil,bend right=70] node[text width= 15em, anchor=west] {$\cdot 1000$} (meter.east);
        \node[punkt, above=of kilometer] (meterdummy) {}
            edge[pil,bend right=70] node[text width= 11em, anchor=west] {$\cdot 10$} (dezimeter.east)
            edge[pil,bend right=70] node[text width= 3em, anchor=east] {$\div 1000$} (kilometer.west);
        \node[punkt, above=of meter] (dezimeterdummy) {}
            edge[pil,bend right=70] node[text width= 11em, anchor=west] {$\cdot 10$} (zentimeter.east)
            edge[pil,bend right=70] node[text width= 2em, anchor=east] {$\div 10$} (meter.west);
        \node[punkt, above=of dezimeter] (zentimeterdummy) {}
            edge[pil,bend right=70] node[text width= 2em, anchor=east] {$\div 10$} (dezimeter.west)
            edge[pil,bend right=70] node[text width= 11em, anchor=west] {$\cdot 10$} (millimeter.east);
        \node[punkt, above=of zentimeter] (millimeterdummy) {}
            edge[pil,bend right=70] node[text width= 2em, anchor=east] {$\div 10$} (zentimeter.west)
            edge[pil,bend right=70] node[text width= 11em, anchor=west] {$\cdot 1000$} (mikrometer.east);
        \node[punkt, above=of millimeter] (mikrometerdummy) {}
            edge[pil,bend right=70] node[text width= 11em, anchor=west] {$\cdot 1000$} (nanometer.east)
            edge[pil,bend right=70] node[text width= 3em, anchor=east] {$\div 1000$} (millimeter.west);
        \node[punkt, above=of mikrometer] (nanometerdummy) {}
            edge[pil,bend right=70] node[text width= 3em, anchor=east] {$\div 1000$} (mikrometer.west);

        %\node[punkt, below=of kilometer] (lowerpaddingdummy) {};
        \node[punkt, left=of kilometer] (leftpaddingdummy) {};
        \node[punkt, left=of leftpaddingdummy] (leftpaddingdummy2) {};
    \end{tikzpicture}
  \caption{Umwandlungen von Strecken (m)}\label{fig:sisystemMUmwandlungenM}
\end{figure}


\section{Das Kilogramm, Maßeinheit für Massen}

\textbf{Das Kilogramm (kg)}\index{Kilogramm}\index{kg|see{Kilogramm}} ist die \textbf{Maßeinheit} des \glsdisp{SISystem}{SI-Systems} für \textbf{Massen}. In der Prüfung werden damit auch Gewichte gemessen.\footnote{Tatsächlich ist das Kilogramm die Einheit für Massen. Da alle Massen im Hauptschulabschluss auf der Erde, mehr oder minder in gleich bleibendem Abstand vom Mittelpunkt der Erde, gemessen werden können wir die Unterscheidung vernachlässigen, auch wenn das dem Physiker weh tut...}

\begin{figure}
  \centering
    \begin{tikzpicture}[node distance=1cm, auto,]
        \node[punkt] (kilogramm) {kg};
        \node[punkt, above=of kilogramm] (gramm) {g};
        \node[punkt, above=of gramm] (milligramm) {mg};
        \node[punkt, above=of milligramm] (mikrogramm) {$\mu$g};
        \node[punkt, below=of kilogramm] (tonne) {t};
        \node[punkt, below=of gramm] (kilogrammdummy) {}
            edge[pil,bend right=70] node[text width= 11em, anchor=west] {$\cdot 1000$} (gramm.east)
            edge[pil,bend right=70] node[text width= 3em, anchor=east] {$\div 1000$} (tonne.west);
        \node[punkt, above=of kilogramm] (grammdummy) {}
            edge[pil,bend right=70] node[text width= 11em, anchor=west] {$\cdot 1000$} (milligramm.east)
            edge[pil,bend right=70] node[text width= 3em, anchor=east] {$\div 1000$} (kilogramm.west);
        \node[punkt, above=of gramm] (milligrammdummy) {}
            edge[pil,bend right=70] node[text width= 3em, anchor=east] {$\div 1000$} (gramm.west)
            edge[pil,bend right=70] node[text width= 11em, anchor=west] {$\cdot 1000$} (mikrogramm.east);
        \node[punkt, above=of milligramm] (mikrogrammdummy) {}
            edge[pil,bend right=70] node[text width= 3em, anchor=east] {$\div 1000$} (milligramm.west);
        \node[punkt, below=of kilogramm] (tonnedummy) {}
            edge[pil,bend right=70] node[text width= 15em, anchor=west] {$\cdot 1000$} (kilogramm.east);

        \node[punkt, left=of kilogramm] (leftpaddingdummy) {};
        \node[punkt, left=of leftpaddingdummy] (leftpaddingdummy2) {};
    \end{tikzpicture}

  \caption{Umwandlungen von Massen (kg)}\label{fig:sisystemMUmwandlungenKG}
\end{figure}


\section{Die Sekunde, Maßeinheit für Zeitspannen}

\textbf{Die Sekunde}\index{Sekunde}\index{s|see{Sekunde}} ist die \textbf{Maßeinheit} des \glsdisp{SISystem}{SI-Systems} für Zeitspannen. Bei den Zeiteinheiten haben sich die Schritte des 60er-Stellensystem der Babylonier gehalten. In der Prüfung müssen wir mit den 60er-Schritten zwischen Sekunden, Minuten und Stunden zurechtkommen. Technisch und Wissenschaftlich rechnet man in Sekunden und den im \gls{SISystem} regelmäßigen Präfixen.

\begin{figure}
  \centering
    \begin{tikzpicture}[node distance=1cm, auto,]
        \node[punkt] (sekunde) {s};
        \node[punkt, above=of sekunde] (millisekunde) {ms};
        \node[punkt, above=of millisekunde] (mikrosekunde) {$\mu$s};
        \node[punkt, above=of mikrosekunde] (nanosekunde) {ns};
        \node[punkt, below=of sekunde] (minute) {min};
        \node[punkt, below=of minute] (stunde) {h};
        \node[punkt, below=of stunde] (tag) {d};
        \node[punkt, below=of tag] (jahr) {y};

        \node[punkt, below=of millisekunde] (sekundedummy) {}
            edge[pil,bend right=70] node[text width= 11em, anchor=west] {$\cdot 1000$} (millisekunde.east)
            edge[pil,bend right=70] node[text width= 3em, anchor=east] {$\div 60$} (minute.west);
        \node[punkt, below=of sekunde] (minutedummy) {}
            edge[pil,bend right=70] node[text width= 11em, anchor=west] {$\cdot 60$} (sekunde.east)
            edge[pil,bend right=70] node[text width= 3em, anchor=east] {$\div 60$} (stunde.west);
        \node[punkt, below=of minute] (stundedummy) {}
            edge[pil,bend right=70] node[text width= 11em, anchor=west] {$\cdot 60$} (minute.east)
            edge[pil,bend right=70] node[text width= 3em, anchor=east] {$\div 24$} (tag.west);
        \node[punkt, below=of stunde] (tagdummy) {}
            edge[pil,bend right=70] node[text width= 11em, anchor=west] {$\cdot 24$} (stunde.east)
            edge[pil,bend right=70] node[text width= 3em, anchor=east] {$\div 365$} (jahr.west);
        \node[punkt, below=of tag] (jahrdummy) {}
            edge[pil,bend right=70] node[text width= 11em, anchor=west] {$\cdot 365$} (tag.east);
        \node[punkt, above=of sekunde] (millisekundedummy) {}
            edge[pil,bend right=70] node[text width= 11em, anchor=west] {$\cdot 1000$} (mikrosekunde.east)
            edge[pil,bend right=70] node[text width= 3em, anchor=east] {$\div 1000$} (sekunde.west);
        \node[punkt, above=of millisekunde] (mikroskundedummy) {}
            edge[pil,bend right=70] node[text width= 11em, anchor=west] {$\cdot 1000$} (nanosekunde.east)
            edge[pil,bend right=70] node[text width= 3em, anchor=east] {$\div 1000$} (millisekunde.west);
        \node[punkt, above=of mikrosekunde] (nanoskundedummy) {}
            edge[pil,bend right=70] node[text width= 3em, anchor=east] {$\div 1000$} (mikrosekunde.west);
        
        \node[punkt, left=of sekunde] (leftpaddingdummy) {};
        \node[punkt, left=of leftpaddingdummy] (leftpaddingdummy2) {};
    \end{tikzpicture}
  \caption{Umwandlungen von Zeitspannen (s)}\label{fig:sisystemMUmwandlungenS}
\end{figure}

Monate sind nicht eindeutig. Kalendarische Monate haben 28 bis 31 Tage, betriebswirtschaftliche häufig 30. Wichtig für die Prüfung sollte höchstens sein dass ein Jahr zwölf Monate hat (Zinsen, Gehalt, ...).


\chapter{1 mal 1}

Das \enquote{$1 \times 1$}, sprich \enquote{ein mal eins}, wird perfekt auswendig gelernt.

\begin{tikzpicture}[border style/.style={
    draw,fill=#1,minimum size=0.8cm,anchor=center,outer sep=0,
    name=\tikzmatrixname-\the\mcr-\the\mcc
}]
\matrix[row sep=-.5*\pgflinewidth,column sep=-.5*\pgflinewidth,
   execute at empty cell={%
       \ifnum1=\mcr\relax%
           \ifnum1=\mcc\relax\node[border style=black!10!white]{$\cdot$};%
           \else\node[border style=black!10!white]{$\number\numexpr\the\mcc-1\relax$};\fi
       \else%
         \ifnum1=\mcc\relax\node[border style=black!10!white]{$\cdot$};%
           \node[border style=black!10!white]{$\number\numexpr\the\mcr-1\relax$};%
         \else%
           \pgfmathparse{int(abs(\mcr-\mcc))}%
           \ifnum5=\pgfmathresult\relax\def\temp{white}\else\def\temp{none}\fi%
           \node[border style=\temp]{\number\numexpr\numexpr\the\mcc-1\relax*\numexpr\the\mcr-1\relax\relax};%
         \fi%
       \fi}
] (a) {
&&&&&&&&&&&\\
&&&&&&&&&&&\\
&&&&&&&&&&&\\
&&&&&&&&&&&\\
&&&&&&&&&&&\\
&&&&&&&&&&&\\
&&&&&&&&&&&\\
&&&&&&&&&&&\\
&&&&&&&&&&&\\
&&&&&&&&&&&\\
&&&&&&&&&&&\\
&&&&&&&&&&&\\
};
\end{tikzpicture}

Die perfekte Beherrschung des \enquote{$1 \times 1$} ist \textbf{notwendige Voraussetzung} für sicheres und schnelles schriftliches Rechnen.

\chapter{Schriftlich Rechnen}\index{Schriftlich Rechnen}

Nochmal: Die perfekte Beherrschung des \enquote{$1 \times 1$} ist \textbf{notwendige Voraussetzung} für sicheres und schnelles schriftliches Rechnen!


\section{Schriftliche Addition}\index{Addition!schriftliche}

Wir addieren einen Term von \textbf{Summanden} indem wir alle \textbf{Summanden} korrekt nach 10er-\textbf{Stellenwertsystem} untereinander schreiben und von rechts nach links, also von der 1er-\textbf{Stelle} beginnend, alle gleichen Stellen addieren. Für Ergebnisse ab 10 ist die 1er-\textbf{Stelle} des Ergebnisses die 1er-\textbf{Stelle} der \textbf{Summe} der 1er-\textbf{Stellen} aller \textbf{Summanden}. Weitere \textbf{Stellen} werden an den \textbf{Stellen ihres Stellenwertes} vermerkt und beim nach links gehenden \textbf{Addieren} mit \textbf{addiert} (kleine Beispiele übertragen i.d.R. nur einen Zehner. Das ist die  \enquote{1}, die häufig auf dem Strich über dem Ergebnis erscheint. Bei \textbf{Summen} mit vielen \textbf{Summanden} kommen aber auch größere \textbf{Überträge} zustande.).

\begin{beispiel}[ausführliche Beschreibung schriftlich addieren]
    Wir addieren $12345$ und $6789$, $12345 + 6789$ = x (s. Abb. \ref{fig:schriftlichAddieren}, S. \pageref{fig:schriftlichAddieren}). Wir schreiben die beiden Summanden sauber ausgerichtet (1er unter 1er, 10er unter 10er, ...) untereinander.  Danach addieren wir Stellenweise, zuerst die 1er-Stellen: $9+5=14$. Wir schreiben die $4$ als 1er-Stelle des Ergebnisses auf und notieren die $1$ als Übertrag auf die 10er-Stelle. Für die 10er-Stelle des Ergebnisses addieren wir den Übertrag und die 10er-Stellen der Summanden: $1+8+4=13$. Die letzte Stelle der Summe der 10er-Stellen wird als 10er-Stelle des Ergebnisses notiert.\topicend
\end{beispiel}

\begin{beispiel}[ausführliche Beschreibung schriftlich addieren]
    Wir addieren eine ganze Reihe von Zahlen. Insbesondere beim Addieren (und Subtrahieren) von vielen Zahlen sagen wir auch dass wir \textbf{die Summe -n}\index{Summe}\index{summieren} bilden. Die Zahlen bezeichnen wir dann als \textbf{der Summand -en}\index{Summand}. Wir sagen: \enquote{Wir summieren die Summanden zur Summe.} oder \enquote{Wir bilden die Summe.}

    Wir bilden die Summe von \{83337, 47466, 43895, 18792, 68914, 36092, 644, 17669, 849, 825, 23559, 29410, 423, 29403, 69104, 44024, 220, 82771, 23239\} (s. Abb. \ref{fig:schriftlichAddieren}, S. \pageref{fig:schriftlichAddieren}):

    Wir schreiben alle Summanden sauber der Stellenwertigkeiten nach an den 1er-Stellen ausgerichtet untereinander. Wir summieren die Stellen jeweils einzeln von rechts nach links (und damit von den 1er-Stellen aufwärts). Die 1er-Stelle des Ergebnisses der Summe aller 1er-Stellen schreiben wir als 1er-Stelle des Ergebnisses auf. Die größeren Stellen der Summe der 1er-Stellen schreiben wir (klein als Merkhilfe) unter die entsprechenden Stellen der 10er-Stellen (oder bei sehr langen Listen von Summanden auch darüber hinaus).

    Die Summe der 1er-Stellen ist 85. Wir notieren die 5 als 1er-Stelle des Ergebnisses und die 8 als \textbf{Übertrag} zu den 10er-Stellen. Die Summe der 10er-Stellen und des \textbf{Übertrags} ist 83. Wir notieren die 3 als 10er-Stelle des Ergebnisses und die 8 als \textbf{Übertrag} zu den 100er-Stellen. Die Summe der 100er-Stellen und des Übertrags ist 96. Wir notieren die 6 als 100er-Stelle des Ergebnisses und die 9 als Übertrag zu den 1000er-Stellen. Die Summe der 1000er-Stellen und des Übertrags ist 90. Wir notieren die 0 als 1000er-Stelle des Übertrags und die 9 als Übertrag zu den 10000er-Stellen. Die Summe der 10000er-Stellen und des Übertrags ist 62. Wir notieren die 2 als 10000er-Stellen des Ergebnisses und die 6 als Übertrag zu den 100000er-Stellen. Die Summe der 100000er-Stellen und des Übertrags ist 6. Wir notieren die 6 als 100000er-Stelle des Ergebnisses und lesen das Ergebnis ab: 620635.\topicend
\end{beispiel}

Die Beispiele zeigen, dass schriftlich zu addieren bei Nutzung des karierten Papiers, ordentlicher Notation und etwas Geduld, mit 20 Summanden nicht schwieriger ist als mit zwei, nur längere Konzentration erfordert.

\begin{figure}
  \centering
  \begin{tikzpicture}
    %\draw[step=1mm, line width=0.1mm, black!30!white] (0,0) grid (\width,\hauteur);
    \draw[step=5mm, line width=0.1mm, black!40!white] (0,0) grid (\width,\hauteur);
    %\draw[step=5cm, line width=0.5mm, black!50!white] (0,0) grid (\width,\hauteur);
    %\draw[step=1cm, line width=0.3mm, black!90!white] (0,0) grid (\width,\hauteur);

    \node at (0.75cm, 11.75cm) {1};
    \node at (1.25cm, 11.75cm) {2};
    \node at (1.75cm, 11.75cm) {3};
    \node at (2.25cm, 11.75cm) {4};
    \node at (2.75cm, 11.75cm) {5};
    \node at (0.25cm, 11.25cm) {+};
    \node at (1.25cm, 11.25cm) {6};
    \node at (1.75cm, 11.25cm) {7};
    \node at (2.25cm, 11.25cm) {8};
    \node at (2.75cm, 11.25cm) {9};

    \draw (0.2cm,10.7cm) -- (3cm,10.7cm);

    \node at (2.75cm, 10.25cm) {4};
    \node at (2.35cm, 10.8cm) {\tiny 1};
    \node at (2.25cm, 10.25cm) {3};
    \node at (1.85cm, 10.8cm) {\tiny 1};
    \node at (1.75cm, 10.25cm) {1};
    \node at (1.35cm, 10.8cm) {\tiny 1};
    \node at (1.25cm, 10.25cm) {9};
    \node at (0.75cm, 10.25cm) {1};

    \node at (5.75cm, 11.75cm) {};
    \node at (6.25cm, 11.75cm) {8};
    \node at (6.75cm, 11.75cm) {3};
    \node at (7.25cm, 11.75cm) {3};
    \node at (7.75cm, 11.75cm) {3};
    \node at (8.25cm, 11.75cm) {7};

    \node at (5.25cm, 11.25cm) {+};
    \node at (5.75cm, 11.25cm) {};
    \node at (6.25cm, 11.25cm) {4};
    \node at (6.75cm, 11.25cm) {7};
    \node at (7.25cm, 11.25cm) {4};
    \node at (7.75cm, 11.25cm) {6};
    \node at (8.25cm, 11.25cm) {6};

    \node at (5.25cm, 10.75cm) {+};
    \node at (5.75cm, 10.75cm) {};
    \node at (6.25cm, 10.75cm) {4};
    \node at (6.75cm, 10.75cm) {3};
    \node at (7.25cm, 10.75cm) {8};
    \node at (7.75cm, 10.75cm) {9};
    \node at (8.25cm, 10.75cm) {5};

    \node at (5.25cm, 10.25cm) {+};
    \node at (5.75cm, 10.25cm) {};
    \node at (6.25cm, 10.25cm) {1};
    \node at (6.75cm, 10.25cm) {8};
    \node at (7.25cm, 10.25cm) {7};
    \node at (7.75cm, 10.25cm) {9};
    \node at (8.25cm, 10.25cm) {1};

    \node at (5.25cm, 9.75cm) {+};
    \node at (5.75cm, 9.75cm) {};
    \node at (6.25cm, 9.75cm) {6};
    \node at (6.75cm, 9.75cm) {8};
    \node at (7.25cm, 9.75cm) {9};
    \node at (7.75cm, 9.75cm) {1};
    \node at (8.25cm, 9.75cm) {4};

    \node at (5.25cm, 9.25cm) {+};
    \node at (5.75cm, 9.25cm) {};
    \node at (6.25cm, 9.25cm) {3};
    \node at (6.75cm, 9.25cm) {6};
    \node at (7.25cm, 9.25cm) {0};
    \node at (7.75cm, 9.25cm) {9};
    \node at (8.25cm, 9.25cm) {2};

    \node at (5.25cm, 8.75cm) {+};
    \node at (5.75cm, 8.75cm) {};
    \node at (6.25cm, 8.75cm) {};
    \node at (6.75cm, 8.75cm) {};
    \node at (7.25cm, 8.75cm) {6};
    \node at (7.75cm, 8.75cm) {4};
    \node at (8.25cm, 8.75cm) {4};

    \node at (5.25cm, 8.25cm) {+};
    \node at (5.75cm, 8.25cm) {};
    \node at (6.25cm, 8.25cm) {1};
    \node at (6.75cm, 8.25cm) {7};
    \node at (7.25cm, 8.25cm) {6};
    \node at (7.75cm, 8.25cm) {6};
    \node at (8.25cm, 8.25cm) {9};

    \node at (5.25cm, 7.75cm) {+};
    \node at (5.75cm, 7.75cm) {};
    \node at (6.25cm, 7.75cm) {};
    \node at (6.75cm, 7.75cm) {};
    \node at (7.25cm, 7.75cm) {8};
    \node at (7.75cm, 7.75cm) {4};
    \node at (8.25cm, 7.75cm) {9};

    \node at (5.25cm, 7.25cm) {+};
    \node at (5.75cm, 7.25cm) {};
    \node at (6.25cm, 7.25cm) {};
    \node at (6.75cm, 7.25cm) {};
    \node at (7.25cm, 7.25cm) {8};
    \node at (7.75cm, 7.25cm) {2};
    \node at (8.25cm, 7.25cm) {5};

    \node at (5.25cm, 6.75cm) {+};
    \node at (5.75cm, 6.75cm) {};
    \node at (6.25cm, 6.75cm) {2};
    \node at (6.75cm, 6.75cm) {3};
    \node at (7.25cm, 6.75cm) {5};
    \node at (7.75cm, 6.75cm) {5};
    \node at (8.25cm, 6.75cm) {9};

    \node at (5.25cm, 6.25cm) {+};
    \node at (5.75cm, 6.25cm) {};
    \node at (6.25cm, 6.25cm) {2};
    \node at (6.75cm, 6.25cm) {9};
    \node at (7.25cm, 6.25cm) {4};
    \node at (7.75cm, 6.25cm) {1};
    \node at (8.25cm, 6.25cm) {0};

    \node at (5.25cm, 5.75cm) {+};
    \node at (5.75cm, 5.75cm) {};
    \node at (6.25cm, 5.75cm) {};
    \node at (6.75cm, 5.75cm) {};
    \node at (7.25cm, 5.75cm) {4};
    \node at (7.75cm, 5.75cm) {2};
    \node at (8.25cm, 5.75cm) {3};

    \node at (5.25cm, 5.25cm) {+};
    \node at (5.75cm, 5.25cm) {};
    \node at (6.25cm, 5.25cm) {2};
    \node at (6.75cm, 5.25cm) {9};
    \node at (7.25cm, 5.25cm) {4};
    \node at (7.75cm, 5.25cm) {0};
    \node at (8.25cm, 5.25cm) {3};

    \node at (5.25cm, 4.75cm) {+};
    \node at (5.75cm, 4.75cm) {};
    \node at (6.25cm, 4.75cm) {6};
    \node at (6.75cm, 4.75cm) {9};
    \node at (7.25cm, 4.75cm) {1};
    \node at (7.75cm, 4.75cm) {0};
    \node at (8.25cm, 4.75cm) {4};

    \node at (5.25cm, 4.25cm) {+};
    \node at (5.75cm, 4.25cm) {};
    \node at (6.25cm, 4.25cm) {4};
    \node at (6.75cm, 4.25cm) {4};
    \node at (7.25cm, 4.25cm) {0};
    \node at (7.75cm, 4.25cm) {2};
    \node at (8.25cm, 4.25cm) {4};

    \node at (5.25cm, 3.75cm) {+};
    \node at (5.75cm, 3.75cm) {};
    \node at (6.25cm, 3.75cm) {};
    \node at (6.75cm, 3.75cm) {};
    \node at (7.25cm, 3.75cm) {2};
    \node at (7.75cm, 3.75cm) {2};
    \node at (8.25cm, 3.75cm) {0};

    \node at (5.25cm, 3.25cm) {+};
    \node at (5.75cm, 3.25cm) {};
    \node at (6.25cm, 3.25cm) {8};
    \node at (6.75cm, 3.25cm) {2};
    \node at (7.25cm, 3.25cm) {7};
    \node at (7.75cm, 3.25cm) {7};
    \node at (8.25cm, 3.25cm) {1};

    \node at (5.25cm, 2.75cm) {+};
    \node at (5.75cm, 2.75cm) {};
    \node at (6.25cm, 2.75cm) {2};
    \node at (6.75cm, 2.75cm) {3};
    \node at (7.25cm, 2.75cm) {2};
    \node at (7.75cm, 2.75cm) {3};
    \node at (8.25cm, 2.75cm) {9};

    \draw (5.2cm,2.2cm) -- (9cm,2.2cm);

    \node at (5.75cm, 1.75cm) {6};
    \node at (5.8cm, 2.3cm) {\tiny 6};
    \node at (6.25cm, 1.75cm) {2};
    \node at (6.3cm, 2.3cm) {\tiny 9};
    \node at (6.75cm, 1.75cm) {0};
    \node at (6.8cm, 2.3cm) {\tiny 9};
    \node at (7.25cm, 1.75cm) {6};
    \node at (7.3cm, 2.3cm) {\tiny 8};
    \node at (7.75cm, 1.75cm) {3};
    \node at (7.8cm, 2.3cm) {\tiny 8};
    \node at (8.25cm, 1.75cm) {5};//85

  \end{tikzpicture}
  \caption{schriftlich addieren}\label{fig:schriftlichAddieren}
\end{figure}


\section{Schriftliche Subtraktion}\index{Subtraktion!schriftliche}

Wir schreiben $3-2=1$ und lesen \enquote{drei minus zwei (ist) gleich 1}. In \glsdisp{symb:Sum}{Summen} können positive und negative Summanden gemischt vorkommen.

\begin{figure}
  \centering
  \begin{tikzpicture}
    %\draw[step=1mm, line width=0.1mm, black!30!white] (0,0) grid (\width,\hauteur);
    \draw[step=5mm, line width=0.1mm, black!40!white] (0,0) grid (\width,\hauteur);
    %\draw[step=5cm, line width=0.5mm, black!50!white] (0,0) grid (\width,\hauteur);
    %\draw[step=1cm, line width=0.3mm, black!90!white] (0,0) grid (\width,\hauteur);

    \node at (0.75cm, 11.75cm) {1};
    \node at (1.25cm, 11.75cm) {2};
    \node at (1.75cm, 11.75cm) {3};
    \node at (2.25cm, 11.75cm) {4};
    \node at (2.75cm, 11.75cm) {5};
    \node at (0.25cm, 11.25cm) {-};
    \node at (1.25cm, 11.25cm) {6};
    \node at (1.75cm, 11.25cm) {7};
    \node at (2.25cm, 11.25cm) {8};
    \node at (2.75cm, 11.25cm) {9};

    \draw (0.2cm,10.7cm) -- (3cm,10.7cm);

    \node at (2.75cm, 10.25cm) {6};
    \node at (2.35cm, 10.8cm) {\tiny 1};
    \node at (2.25cm, 10.25cm) {5};
    \node at (1.85cm, 10.8cm) {\tiny 1};
    \node at (1.75cm, 10.25cm) {5};
    \node at (1.35cm, 10.8cm) {\tiny 1};
    \node at (1.25cm, 10.25cm) {5};
    \node at (0.85cm, 10.8cm) {\tiny 1};
    \node at (0.75cm, 10.25cm) {};

    \node at (4.75cm, 11.75cm) {};
    \node at (5.25cm, 11.75cm) {8};
    \node at (5.75cm, 11.75cm) {3};
    \node at (6.25cm, 11.75cm) {3};
    \node at (6.75cm, 11.75cm) {3};
    \node at (7.25cm, 11.75cm) {7};

    \node at (4.25cm, 11.25cm) {-};
    \node at (4.75cm, 11.25cm) {};
    \node at (5.25cm, 11.25cm) {4};
    \node at (5.75cm, 11.25cm) {7};
    \node at (6.25cm, 11.25cm) {4};
    \node at (6.75cm, 11.25cm) {6};
    \node at (7.25cm, 11.25cm) {6};

    \node at (4.25cm, 10.75cm) {-};
    \node at (4.75cm, 10.75cm) {};
    \node at (5.25cm, 10.75cm) {4};
    \node at (5.75cm, 10.75cm) {3};
    \node at (6.25cm, 10.75cm) {8};
    \node at (6.75cm, 10.75cm) {9};
    \node at (7.25cm, 10.75cm) {5};

    \node at (4.25cm, 10.25cm) {-};
    \node at (4.75cm, 10.25cm) {};
    \node at (5.25cm, 10.25cm) {1};
    \node at (5.75cm, 10.25cm) {8};
    \node at (6.25cm, 10.25cm) {7};
    \node at (6.75cm, 10.25cm) {9};
    \node at (7.25cm, 10.25cm) {1};

    \node at (4.25cm, 9.75cm) {-};
    \node at (4.75cm, 9.75cm) {};
    \node at (5.25cm, 9.75cm) {6};
    \node at (5.75cm, 9.75cm) {8};
    \node at (6.25cm, 9.75cm) {9};
    \node at (6.75cm, 9.75cm) {1};
    \node at (7.25cm, 9.75cm) {4};

    \node at (4.25cm, 9.25cm) {-};
    \node at (4.75cm, 9.25cm) {};
    \node at (5.25cm, 9.25cm) {3};
    \node at (5.75cm, 9.25cm) {6};
    \node at (6.25cm, 9.25cm) {0};
    \node at (6.75cm, 9.25cm) {9};
    \node at (7.25cm, 9.25cm) {2};

    \node at (4.25cm, 8.75cm) {-};
    \node at (4.75cm, 8.75cm) {};
    \node at (5.25cm, 8.75cm) {};
    \node at (5.75cm, 8.75cm) {};
    \node at (6.25cm, 8.75cm) {6};
    \node at (6.75cm, 8.75cm) {4};
    \node at (7.25cm, 8.75cm) {4};

    \node at (4.25cm, 8.25cm) {-};
    \node at (4.75cm, 8.25cm) {};
    \node at (5.25cm, 8.25cm) {1};
    \node at (5.75cm, 8.25cm) {7};
    \node at (6.25cm, 8.25cm) {6};
    \node at (6.75cm, 8.25cm) {6};
    \node at (7.25cm, 8.25cm) {9};

    \node at (4.25cm, 7.75cm) {-};
    \node at (4.75cm, 7.75cm) {};
    \node at (5.25cm, 7.75cm) {};
    \node at (5.75cm, 7.75cm) {};
    \node at (6.25cm, 7.75cm) {8};
    \node at (6.75cm, 7.75cm) {4};
    \node at (7.25cm, 7.75cm) {9};

    \node at (4.25cm, 7.25cm) {-};
    \node at (4.75cm, 7.25cm) {};
    \node at (5.25cm, 7.25cm) {};
    \node at (5.75cm, 7.25cm) {};
    \node at (6.25cm, 7.25cm) {8};
    \node at (6.75cm, 7.25cm) {2};
    \node at (7.25cm, 7.25cm) {5};

    \node at (4.25cm, 6.75cm) {-};
    \node at (4.75cm, 6.75cm) {};
    \node at (5.25cm, 6.75cm) {2};
    \node at (5.75cm, 6.75cm) {3};
    \node at (6.25cm, 6.75cm) {5};
    \node at (6.75cm, 6.75cm) {5};
    \node at (7.25cm, 6.75cm) {9};

    \node at (4.25cm, 6.25cm) {-};
    \node at (4.75cm, 6.25cm) {};
    \node at (5.25cm, 6.25cm) {2};
    \node at (5.75cm, 6.25cm) {9};
    \node at (6.25cm, 6.25cm) {4};
    \node at (6.75cm, 6.25cm) {1};
    \node at (7.25cm, 6.25cm) {0};

    \node at (4.25cm, 5.75cm) {-};
    \node at (4.75cm, 5.75cm) {};
    \node at (5.25cm, 5.75cm) {};
    \node at (5.75cm, 5.75cm) {};
    \node at (6.25cm, 5.75cm) {4};
    \node at (6.75cm, 5.75cm) {2};
    \node at (7.25cm, 5.75cm) {3};

    \node at (4.25cm, 5.25cm) {-};
    \node at (4.75cm, 5.25cm) {};
    \node at (5.25cm, 5.25cm) {2};
    \node at (5.75cm, 5.25cm) {9};
    \node at (6.25cm, 5.25cm) {4};
    \node at (6.75cm, 5.25cm) {0};
    \node at (7.25cm, 5.25cm) {3};

    \node at (4.25cm, 4.75cm) {-};
    \node at (4.75cm, 4.75cm) {};
    \node at (5.25cm, 4.75cm) {6};
    \node at (5.75cm, 4.75cm) {9};
    \node at (6.25cm, 4.75cm) {1};
    \node at (6.75cm, 4.75cm) {0};
    \node at (7.25cm, 4.75cm) {4};

    \node at (4.25cm, 4.25cm) {-};
    \node at (4.75cm, 4.25cm) {};
    \node at (5.25cm, 4.25cm) {4};
    \node at (5.75cm, 4.25cm) {4};
    \node at (6.25cm, 4.25cm) {0};
    \node at (6.75cm, 4.25cm) {2};
    \node at (7.25cm, 4.25cm) {4};

    \node at (4.25cm, 3.75cm) {-};
    \node at (4.75cm, 3.75cm) {};
    \node at (5.25cm, 3.75cm) {};
    \node at (5.75cm, 3.75cm) {};
    \node at (6.25cm, 3.75cm) {2};
    \node at (6.75cm, 3.75cm) {2};
    \node at (7.25cm, 3.75cm) {0};

    \node at (4.25cm, 3.25cm) {-};
    \node at (4.75cm, 3.25cm) {};
    \node at (5.25cm, 3.25cm) {8};
    \node at (5.75cm, 3.25cm) {2};
    \node at (6.25cm, 3.25cm) {7};
    \node at (6.75cm, 3.25cm) {7};
    \node at (7.25cm, 3.25cm) {1};

    \node at (4.25cm, 2.75cm) {-};
    \node at (4.75cm, 2.75cm) {};
    \node at (5.25cm, 2.75cm) {2};
    \node at (5.75cm, 2.75cm) {3};
    \node at (6.25cm, 2.75cm) {2};
    \node at (6.75cm, 2.75cm) {3};
    \node at (7.25cm, 2.75cm) {9};

    \draw (4.2cm,2.2cm) -- (8cm,2.2cm);

    \node at (4.75cm, 1.75cm) {};
    \node at (4.8cm, 2.3cm) {\tiny };
    \node at (5.25cm, 1.75cm) {};
    \node at (5.3cm, 2.3cm) {\tiny };
    \node at (5.75cm, 1.75cm) {};
    \node at (5.8cm, 2.3cm) {\tiny };
    \node at (6.25cm, 1.75cm) {};
    \node at (6.3cm, 2.3cm) {\tiny };
    \node at (6.75cm, 1.75cm) {};
    \node at (6.8cm, 2.3cm) {\tiny };
    \node at (7.25cm, 1.75cm) {?};

    \node at (8.75cm, 11.75cm) {};
    \node at (9.25cm, 11.75cm) {8};
    \node at (9.75cm, 11.75cm) {3};
    \node at (10.25cm, 11.75cm) {3};
    \node at (10.75cm, 11.75cm) {3};
    \node at (11.25cm, 11.75cm) {7};

    \node at (8.25cm, 11.25cm) {-};
    \node at (8.75cm, 11.25cm) {(};
    \node at (8.75cm, 11.25cm) {};
    \node at (9.25cm, 11.25cm) {4};
    \node at (9.75cm, 11.25cm) {7};
    \node at (10.25cm, 11.25cm) {4};
    \node at (10.75cm, 11.25cm) {6};
    \node at (11.25cm, 11.25cm) {6};

    \node at (8.25cm, 10.75cm) {+};
    \node at (8.75cm, 10.75cm) {};
    \node at (9.25cm, 10.75cm) {4};
    \node at (9.75cm, 10.75cm) {3};
    \node at (10.25cm, 10.75cm) {8};
    \node at (10.75cm, 10.75cm) {9};
    \node at (11.25cm, 10.75cm) {5};

    \node at (8.25cm, 10.25cm) {+};
    \node at (8.75cm, 10.25cm) {};
    \node at (9.25cm, 10.25cm) {1};
    \node at (9.75cm, 10.25cm) {8};
    \node at (10.25cm, 10.25cm) {7};
    \node at (10.75cm, 10.25cm) {9};
    \node at (11.25cm, 10.25cm) {1};

    \node at (8.25cm, 9.75cm) {+};
    \node at (8.75cm, 9.75cm) {};
    \node at (9.25cm, 9.75cm) {6};
    \node at (9.75cm, 9.75cm) {8};
    \node at (10.25cm, 9.75cm) {9};
    \node at (10.75cm, 9.75cm) {1};
    \node at (11.25cm, 9.75cm) {4};

    \node at (8.25cm, 9.25cm) {+};
    \node at (8.75cm, 9.25cm) {};
    \node at (9.25cm, 9.25cm) {3};
    \node at (9.75cm, 9.25cm) {6};
    \node at (10.25cm, 9.25cm) {0};
    \node at (10.75cm, 9.25cm) {9};
    \node at (11.25cm, 9.25cm) {2};

    \node at (8.25cm, 8.75cm) {+};
    \node at (8.75cm, 8.75cm) {};
    \node at (9.25cm, 8.75cm) {};
    \node at (9.75cm, 8.75cm) {};
    \node at (10.25cm, 8.75cm) {6};
    \node at (10.75cm, 8.75cm) {4};
    \node at (11.25cm, 8.75cm) {4};

    \node at (8.25cm, 8.25cm) {+};
    \node at (8.75cm, 8.25cm) {};
    \node at (9.25cm, 8.25cm) {1};
    \node at (9.75cm, 8.25cm) {7};
    \node at (10.25cm, 8.25cm) {6};
    \node at (10.75cm, 8.25cm) {6};
    \node at (11.25cm, 8.25cm) {9};

    \node at (8.25cm, 7.75cm) {+};
    \node at (8.75cm, 7.75cm) {};
    \node at (9.25cm, 7.75cm) {};
    \node at (9.75cm, 7.75cm) {};
    \node at (10.25cm, 7.75cm) {8};
    \node at (10.75cm, 7.75cm) {4};
    \node at (11.25cm, 7.75cm) {9};

    \node at (8.25cm, 7.25cm) {+};
    \node at (8.75cm, 7.25cm) {};
    \node at (9.25cm, 7.25cm) {};
    \node at (9.75cm, 7.25cm) {};
    \node at (10.25cm, 7.25cm) {8};
    \node at (10.75cm, 7.25cm) {2};
    \node at (11.25cm, 7.25cm) {5};

    \node at (8.25cm, 6.75cm) {+};
    \node at (8.75cm, 6.75cm) {};
    \node at (9.25cm, 6.75cm) {2};
    \node at (9.75cm, 6.75cm) {3};
    \node at (10.25cm, 6.75cm) {5};
    \node at (10.75cm, 6.75cm) {5};
    \node at (11.25cm, 6.75cm) {9};

    \node at (8.25cm, 6.25cm) {+};
    \node at (8.75cm, 6.25cm) {};
    \node at (9.25cm, 6.25cm) {2};
    \node at (9.75cm, 6.25cm) {9};
    \node at (10.25cm, 6.25cm) {4};
    \node at (10.75cm, 6.25cm) {1};
    \node at (11.25cm, 6.25cm) {0};

    \node at (8.25cm, 5.75cm) {+};
    \node at (8.75cm, 5.75cm) {};
    \node at (9.25cm, 5.75cm) {};
    \node at (9.75cm, 5.75cm) {};
    \node at (10.25cm, 5.75cm) {4};
    \node at (10.75cm, 5.75cm) {2};
    \node at (11.25cm, 5.75cm) {3};

    \node at (8.25cm, 5.25cm) {+};
    \node at (8.75cm, 5.25cm) {};
    \node at (9.25cm, 5.25cm) {2};
    \node at (9.75cm, 5.25cm) {9};
    \node at (10.25cm, 5.25cm) {4};
    \node at (10.75cm, 5.25cm) {0};
    \node at (11.25cm, 5.25cm) {3};

    \node at (8.25cm, 4.75cm) {+};
    \node at (8.75cm, 4.75cm) {};
    \node at (9.25cm, 4.75cm) {6};
    \node at (9.75cm, 4.75cm) {9};
    \node at (10.25cm, 4.75cm) {1};
    \node at (10.75cm, 4.75cm) {0};
    \node at (11.25cm, 4.75cm) {4};

    \node at (8.25cm, 4.25cm) {+};
    \node at (8.75cm, 4.25cm) {};
    \node at (9.25cm, 4.25cm) {4};
    \node at (9.75cm, 4.25cm) {4};
    \node at (10.25cm, 4.25cm) {0};
    \node at (10.75cm, 4.25cm) {2};
    \node at (11.25cm, 4.25cm) {4};

    \node at (8.25cm, 3.75cm) {+};
    \node at (8.75cm, 3.75cm) {};
    \node at (9.25cm, 3.75cm) {};
    \node at (9.75cm, 3.75cm) {};
    \node at (10.25cm, 3.75cm) {2};
    \node at (10.75cm, 3.75cm) {2};
    \node at (11.25cm, 3.75cm) {0};

    \node at (8.25cm, 3.25cm) {+};
    \node at (8.75cm, 3.25cm) {};
    \node at (9.25cm, 3.25cm) {8};
    \node at (9.75cm, 3.25cm) {2};
    \node at (10.25cm, 3.25cm) {7};
    \node at (10.75cm, 3.25cm) {7};
    \node at (11.25cm, 3.25cm) {1};

    \node at (8.25cm, 2.75cm) {+};
    \node at (8.75cm, 2.75cm) {};
    \node at (9.25cm, 2.75cm) {2};
    \node at (9.75cm, 2.75cm) {3};
    \node at (10.25cm, 2.75cm) {2};
    \node at (10.75cm, 2.75cm) {3};
    \node at (11.25cm, 2.75cm) {9};
    \node at (11.75cm, 2.75cm) {)};

    \draw (8.2cm,2.2cm) -- (11.9cm,2.2cm);

    \node at (8.75cm, 1.75cm) {5};
    \node at (8.8cm, 2.3cm) {\tiny 5};
    \node at (9.25cm, 1.75cm) {3};
    \node at (9.3cm, 2.3cm) {\tiny 8};
    \node at (9.75cm, 1.75cm) {7};
    \node at (9.8cm, 2.3cm) {\tiny 9};
    \node at (10.25cm, 1.75cm) {2};
    \node at (10.3cm, 2.3cm) {\tiny 7};
    \node at (10.75cm, 1.75cm) {9};
    \node at (10.8cm, 2.3cm) {\tiny 7};
    \node at (11.25cm, 1.75cm) {8};

    \node at (8.25cm, 1.25cm) {-};
    \node at (8.75cm, 1.25cm) {};
    \node at (9.25cm, 1.25cm) {8};
    \node at (9.75cm, 1.25cm) {3};
    \node at (10.25cm, 1.25cm) {3};
    \node at (10.75cm, 1.25cm) {3};
    \node at (11.25cm, 1.25cm) {7};

    \draw (8.2cm,0.7cm) -- (11.9cm,0.7cm);

    \node at (8.25cm, 0.25cm) {-};
    \node at (8.75cm, 0.25cm) {4};
    \node at (8.8cm, 0.8cm) {\tiny 1};
    \node at (9.25cm, 0.25cm) {5};
    \node at (9.3cm, 0.8cm) {\tiny };
    \node at (9.75cm, 0.25cm) {3};
    \node at (9.8cm, 0.8cm) {\tiny 1};
    \node at (10.25cm, 0.25cm) {9};
    \node at (10.3cm, 0.8cm) {\tiny };
    \node at (10.75cm, 0.25cm) {6};
    \node at (10.8cm, 0.8cm) {\tiny };
    \node at (11.25cm, 0.25cm) {1};

  \end{tikzpicture}
  \caption{schriftlich subtrahieren}\label{fig:schriftlichSubtrahieren}
\end{figure}

\section{Schriftliche Multiplikation}\index{Multiplikation!schriftliche}

\begin{figure}
  \centering
  \begin{tikzpicture}
    %\draw[step=1mm, line width=0.1mm, black!30!white] (0,0) grid (\width,\hauteur);
    \draw[step=5mm, line width=0.1mm, black!40!white] (0,0) grid (\width,\hauteur);
    %\draw[step=5cm, line width=0.5mm, black!50!white] (0,0) grid (\width,\hauteur);
    %\draw[step=1cm, line width=0.3mm, black!90!white] (0,0) grid (\width,\hauteur);

    \node at (1.25cm, 11.25cm) {1};
    \node at (1.75cm, 11.25cm) {2};
    \node at (2.25cm, 11.25cm) {3};
    \node at (2.75cm, 11.25cm) {4};
    \node at (3.25cm, 11.25cm) {5};
    \node at (3.75cm, 11.25cm) {$\cdot$};
    \node at (4.25cm, 11.25cm) {6};
    \node at (4.75cm, 11.25cm) {7};
    \node at (5.25cm, 11.25cm) {8};
    \node at (5.75cm, 11.25cm) {9};
    \node at (6.25cm, 11.25cm) {=};
    \node at (6.75cm, 11.25cm) {8};
    \node at (7.25cm, 11.25cm) {3};
    \node at (7.75cm, 11.25cm) {8};
    \node at (8.25cm, 11.25cm) {1};
    \node at (8.75cm, 11.25cm) {0};
    \node at (9.25cm, 11.25cm) {2};
    \node at (9.75cm, 11.25cm) {0};
    \node at (10.25cm, 11.25cm) {5};

    \draw (0.2cm,10.75cm) -- (6.2cm,10.75cm);
    \node at (1.25cm, 9.75cm) {+};
    \node at (1.25cm, 9.25cm) {+};
    \node at (1.25cm, 8.75cm) {+};

    \node at (2.25cm, 10.25cm) {7};
    \node at (2.75cm, 10.25cm) {4};
    \node at (3.25cm, 10.25cm) {0};
    \node at (3.75cm, 10.25cm) {7};
    \node at (4.25cm, 10.25cm) {0};

    \node at (2.75cm, 9.75cm) {8};
    \node at (3.25cm, 9.75cm) {6};
    \node at (3.75cm, 9.75cm) {4};
    \node at (4.25cm, 9.75cm) {1};
    \node at (4.75cm, 9.75cm) {5};

    \node at (3.25cm, 9.25cm) {9};
    \node at (3.75cm, 9.25cm) {8};
    \node at (4.25cm, 9.25cm) {7};
    \node at (4.75cm, 9.25cm) {6};
    \node at (5.25cm, 9.25cm) {0};

    \node at (3.25cm, 8.75cm) {1};
    \node at (3.75cm, 8.75cm) {1};
    \node at (4.25cm, 8.75cm) {1};
    \node at (4.75cm, 8.75cm) {1};
    \node at (5.25cm, 8.75cm) {0};
    \node at (5.75cm, 8.75cm) {5};

    \draw (0.2cm,8.2cm) -- (6.2cm,8.2cm);

    \node at (2.35cm, 8.3cm) {\tiny 1};
    \node at (2.25cm, 7.75cm) {8};
    \node at (2.85cm, 8.3cm) {\tiny 1};
    \node at (2.75cm, 7.75cm) {3};
    \node at (3.35cm, 8.3cm) {\tiny 2};
    \node at (3.25cm, 7.75cm) {8};
    \node at (3.85cm, 8.3cm) {\tiny 1};
    \node at (3.75cm, 7.75cm) {1};
    \node at (4.35cm, 8.3cm) {\tiny 1};
    \node at (4.25cm, 7.75cm) {0};
    \node at (4.75cm, 7.75cm) {2};
    \node at (5.25cm, 7.75cm) {0};
    \node at (5.75cm, 7.75cm) {5};

  \end{tikzpicture}
  \caption{schriftlich multiplizieren}\label{fig:schriftlichMultiplizieren}
\end{figure}


\section{Schriftliche Division}\index{Division!schriftliche}

\begin{figure}
  \centering
  \begin{tikzpicture}
    %\draw[step=1mm, line width=0.1mm, black!30!white] (0,0) grid (\width,\hauteur);
    \draw[step=5mm, line width=0.1mm, black!40!white] (0,0) grid (\width,\hauteur);
    %\draw[step=5cm, line width=0.5mm, black!50!white] (0,0) grid (\width,\hauteur);
    %\draw[step=1cm, line width=0.3mm, black!90!white] (0,0) grid (\width,\hauteur);

    \node at (1.25cm, 11.25cm) {1};
    \node at (1.75cm, 11.25cm) {:};
    \node at (2.25cm, 11.25cm) {2};
    \node at (2.75cm, 11.25cm) {=};
    \node at (3.25cm, 11.25cm) {0};
    \node at (3.75cm, 11.10cm) {,};
    \node at (4.25cm, 11.25cm) {5};

    \node at (0.75cm, 10.75cm) {-};
    \node at (1.25cm, 10.75cm) {0};

    \draw (0.7cm,10.25cm) -- (1.8cm,10.25cm);
    \node at (1.25cm, 9.75cm) {1};
    \node at (1.75cm, 9.75cm) {0};

    \node at (0.75cm, 9.25cm) {-};
    \node at (1.25cm, 9.25cm) {1};
    \node at (1.75cm, 9.25cm) {0};

    \draw (0.7cm,8.75cm) -- (2.3cm,8.75cm);
    \node at (1.75cm, 8.25cm) {0};


    \node at (6.25cm, 11.25cm) {1};
    \node at (6.75cm, 11.25cm) {:};
    \node at (7.25cm, 11.25cm) {1};
    \node at (7.75cm, 11.25cm) {6};
    \node at (8.25cm, 11.25cm) {=};
    \node at (8.75cm, 11.25cm) {0};
    \node at (9.25cm, 11.10cm) {,};
    \node at (9.75cm, 11.25cm) {0};
    \node at (10.25cm, 11.25cm) {6};
    \node at (10.75cm, 11.25cm) {2};
    \node at (11.25cm, 11.25cm) {5};

    \node at (5.75cm, 10.75cm) {-};
    \node at (6.25cm, 10.75cm) {0};
    \draw (5.75cm,10.25cm) -- (6.75cm,10.25cm);

    \node at (6.25cm, 9.75cm) {1};
    \node at (6.75cm, 9.75cm) {0};
    \node at (5.75cm, 9.25cm) {-};
    \node at (6.75cm, 9.25cm) {0};
    \draw (5.75cm,8.75cm) -- (7.25cm,8.75cm);
    \node at (6.25cm, 8.25cm) {1};
    \node at (6.75cm, 8.25cm) {0};
    \node at (7.25cm, 8.25cm) {0};

    \node at (5.75cm, 7.75cm) {-};
    \node at (6.75cm, 7.75cm) {9};
    \node at (7.25cm, 7.75cm) {6};
    \draw (5.75cm,7.25cm) -- (7.75cm,7.25cm);
    \node at (7.25cm, 6.75cm) {4};
    \node at (7.75cm, 6.75cm) {0};

    \node at (6.25cm, 6.25cm) {-};
    \node at (7.25cm, 6.25cm) {3};
    \node at (7.75cm, 6.25cm) {2};
    \draw (6.75cm,5.75cm) -- (8.25cm,5.75cm);
    \node at (7.75cm, 5.25cm) {8};
    \node at (8.25cm, 5.25cm) {0};

    \node at (7.25cm, 4.75cm) {-};
    \node at (7.75cm, 4.75cm) {8};
    \node at (8.25cm, 4.75cm) {0};
    \draw (6.75cm,4.25cm) -- (8.75cm,4.25cm);
    \node at (8.25cm, 3.75cm) {0};

  \end{tikzpicture}
  \caption{schriftlich dividieren}\label{fig:schriftlichDividieren}
\end{figure}



\chapter{Bruchrechnung}\index{Bruchrechnung}

\section{Motivation}

Mit Brüchen zu rechnen vereinfacht das Leben mit Mathematik sehr. Natürliche Zahlen sind uns eben natürlich. Ein Ganzes, ein Halbes oder ein Drittel von etwas, einer Pizza z.B., ist uns unmittelbar verständlich. $33,\overline{3}\%$ sind es nicht und machen erst durch den Gedanken \enquote{$33,\overline{3}$\% \textbf{ist} ein Drittel} gedanklich Sinn. Brüche sind einfach zu begreifen, Dezimalentwicklungen nicht. Unter $0,\overline{142857}$ kann sich Niemand etwas vorstellen, der nicht weiß, dass das $\frac{1}{7}$ ist.


\section{Grundlagen}\index{Bruchrechnung!Grundlagen}\label{Bruchrechnung}

Wir sprechen jeweils über die grauen Teile/ Flächen relativ zur gesamten Fläche, dem ganzen Kreis.

\begin{longtable}{|m{0.3\linewidth}|m{0.6\linewidth}|}
\hline
$
    \begin{tikzpicture}
        \filldraw[fill=gray!20] circle(0.75cm);
    \end{tikzpicture}
$
& Das ist eine Pizza. Das ist eine ganze Pizza. Das ist ein Ganzes. Das ist ein Stück einer Pizza, die aus einem Stück besteht. Es ist 1. Es ist $\frac{1}{1}$.\\
\hline
$
    \begin{tikzpicture}[scale=0.5, .style={fontsize=\footnotesize}]
            \filldraw[fill=gray!20] (0,0) circle(0.75cm);
            \draw (1.1,0) node{+};
            \filldraw[fill=gray!20] (2.2,0) circle(0.75cm);
            \draw (3.3,0) node{=};
            \filldraw[fill=gray!20] (4.4,0) circle(0.75cm);
            \filldraw[fill=gray!20] (5.9,0) circle(0.75cm);
    \end{tikzpicture}
$
& Das sind zwei Ganze (Kreise, Pizzen, Dinge, ...). Ein Kreis war $\frac{1}{1}$. Zwei sind $\frac{1}{1}+\frac{1}{1}=\frac{2}{1} = 2$ oder $2 \cdot \frac{1}{1} = \frac{2}{1} = 2$.\\
\hline
$
    \begin{tikzpicture}
            \filldraw[fill=gray!20] (0,0) circle(0.75cm);
            \draw (0, -0.75) -- (0,0.75);
    \end{tikzpicture}
$
& Das ist immer noch ein Kreis. Dass er durchgeschnitten ist ändert das nicht. Es sind zwei (gleich große) Teile eines Kreises aus zwei Teilen. Das sind $\frac{2}{2}$. Also sind $\frac{2}{2} = \frac{1}{1} = 1$.\\
\hline
$
    \begin{tikzpicture}[radius=7.5mm, delta angle=120]
        \filldraw[fill=black!10!white, draw=black!70!white, rotate=90]
            (0,0) -- (7.5mm, 0) arc (0:180:7.5mm) -- cycle;
        \filldraw[fill=white, draw=black!70!white, rotate=270]
            (0,0) -- (7.5mm, 0) arc (0:180:7.5mm) -- cycle;
    \end{tikzpicture}
$
& Das ist immer noch ein Kreis. Dass er durchgeschnitten ist ändert das nicht. Es ist ein Teil von zwei (gleich großen) Teilen eines Kreises aus zwei Teilen. Das sind $\frac{1}{2}$.\\
\hline
$
    \begin{tikzpicture}[radius=7.5mm, delta angle=120]
        \filldraw[fill=black!10!white, draw=black!70!white]
            (0,0) -- (7.5mm, 0) arc (0:120:7.5mm) -- cycle;
        \filldraw[fill=white, draw=black!70!white, rotate=120]
            (0,0) -- (7.5mm, 0) arc (0:120:7.5mm) -- cycle;
        \filldraw[fill=white, draw=black!70!white, rotate=240]
            (0,0) -- (7.5mm, 0) arc (0:120:7.5mm) -- cycle;
    \end{tikzpicture}
$
& Das ist immer noch ein Kreis. Dass er durchgeschnitten ist ändert das nicht. Es ist ein Teil von drei (gleich großen) Teile eines Kreises aus drei Teilen. Das sind $\frac{1}{3}$.\\\hline
$
    \begin{tikzpicture}[radius=7.5mm, delta angle=120]
        \filldraw[fill=black!10!white, draw=black!70!white]
            (0,0) -- (7.5mm, 0) arc (0:120:7.5mm) -- cycle;
        \filldraw[fill=black!10!white, draw=black!70!white, rotate=120]
            (0,0) -- (7.5mm, 0) arc (0:120:7.5mm) -- cycle;
        \filldraw[fill=white, draw=black!70!white, rotate=240]
            (0,0) -- (7.5mm, 0) arc (0:120:7.5mm) -- cycle;
    \end{tikzpicture}
$
& Das ist immer noch ein Kreis. Dass er durchgeschnitten ist ändert das nicht. Es sind zwei (gleich große) Teile von drei (gleich großen) Teilen eines Kreises aus drei Teilen. Das sind $\frac{2}{3}$.\\\hline
$
    \begin{tikzpicture}[baseline=-1mm,scale=0.5, .style={fontsize=\footnotesize}]
        \filldraw[fill=black!10!white, draw=black!70!white]
            (0,0) -- (7.5mm, 0) arc (0:120:7.5mm) -- cycle;
        \filldraw[fill=white, draw=black!70!white, rotate=120]
            (0,0) -- (7.5mm, 0) arc (0:120:7.5mm) -- cycle;
        \filldraw[fill=white, draw=black!70!white, rotate=240]
            (0,0) -- (7.5mm, 0) arc (0:120:7.5mm) -- cycle;
    \end{tikzpicture}
    +
    \begin{tikzpicture}[baseline=-1mm,scale=0.5, .style={fontsize=\footnotesize}]
        \filldraw[fill=white, draw=black!70!white]
            (0,0) -- (7.5mm, 0) arc (0:120:7.5mm) -- cycle;
        \filldraw[fill=black!10!white, draw=black!70!white, rotate=120]
            (0,0) -- (7.5mm, 0) arc (0:120:7.5mm) -- cycle;
        \filldraw[fill=black!10!white, draw=black!70!white, rotate=240]
            (0,0) -- (7.5mm, 0) arc (0:120:7.5mm) -- cycle;
    \end{tikzpicture}
    =
    \begin{tikzpicture}[baseline=-1mm,scale=0.5, .style={fontsize=\footnotesize}]
        \filldraw[fill=black!10!white, draw=black!70!white]
            (0,0) -- (7.5mm, 0) arc (0:120:7.5mm) -- cycle;
        \filldraw[fill=black!10!white, draw=black!70!white, rotate=120]
            (0,0) -- (7.5mm, 0) arc (0:120:7.5mm) -- cycle;
        \filldraw[fill=black!10!white, draw=black!70!white, rotate=240]
            (0,0) -- (7.5mm, 0) arc (0:120:7.5mm) -- cycle;
    \end{tikzpicture}
$
& S.o. $\frac{1}{3} + \frac{2}{3} = \frac{3}{3} = 1$. Gesprochen wir dies als \enquote{ein Drittel plus zwei Drittel gleich drei Drittel gleich eins}.\\\hline
$
    \begin{tikzpicture}[baseline=-1mm,scale=0.5, .style={fontsize=\footnotesize}]
        \filldraw[fill=black!10!white, draw=black!70!white]
            (0,0) -- (7.5mm, 0) arc (0:120:7.5mm) -- cycle;
        \filldraw[fill=white, draw=black!70!white, rotate=120]
            (0,0) -- (7.5mm, 0) arc (0:120:7.5mm) -- cycle;
        \filldraw[fill=white, draw=black!70!white, rotate=240]
            (0,0) -- (7.5mm, 0) arc (0:120:7.5mm) -- cycle;
    \end{tikzpicture}
    +
    \begin{tikzpicture}[baseline=-1mm,scale=0.5, .style={fontsize=\footnotesize}]
        \filldraw[fill=black!10!white, draw=black!70!white, rotate=120]
            (0,0) -- (7.5mm, 0) arc (0:60:7.5mm) -- cycle;
        \filldraw[fill=black!10!white, draw=black!70!white, rotate=180]
            (0,0) -- (7.5mm, 0) arc (0:60:7.5mm) -- cycle;
        \filldraw[fill=white, draw=black!70!white]
            (0,0) -- (7.5mm, 0) arc (0:120:7.5mm) -- cycle;
        \filldraw[fill=white, draw=black!70!white, rotate=240]
            (0,0) -- (7.5mm, 0) arc (0:60:7.5mm) -- cycle;
        \filldraw[fill=white, draw=black!70!white, rotate=60]
            (0,0) -- (7.5mm, 0) arc (0:60:7.5mm) -- cycle;
        \filldraw[fill=white, draw=black!70!white, rotate=300]
            (0,0) -- (7.5mm, 0) arc (0:60:7.5mm) -- cycle;
    \end{tikzpicture}
    =
    \begin{tikzpicture}[baseline=-1mm,scale=0.5, .style={fontsize=\footnotesize}]
        \filldraw[fill=black!10!white, draw=black!70!white, rotate=120]
            (0,0) -- (7.5mm, 0) arc (0:60:7.5mm) -- cycle;
        \filldraw[fill=black!10!white, draw=black!70!white, rotate=180]
            (0,0) -- (7.5mm, 0) arc (0:60:7.5mm) -- cycle;
        \filldraw[fill=black!10!white, draw=black!70!white]
            (0,0) -- (7.5mm, 0) arc (0:120:7.5mm) -- cycle;
        \filldraw[fill=white, draw=black!70!white, rotate=240]
            (0,0) -- (7.5mm, 0) arc (0:120:7.5mm) -- cycle;
    \end{tikzpicture}
$ & $\frac{1}{3} + \frac{2}{6} = \frac{2}{3}$ und $\frac{2}{6} = \frac{1}{3}$\\\hline
\end{longtable}

Die Zahl über dem Bruchstrich nennen wir \enquote{\textbf{Zähler}}, die Zahl unter dem Bruchstrich \enquote{\textbf{Nenner}}. Wir lesen Brüche als $\frac{1}{2}$\enquote{ein Halb} (oder \enquote{ein Halbes}), $\frac{1}{3}$ als \enquote{ein Drittel}, $\frac{1}{4}$ als \enquote{ein Viertel}, $\frac{1}{12}$ als \enquote{ein Zwölftel}, $\frac{13}{1000}$ als \enquote{dreizehn Tausendstel}. Wird der Bruch nicht als Nomen (der Zahl) verwendet, sondern als adverbiale Bestimmung der Anzahl eines benannten Etwas, wie in \enquote{dreiviertel Pizza}, dann wird klein und mit den unter Numeralia (s. \ref{numeralia}) Regeln zusammen oder getrennt geschrieben. Da wir solche Zahlen stets mit Ziffern schreiben, und beim Lesen/ Sprechen Groß- und Kleinschreibung und Trennungen wenig interessant sind, ist das nicht besonders wichtig. Wir machen einen großen Sprung und geben die Rechenregeln der Bruchrechnung an. Keine Angst, wir kehren nach Angabe einer Regel jeweils zu Beispielen zurück und nehmen uns alle Zeit der Welt um an diesen wichtigen Stellen niemanden zu verlieren.

\section{Rechenregeln}

Wir \textbf{addieren} Brüche indem wir sie auf gleiche \textbf{Nenner} bringen und dann die angepassten \textbf{Zähler} \textbf{addieren}. Den kleinsten gemeinsamen \textbf{Nenner} finden wir indem wir das \textbf{kleinste gemeinsame Vielfache (KGV)} der \textbf{Nenner} bestimmen. Das \textbf{KGV} finden wir indem wir die \textbf{Primfaktorzerlegungen} der \textbf{Nenner} \glsdisp{symb:Vereinigung}{vereinigen}\footnote{Wir vereinfachen hier etwas sprachlich, denn die Vereinigung von Mengen ist es nur dann genau wenn wir die Faktoren $m^n$ für verschiedene $n$ aufführen, also z.B. $Pfz(4)=\{2^1, 2^2\}$}.

Wir dürfen nur Brüche mit gleichem \textbf{Nenner} (direkt) \textbf{addieren} oder \textbf{subtrahieren}. Ansonsten müssen wir zuerst alle \textbf{Summanden} auf den gleichen \textbf{Nenner} bringen (\enquote{erweitern}).

\begin{equation}
    \frac{a}{c} + \frac{b}{c} = \frac{a+b}{c}
\end{equation}

\begin{figure}
  \centering
  \begin{tikzpicture}
    %\draw[step=1mm, line width=0.1mm, black!30!white] (0,0) grid (\width,\hauteur);
    \draw[step=5mm, line width=0.1mm, black!40!white] (0,0) grid (\width,\hauteur);
    %\draw[step=5cm, line width=0.5mm, black!50!white] (0,0) grid (\width,\hauteur);
    %\draw[step=1cm, line width=0.3mm, black!90!white] (0,0) grid (\width,\hauteur);

    \node at (1.25cm, 11.25cm) {1};
    \draw (0.75cm, 10.75) -- (1.75cm, 10.75);
    \node at (1.25cm, 10.25cm) {2};

    \node at (2.25cm, 10.75cm) {+};

    \node at (3.25cm, 11.25cm) {1};
    \draw (2.75cm, 10.75) -- (3.75cm, 10.75);
    \node at (3.25cm, 10.25cm) {3};

    \node at (4.25cm, 10.75cm) {=};

    \node at (4.75cm, 10.75cm) {?};


    \node at (5.75cm, 10.75cm) {K};
    \node at (6.25cm, 10.75cm) {G};
    \node at (6.75cm, 10.75cm) {V};
    \node at (7.25cm, 10.75cm) {(};
    \node at (7.75cm, 10.75cm) {2};
    \node at (8.25cm, 10.75cm) {;};
    \node at (8.75cm, 10.75cm) {3};
    \node at (9.25cm, 10.75cm) {)};
    \node at (9.75cm, 10.75cm) {=};
    \node at (10.25cm, 10.75cm) {?};

    \node at (5.75cm, 9.75cm) {P};
    \node at (6.25cm, 9.75cm) {F};
    \node at (6.75cm, 9.75cm) {Z};
    \node at (7.25cm, 9.75cm) {(};
    \node at (7.75cm, 9.75cm) {2};
    \node at (8.25cm, 9.75cm) {)};
    \node at (8.75cm, 9.75cm) {=};
    \node at (9.25cm, 9.75cm) {\{};
    \node at (9.75cm, 9.75cm) {2};
    \node at (10.25cm, 9.75cm) {\}};

    \node at (5.75cm, 9.25cm) {P};
    \node at (6.25cm, 9.25cm) {F};
    \node at (6.75cm, 9.25cm) {Z};
    \node at (7.25cm, 9.25cm) {(};
    \node at (7.75cm, 9.25cm) {3};
    \node at (8.25cm, 9.25cm) {)};
    \node at (8.75cm, 9.25cm) {=};
    \node at (9.25cm, 9.25cm) {\{};
    \node at (9.75cm, 9.25cm) {3};
    \node at (10.25cm, 9.25cm) {\}};

    \node at (1.25cm, 8.25cm) {P};
    \node at (1.75cm, 8.25cm) {F};
    \node at (2.25cm, 8.25cm) {Z};
    \node at (2.75cm, 8.25cm) {(};
    \node at (3.25cm, 8.25cm) {2};
    \node at (3.75cm, 8.25cm) {)};
    \node at (4.25cm, 8.25cm) {$\bigcup$};
    \node at (4.75cm, 8.25cm) {P};
    \node at (5.25cm, 8.25cm) {F};
    \node at (5.75cm, 8.25cm) {Z};
    \node at (6.25cm, 8.25cm) {(};
    \node at (6.75cm, 8.25cm) {3};
    \node at (7.25cm, 8.25cm) {)};
    \node at (7.75cm, 8.25cm) {=};
    \node at (8.25cm, 8.25cm) {\{};
    \node at (8.75cm, 8.25cm) {2};
    \node at (9.25cm, 8.25cm) {;};
    \node at (9.75cm, 8.25cm) {3};
    \node at (10.25cm, 8.25cm) {\}};

    \node at (1.25cm, 7.25cm) {$\Rightarrow$};
    \node at (1.75cm, 7.25cm) {K};
    \node at (2.25cm, 7.25cm) {G};
    \node at (2.75cm, 7.25cm) {V};
    \node at (3.25cm, 7.25cm) {(};
    \node at (3.75cm, 7.25cm) {2};
    \node at (4.25cm, 7.25cm) {;};
    \node at (4.75cm, 7.25cm) {3};
    \node at (5.25cm, 7.25cm) {)};
    \node at (5.75cm, 7.25cm) {=};
    \node at (6.25cm, 7.25cm) {2};
    \node at (6.75cm, 7.25cm) {$\cdot$};
    \node at (7.25cm, 7.25cm) {3};
    \node at (7.75cm, 7.25cm) {=};
    \node at (8.25cm, 7.25cm) {6};

    \node at (1.25cm, 6.25cm) {1};
    \draw (0.75cm, 5.75) -- (1.75cm, 5.75);
    \node at (1.25cm, 5.25cm) {2};

    \node at (2.25cm, 5.75cm) {+};

    \node at (3.25cm, 6.25cm) {1};
    \draw (2.75cm, 5.75) -- (3.75cm, 5.75);
    \node at (3.25cm, 5.25cm) {3};

    \node at (4.25cm, 5.75cm) {=};

    \node at (5.25cm, 6.25cm) {3};
    \draw (4.75cm, 5.75) -- (5.75cm, 5.75);
    \node at (5.25cm, 5.25cm) {6};

    \node at (6.25cm, 5.75cm) {+};

    \node at (7.25cm, 6.25cm) {2};
    \draw (6.75cm, 5.75) -- (7.75cm, 5.75);
    \node at (7.25cm, 5.25cm) {6};

    \node at (8.25cm, 5.75cm) {=};

    \node at (9.25cm, 6.25cm) {5};
    \draw (8.75cm, 5.75) -- (9.75cm, 5.75);
    \node at (9.25cm, 5.25cm) {6};
  \end{tikzpicture}
  \caption{Brüche addieren}\label{fig:BruecheAddieren}
\end{figure}

\begin{figure}
  \centering
  \begin{tikzpicture}
    %\draw[step=1mm, line width=0.1mm, black!30!white] (0,0) grid (\width,\hauteur);
    \draw[step=5mm, line width=0.1mm, black!40!white] (0,0) grid (\width,\hauteur);
    %\draw[step=5cm, line width=0.5mm, black!50!white] (0,0) grid (\width,\hauteur);
    %\draw[step=1cm, line width=0.3mm, black!90!white] (0,0) grid (\width,\hauteur);

    \node at (0.75cm, 11.25cm) {};
    \node at (1.25cm, 11.25cm) {3};
    \draw (0.25cm, 10.75) -- (1.75cm, 10.75);
    \node at (0.75cm, 10.25cm) {1};
    \node at (1.25cm, 10.25cm) {4};

    \node at (2.25cm, 10.75cm) {+};

    \node at (2.75cm, 11.25cm) {};
    \node at (3.75cm, 11.25cm) {5};
    \draw (2.75cm, 10.75) -- (4.25cm, 10.75);
    \node at (3.25cm, 10.25cm) {1};
    \node at (3.75cm, 10.25cm) {2};

    \node at (4.75cm, 10.75cm) {=};

    \node at (5.25cm, 10.75cm) {?};


    \node at (9.75cm, 10.75cm) {K};
    \node at (10.25cm, 10.75cm) {G};
    \node at (10.75cm, 10.75cm) {V};
    \node at (11.25cm, 10.75cm) {=};
    \node at (11.75cm, 10.75cm) {?};

    \node at (0.25cm, 9.25cm) {P};
    \node at (0.75cm, 9.25cm) {F};
    \node at (1.25cm, 9.25cm) {Z};
    \node at (1.75cm, 9.25cm) {(};
    \node at (2.25cm, 9.25cm) {1};
    \node at (2.75cm, 9.25cm) {4};
    \node at (3.25cm, 9.25cm) {)};
    \node at (3.75cm, 9.25cm) {=};
    \node at (4.25cm, 9.25cm) {\{};
    \node at (4.75cm, 9.25cm) {2};
    \node at (5.25cm, 9.25cm) {;};
    \node at (5.75cm, 9.25cm) {7};
    \node at (6.25cm, 9.25cm) {\}};

    \node at (0.25cm, 8.75cm) {P};
    \node at (0.75cm, 8.75cm) {F};
    \node at (1.25cm, 8.75cm) {Z};
    \node at (1.75cm, 8.75cm) {(};
    \node at (2.25cm, 8.75cm) {1};
    \node at (2.75cm, 8.75cm) {2};
    \node at (3.25cm, 8.75cm) {)};
    \node at (3.75cm, 8.75cm) {=};
    \node at (4.25cm, 8.75cm) {\{};
    \node at (4.75cm, 8.75cm) {2};
    \node at (5.25cm, 8.75cm) {;};
    \node at (5.75cm, 8.75cm) {2};
    \node at (6.25cm, 8.75cm) {;};
    \node at (6.75cm, 8.75cm) {3};
    \node at (7.25cm, 8.75cm) {\}};


    \node at (0.25cm, 8.25cm) {P};
    \node at (0.75cm, 8.25cm) {F};
    \node at (1.25cm, 8.25cm) {Z};
    \node at (1.75cm, 8.25cm) {(};
    \node at (2.25cm, 8.25cm) {1};
    \node at (2.75cm, 8.25cm) {4};
    \node at (3.25cm, 8.25cm) {)};
    \node at (3.75cm, 8.25cm) {$\bigcup$};
    \node at (4.25cm, 8.25cm) {P};
    \node at (4.75cm, 8.25cm) {F};
    \node at (5.25cm, 8.25cm) {Z};
    \node at (5.75cm, 8.25cm) {(};
    \node at (6.25cm, 8.25cm) {1};
    \node at (6.75cm, 8.25cm) {2};
    \node at (7.25cm, 8.25cm) {)};
    \node at (0.75cm, 7.75cm) {=};
    \node at (1.25cm, 7.75cm) {\{};
    \node at (1.75cm, 7.75cm) {2};
    \node at (2.25cm, 7.75cm) {;};
    \node at (2.75cm, 7.75cm) {2};
    \node at (3.25cm, 7.75cm) {;};
    \node at (3.75cm, 7.75cm) {3};
    \node at (4.25cm, 7.75cm) {;};
    \node at (4.75cm, 7.75cm) {7};
    \node at (5.25cm, 7.75cm) {\}};

    \node at (0.25cm, 7.25cm) {K};
    \node at (0.75cm, 7.25cm) {G};
    \node at (1.25cm, 7.25cm) {V};
    \node at (1.75cm, 7.25cm) {=};
    \node at (2.25cm, 7.25cm) {2};
    \node at (2.75cm, 7.25cm) {$\cdot$};
    \node at (3.25cm, 7.25cm) {2};
    \node at (3.75cm, 7.25cm) {$\cdot$};
    \node at (4.25cm, 7.25cm) {3};
    \node at (4.75cm, 7.25cm) {$\cdot$};
    \node at (5.25cm, 7.25cm) {7};
    \node at (5.75cm, 7.25cm) {=};
    \node at (6.25cm, 7.25cm) {8};
    \node at (6.75cm, 7.25cm) {4};

    \node at (0.75cm, 6.25cm) {1};
    \node at (1.25cm, 6.25cm) {8};
    \draw (0.25cm, 5.75) -- (1.75cm, 5.75);
    \node at (0.75cm, 5.25cm) {8};
    \node at (1.25cm, 5.25cm) {4};

    \node at (2.25cm, 5.75cm) {+};

    \node at (3.25cm, 6.25cm) {3};
    \node at (3.75cm, 6.25cm) {5};
    \draw (2.75cm, 5.75) -- (4.25cm, 5.75);
    \node at (3.25cm, 5.25cm) {8};
    \node at (3.75cm, 5.25cm) {4};

    \node at (4.75cm, 5.75cm) {=};

    \node at (5.75cm, 6.25cm) {5};
    \node at (6.25cm, 6.25cm) {3};
    \draw (5.25cm, 5.75) -- (6.75cm, 5.75);
    \node at (5.75cm, 5.25cm) {8};
    \node at (6.25cm, 5.25cm) {4};
  \end{tikzpicture}
  \caption{Brüche addieren}\label{fig:BruecheAddieren}
\end{figure}


\begin{equation}
    \frac{a}{c} - \frac{b}{c} = \frac{a-b}{c}
\end{equation}

\begin{equation}
    \frac{a}{c} \cdot \frac{b}{d} = \frac{a b}{c d}
\end{equation}

Wir dividieren durch Brüche wir indem wir mit dem \textbf{Kehrwert} multiplizieren.

\begin{equation}\label{dividieren durch multiplizieren mit Kehrwert}
    \frac{a}{c} \div \frac{b}{d} = \frac{a}{c} \cdot \frac{d}{b} = \frac{a d}{b c}
\end{equation}

Brüche deren \textbf{Zähler} und \textbf{Nenner} einen gemeinsamen \textbf{Teiler} haben kann man \textbf{kürzen}.

\begin{equation}
    \frac{n a}{n b} = \frac{a}{b}
\end{equation}

Wir sagen \enquote{$\frac{n a}{n b}$ lässt sich mit $n$ \textbf{kürzen} und ist als \textbf{gekürzter Bruch} $\frac{a}{b}$}.

\begin{beispiel}
    Der Bruch $r=\frac{2}{6}$ lässt sich \textbf{kürzen}, denn $2$ und $6$ haben den \textbf{gemeinsamen Teiler} $2$. $\frac{2}{6} = \frac{1}{3}$.\topicend
\end{beispiel}

Brüche deren \textbf{Zähler} größer als ihr \textbf{Nenner} ist können als \textbf{gemischter Bruch} geschrieben werden.

\begin{equation}\glsadd{symb:Abrunden}
    \frac{a}{b} = \left\lfloor\frac{a}{b}\right\rfloor \frac{a-b \left\lfloor\frac{a}{b}\right\rfloor}{b}\footnote{$\left\lfloor \frac{a}{b} \right\rfloor$ bedeutet das auf die nächst kleinere ganze Zahl \textbf{abgerundete} Ergebnis von  $a \div b$. $\frac{5}{2} = 2,5$. $\left\lfloor 2,5 \right\rfloor = 2$. Also $\left\lfloor\frac{5}{2} \right\rfloor = 2$.}
\end{equation}

\begin{beispiel}
    $\frac{13}{2} = 6 \frac{1}{2}$
\end{beispiel}


\section{Prozentrechnung}\label{Prozentrechnung}\index{Prozentrechnung}

Wir betrachten \enquote{Prozentrechnung} nicht als eigenes (Haupt-)Thema. \enquote{Prozentrechnung} ist einfach Bruchrechnung mit Hundertsteln.

\begin{eqnarray}
% \nonumber to remove numbering (before each equation)
  \text{\textbf{Grundwert}} (GW) \cdot \text{\textbf{Prozentsatz}} (PS) &=& \text{\textbf{Prozentwert}} (PW)\\
  PS &=& \frac{PW}{GW} \\
  GW &=& \frac{PW}{PS}
\end{eqnarray}

Beim Rechnen mit Geld sind \textbf{Kapital} und \textbf{Guthaben}\footnote{sowie \textbf{Schulden} als negatives \textbf{Guthaben}} Synonyme für \textbf{Grundwert}, \textbf{Zinssatz} für \textbf{Prozentsatz} und \textbf{Zinsen} für \textbf{Prozentwert}. Außerdem ist \textbf{Ist} synonym mit positivem \textbf{Guthaben} und \textbf{Soll} mit negativem \textbf{Guthaben} (\textbf{Schulden}).

\begin{beispiel}
    Zu bestimmen sei wie viel Gramm die 23 Prozent Zucker eines 35 Gramm Schokolandenriegels sind.
    \begin{eqnarray}
    % \nonumber to remove numbering (before each equation)
      Prozentwert &=& Grundwert \cdot Prozentsatz \\
       &=& 35g \cdot \frac{7}{100} \\
       &=& \frac{35}{100}g\\
       &=& 0,35g
    \end{eqnarray}\topicend
\end{beispiel}


\chapter{Geometrie}\index{Geometrie}

In der prüfungsrelevanten Geometrie konstruieren und berechnen wir Dreiecke, elementare 2D-Flächen und \textbf{Prismen}\index{Prisma}.


\chapter{Proportionale und Antiproportionale Zuordnungen}\index{proportionale Zuordnung}\index{antiproportionale Zuordnung|see{proportionale Zuordnung}}


\chapter{Term Replacement Systems (TRS)}\label{TRS}\index{TRS}

Um effizient Mathematik zu betreiben ist eine nützliche Sichtweise das Rechnen als System zu verstehen in dem Ausdrücke durch anders geformte Ausdrücke ersetzt werden können, ohne den ursprünglichen Ausdruck (wertmäßig) zu verändern. Insbesondere benutzen wir \textbf{Formeln} und ersetzen die in ihnen enthaltenen allgemeinen \textbf{Variablen} durch konkrete Werte (\textbf{einsetzen}), wodurch wir konkrete Ergebnisse \textbf{ausrechnen} können.


\section{Operatorrangfolge}

Die Regel, die in deutschen (Grundschulen) gelehrt wird lautet \enquote{Punkt- vor Strichrechnung}. D.h. \textbf{Multiplikation} und \textbf{Division} werden vor \textbf{Addition} und \textbf{Subtraktion} ausgeführt. Dieser Regel fehlen unter anderem \textbf{Potenz} und \textbf{Wurzel}.

Eine vollständige Regel lautet: \enquote{Arbeite die List von oben nach unten ab und Operationen mit der selben Priorität/ dem selben Rang von links nach rechts in der Reihenfolge ihres Auftretens im Ausdruck (nicht in der Liste):}.

\begin{enumerate}
  \item \enquote{$(t)$}
  \item \enquote{${t_1}^{t_2}$}, \enquote{$\sqrt[t_2]{t_1}$}
  \item \enquote{$\times$}, \enquote{$\div$}
  \item \enquote{$+$}, \enquote{$-$}
\end{enumerate}

\begin{beispiel}
    \begin{equation}
        2 + 3 \cdot 4 = 2 + (3 \cdot 4) = 2 + 12 = 14
    \end{equation}\topicend
\end{beispiel}


\chapter{Prüfungstaktik}

\section{Vorbereitung}

Der wichtigste Test ob die Vorbereitung für die schriftliche Prüfung in Mathematik erfolgreich abgeschlossen ist, ist das \textit{Stark-Heft} für Mathematik ohne Lösungsheft unter realistischen Bedingungen zu bearbeiten. Wir werden in den Monaten vor der Prüfung alle darin enthaltenen Prüfungen (die tatsächlichen Prüfungen der letzten fünf Jahre) bearbeiten. Um dadurch eine realistische Einschätzung zu bekommen bearbeitet die alten Prüfungen bitte nur wenn die Dozenten Euch das aufgeben - und dann i.d.R. für realistische Bedingungen sorgen. Braucht Ihr Material zum Üben sprecht die Dozenten an. Wir geben Euch dann passendes Material, das aber nicht in eine komplette Prüfung eingebunden ist, die wir später noch für die Prüfungsvorbereitung brauchen.

Noch vor diesen Tests in der eigentlichen Prüfungsvorbereitung ist die beste Vorbereitung (in Mathematik) jede Stunde mitzuschreiben, die Hausaufgaben zu machen, gemeinsam mit anderen Teilnehmern zu lernen und sich die Inhalte gegenseitig zu erklären, und jedes unbekannte Wort in die Vokabelliste (für die Mathematik) aufzunehmen und zu lernen.


\section{Priorisierung der Aufgaben}

In den Monaten vor der Prüfung besprechen wir welche Aufgaben einfach und welche schwierig sind und welche damit leicht zu bekommende Punkte für die Note sind und an welchen man sich besser nicht aufhält, sondern sie macht wenn man mit den einfachen Aufgaben, auf die es viele Punkte gibt (z.B. ein Dreieck zu konstruieren), fertig ist.


\section{Plausibilitätsprüfung}

Die Aufgaben in den Prüfungen sind mit dem Bemühen gestellt plausible Anwendungen der Realität zu modellieren und ihre Beherrschung zu prüfen. Insbesondere sind damit auch die Zahlen i.d.R. so gewählt, dass eine Plausibilitätsprüfung mit gesundem Menschenverstand möglich ist. Errechnet man als Prüfling z.B. für das Volumen eines Stausees einige Liter (nicht einmal tausende), dann kann das nicht stimmen. Mit griffigen Beispielen für Größenordnungen, u.a. dem Liter als Flasche Cola, Packung Milch o.ä., und dem Kubikmeter als immer wieder benutztes (gerne mit Armen visualisierend) konkretes Beispiel (1.000 Liter) und dem Lehren von fortwährender skeptischer Plausibilitätsprüfung des (rechnenden) Tuns kann ein solcher Fehler sicher aufgedeckt werden und Zeit und Einsicht vorausgesetzt vom Prüfling korrigiert werden. Realistische Beispiele von Größenordnungen sollten möglichst von allen Dozenten häufig angeboten werden, insbesondere bietet sich neben der Mathematik selbst der naturwissenschaftliche Unterricht, hier zur Zeit der Geographie, an.

\begin{uebung}
    Wiederhole falls unsicher die Maßeinheiten und finde zu jeder Maßeinheiten für die Bereiche von $\frac{1}{1000}$, $\frac{1}{100}$, $\frac{1}{10}$, eins, zehn, hundert und tausend mal der Einheit ein für Dich gut bildlich vorstellbares Beispiel. Z.B. ein Liter Cola oder Milch und ein tausendstel Meter als Millimeter auf dem Geodreieck.\topicend
\end{uebung}

\begin{uebung}
    Schätze ab: das Volumen des Raums in dem Du Dich gerade befindest, die Masse/ das Gewicht des Gebäudes in dem Du Dich gerade befindest, das Volumen Deines Körpers, die Entfernungen zum Rathaus, nach Berlin, nach New York, zum Mond, zur Sonne, die Länge Deiner Hand, deines Unterarms, der Spanne Deiner beiden ausgestreckten Arme, die Höhe eines Gebäudes, das Du aus dem Fenster sehen kannst, die Dichte von Wasser, Stahl, Holz, einem Menschen, wie viel Liter Wasser der Eder-Stausee fasst, wie viele Liter Wasser Du durchschnittlich pro Tag verbrauchst, die Masse der Erde, ...\topicend
\end{uebung}


\section{Proben rechnen}

\appendix

\chapter{Vokabeln}

Alle diese Vokabeln müssen gelernt werden! Zusätzlich werden noch viele gebraucht, die im Unterricht auftauchen. Diese hier haben perfekt \textbf{verstanden} zu werden (fehlerfreies Schreiben ist für Mathematik nicht notwendig).

die Achse (pl. die Achsen),
die Abbuchung (pl. die Abbuchungen),
abheben,
abrunden,
abziehen,
addieren,
die Addition (pl. die Additionen),
der Algorithmus (pl. die Algorithmen),
antiproportional,
aufrunden,
der Ausdruck (pl. die Ausdrücke),
ausdrücken,
die Auszahlung (pl. die Auszahlungen),
der Beweis (pl. die Beweise),
beweisen,
der Bruch (pl. die Brüche),
die Bruchrechnung (kein pl.),
die Buchung (pl. die Buchungen),
die Differenz (pl. die Differenzen),
die Division (pl. die Divisionen),
dividieren,
die Einheit (pl. die Einheiten),
die Einzahlung (pl. die Einzahlungen),
das Ergebnis (pl. die Ergebnisse),
die Ganze Zahl (pl. die Ganzen Zahlen),
das Gewicht (pl. die Gewichte),
das Gramm (kein pl.),
die Grundrechenart (pl. die Grundrechenarten),
die Größenordnung (pl. die Größenordnungen),
das Guthaben (pl. die Guthaben),
hinreichend,
das Kapital (kein pl.),
das Kilogramm (kein pl.),
die Kommunikation (pl. die Kommunikationen),
konstruieren,
kommunizieren,
der Kreis (pl. die Kreise),
kürzen,
die Masse (pl. die Massen),
das Maß (pl. die Maße),
die Maßeinheit (pl. die Maßeinheiten),
der Maßstab (pl. die Maßstäbe),
maßstabsgetreu,
messen,
die Messung (pl. die Messungen),
die Multiplikation (pl. die Multiplikationen),
der Nenner (pl. die Nenner),
die Notation (pl. die Notationen),
notwendig,
die Ordnung (pl. die Ordnungen),
plausibel,
die Plausibilität (kein Plural),
das Prisma (pl. die Prismen),
die Probe (pl. die Proben),
die Proberechnung (pl. die Proberechnungen),
proportional,
das Prozent (pl. die Prozente),
die Prozentrechnung (kein pl.),
rechnen,
der Quader (pl. die Quader),
der Quotient (pl. die Quotienten),
die Rechnung (pl. die Rechnungen),
das Recht (pl. die Rechte),
runden,
die Schulden (kein Singular),
die Skizze (pl. die Skizzen),
skizzieren,
die Stelle (pl. die Stellen),
der Stellenwert (pl. die Stellenwerte),
das Stellenwertsystem (pl. die Stellenwertsysteme),
die Strecke (pl. die Strecken),
subtrahieren,
die Subtraktion (pl. die Subtraktionen),
der Summand (pl. die Summanden),
die Summe (pl. die Summen),
der Übertrag (pl. die Überträge),
die Voraussetzung (pl. die Voraussetzungen),
die Waage (pl. die Waagen),
der Wert (pl. die Werte),
wiegen,
der Würfel (pl. die Würfel),
zählen,
der Zähler (pl. die Zähler),
die Zahl (pl. die Zahlen),
die Zahlung (pl. die Zahlungen),
zeichnen,
die Zeichnung (pl. die Zeichnungen),
zeigen,
die Ziffer (pl. die Ziffern),
der Zins (pl. die Zinsen),
die Zinsrechnung (kein pl.),
zeigen,
das Zwischenergebnis (pl. die Zwischenergebnisse),
der Zylinder (pl. die Zylinder)


\printglossary
\printglossary[type=symbols,style=long]
\printindex
\printbibliography

\end{document}


