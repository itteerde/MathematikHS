
\documentclass[a4paper]{book}%[a4paper,oneside]
\usepackage[pass]{geometry}%[hmarginratio=1:1]{geometry}
%\usepackage[utf8]{inputenc}

\usepackage{polyglossia}
\setdefaultlanguage[spelling=new]{german}
\setotherlanguage{english}
%\setotherlanguage[numerals=western]{farsi}
\usepackage{fontspec}
\usepackage{xeCJK}
\usepackage{csquotes}

\usepackage[
    backend=biber,
    style=numeric,
    sortlocale=de_DE,
    natbib=true,
    url=false,
    doi=true,
    eprint=false
]{biblatex}
\addbibresource{itteerde.bib}
\bibliography{itteerde}	

\usepackage{makeidx}
\usepackage{mdframed}
\usepackage{graphics}
\usepackage{graphicx}
\usepackage{pictex}
\usepackage{subfig}
\usepackage{float}
\usepackage{array}
\usepackage{xspace}
\usepackage{xcolor}
\usepackage{pdfpages}

\usepackage{longtable}

\usepackage{tikz}
\usetikzlibrary{shapes,backgrounds,arrows,positioning,calc,decorations.markings,matrix,intersections}


\usepackage{verse}

\usepackage{amsmath}
\usepackage{amssymb}
\usepackage{amstext}
\usepackage{amsfonts}
\usepackage{mathrsfs}
\usepackage{amsthm}
\usepackage{mathtools}

\usepackage{chemfig}
\usepackage{listings}
\usepackage{soul}
\usepackage{calc}
\usepackage{metalogo}
\usepackage{hologo}

%\usepackage{stmaryrd}
%\usepackage{marvosym}

\usepackage{url}
\usepackage[position=top]{caption}

\usepackage{etoolbox}
\usepackage{etaremune}

\usepackage[citecolor=black,urlcolor=black,linkcolor=black]{hyperref}
\hypersetup{
    colorlinks=true,
}
\usepackage[xindy={language=german,codepage=duden-utf8},
    nonumberlist,
    toc,
    nopostdot,
    style=altlist,
    nogroupskip
    ]{glossaries}
    \GlsSetXdyCodePage{duden-utf8}
\apptocmd{\thebibliography}{\raggedright}{}{}

\newfontfamily\arabicfont{Amiri}



\def\UrlBreaks{\do\A\do\B\do\C\do\D\do\E\do\F\do\G\do\H\do\I\do\J\do\K
\do\L%
\do\M\do\N\do\O\do\P\do\Q\do\R\do\S\do\T\do\U\do\V\do\W\do\X\do\Y\do\Z
\do\0%
\do\a\do\b\do\c\do\d\do\e\do\f\do\g \do\h\do\i\do\j\do\k\do\l%
\do\m\do\n\do\o\do\p\do\q\do\r\do\s\do\t\do\u\do\v \do\w\do\x\do\y\do\z
%
\do\1\do\2\do\3\do\4\do\5\do\6\do\7\do\8\do\9\do\-}%
\def\UrlBigBreaks{\do\_}

\def\centerarc[#1](#2)(#3:#4:#5)% Syntax: [draw options] (center) (initial angle:final angle:radius)
    { \draw[#1] ($(#2)+({#5*cos(#3)},{#5*sin(#3)})$) arc (#3:#4:#5); }

\newcommand{\uu}[1]{\underline{#1}}
\newcommand{\ii}[1]{\textit{#1}}
\newcommand{\tm}{\textsuperscript{\tiny{TM}}}
\newcommand{\rtm}{\textsuperscript{\tiny{\circledR}}}
\newcommand{\whatsapp}{\textit{WhatsApp}\xspace}

\newcommand{\youtube}{\url{http://www.youtube.com}\xspace}
\newcommand{\tagesschau}{\url{http://www.tagesschau.de}\xspace}
\newcommand{\wikipedia}{\url{http://en.wikipedia.org}\xspace}
\newcommand{\cosmos}{\textit{Cosmos - A Space Time Odyssee}\index{Unterhaltung!Cosmos - A Space Time Odyssee}\xspace}

\newcommand{\vcenteredincludeIcon}[1]{\begingroup
    \setbox0=\hbox{\includegraphics[height=1em]{#1}}%
    \parbox{\wd0}{\box0}\endgroup
}
\newcommand\crule[3][black]{\textcolor{#1}{\rule{#2}{#3}}}

\newcommand{\TikZ}{Ti\textit{k}z\xspace}
\newcommand*\circled[1]{\tikz[baseline=(char.base)]{
            \node[shape=circle,draw,inner sep=2pt] (char) {#1};}}

\newcommand{\emojiSmile}{\vcenteredincludeIcon{graphics/smile.jpg}\xspace}
\newcommand{\emojiSmileT}{\vcenteredincludeIcon{graphics/smile_st.jpg}\xspace}
\newcommand{\emojiSet}{\vcenteredincludeIcon{graphics/smile_set.jpg}\xspace}
\newcommand{\emojiSaint}{\vcenteredincludeIcon{graphics/smile_saint.jpg}\xspace}
\newcommand{\emojiTears}{\vcenteredincludeIcon{graphics/smile_tears.jpg}\xspace}
\newcommand{\emojiGrin}{\vcenteredincludeIcon{graphics/smile_grin.jpg}\xspace}
\newcommand{\emojiWink}{\vcenteredincludeIcon{graphics/smile_wink.jpg}\xspace}
\newcommand{\emojiTOE}{\vcenteredincludeIcon{graphics/smile_toe.jpg}\xspace}
\newcommand{\emojiTES}{\vcenteredincludeIcon{graphics/smile_tes.jpg}\xspace}
\newcommand{\emojiSunglasses}{\vcenteredincludeIcon{graphics/smile_sunglasses.jpg}\xspace}
\newcommand{\emojiBlushLips}{\vcenteredincludeIcon{graphics/smile_blushlips.jpg}\xspace}
\newcommand{\emojiBlushSmile}{\vcenteredincludeIcon{graphics/smile_blushsmile.jpg}\xspace}
\newcommand{\emojiDevil}{\vcenteredincludeIcon{graphics/smile_devil.jpg}\xspace}
\newcommand{\emojiBlusBigEyes}{\vcenteredincludeIcon{graphics/smile_blushbigeyes.jpg}\xspace}
\newcommand{\emojiSmileTeeth}{\vcenteredincludeIcon{graphics/smile_teeth.jpg}\xspace}
\newcommand{\emojiSmileTears}{\vcenteredincludeIcon{graphics/smile_tears.jpg}\xspace}
\newcommand{\emojiSmileKiss}{\vcenteredincludeIcon{graphics/smile_kiss.jpg}\xspace}
\newcommand{\emojiSeeNoEvil}{\vcenteredincludeIcon{graphics/emojiSeeNoEvil.jpg}\xspace}
\newcommand{\emojiSmileWink}{\vcenteredincludeIcon{graphics/smile_wink.jpg}\xspace}
\newcommand{\emojiSmileSad}{\vcenteredincludeIcon{graphics/smile_sad.jpg}\xspace}
\newcommand{\emojiHandPointUp}{\vcenteredincludeIcon{graphics/emoji_hand_point_up.jpg}\xspace}

\newcommand{\emojiBWSmile}{\vcenteredincludeIcon{graphics/smile_bw.jpg}\xspace}
\newcommand{\emojiBWSmileT}{\vcenteredincludeIcon{graphics/smile_bw_st.jpg}\xspace}
\newcommand{\emojiBWSet}{\vcenteredincludeIcon{graphics/smile_bw_set.jpg}\xspace}
\newcommand{\emojiBWSaint}{\vcenteredincludeIcon{graphics/smile_bw_saint.jpg}\xspace}
\newcommand{\emojiBWTears}{\vcenteredincludeIcon{graphics/smile_bw_tears.jpg}\xspace}
\newcommand{\emojiBWGrin}{\vcenteredincludeIcon{graphics/smile_bw_grin.jpg}\xspace}
\newcommand{\emojiBWWink}{\vcenteredincludeIcon{graphics/smile_bw_wink.jpg}\xspace}
\newcommand{\emojiBWTOE}{\vcenteredincludeIcon{graphics/smile_bw_toe.jpg}\xspace}
\newcommand{\emojiBWTES}{\vcenteredincludeIcon{graphics/smile_bw_tes.jpg}\xspace}
\newcommand{\emojiBWSunglasses}{\vcenteredincludeIcon{graphics/smile_bw_sunglasses.jpg}\xspace}
\newcommand{\emojiBWBlushLips}{\vcenteredincludeIcon{graphics/smile_bw_blushlips.jpg}\xspace}
\newcommand{\emojiBWBlushSmile}{\vcenteredincludeIcon{graphics/smile_bw_blushsmile.jpg}\xspace}
\newcommand{\emojiBWDevil}{\vcenteredincludeIcon{graphics/smile_bw_devil.jpg}\xspace}
\newcommand{\emojiBWBlusBigEyes}{\vcenteredincludeIcon{graphics/smile_bw_blushbigeyes.jpg}\xspace}
\newcommand{\emojiBWSmileTeeth}{\vcenteredincludeIcon{graphics/smile_bw_teeth.jpg}\xspace}
\newcommand{\emojiBWSmileBWTears}{\vcenteredincludeIcon{graphics/smile_bw_tears.jpg}\xspace}
\newcommand{\emojiBWSmileKiss}{\vcenteredincludeIcon{graphics/smile_bw_kiss.jpg}\xspace}
\newcommand{\emojiBWSeeNoEvil}{\vcenteredincludeIcon{graphics/emoji_bw_SeeNoEvil.jpg}\xspace}
\newcommand{\emojiBWSmileWink}{\vcenteredincludeIcon{graphics/smile_bw_wink.jpg}\xspace}
\newcommand{\emojiBWSmileSad}{\vcenteredincludeIcon{graphics/smile_bw_sad.jpg}\xspace}
\newcommand{\emojiBWHandPointUp}{\vcenteredincludeIcon{graphics/emoji_bw_hand_point_up.jpg}\xspace}

\newcommand{\proofsquare}{
    \begin{flushright}
      $\square$
    \end{flushright}\xspace
}
\newcommand{\done}{
    \begin{flushright}
      $\bullet$
    \end{flushright}\xspace
}
\newcommand{\donenot}{
    \begin{flushright}
      $\ldots$
    \end{flushright}\xspace
}
\newcommand{\noresult}{
    \begin{flushright}
      $\varnothing$
    \end{flushright}\xspace
}

\newcommand{\progress}{
    \begin{flushright}
      $\Delta$
    \end{flushright}\xspace
}

\newcommand{\topicend}{
      $\blacktriangleleft$
}

\newcommand{\meineChance}{\textit{Meine Chance}\xspace\index{Meine Chance}}

\newcommand{\anweisungArbeitsblatt}{Löse für $x$ auf kariertem Papier. Achte auf eine gut lesbare/ nachvollziehbare Notation und Aufteilung. Nummeriere die Lösungen wie auf dem Arbeitsblatt und gibt sie mit dem Arbeitsblatt ab. Wenn Du die Aufgaben in einer Gruppe bearbeitest notiere dennoch selbst die Lösungen und gibt sie ab, so dass die Dozenten Tipps zur Notation geben können. In diesem Fall gibt bitte zusätzlich zu Deinem Namen an wer in der Gruppe zusammen gearbeitet hat.}
\newlength\myLength

\tikzstyle{every node}=[font=\small]
\tikzstyle{every node}=[inner sep=1pt]
\def\mcr{\pgfmatrixcurrentrow}\def\mcc{\pgfmatrixcurrentcolumn}
\def\width{12}
\def\hauteur{12}

\definecolor{rosso}{RGB}{220,57,18}
\definecolor{giallo}{RGB}{255,153,0}
\definecolor{blu}{RGB}{102,140,217}
\definecolor{verde}{RGB}{16,150,24}
\definecolor{viola}{RGB}{153,0,153}

\makeatletter

\pgfdeclarelayer{background}
\pgfdeclarelayer{foreground}
\pgfsetlayers{background,main,foreground}


\newcommand{\pie}[3][]{
    \begin{scope}[#1]
    \pgfmathsetmacro{\curA}{90}
    \pgfmathsetmacro{\r}{1}
    \def\c{(0,0)}
    \node[pie title] at (90:1.3) {#2};
    \foreach \v/\s in{#3}{
        \pgfmathsetmacro{\deltaA}{\v/100*360}
        \pgfmathsetmacro{\nextA}{\curA + \deltaA}
        \pgfmathsetmacro{\midA}{(\curA+\nextA)/2}

        \path[slice,\s] \c
            -- +(\curA:\r)
            arc (\curA:\nextA:\r)
            -- cycle;
        \pgfmathsetmacro{\d}{max((\deltaA * -(.5/50) + 1) , .5)}

        \begin{pgfonlayer}{foreground}
        \path \c -- node[pos=\d,pie values,values of \s]{$\v\%$} +(\midA:\r);
        \end{pgfonlayer}

        \global\let\curA\nextA
    }
    \end{scope}
}

\newcommand{\slice}[4]{
  \pgfmathparse{0.5*#1+0.5*#2}
  \let\midangle\pgfmathresult

  % slice
  \draw[thick,fill=black!10] (0,0) -- (#1:1) arc (#1:#2:1) -- cycle;

  % outer label
  \node[label=\midangle:#4] at (\midangle:1) {};

  % inner label
  \pgfmathparse{min((#2-#1-10)/110*(-0.3),0)}
  \let\temp\pgfmathresult
  \pgfmathparse{max(\temp,-0.5) + 0.8}
  \let\innerpos\pgfmathresult
  \node at (\midangle:\innerpos) {#3};
}

\newcommand{\legend}[2][]{
    \begin{scope}[#1]
    \path
        \foreach \n/\s in {#2}
            {
                  ++(0,-10pt) node[\s,legend box] {} +(5pt,0) node[legend label] {\n}
            }
    ;
    \end{scope}
}


\theoremstyle{definition}
%\theorembodyfont{\small}
\newtheorem{definition}{Definition}
\newtheorem{uebung}{Übung}
\newtheorem{beispiel}{Beispiel}
\newtheorem{satz}{Satz}
\newtheorem{exkurs}{Exkurs}

\newglossary*{symbols}{Symbolverzeichnis}

\lstset{
    language={java},
    basicstyle=\ttfamily\footnotesize,
    breaklines=true,
    numbers=left,
    stepnumber=5,
    numberstyle=\tiny\color{gray},
    mathescape=true,
    showstringspaces=false
}

\newglossaryentry{CAS}{
    name={Computeralgebrasystem},
    description={\url{https://de.wikipedia.org/wiki/Computeralgebrasystem}: ''Ein Computeralgebrasystem (CAS) ist ein Computerprogramm, das der Bearbeitung algebraischer Ausdrücke dient. Es löst nicht nur mathematische Aufgaben mit Zahlen (wie ein einfacher Taschenrechner), sondern auch solche mit symbolischen Ausdrücken (wie Variablen, Funktionen, Polynomen und Matrizen).''
    }
}

\newglossaryentry{GGT}{
    name={ggT},
    description={Größter gemeinsamer Teiler.}
}

\newglossaryentry{KGV}{
    name={kgV},
    description={Das kleinste Gemeinsame Vielfache einer Zahl.}
}

\newglossaryentry{SISystem}{
    name={SI-System},
    description={\enquote{Das Internationale Einheitensystem oder SI (frz. Système international d’unités) ist das am weitesten verbreitete Einheitensystem für physikalische Größen.} (\url{https://de.wikipedia.org/wiki/Internationales_Einheitensystem}, abgerufen 2017-09-13 15:27)}
}

\newglossaryentry{TRS}{
    name={Term Replacement System (TRS)},
    description={
        \enquote{In mathematics, computer science, and logic, rewriting covers a wide range of (potentially non-\-de\-term\-inistic) methods of replacing subterms of a formula with other terms. What is considered are rewriting systems (also known as rewrite systems, rewrite engines or reduction systems). In their most basic form, they consist of a set of objects, plus relations on how to transform those objects. (\url{https://en.wikipedia.org/wiki/Rewriting}, abgerufen 2017-09-17 14:15)}
    }
}

\newglossaryentry{Uhrzeigersinn}{
    name={Uhrzeigersinn},
    description={
        \enquote{im Uhrzeigersinn} und \enquote{gegen den Uhrzeigersinn} bezeichnet die Richtung in der ein Rundweg in der Ebene betrachtet wird. Im Uhrzeigersinn ist die Richtung die die Zeiger einer Uhr beschreiben, gegen den Uhrzeigersinn die entgegengesetzte Richtung.
    }
} 
\newglossaryentry{symb:Abrunden}{
    type=symbols,
    name={$\left\lfloor x \right\rfloor$},
    description={Abrunden. $\left\lfloor 1,99 \right\rfloor = 1$.},
    sort={abrunden}
}

\newglossaryentry{symb:alpha}{
    type=symbols,
    name={$\alpha$},
    description={Der griechische Buchstabe $\alpha$, sprich \enquote{alpha} bezeichnet insbesondere den Winkel am Punkt $A$ in Polygonen.},
    sort={alpha}
}

\newglossaryentry{symb:Area}{
    type=symbols,
    name={$A_F$},
    description={Flächeninhalt (engl. area) einer Fläche $F$. Wir ersetzen $F$ für elementare Flächen durch den ersten Buchstaben des Namens der Fläche.},
    sort={A}
}

\newglossaryentry{symb:Aufrunden}{
    type=symbols,
    name={$\left\lceil x \rceil$},
    description={Abrunden. $\left\lfloor 1,99 \right\rfloor = 1$.},
    sort={abrunden}
}

\newglossaryentry{symb:beta}{
    type=symbols,
    name={$\beta$},
    description={Der griechische Buchstabe $\beta$, sprich \enquote{beta} bezeichnet insbesondere den Winkel am Punkt $B$ in Polygonen.},
    sort={beta}
}

\newglossaryentry{symb:delta}{
    type=symbols,
    name={$\delta$},
    description={Der griechische Buchstabe $\delta$, sprich \enquote{delta} bezeichnet insbesondere den Winkel am Punkt $D$ in Polygonen.},
    sort={delta}
}

\newglossaryentry{symb:Div}{
    type=symbols,
    name={$\div$},
    description={Symbol für die Division. Gelegentilich wird \enquote{/} benutzt, insbesondere in Texten in denen keine Brüche gesetzt werden können sowieo auf vielen Taschenrechnern.
    },
    sort={Division}
}

\newglossaryentry{symb:gamma}{
    type=symbols,
    name={$\gamma$},
    description={Der griechische Buchstabe $\gamma$, sprich \enquote{gamma} bezeichnet insbesondere den Winkel am Punkt $C$ in Polygonen.},
    sort={gamma}
}

\newglossaryentry{symb:Gleich}{
    type=symbols,
    name={$=$},
    description={
        Wir sagen \enquote{gleich} oder \enquote{ist gleich} und bezeichnen damit die Gleichheit der seiten einer Gleichung, insbesondere die Gleichheit eines zu berechnenden Ausdrucks und des Ergebnisses der Rechnung. $2-1=1$, \enquote{zwei minus eins (ist) gleich eins.}
        },
    sort={gleich}
}

\newglossaryentry{symb:Lambda}{
    name={$\lambda$},
    description={Eine beliebige Zahl, mit der der nachfolgende Ausdruck multipliziert wird.},
    type=symbols,
    sort={lambda}
}

\newglossaryentry{symb:Mal}{
    type=symbols,
    name={$\cdot$},
    description={Symbol für die Multiplikation. Gelegentlich wird, insbesondere für die Lesbarkeit $\times$ benutzt},
    sort={mal}
}

\newglossaryentry{symb:Minus}{
    type = symbols,
    name={$-$},
    description={Wir sagen \enquote{minus} und bezeichnen damit die Operation der Subtraktion. $2-1=1$, \enquote{zwei minus eins (ist) gleich eins.}},
    sort={minus}
}

\newglossaryentry{symb:N}{
    type=symbols,
    name={$\mathbb{N}$},
    description={Die Menge der Natürlichen Zahlen $\{0, 1, 2, 3, 4, ...\}$},
    sort={N}
}

\newglossaryentry{symb:Phi}{
    name={$\varphi$},
    description={Ein beliebiger Winkel.},
    type=symbols,
    sort={phi}
}

\newglossaryentry{symb:Pi}{
    name={$\pi$},
    description={Die Kreiszahl.},
    type=symbols,
    sort={pi}
}

\newglossaryentry{symb:Plus}{
    type=symbols,
    name={$+$},
    description={Wir sagen \enquote{plus} und bezeichnen damit die Operation der Addition. $1+2=3$, \enquote{eins plus zwei (ist) gleich drei}.},
    sort={plus}
}

\newglossaryentry{symb:Sum}{
    type=symbols,
    name={$\Sigma$},
    description={Summe},
    sort=Sigma
}

\newglossaryentry{symb:Vereinigung}{
    type=symbols,
    name={$\bigcup$},
    description={Vereinigung (von Mengen)},
    sort=Vereinigung
} 
\makeglossaries

\setlength{\parskip}{1ex}

\let\origitemize\itemize
\def\itemize{\origitemize\itemsep0pt}

%\setcounter{tocdepth}{1}
%\setcounter{secnumdepth}{0}

\makeindex

\begin{titlepage}
%\title{Meine Chance II - 2017\\Mathematik}
\centering
\title{Mathematik\\ \vspace{1cm} \normalsize \centering für den Mathematik-Unterricht, Lernen und Üben in Vorbereitung für die Abschlussprüfung für den Hauptschulabschluss als Nichtschülerprüfung in Hessen, konkretisiert für das Projekt \enquote{Meine Chance} ($mc^2$)\\\vspace{3em}\footnotesize Die freie, insbesondere unentgeltliche, Nutzung im Rahmen von \enquote{Meine Chance} ist auf Dauer vom Urheber genehmigt. Sollte der Autor dauerhaft nicht erreichbar sein geht das komplette Material inklusive der \LaTeX-Sources (\url{https://github.com/itteerde/MathematikHS}) in public domain über.}

\centering
\author{Erik Itter}
\end{titlepage}


\begin{document}

\newgeometry{hmarginratio=1:1}
\maketitle
\restoregeometry
\tableofcontents

\chapter*{Vorbemerkungen und Umgang mit dem Text}

Dieser Text hilft bei der Vorbereitung auf die Prüfung in Mathematik für den Hauptschulabschluss (Abendhauptschule in Hessen). Der Text ist einfach geschrieben. Er kann zum alleine Lernen, zur Wiederholung des im Unterricht Gelernten und zum Üben benutzt werden.

Wer den ganzen Text wirklich versteht, alle Beispiele versteht, alle Aufgaben lösen kann, ist gut auf die Mathematik in der Prüfung vorbereitet. Fußnoten\footnote{Das hier ist eine Fußnote. Die musst Du nicht lesen. Sie ist für tieferes/ weiteres Verstehen und Lehrer/ Dozenten.} brauchst Du nicht zu lesen. Sie sind für weitergehendes Verstehen nach der Prüfung gedacht\footnote{... und um den Text zwar sehr bewusst unvollständig hinsichtlich der Mathematik zu verfassen aber wenigstens naturwissenschaftlich falsche Texte zu erläutern, die in der Prüfung so erwartet werden, wie Massen konsequent Gewichte zu nennen.}. Einige Abschnitte werden extra gekennzeichnet als nicht für die Prüfung benötigt (\enquote{Exkurs}), aber aus anderen Gründen so wichtig, dass wir ihre Inhalte anbieten wollen. Ein Beispiel dafür ist der \textsc{Satz des Pythagoras}, der in der Prüfung der Abendhauptschule nicht gebraucht wird, in der \enquote{normalen} Hauptschule aber gelehrt (und geprüft) wird, und den Unternehmen als bekannt erwarten. Bei solchen Inhalten schreiben wir warum wir sie Dir anbieten. Du kannst dann wählen sie gar nicht zu beachten, weil Du sie nicht für die Prüfung brauchst. Du kannst wählen sie zu lesen aber nicht beherrschen zu müssen, z.B. um Erwartungen Deines zukünftigen Betriebes zu erfüllen. Du kannst sie als Grundlage für mehr (mathematische) Bildung benutzen, z.B. vorbereitend auf eine Ausbildung mit hohem mathematischen Anteil. Solche Abschnitte beginnen mit einer Überschrift wie \enquote{\textbf{Exkurs 1}} und enden mit einer Markierung, \enquote{\topicend}.

Alle Worte im Text, der nicht besonders als Ausnahme gezeichnet ist, müssen als Vokabeln gelernt werden. Besonders wichtig ist das bei der Fachsprache der Mathematik, die auch in der Prüfung benutzt wird. Zusätzlich sollen alle Wörter in Beispielen und Aufgaben gelernt werden, da diese besonders häufig im Text von Prüfungen vorkommen. Der Text enthält ein Vokabelverzeichnis der Fachsprache und häufig verwendeter Vokabeln. Diese Vokabeln müssen sicher gelernt und gut verstanden werden. Das Vokabular für die Mathematik ist umfangreich und muss über die ganze Schulzeit verteilt gelernt werden. Dafür seid Ihr selbst verantwortlich. In der Mathematik schreiben wir keine Vokabeltests. Ohne die Vokabeln zu lernen wirst Du aber die Prüfung nicht schaffen.

\begin{exkurs}[mathematische Methodik]
    Eine Besonderheit der Mathematik\footnote{bezogen auf die Schule, universitär arbeiten alle Wissenschaften so} ist die Arbeit mit Definitionen und Sätzen. Dinge, wie eine Zahl erfüllen eine Definition oder tun das nicht. Wenn sie die Definition erfüllen kann man Methoden, Regeln, Techniken auf sie anwenden, die in Sätzen (mit einem Namen wie \textsc{Satz des Pythagoras}) oder Rechenregeln exakt beschrieben sind. Dieses anfangs scheinbar umständliche und schwierige Vorgehen stellt sich später als große Stärke der Mathematik heraus, die dazu führt, dass alle Naturwissenschaften (und zunehmend auch alle anderen Wissenschaften und vermehrt auch nicht akademisch gelernte Berufe) sich in ihrer Methodik der Mathematik als Mittel der Kommunikation bedienen.
\end{exkurs}


\chapter{Was ist Mathematik?}

\section{Also was ist Mathematik?}

Die \textbf{Mathematik} im Hauptschulabschluss\footnote{Für den Realschulabschluss kommt schon einiges zu Funktionen, auch wenn man noch ohne das durch die Prüfung kommt.} verstehen wir als
\begin{itemize}
  \item \textbf{Zählen}\index{zählen},
  \item \textbf{Messen}\index{messen},
  \item \textbf{Rechnen}\index{rechnen},
  \item \textbf{Kommunikation} (darüber reden, sich unterhalten und schreiben) dessen.
\end{itemize}

\textbf{Zählen} brauchen wir um genau angeben zu können wie viel mal etwas/ ein Ding vorhanden ist. \textbf{Messen} müssen wir etwas, das nicht zählbar ist, sondern eine Menge mit Einheit, z.B. Wasser, das als Liter Wasser gemessen wird, und auch 3,14159 Liter Wasser sein kann. \textbf{Rechnen} müssen wir wenn etwas/ Dinge so viel ist, dass wir nicht zählen können. (Mathematische) \textbf{Kommunikation} brauchen wir um etwas erklären zu können/ etwas jemandem (direkt oder schriftlich) mitteilen/ kommunizieren zu können. Neben den Symbolen, in Notation und wie man sie Spricht, brauchen wir dafür die Zahlwörter (s. Numeralia, S. \pageref{numeralia}).

In der Prüfung im Hauptschulabschluss wird geprüft ob Du mathematisch \textbf{formulierten}/ \textbf{notierten} Text, \enquote{mathematische \textbf{Notation}}, lesen und verstehen könnt. Es wird geprüft ob Ihr aus Notation oder Text erarbeiten könnt wie die Aufgabe mit Mathematik gelöst werden kann. Es wird geprüft ob Ihr selbst mathematisch kommunizieren könnt (das insbesondere in der mündlichen Prüfung). Der Schwerpunkt der Prüfung ist das Rechnen. Nur zusammen mit dem Textverständnis reicht rechnen zu können zum Bestehen. Weitergehendes Verständnis ist für eine gute Note in der Prüfung notwendig.

Mit Mathematik können wir (scheinbar) schwierige Probleme der Wirklichkeit/ Realität lösen. Wir beschreiben zu einem Problem ein \textbf{Modell} (mathematisch notiert), lösen dieses mathematische Problem, und übertragen die Lösung zurück auf Wirklichkeit/ Realität. Ein solches Verständnis für die Nützlichkeit/ Anwendbarkeit der Mathematik erlaubt zusammen mit dem Rechnen eine gute Note zu erreichen\footnote{... und ist eine gute Grundlage für die Ausbildung in einem technischen Beruf oder einem Beruf, der direkt mit Mathematik zu tun hat.}. Eine besondere Leistung der Mathematik ist, dass ihre Modelle erlauben die Zukunft vorher zu sagen. Ein Physiker kann ausrechnen wie eine Rakete zum Mars fliegen wird, und dann wird sie das auch wirklich so tun. Nur die (mathematisch notierten/ Mathematik benutzenden) Naturwissenschaften können die Zukunft zuverlässig vorhersagen.

\begin{exkurs}[Mathematik, Sprache des Fortschritts]
    Eine frühe Erfolgsgeschichte der Mathematik ist zum Beispiel zukünftige Positionen der Planeten des Sonnensystems vorherzusagen, Sonnenfinsternisse, Mondphasen, Jahreszeiten und so weiter. Die selbe Mathematik, die für diese astronomischen Berechnungen benutzt wird erlaubte viel später Flugbahnen, zunächst von Artillerie-Geschossen, später auch der Apollo-Missionen, die 1969 Menschen ermöglicht hat auf dem Mond zu gehen, fahren, Mondgestein zur Erde zurück zu bringen usw. Auch die gesamte Informatik, also alles was mit Computern und ihren Berechnungen zu tun hat, inklusive der zugrundeliegenden Theorie, ist eine Abspaltung der Mathematik, die so relevant geworden ist, dass wir ihr einen eigenen Namen gegeben haben. Mathematik und Philosophie sind das Fundament auf das sich alle Wissenschaften (außer Theologie, die, obwohl historisch eine der ersten akademischen Disziplinen, der Autor nicht zu den Wissenschaften zählt) stützen und die Menschheit zu den Fortschritten der letzten 2500 Jahre, insbesondere aber der letzten 500 Jahre, gebracht haben.

    Während die Schule in der Tat häufig auf das Rechnen konzentriert ist, ist die Aufgabe der Mathematik eine weiter gehende (für die allerdings rechnen zu können Voraussetzung ist). Die Mathematik erlaubt uns \textbf{Probleme der Wirklichkeit} zu lösen indem wir sie \textbf{abstrahieren und in dem Modell eine Lösung finden}, die wir zurück auf die Realität übertragen können und die in der Weise erfolgreich ist, dass sie einen Vorgang beschreiben kann inklusive Vorhersagen für die Zukunft\footnote{die deutlich häufiger als zufällig zu erwarten eintreffen}.\topicend
\end{exkurs}

Wir konzentrieren uns auf die Arten von Aufgaben, die in den Prüfungen gestellt werden. Die Beispiele werden so gelöst wie wir das für eine perfekte Prüfung haben möchten. Zur Übung hilft deshalb auch mal ein Beispielt.

\begin{beispiel}[Heizung...]
    Seit einigen Jahren baut man Heizungen (abgesehen von Fernwärme) so, dass sie möglichst exakt die Leistung haben, die für die Räumlichkeiten benötigt wird. In einfachen Fällen wie einem Einfamilienhaus reicht es vielleicht die Volumina der Etagen zu berechnen. Im gewerblichen Bereich braucht man in der Regel wesentlich genauere Modelle, die bei aller Software-Unterstützung doch immerhin noch das mathematische Verständnis voraussetzen diese unterstützende Software mit den korrekten Zahlen zu füttern.
\end{beispiel}

\begin{beispiel}[Exkurs: Was ist Mathematik? Enigma!]
    Würde man den Autor fragen wer den zweiten Weltkrieg gewonnen hat, dann würde er vielleicht sagen: \enquote{Alan Turing}. Die meisten Menschen haben keine Ahnung wer das ist. Turing war ein Mathematiker, der das Team geleitet hat, das die Verschlüsselung der Deutschen gebrochen hat indem es die ersten praktisch genutzten Computer entwickelt hat. Vielleicht würde er auch sagen: \enquote{Die Mathematiker und Physiker des Manhattan-Projektes}, die Entwickler der ersten Atombomben. Vielleicht würde er auch gute Laune haben und fragen: \enquote{In welchem Universum?} und damit auf Mathematiker und Physiker des 20. Jahrhunderts referenzieren, die das gesamte Weltbild der (gebildeten) Menschheit gleich mehrfach umgestoßen haben.\topicend
\end{beispiel}

\begin{beispiel}[Exkurs: Regale]
    Der Autor kennt dieses Beispiel nur aus Erzählungen... Im Alter von fünf Jahren hat der Autor sein neues Zimmer im neu gebauten Haus seiner Eltern bezogen. Zu den ersten Schritten das neue Zuhause einzurichten gehörte, dass der Vater des Autors ein Regal an die Wand schraubte für fast alles Spielzeug des Autors. Die Mutter des Autors kritisierte seinen Vater deutlich, lautstark und wohl wenig angemessen für die Wahl von langen und dicken Schrauben, also ziemlich tiefen und großen Löchern in der Wand. Der Vater des Autors ignorierte die Einwände (und benutzte Dübel, die nicht für Innenausstattung gedacht waren), schraubte Befestigungen, die zusätzlich zu Brettern und Spielzeug noch wenigstens eine Tonne Zugkraft aushalten würden. WTF?! Eine Tonne? Nun ja, bereits am ersten Abend zogen die vielleicht 20kg des Autors über einen Hebel der Länge der Bretter gegen einen kürzeren Arm weniger Zentimeter Abstand von Winkel zu Schrauben. Der Autor schätzt etwa 8000 Newton Zugkraft (entspricht 800kg an der Schraube hängend) aufgebracht zu haben bis er auf dem Schrank neben dem Regal angelangt war. Die Erfahrung eines Sturzes, gefolgt von einer Dusche aus Spielzeug und Regalbrettern, ist dem Autor erspart geblieben (auch dazu lernen Schlosser in Deutschland einiges an klassischer Mechanik)...
\end{beispiel}


\section{Wieso müssen wir Mathematik lernen?}\label{MathematikWieso}

\begin{enumerate}
    \item Weil die Mathematik in der Abschlussprüfung geprüft werden muss, mit Prüfungen, die für ganz Hessen vom Kultusministerium (Ministerium für Bildung und...) festgelegt werden. Ohne deren Bestehen man keinen Schulabschluss bekommen (vgl. \citep{LehrplanMathematikHauptschuleHessen2017}).
    \item Mathematik spielt in allen Ausbildungsberufen eine Rolle. In einigen spielt sie eine wichtige, z.B. in allen technischen Berufen. Zusätzlich zur Mathematik selbst braucht man sie in der Physik, in allen handwerklichen in Form zumindest der Geometrie, oft aber auch dort wieder Physik, nämlich klassische Mechanik. In allen kaufmännischen Berufen (und dazu gehören in diesem Sinn alle Berufe, die eine Büro-/ Verwaltungstätigkeit beinhalten) wird u.a. Zinsrechnung gebraucht, die unmittelbar an die Mathematik im Hauptschuleabschluss anschließt\footnote{Sie ist trivial wenn man Dezimalbrüche verstanden hat ohne Zinseszins und mit kommt die Potenzrechnung dazu.}.
    \item Mathematik erlaubt es sicheres Wissen zu erlangen, z.B. sicher sein zu können, dass ein Vertrag billiger ist als ein anderer.
    \item Mathematik erlaubt es zwingend zu argumentieren, z.B. nachzuweisen dass ein Angebot eines Konkurrenten günstiger ist (und der Verkäufer, dem man das zeigt, vielleicht auf diesen Preis herunter geht).
    \item \enquote{Mathematiker sind intelligent.} Mal davon abgesehen dass das stimmt\footnote{... aber ziemlich unklar ist was das bedeutet über das Abschneiden im Intelligenztest hinaus...} lässt sich immer wieder feststellen, dass wer eine solide Basis mathematischer Kompetenzen zeigt, für intelligent gehalten wird - mit entsprechend vorteilhaften Folgen z.B. hinsichtlich beruflicher Karriere. Eine Abwandlung ist die Annahme von vielen Personalverantwortlichen, dass wer sich durch die Mathematik durchgekämpft hat, mit guten Ergebnissen, auch alles lernen kann was man (in einer nicht handwerklichen) beruflichen Tätigkeit braucht.
    \item Mathematik macht Spaß. (WTF?) Leider können wir mit der schulischen Mathematik interessante Themen, die Spaß machen, nur gelegentlich streifen/ andeuten. Der ein oder andere wird aber vielleicht später einmal \enquote{richtige} Mathematik betreiben (können oder müssen..?), die in der Tat viel Freude machen kann.
    \item Mathematik ist die Sprache aller (Natur-) Wissenschaften und darüber hinaus auch in fast allen anderen Wissenschaften massiv im Vormarsch - ausgenommen vielleicht die Theologien.
    \item Ohne elementare Kompetenzen der Mathematik kann man fast keinen Beruf ausüben. Auch hier seien nochmals Verträge erwähnt und die zugrunde liegenden Angebote und Rechnungen, wobei letztere nicht umsonst \enquote{Rechnungen} genannt werden - sie berechnen die Kosten.
    \item \enquote{Made in Germany} ist wegen deutschen Mathematikern, Physikern und Ingenieuren zu einer Erfolgsgeschichte für Deutschland geworden.\footnote{Eigentlich hätte es den Britten als Kennzeichnungspflicht ermöglichen sollen den Produkten aus dem konkurrierenden Deutschland aus dem Weg zu gehen. Stattdessen haben es die Britten als Qualitätszeichen interpretiert und sind ihren eigenen Produkten zugunsten \enquote{Made in Germany} ausgewichen... (Protektionismus ist selten eine gute Idee.)}
\end{enumerate}


\begin{exkurs}[Modelle und Wissenschaften]\label{ModelleUndWissenschaften}
    Die Realität (Wirklichkeit/ Welt/ Natur) ist immer zu komplex um sie vollständig zu beschreiben. Ein Teil der wissenschaftlichen Methodik ist es für die Beobachtungen und vermuteten Zusammenhänge in der Realität \textbf{vereinfachte} (\textbf{abstrahierte}) \textbf{Modelle} zu konstruieren, die \textbf{Vorhersagen} über die Realität ermöglichen. Treffen diese Vorhersagen ein, so geht man \textbf{vorläufig} davon aus, dass das Modell \enquote{richtig/ wahr} ist. Damit meint man dass es die nützlichste Quelle für Vorhersagen ist. Man stellt die \textbf{Hypothese} auf dass sich die Realität wie das Modell (im einfachsten Fall eine Formel) verhält. Ein widerspruchsfreies System von Modellen und Methoden damit umzugehen, das sich vielfach bewährt hat, und zu dem die Realität in keinem einzigen bekannten Fall im Widerspruch steht, nennen wir \textbf{Theorie}. Eine wissenschaftliche Theorie ist die höchste Sicherheit, die wir über Realität erreichen können.

    Darüber hinaus können wir in der Mathematik dadurch dass wir das Modell definieren sicheres Wissen über das Modell gewinnen (beweisen). Wir wissen jedoch nie sicher, dass das Modell auch wirklich etwas mit der Realität zu tun hat. Es gibt immer die Möglichkeit dass im nächsten Augenblick die Realität ein Ereignis zeigt, dass der Theorie widerspricht. In dem Augenblick wird die Theorie verworfen und man muss nach einer neuen (oft einer Verbesserung der alten) Theorie suchen, die auch das zur alten Theorie widersprüchliche \textbf{Phänomen} erklärt (genauer: mit der Beobachtung vereinbar ist). Gibt es mehrere Theorien, die sich zur Zeit nicht experimentell falsifizieren (als falsch widerlegen) lassen, so geht man i.d.R. von der Theorie aus, die am wenigsten kompliziert ist.
\end{exkurs}


\chapter{Mengen}\label{chapter:Mengen}

Wir machen Mengen nicht zu einem großen zentralen Thema, da sie meistens nur mittelbar für die Prüfung relevant sind. Wir benutzen sie aber bei Erklärungen und auch bei Definitionen und führen sie deshalb (nicht rigoros) ein. Mengen sind grafisch mit Venn-Diagrammen (s. \ref{VennDiagramme}, S. \pageref{VennDiagramme}) darstellbar, die auch für Beweise verwendet werden dürfen.

\begin{definition}[Menge]\index{Menge}
    Wir nennen eine \enquote{Ansammlung/ Sammlung} von \enquote{Dingen} eine Menge. Die Dinge in einer Menge nennen wir die \enquote{Elemente} der Menge. Wir schreiben $m \in \mathbb{M}$ (und sagen \enquote{m ist Element von M} oder \enquote{m ist in M enthalten}.) falls das Element $m$ in der Menge $\mathbb{M}$ enthalten ist, und $m \notin \mathbb{M}$ (und sagen \enquote{m ist nicht Element von M} oder \enquote{m ist nicht in M enthalten}) falls das Element $m$ nicht in der Menge $\mathbb{M}$ enthalten ist.
\end{definition}

\begin{definition}[Teilmenge]\index{Teilmenge}\index{Element}
    Wir nennen eine Menge $M_t$ \enquote{Teilmenge} der Menge $M$ wenn alle Elemente der Menge $M_t$ in der Menge $M$ enthalten ist. Wir schreiben $M_t \subset M$ oder $M_t \subseteq M$ und lesen \enquote{$M_t$ ist eine Teilmenge von $M$}.
\end{definition}

\begin{definition}[Vereinigung]\index{Vereinigung}
    Wir nennen die Menge $M_\cup$, die genau\footnote{genau diese und keine darüber hinaus, alle diese und nur genau alle diese} alle Elemente aus den Mengen $M_1$ und $M_2$ enthält \enquote{Vereinigung} von $M_1$ und $M_2$ und schreiben $M_\cup = M_1 \cup M_2$.
\end{definition}

\begin{definition}[Schnittmenge]\index{Schnittmenge}
    Wir nennen eine Menge $M_\cap$, die genau die Elemente enthält, die sowohl in der Menge $M_1$ als auch in der Menge $M_2$ enthalten sind, \enquote{Schnittmenge} oder \enquote{Schnitt} der Mengen $M_1$ und $M_2$ und schreiben $M_\cap = M_1 \cap M_2$.
\end{definition}

\begin{beispiel}[Mengen]
    Sei $M_1$ die Menge $M_1 = \{1,2,4,5,6\}$ und $M_2$ die Menge $M_2 = \{2,4,6,8\}$.

    Dann sind $M_a = \{1,4\}$, $M_b = \{2,4\}$ Teilmengen von $M_1$, in mathematischer Notation $M_a \subset M_1$ und $M_b \subset M_1$. $M_a$ ist auch Teilmenge von $M_1$, geschrieben $M_a \subset M_1$, aber ist nicht Teilmenge von $M_2$, geschrieben $M_a \not\subset M_2$ oder $M_a \nsubseteq M_2$, weil 1 in $M_b$ enthalten ist, nicht aber in $M_2$. $\{1,2\}$ ist Teilmenge von $\{1,2\}$, geschrieben $\{1,2\} \subset \{1,2\}$. Die Möglichkeit der Gleichheit/ Identität betont man mit den Symbolen $\not\subset$ und $\nsubseteq$. Eine Menge ohne Elemente nennen wir \enquote{leere Menge}\index{leere Menge} und schreiben \glsdisp{symb:LeereMenge}{$\varnothing$} (auch $\emptyset$). Z.B. ist die Schnittmenge $M_\cap = \{1,2\} \cap \{3,4\}$ leer, geschrieben als $\{1,2\} \cap \{3,4\} = \varnothing$ oder $\{ \; \}$.
\end{beispiel}


\chapter{Zahlen}\label{chapter:Zahlen}

\section{Natürliche Zahlen ($\mathbb{N}$)}\index{natürliche Zahlen}

\textbf{\glsdisp{symb:N}{Natürliche Zahlen}} benutzen wir zum Zählen. Ein Apfel, zwei Äpfel, drei Äpfel, vier Äpfel, ... Wir schreiben die Menge der Natürlichen Zahlen \enquote{$\mathbb{N} = \{ 0$\footnote{Viele Mathematiker betrachten die Null nicht als natürliche Zahl. Für uns spielt keine Rolle ob Null eine natürliche Zahl ist oder nicht.}\textsuperscript{,}\footnote{Wir verzichten auf eine Definition und gehen davon aus, dass Beispiele ausreichen und spätestens mit der Teilbarkeit Klarheit intuitiv (natürlich!) entsteht.}$, 1, 2, 3, 4, 5, 6, 7, 8, 9, 10, 11, ...\}$} und sagen \enquote{die Natürlichen Zahlen}.


\section{Teilbarkeit}\index{Teilbarkeit}\label{def:teilbarkeit}

\begin{definition}[Teilbarkeit]\index{Teilbarkeit}
    Wir sagen dass eine natürliche Zahl $a$ eine natürliche Zahl $b$ \textbf{teilt} wenn $b \div a$ eine natürliche Zahl ist. Wir sagen \enquote{a teilt b} und schreiben \enquote{\glsdisp{symb:teilt}{$a | b$}} und \enquote{a teilt nicht b} und schreiben \glsdisp{symb:teiltnicht}{\enquote{$a \nmid b$}}. Jede natürliche Zahl ist durch sich selbst und durch 1 teilbar. Wir nennen 1 und die Zahl selbst triviale Teiler der Zahl.
\end{definition}

Wir brauchen die Teilbarkeit beim Kürzen von Brüchen (s. Bruchrechnung, S. \pageref{Bruchrechnung}).

\begin{beispiel}[Teilbarkeit]\index{Teilbarkeit}
    $2 | 4$, $1 | 2$, $3 \nmid 4$, $3 | 9$, $3 \nmid 7$, $3 | 27$, $15 | 60$, $3 \nmid 10$, $10 | 10$, $ 5 \nmid 7$, $16 | 32$.
\end{beispiel}

\begin{uebung}[Teilbarkeit und Dezimal- vs. 60er-System]\index{Teilbarkeit}
    Liste die Teiler von 60, 10, 120 und 100 auf und denke darüber nach wieso die Babylonier ein 60er-System statt eines 10er-Systems verwendet haben (und wir das bei den Stunden [und etwas weniger offensichtlich bei den Tagen] beibehalten haben).
\end{uebung}

\begin{uebung}[Teilermengen]
    Liste die Teiler der folgenden Zahlen auf: $12$, $30$, $24$, $33$, $47$, $51$, $25$, $49$, $8640$.
\end{uebung}


\subsection{Größter gemeinsamer Teiler (ggT)}\index{ggT}

\begin{definition}[ggT]
    Wir nennen die größte Zahl, die alle Zahlen einer Menge \textbf{teilt}, den \enquote{\textbf{\glsdisp{GGT}{größten gemeinsamen Teiler} (ggT)}} der Menge.
\end{definition}

\begin{beispiel}[ggT]
    Die Teiler von 60 sind $\text{T}_{60} = \{2,3,4,5,6,10,12,15,20,30\}$ und die Teiler von 40 sind $\text{T}_{40}\{2,4,5,8,10,20\}$. Die \textbf{gemeinsamen Teiler} von 60 und 40 sind $\text{gT}_{\{60,40\}} = \text{T}_{60} \bigcup \text{T}_{40} = \{2,4,5,10,20\}$. Der \textbf{\glsdisp{GGT}{größte gemeinsame Teiler (ggT)}} von 60 und 40 ist das größte \textbf{Element} von $\text{gT}_{\{60,40\}}$: $\text{ggT}_{\{60,40\}}=20$.
\end{beispiel}


\subsection{Kleinstes gemeinsames Vielfaches (kgV)}\index{kgV}

\begin{definition}[kgV]
    Wir nennen die kleinste Zahl durch die jede Zahl der Menge deren kleinstes gemeinsames Vielfaches (kgV) wir suchen \enquote{\textbf{kleinstes gemeinsames Vielfaches (kgV)}}.
\end{definition}

Wir brauchen das kgV beim Addieren und Subtrahieren von Brüchen (s. Brüche, S. \pageref{Bruchrechnung}).

Systematisch finden wir das kgV einer Menge von Zahlen, insbesondere von zwei Zahlen, über die Primfaktorzerlegungen der Zahlen. Für kleine Zahlen (so wie die im Teil der Prüfung ohne Taschenrechner) reicht eine Tabelle der Vielfachen aus in der man die erste und damit kleinste Zahl sucht, die in den Reihen der Vielfachen aller Zahlen deren kgV man sucht, vorkommt.

{\centering
\begin{tabular}{|r|r|r|}
  % after \\: \hline or \cline{col1-col2} \cline{col3-col4} ...
  1 & 7 & 9 \\
  \hline
  2 & 14 & 18 \\
  3 & 21 & 27 \\
  4 & 28 & 36 \\
  5 & 35 & 45 \\
  6 & 42 & 54 \\
  7 & 49 & \textbf{63} \\
  8 & 56 & 72 \\
  9 & \textbf{63} &  \\
  10 & &
\end{tabular}\hspace{1em}
\begin{tabular}{|r|r|r|}
  % after \\: \hline or \cline{col1-col2} \cline{col3-col4} ...
  1 & 3 & \textbf{15} \\
  \hline
  2 & 6 &  \\
  3 & 9 &  \\
  4 & 12 &  \\
  5 & \textbf{15} &  \\
  6 & & \\
  7 & & \\
  8 & & \\
  9 & & \\
  10 & &
\end{tabular}\hspace{1em}
\begin{tabular}{|r|r|r|r|}
  % after \\: \hline or \cline{col1-col2} \cline{col3-col4} ...
  1 & 12 & 16 \\
  \hline
  2 & 12 & 32 \\
  3 & 36 & 48 \\
  4 & 48 & 64 \\
  5 & 60 & 80 \\
  6 & 72 & \textbf{96} \\
  7 & 84 & \\
  8 & \textbf{96} & \\
  9 & & \\
  10 & &
\end{tabular}\\}


\section{Primzahlen ($\mathbb{P}$)}\index{Primzahlen}

\begin{definition}[Primzahlen]\label{def:Primzahl}
    Wir nennen eine natürliche Zahl $p$ größer als Eins, die nur durch sich selbst und eins ganzzahlig teilbar ist, \enquote{Primzahl}.
\end{definition}

Die Primzahlen bis 50 sind auswendig zu lernen: $\mathbb{P}=\{2$, $3$, $5$, $7$, $11$, $13$, $17$, $19$, $23$, $29$, $31$, $37$, $41$, $43$, $47$,...$\}$. Wir benötigen sie beim Kürzen von Brüchen und es steht im Lehrplan (\citep[S. 11]{LehrplanMathematikHauptschuleHessen2017}). In den Videos zum Unterricht wird zum besseren Verständnis das Verfahren zur Bestimmung der Menge der Primzahlen kleiner einer bestimmten Zahl gezeigt, das \textsc{Sieb des Eratosthenes}: \url{https://www.youtube.com/watch?v=oiQ5pxdQ8l0&list=PLVBzQZPmcRQL6vbGv1uDj7styo4I0nzHr&index=4}.

\begin{beispiel}
    Die \textbf{Primfaktorzerlegung}\index{Primfaktorzerlegung} von $6$ ist $\{2,3\}$, die von $120$ ist $\{2,2,2,3,5\}$. Wir schreiben $\{2,2,2,3,5\}$ auch als $\{2^3,3^1,5^1\}$ oder $\{2^3,3,5\}$.\topicend
\end{beispiel}


\section{Primfaktorzerlegung}\index{Primfaktorzerlegung}

Jede natürliche Zahl $n$ kann (in einer eindeutigen Weise) in \textbf{Primfaktoren}\index{Primfaktor} zerlegt werden (\textbf{Faktorisierung}). 6 ist \textbf{teilbar} durch 2 und 3. 1 und die Zahl selbst führen wir nicht auf\footnote{Z.B. \texttt{FactorInteger} in \textit{Mathematica}/ \textit{Wolfram Language} liefert für Primzahlen die Teiler und sonst die Primfaktoren, also uneinheitlich als Faktorisierung, eindeutig nur in der Primfaktorzerlegung.}.

\begin{beispiel}[Primfaktorzerlegung]
    \begin{equation}\label{eqn:primfaktorzerlegung01}
        6 = 2 \cdot 3
    \end{equation}
\end{beispiel}

Die Reihenfolge der \textbf{Faktoren} ist egal. $3 \cdot 2$ und $2 \cdot 3$ werden als die selbe \textbf{Zerlegung} (\textbf{Primfaktorzerlegung}) betrachtet. Für bessere Lesbarkeit schreiben wir die Faktoren meistens von klein nach groß aufsteigend.

\begin{beispiel}[Primfaktorzerlegung]
    \begin{equation}\label{eqn:primfaktorzerlegung02}
        60 = 2 \cdot 2 \cdot 3 \cdot 5
    \end{equation}
\end{beispiel}


\begin{uebung}[Gerade Primzahlen]
    Wie viele gerade Primzahlen gibt es? Versuche Deine Antwort zu begründen.
\end{uebung}


\section{Ganze Zahlen ($\mathbb{Z}$)}\index{ganze Zahlen}

\begin{definition}[Betrag]\index{Betrag}
    Wir nennen $|x|$ den Betrag (engl. abs) von $x$. Es sei $|-x| = x$, $|x|=x$ und damit $|-x|=|x|$.
\end{definition}

Die Menge der ganzen Zahlen $\mathbb{Z}$ besteht aus den natürlichen Zahlen und die negativen Zahlen gleichen Betrages: $\mathbb{Z}=\{..., -2, -1, 0, 1, 2, 3, ...\}$. Die \textbf{Ordnungsrelation} \enquote{$<$}\glsadd{symb:Kleiner}\index{$<$} sprechen wir \enquote{kleiner als}. \enquote{$1<2$} ist eine wahre Aussage; \enquote{$2<1$} ist eine falsche Aussage. Die Ordnungsrelation \enquote{$>$}\glsadd{symb:Groeszer}\index{$>$} sprechen wir \enquote{größer als}. \enquote{$2>1$} ist eine wahre Aussage; \enquote{$1>2$} ist eine falsche Aussage. Auf dem Zahlenstrahl ist eine Zahl $a$ mit $a < b$ links von der Zahl $b$.


\section{Rationale Zahlen ($\mathbb{Q}$)}\index{rationale Zahlen}

Die Menge der \textbf{rationalen Zahlen} $\mathbb{Q}$ besteht aus den Zahlen der Form $\frac{a}{b}, a, b \in \mathbb{Z}, b \neq 0$. Wir nennen diese Zahlen (insbesondere in dieser Form aufgeschrieben) \textbf{Brüche}. Besondere Bedeutung haben im Alltag \textbf{Dezimalbrüche}, die wir mit Komma schreiben im \textbf{Dezimalsystem} (s. Dezimalsystem, S. \pageref{Dezimalsystem}).

\subsection{Motivation}

Mit Brüchen statt Dezimalbrüchen (Kommazahlen) zu rechnen vereinfacht das Leben mit Mathematik sehr. Natürliche Zahlen sind uns eben natürlich. Ein Ganzes, ein Halbes oder ein Drittel von etwas, einer Pizza z.B., ist uns unmittelbar verständlich. $33,\overline{3}\%$ sind es nicht und machen erst durch den Gedanken \enquote{$33,\overline{3}$\% \textbf{ist} ein Drittel} gedanklich Sinn. Brüche sind einfach zu begreifen, Dezimalentwicklungen nicht. Unter $0,\overline{142857}$ kann sich Niemand etwas vorstellen, der nicht weiß, dass das $\frac{1}{7}$ ist.


\subsection{Grundlagen}\index{Bruchrechnung!Grundlagen}\label{Bruchrechnung}

Wir sprechen jeweils über die grauen Teile/ Flächen relativ zur gesamten Fläche, dem ganzen Kreis.

\begin{longtable}{|m{0.3\linewidth}|m{0.6\linewidth}|}
\hline
$
    \begin{tikzpicture}
        \filldraw[fill=gray!20] circle(0.75cm);
    \end{tikzpicture}
$
& Das ist eine Pizza. Das ist eine ganze Pizza. Das ist ein Ganzes. Das ist ein Stück einer Pizza, die aus einem Stück besteht. Es ist 1. Es ist $\frac{1}{1}$.\\
\hline
$
    \begin{tikzpicture}[scale=0.5, .style={fontsize=\footnotesize}]
            \filldraw[fill=gray!20] (0,0) circle(0.75cm);
            \draw (1.1,0) node{+};
            \filldraw[fill=gray!20] (2.2,0) circle(0.75cm);
            \draw (3.3,0) node{=};
            \filldraw[fill=gray!20] (4.4,0) circle(0.75cm);
            \filldraw[fill=gray!20] (5.9,0) circle(0.75cm);
    \end{tikzpicture}
$
& Das sind zwei Ganze (Kreise, Pizzen, Dinge, ...). Ein Kreis war $\frac{1}{1}$. Zwei sind $\frac{1}{1}+\frac{1}{1}=\frac{2}{1} = 2$ oder $2 \cdot \frac{1}{1} = \frac{2}{1} = 2$.\\
\hline
$
    \begin{tikzpicture}
            \filldraw[fill=gray!20] (0,0) circle(0.75cm);
            \draw (0, -0.75) -- (0,0.75);
    \end{tikzpicture}
$
& Das ist immer noch ein Kreis. Dass er durchgeschnitten ist ändert das nicht. Es sind zwei (gleich große) Teile eines Kreises aus zwei Teilen. Das sind $\frac{2}{2}$. Also sind $\frac{2}{2} = \frac{1}{1} = 1$.\\
\hline
$
    \begin{tikzpicture}[radius=7.5mm, delta angle=120]
        \filldraw[fill=black!10!white, draw=black!70!white, rotate=90]
            (0,0) -- (7.5mm, 0) arc (0:180:7.5mm) -- cycle;
        \filldraw[fill=white, draw=black!70!white, rotate=270]
            (0,0) -- (7.5mm, 0) arc (0:180:7.5mm) -- cycle;
    \end{tikzpicture}
$
& Das ist immer noch ein Kreis. Dass er durchgeschnitten ist ändert das nicht. Es ist ein Teil von zwei (gleich großen) Teilen eines Kreises aus zwei Teilen. Das sind $\frac{1}{2}$.\\
\hline
$
    \begin{tikzpicture}[radius=7.5mm, delta angle=120]
        \filldraw[fill=black!10!white, draw=black!70!white]
            (0,0) -- (7.5mm, 0) arc (0:120:7.5mm) -- cycle;
        \filldraw[fill=white, draw=black!70!white, rotate=120]
            (0,0) -- (7.5mm, 0) arc (0:120:7.5mm) -- cycle;
        \filldraw[fill=white, draw=black!70!white, rotate=240]
            (0,0) -- (7.5mm, 0) arc (0:120:7.5mm) -- cycle;
    \end{tikzpicture}
$
& Das ist immer noch ein Kreis. Dass er durchgeschnitten ist ändert das nicht. Es ist ein Teil von drei (gleich großen) Teile eines Kreises aus drei Teilen. Das sind $\frac{1}{3}$.\\\hline
$
    \begin{tikzpicture}[radius=7.5mm, delta angle=120]
        \filldraw[fill=black!10!white, draw=black!70!white]
            (0,0) -- (7.5mm, 0) arc (0:120:7.5mm) -- cycle;
        \filldraw[fill=black!10!white, draw=black!70!white, rotate=120]
            (0,0) -- (7.5mm, 0) arc (0:120:7.5mm) -- cycle;
        \filldraw[fill=white, draw=black!70!white, rotate=240]
            (0,0) -- (7.5mm, 0) arc (0:120:7.5mm) -- cycle;
    \end{tikzpicture}
$
& Das ist immer noch ein Kreis. Dass er durchgeschnitten ist ändert das nicht. Es sind zwei (gleich große) Teile von drei (gleich großen) Teilen eines Kreises aus drei Teilen. Das sind $\frac{2}{3}$.\\\hline
$
    \begin{tikzpicture}[baseline=-1mm,scale=0.5, .style={fontsize=\footnotesize}]
        \filldraw[fill=black!10!white, draw=black!70!white]
            (0,0) -- (7.5mm, 0) arc (0:120:7.5mm) -- cycle;
        \filldraw[fill=white, draw=black!70!white, rotate=120]
            (0,0) -- (7.5mm, 0) arc (0:120:7.5mm) -- cycle;
        \filldraw[fill=white, draw=black!70!white, rotate=240]
            (0,0) -- (7.5mm, 0) arc (0:120:7.5mm) -- cycle;
    \end{tikzpicture}
    +
    \begin{tikzpicture}[baseline=-1mm,scale=0.5, .style={fontsize=\footnotesize}]
        \filldraw[fill=white, draw=black!70!white]
            (0,0) -- (7.5mm, 0) arc (0:120:7.5mm) -- cycle;
        \filldraw[fill=black!10!white, draw=black!70!white, rotate=120]
            (0,0) -- (7.5mm, 0) arc (0:120:7.5mm) -- cycle;
        \filldraw[fill=black!10!white, draw=black!70!white, rotate=240]
            (0,0) -- (7.5mm, 0) arc (0:120:7.5mm) -- cycle;
    \end{tikzpicture}
    =
    \begin{tikzpicture}[baseline=-1mm,scale=0.5, .style={fontsize=\footnotesize}]
        \filldraw[fill=black!10!white, draw=black!70!white]
            (0,0) -- (7.5mm, 0) arc (0:120:7.5mm) -- cycle;
        \filldraw[fill=black!10!white, draw=black!70!white, rotate=120]
            (0,0) -- (7.5mm, 0) arc (0:120:7.5mm) -- cycle;
        \filldraw[fill=black!10!white, draw=black!70!white, rotate=240]
            (0,0) -- (7.5mm, 0) arc (0:120:7.5mm) -- cycle;
    \end{tikzpicture}
$
& S.o. $\frac{1}{3} + \frac{2}{3} = \frac{3}{3} = 1$. Gesprochen wir dies als \enquote{ein Drittel plus zwei Drittel gleich drei Drittel gleich eins}.\\\hline
$
    \begin{tikzpicture}[baseline=-1mm,scale=0.5, .style={fontsize=\footnotesize}]
        \filldraw[fill=black!10!white, draw=black!70!white]
            (0,0) -- (7.5mm, 0) arc (0:120:7.5mm) -- cycle;
        \filldraw[fill=white, draw=black!70!white, rotate=120]
            (0,0) -- (7.5mm, 0) arc (0:120:7.5mm) -- cycle;
        \filldraw[fill=white, draw=black!70!white, rotate=240]
            (0,0) -- (7.5mm, 0) arc (0:120:7.5mm) -- cycle;
    \end{tikzpicture}
    +
    \begin{tikzpicture}[baseline=-1mm,scale=0.5, .style={fontsize=\footnotesize}]
        \filldraw[fill=black!10!white, draw=black!70!white, rotate=120]
            (0,0) -- (7.5mm, 0) arc (0:60:7.5mm) -- cycle;
        \filldraw[fill=black!10!white, draw=black!70!white, rotate=180]
            (0,0) -- (7.5mm, 0) arc (0:60:7.5mm) -- cycle;
        \filldraw[fill=white, draw=black!70!white]
            (0,0) -- (7.5mm, 0) arc (0:120:7.5mm) -- cycle;
        \filldraw[fill=white, draw=black!70!white, rotate=240]
            (0,0) -- (7.5mm, 0) arc (0:60:7.5mm) -- cycle;
        \filldraw[fill=white, draw=black!70!white, rotate=60]
            (0,0) -- (7.5mm, 0) arc (0:60:7.5mm) -- cycle;
        \filldraw[fill=white, draw=black!70!white, rotate=300]
            (0,0) -- (7.5mm, 0) arc (0:60:7.5mm) -- cycle;
    \end{tikzpicture}
    =
    \begin{tikzpicture}[baseline=-1mm,scale=0.5, .style={fontsize=\footnotesize}]
        \filldraw[fill=black!10!white, draw=black!70!white, rotate=120]
            (0,0) -- (7.5mm, 0) arc (0:60:7.5mm) -- cycle;
        \filldraw[fill=black!10!white, draw=black!70!white, rotate=180]
            (0,0) -- (7.5mm, 0) arc (0:60:7.5mm) -- cycle;
        \filldraw[fill=black!10!white, draw=black!70!white]
            (0,0) -- (7.5mm, 0) arc (0:120:7.5mm) -- cycle;
        \filldraw[fill=white, draw=black!70!white, rotate=240]
            (0,0) -- (7.5mm, 0) arc (0:120:7.5mm) -- cycle;
    \end{tikzpicture}
$ & $\frac{1}{3} + \frac{2}{6} = \frac{2}{3}$ und $\frac{2}{6} = \frac{1}{3}$\\\hline
\end{longtable}

Die Zahl über dem Bruchstrich nennen wir \enquote{\textbf{Zähler}}, die Zahl unter dem Bruchstrich \enquote{\textbf{Nenner}}. Wir lesen Brüche als $\frac{1}{2}$\enquote{ein Halb} (oder \enquote{ein Halbes}), $\frac{1}{3}$ als \enquote{ein Drittel}, $\frac{1}{4}$ als \enquote{ein Viertel}, $\frac{1}{12}$ als \enquote{ein Zwölftel}, $\frac{13}{1000}$ als \enquote{dreizehn Tausendstel}. Wird der Bruch nicht als Nomen (der Zahl) verwendet, sondern als adverbiale Bestimmung der Anzahl eines benannten Etwas, wie in \enquote{dreiviertel Pizza}, dann wird klein und mit den unter Numeralia (s. \ref{numeralia}) Regeln zusammen oder getrennt geschrieben. Da wir solche Zahlen stets mit Ziffern schreiben, und beim Lesen/ Sprechen Groß- und Kleinschreibung und Trennungen wenig interessant sind, ist das nicht besonders wichtig. Wir machen einen großen Sprung und geben die Rechenregeln der Bruchrechnung an. Keine Angst, wir kehren nach Angabe einer Regel jeweils zu Beispielen zurück und nehmen uns alle Zeit der Welt um an diesen wichtigen Stellen niemanden zu verlieren.


\subsection{Dezimalbrüche}\index{Dezimalbrüche}

Wir notieren Zahlen mit Hilfe eines \textbf{Stellenwertsystems}. Weltweit durchgesetzt hat sich das 10er-System, aus dem Latein \textbf{Dezimalsystem}.


\section{Reelle Zahlen ($\mathbb{R}$)}\index{reelle Zahlen}

Die Menge der reellen Zahlen \glsdisp{symb:rReelleZahlen}{$\mathbb{R}$} \enquote{besteht aus allen Stellen auf dem Zahlenstrahl\index{Zahlenstrahl}}\footnote{Eine rigorose Einführung/ Definition ist für alle Schulabschlüsse überflüssig und wird in naturwissenschaftlichen Studiengängen im ersten Semester an der Universität unterrichtet.}. Wenn wir nicht ausdrücklich eine andere Domäne angeben betrachten wir alle Zahlen mit denen wir arbeiten als reelle Zahlen.


\section{Rundungen}\index{Rundung}

Wir können Zahlen \textbf{runden}\index{runden}. Wir sprechen von \enquote{runden auf $n$ Stellen (nach dem Komma)} oder \enquote{runden auf ganze Tausend, Millionen, ...}. Zu runden bedeutet die nächste Stelle nach der gewünschten Genauigkeit zu betrachten und bei Ziffern kleiner als 5 \textbf{abzurunden} und ab 5 \textbf{aufzurunden}. Oft legt die Bedeutung der Zahl (i.d.R. mit Maßeinheit) eine bestimmte Rundung nahe. Z.B. wird man €-Werte fast immer auf zwei Nachkommastellen gerundet erwarten (auf Cent genau). Wollen wir kennzeichnen dass eine Zahl nicht mehr genau/ exakt ist, sondern (mit bestimmter Präzession) gerundet, schreiben wir statt $=$, \enquote{(ist) gleich}, \glsdisp{symb:Ungefaehr}{$\approx$}, \enquote{(ist) ungefähr (gleich)} oder \enquote{(ist) rund}.


\section{Numeralia (Zahlwörter)}\label{numeralia}

\enquote{eins, zwei, drei, vier, fünf, sechs, sieben, ...} ist der Beginn der Zahlwörter \footnote{als Kardinalzahlen}. \citep[Randzahl 509]{DudenGrammatik2016} gibt eine unvollständige Einleitung. Demnach sind sowohl Schreibweisen als auch gesprochene Zahlen nicht (mehr) streng festgelegt. Eine veraltete Regel lautet dass natürliche Zahlen bis zwölf als Zahlwort geschrieben werden, darüber hinaus als Ziffernfolge. Es bleibt das Problem Zahlwörter für größere Zahlen festzulegen, so dass größere Zahlen gesprochen werden können. Die Numeralia von 0 bis 12 sind \enquote{eins, zwei, drei, vier, fünf, sechs, sieben, acht, neun, zehn, elf, zwölf}. Für Zahlen mit Zehnerstellen\footnote{Mit \enquote{echten} Zehnerstellen, also keine Null in der vorletzten Ziffer der Zahl.} werden die Zehner-Suffixe benötigt, diese sind \enquote{-zehn, -zwanzig, -dreißig, -vierzig, -fünfzig, -sechzig, -siebzig, -achtzig, -neunzig}. Für die Hunderter-Stellen werden die Numerale von 1 bis 9 vor das Numeral \enquote{hundert} gesetzt. Statt \enquote{eins} ist das Numeral hier \enquote{ein}. Sind Hunderter vorhanden wird zwischen ihnen und den Zehnern \enquote{und} eingefügt: \enquote{eins, zwei, drei, vier, fünf, sechs, sieben, acht, neun, zehn, elf, zwölf, dreizehn, vierzehn, ..., neunzehn, zwanzig, einundzwanzig, zweiundzwanzig, ..., neunundzwanzig, dreißig, einunddreißig, ..., neunundneunzig, einhundert, einhundertundeins, ..., einhundertundneunundneunzig\footnote{\cite{DudenGrammatik2016} erkennt ausdrücklich auch hundertneunundneunzig als korrekt an.}, zweihundert, zweihundertundeins, ... neunhundertundneunundneunzig}.

Bei Zahlen ab 1.000 (\enquote{eintausend}) werden die Numeralia in 3er-Paketen (Tripeln) gebildet. Das kleinste Tripel wird mit \enquote{und} verknüpft angehängt. 123.456 wird \enquote{einhundertunddreiundzwanzigtausendundvierhundertundsechsundfünfzig} gelesen\footnote{\citep{DudenGrammatik2016} erkennt hier gleich mehrere unterschiedliche Varianten als korrekt an.}. Es wird zunächst das höchstwertige Tripel gebildet, so dass alle noch folgenden Teile der Zahl echte (dreistellige) Tripel sind. Also wird für 23456 456 abgetrennt und das (unvollständige) höchstwertige Tripel 23 gelesen: \enquote{dreiundzwanzig}. Angehängt wird der Name der Größenordnung (für das vorletzte Tripel 1.000 ($10^3$), \enquote{eintausend}). Danach wird das nächstkleinere Tripel (hier das letzte, 456 gelesen, \enquote{vierhundertundsechsundfünfzig}. Zahlen unter einer Million werden zusammen geschrieben, so dass 23.456 \enquote{dreiundzwanzigtausendundvierhundertundsechsundfünfzig}\footnote{Korrekt wäre u.a. auch \enquote{dreiundzwanzigtausendvierhundertsechsundfünfzig}, aber die Varianten mit weniger \enquote{und} brauchen mehr Bildungsregeln.} heist. Die Namen der ersten Größenordnungen sind (Singular/Plural) $10^3$: tausend/tausend, $10^6$: Million/Millionen, $10^9$: Milliarde/Milliarden, $10^{12}$: Billion/Billionen, $10^{15}$: Billiarde/Billiarden. Für dem Alltag sollten die Namen bis zu den Milliarden bekannt sein (z.B. Bundeshaushalt).

Es ergeben sich z.B. $1.123.456.789.123$: \enquote{eine Billion einhundertunddreiundzwanzig Milliarden vierhundertundsechsundfünfzig Millionen siebenhundertundneunundachtzigtausendundeinhundertunddreiundzwanzig} und $89.001.456.901$: \enquote{neunundachtzig Milliarden eine Million vierhundertundsechsundfünfzigtausendundneunhunderundeins}. Solche Numeralia werden nie geschrieben, die in Ziffern geschrieben Zahl muss aber zumindest bis in den Bereich der Milliarden vorgelesen/ ausgesprochen werden können. Die hier angegebene Variante (mit maximalen \enquote{und}-Verbindungen) scheint die Variante mit den wenigsten Bildungsregeln aus den korrekten Varianten zu sein. Scheinen Stellen nach der größten Größenordnung zu fehlen (Nullen) werden diese nicht gesprochen: $999.999.999$, $1.000.000.000$, $1.000.000.001$ wird \enquote{neunhundertundneunundneunzig Millionen neunhunderundneunungneunzigtausendundneunhundertundneunundneunzig, eine Milliarde, eine Milliarde und eins} gelesen.

\begin{uebung}[Numeralia (Zahlwörter)]
    Schreibe die Numeralia auf für 1 bis 21, 31, 42, 53, 64, 72, 84, 93, 99, 100, 101, 199, 200, 201, 999, 1.000, 1.001, 9.000, 9.999, 123.456, 200.000, 999.999, 1.000.000, 10.000.001, 999.999.999, 1.000.000.000, 1.000.000.001 und 123.456.789.012.345.\topicend
\end{uebung}

\section{Variablen (Unbekannte/ Veränderliche)}\index{Variablen}\index{Unbekannte|see{Variable}}\index{Veränderliche|see{Variable}}

In der Mathematik wollen wir (insbesondere in \textbf{Formeln}) möglichst \textbf{allgemeine} Aussagen treffen um möglichst viel mit diesen anfangen zu können. Oftmals benutzen wir dazu keine Zahlen, sondern Namen wie \enquote{a}, \enquote{b} und \enquote{c} oder \enquote{x} und \enquote{y}. Diese Namen stehen immer für Zahlen. Man kann sie durch beliebige Zahlen ersetzen (aber in einer Rechnung immer durch die Selbe). Die Fläche eines Rechtecks nennen wir z.B. allgemein $a \cdot b$, oder \enquote{Länge mal Breite}. Jeder (positive) Wert ergibt ein Rechteck, dessen Fläche wir so berechnen können: $A_{\text{Rechteck}} = a \cdot b = A_R = a b$\footnote{\enquote{A} steht für \enquote{area}, englisch für \textbf{Fläche/ Flächeninhalt}}.

\begin{beispiel}
    Gegeben sei ein Rechteck mit der Länge $a = 12 \text{cm}$ und Breite $b = 5 \text{cm}$. Zu berechnen sei der Flächeninhalt des Rechtecks $A_R = a b$.
    \begin{align}
      A_R & =  a b && a \rightarrow 12 cm, b \rightarrow 6 cm\\
       & =  12 cm \cdot 6 cm &&\\
       & =  60 cm^2 &&
    \end{align}

    Es ist weitgehend eine sehr gute Weise an die Schulmathematik heran zu gehen indem man \enquote{Rechnen} als \glsdisp{TRS}{Term Replacement System} (System von Ersetzungsregeln/ systematisches Einsetzen) versteht.\topicend
\end{beispiel}

Genau genommen ersetzen wir wie im Beispiel zu sehen, in diesem Fall mit einer reellen Zahl und einer Maßeinheit. Diese Feinheiten können wir jedoch zunächst ignorieren und in den kommenden Monaten intuitiv durch Üben lernen.


\section{Dezimalsystem}\index{Dezimalsystem}\label{Dezimalsystem}

Weltweit hat sich das \textbf{Dezimalsystem} für sowohl alltägliche als auch wis\-sen\-schaft\-lich-\-tech\-nische Rechnungen durchgesetzt. Ausnahmen sind Rechnungen mit Zeiten mit Maßeinheiten größer als Sekunden und die technische Umsetzung in der Informatik (Binärsystem).

Ein Stellenwertsystem funktioniert wie folgt: Es gibt eine 1er-Stelle ($10^0=1$). D.h. die Zahl Eins schreiben wir \enquote{1} da $1 \cdot 10^0 = 1$. Die nächstgrößere Stelle (nach links gehend) ist im Dezimalsystem die 10er-Stelle ($10^1=10$). Es folgen nach links die 100er-Stelle, die 1000er-Stelle usw. Jede Stelle ist im Dezimalsystem\footnote{Das Dezimalsystem ist ein Stellenwert-System mit der Basis 10. Ein wichtiges, aber nicht prüfungsrelevantes, Stellenwertsystem ist das Binärsystem mit 2 als Basis. Das ist das Stellenwertsystem, das allen Computern zugrunde liegt - und damit auch Eurem Taschenrechner. Das Dezimalsystem nutzen wir ziemlich sicher nur weil wir 10 Finger haben und der Anfang des Umgangs mit Zahlen ziemlich sicher das Zählen mit zehn Fingern war.} 10 mal so groß wie ihr rechter Nachbar. Das gilt auch von der 1er-Stelle nach rechts. Die nächste Stelle rechts ist die $\frac{1}{10}$-Stelle, dann die $\frac{1}{100}$-Stelle, die $\frac{1}{1000}$-Stelle und so weiter. Nach links also mal zehn und nach rechts geteilt durch zehn.

$10^0$ ist gleich eins. Das akzeptieren wir bitte, weil es in die Reihe passt und die Dozenten sagen dass das richtig ist, ohne es rigoros einzuführen. $10^1=10$, $10^2=100$, $10^5=100.000$, $10^{-1}=\frac{10}{1}$, $10^{-3}=\frac{1}{1000}$, $10^{-6}=\frac{1}{1.000.000}$, ... Auch hier verzichten wir auf eine Definition und verlassen uns darauf dass die Regel mit Übung natürlich wird.

\begin{beispiel}[Dezimalsystem]\label{bsp:Dezimalsystem01}
    \begin{align}\label{eqn:Dezimalsystem01}
      123,456 &= 1 \cdot 10^2 = 100 \\
       & \quad + 2 \cdot 10^1 = 10  \\
       & \quad + 3 \cdot 10^0 = 3 \\
       & \quad + 4 \cdot 10^{-1} = 0,4\\
       & \quad + 5 \cdot 10^{-2} = 0,05\\
       & \quad + 6 \cdot 10^{-3} = 0,006\\
       &= 123,456
    \end{align}
\end{beispiel}



\chapter{Grundrechenarten}\index{Grundrechenarten}\label{Grundrechenarten}

\enquote{$1+2=3$}\glsadd{symb:Plus}\glsadd{symb:Minus}\glsadd{symb:Gleich} sprechen wir \enquote{eins plus zwei (ist) gleich drei}. \textbf{Plus} zu rechnen nennen wir \textbf{addieren} oder \textbf{die Addition}. \enquote{$3-2=1$} sprechen wir \enquote{drei minus zwei (ist) gleich 1}. \textbf{Minus} zu rechnen nennen wir \textbf{subtrahieren} oder \textbf{die Subtraktion}. \textbf{Das Ergebnis} einer \textbf{Subtraktion} nennen wir \textbf{die Differenz}. Wir sagen auch \enquote{die Differenz von drei und zwei ist eins}\footnote{und umgangssprachlich/ alltagssprachlich auch \enquote{die Differenz von zwei und drei ist eins (obwohl $2-3=-1$). Hierbei kommt die Vorstellung der \textbf{Differenz} als \textbf{Abstand} am \textbf{Zahlenstrahl} $\mathbb{R}^1$ zum intuitiven Ausdruck}}.

\section{Addition und Subtraktion am Zahlenstrahl}

Wir verstehen das Rechnen mit \glsdisp{symb:Plus}{$+$} und \glsdisp{symb:Minus}{$-$} als Bewegung auf dem Zahlenstrahl der reellen Zahlen $\mathbb{R}$\footnote{Der Zahlenstrahl bringt einen natürlichen Weg zu $\mathbb{R}$ mit, da eine kontinuierliche \enquote{Strecke} schlecht in diskrete Abschnitte aufgeteilt sein kann elementar, da man jedes Stück wieder teilen kann ins Unendliche. (Dass vielleicht die Realität gar keine kontinuierlichen Räume beinhaltet wird die Lernenden sicher nicht stören und sei mit einem Hinweis auf die breite Anwendbarkeit der Analysis beiseite gelegt.)}. Von den Lehrenden aus wird das Berechnen von Summen als bekannt vorausgesetzt.

\begin{equation}\label{eqn:00001}
    1 + 2 + 3
\end{equation}

\begin{figure}[H]
  \centering
\begin{tikzpicture}[>=triangle 60]

    \draw (1,1) -- +(10,0);

    \foreach \x in {1,...,11}{
        \pgfmathtruncatemacro{\label}{\x-1}
        \draw (\x,0.8) -- +(0,0.4);
        \node[below] (xa) at (\x,0.7) {\label};
    }

    \coordinate[label={[label distance=10pt]90:1}] (1) at (2,2);
    \coordinate[label={[label distance=10pt]90:3}] (3) at (4,2);
    \coordinate[label={[label distance=10pt]90:6}] (6) at (7,2);
    \fill (1) circle (1pt);
    \fill (3) circle (1pt);
    \draw[->,shorten >=4pt] (1) -- (3) node[midway, below] {+2};
    \fill (6) circle (1pt);
    \draw[->,shorten >=4pt] (3) -- (6) node[midway, below] {+3};

\end{tikzpicture}
  \caption{$+/-$ am Zahlenstrahl $\mathbb{R}$}\label{fig:zahlenstrahlAddition}
\end{figure}

Problematisch wird das Subtrahieren von negativen Zahlen, für das man sich am Zahlenstrahl zwar mit \textquote{Drehungen} um 180° und damit $-(-x) = x$ aus der Affäre ziehen kann, das aber mathematisch so nicht vertretbar ist. Sauber wäre hier schlicht die Regel für die Äquivalenz anzugeben und die Lernenden als \gls{TRS} arbeiten zu lassen. Die Repräsentation kontinuierlicher Strecken vermeidet bei Lernenden, die bereits verfestigt zählen\index{verfestigtes Zählen} (vgl. \citep[S. 112]{Hasemann2014}) statt zu rechnen dies weiter zu bedienen, wie z.B. durch eine Darstellung $\clubsuit + \clubsuit = \clubsuit \clubsuit$ oder $\clubsuit + \clubsuit = 2 \clubsuit$\footnote{Die Darstellung $\clubsuit + \clubsuit = 2 \clubsuit$ hat ihre Berechtigung in der Algebra, wenn die Lernenden dort den abstrakten Umgang mit einer Unbekannten, $x$, erlernen.}. D.h. beim Sprechen über Arithmetik in $\mathbb{R}$ ist wennimmer möglich in Strecken zu reden und Zählen zu vermeiden.

\begin{equation}
    -1 + 5 - 3
\end{equation}

\begin{figure}[H]
  \centering
\begin{tikzpicture}[>=triangle 60]

    \draw (1,1) -- +(10,0);

    \foreach \x in {-2,...,8}{
        \pgfmathtruncatemacro{\label}{\x}
        \pgfmathtruncatemacro{\posx}{\x+3}
        \draw (\posx,0.8) -- +(0,0.4);
        \node[below] (xa) at (\posx,0.7) {\label};
    }

    \coordinate[label={[label distance=10pt]90:-1}] (-1) at (2,3);
    \coordinate[label={[label distance=10pt]90:4}] (4) at (7,3);
    \fill (-1) circle (1pt);
    \fill (4) circle (1pt);
    \draw[->,shorten >=4pt] (-1) -- (4) node[midway, below] {+5};
    \fill (6) circle (1pt);

    \coordinate[label={[label distance=10pt]90:4}] (4l) at (7,2);
    \coordinate[label={[label distance=10pt]90:1}] (1l) at (4,2);
    \fill (1l) circle (1pt);
    \fill (4l) circle (1pt);
    \draw[->,shorten >=4pt] (4l) -- (1l) node[midway, below] {-3};

\end{tikzpicture}
  \caption{$+/-$ am Zahlenstrahl $\mathbb{R}$}\label{fig:zahlenstrahlSubtraktion}
\end{figure}


\begin{beispiel}[Summe mit gemischten Vorzeichen]
    \begin{equation}
        \Sigma = 12345 - 43512 + 31987 - 47614
    \end{equation}

    \begin{figure}[H]
      \centering
    \begin{tikzpicture}[>=triangle 60]

        \draw (1,1) -- +(10,0);

        \foreach \x in {-2,...,8}{
            \pgfmathtruncatemacro{\label}{\x}
            \pgfmathtruncatemacro{\posx}{\x+3}
            \draw (\posx,0.8) -- +(0,0.4);
            %\node[below] (xa) at (\posx*10000,0.7) {\label};
        }

        \node[rotate=90,anchor=center] at (6,0.6) {\small 0};
        \node[rotate=90,anchor=center] at (1,0.1) {\small -50.000};
        \node[rotate=90,anchor=center] at (2,0.1) {\small -40.000};
        \node[rotate=90,anchor=center] at (3,0.1) {\small -30.000};
        \node[rotate=90,anchor=center] at (4,0.1) {\small -20.000};
        \node[rotate=90,anchor=center] at (5,0.1) {\small -10.000};
        \node[rotate=90,anchor=center] at (7,0.1) {\small 10.000};
        \node[rotate=90,anchor=center] at (8,0.1) {\small 20.000};
        \node[rotate=90,anchor=center] at (9,0.1) {\small 30.000};
        \node[rotate=90,anchor=center] at (10,0.1) {\small 40.000};
        \node[rotate=90,anchor=center] at (11,0.1) {\small 50.000};

        \coordinate[label={[label distance=10pt]90:0}] (null) at (6,3);
        \coordinate[label={[label distance=10pt]90:12345}] (a) at (7.2345,3);
        \fill (null) circle (1pt);
        \fill (a) circle (1pt);
        \draw[->,shorten >=4pt] (null) -- (a) node[midway, below] {\tiny +12345};
        %\fill (6) circle (1pt);

        \coordinate[label={[label distance=10pt]90:-31167}] (m31167) at (2.89,2.5);
        \coordinate[label={[label distance=10pt]90:}] (123456a25) at (7.2345,2.5);
        \fill (m31167) circle (1pt);
        \fill (123456a25) circle (1pt);
        \draw[->,shorten >=4pt] (123456a25) -- (m31167) node[midway, below] {\tiny -43512};

        \coordinate[label={[label distance=10pt]90:}] (m31167a2) at (2.89,2);
        \coordinate[label={[label distance=10pt]90:820}] (820) at (6.082,2);
        \fill (m31167a2) circle (1pt);
        \fill (820) circle (1pt);
        \draw[->,shorten >=4pt] (m31167a2) -- (820) node[midway, below] {\tiny +31987};

        \coordinate[label={[label distance=10pt]90:}] (820a15) at (6.082,1.5);
        \coordinate[label={[label distance=10pt]90:-46794}] (m46794) at (1.32,1.5);
        \fill (820a15) circle (1pt);
        \fill (m46794) circle (1pt);
        \draw[->,shorten >=4pt] (820a15) -- (m46794) node[midway, below] {\tiny -47614};

    \end{tikzpicture}
      \caption{$+/-$ am Zahlenstrahl $\mathbb{R}$}\label{fig:zahlenstrahlSumme01}
    \end{figure}
\end{beispiel}



\section{Multiplikation}

Wir schreiben die \textbf{Multiplikation} von \textbf{Faktoren} $f_1 \cdot f_2$ und das Ergebnis einer \textbf{Multiplikation} das \textbf{Produkt}.\glsadd{symb:Mal} Am Zahlenstrahl bedeutet $n \cdot m$ $n$ Schritte der Länge $m$ aneinander zu setzen oder $m$ Schritte der Länge $n$.


\section{Division}

Wir schreiben die \textbf{Division} von \textbf{Faktoren}\footnote{Die Mathematik nennt das in der Tat so (und unterscheidet Multiplikation und Division kaum). Wem das zu unklar ist kann die Begriffe Dividend und Divisor benutzen. Dann bezeichnet der Dividend die zu teilende Zahl und Divisor die Zahl durch die der Dividend geteilt wird.} $f_1 : f_2$, $f_1 \div f_2$ oder $f_1 / f_2$ und nennen das Ergebnis einer \textbf{Division} \textbf{Quotient}.\glsadd{symb:Div} Am Zahlenstrahl bedeutete die Division $p : q$ die Strecke $p$ in $q$ gleich lange Stücke aufzuteilen und der Ausdruck $p : q$, z.B. $6:3=2$ die Länge einer solchen Teilstrecke (ohne Angabe der Maßeinheit).



\chapter{Kopfrechnen}\index{Kopfrechnen}\label{Kopfrechnen}

Auch mit Smartphone ist Kopfrechnen nicht nur für die erste Seite in der Mathematik-Prüfung nützlich. Das Minimum das wir sicher und schnell, nämlich auswendig, können müssen ist das kleine 1x1. Das brauchen wir auch beim schriftlichen Rechnen, es hilft im Alltag positiv statt negativ aufzufallen und es ist notwendig für die erste Seite in der Prüfung. Sicher Kopfrechnen zu können verschafft einem auf generell immer wieder etwas mehr Selbstsicherheit. Geschadet dürfte es noch nie haben.

Außerdem hilft Kopfrechnen im Überschlagsrechnen (s. Überschlagsrechnen, S. \pageref{Überschlagsrechnen}) bei einer Rechnung mitzudenken und dadurch gravierende Fehler relativ wahrscheinlich zu finden (z.B. Vertippen auf Telefon oder Taschenrechner).

\section{1 mal 1}

Das \enquote{$1 \times 1$}, sprich \enquote{ein mal eins}, wird perfekt auswendig gelernt.

\begin{tikzpicture}[border style/.style={
    draw,fill=#1,minimum size=0.8cm,anchor=center,outer sep=0,
    name=\tikzmatrixname-\the\mcr-\the\mcc
}]
\matrix[row sep=-.5*\pgflinewidth,column sep=-.5*\pgflinewidth,
   execute at empty cell={%
       \ifnum1=\mcr\relax%
           \ifnum1=\mcc\relax\node[border style=black!10!white]{$\cdot$};%
           \else\node[border style=black!10!white]{$\number\numexpr\the\mcc-1\relax$};\fi
       \else%
         \ifnum1=\mcc\relax\node[border style=black!10!white]{$\cdot$};%
           \node[border style=black!10!white]{$\number\numexpr\the\mcr-1\relax$};%
         \else%
           \pgfmathparse{int(abs(\mcr-\mcc))}%
           \ifnum5=\pgfmathresult\relax\def\temp{white}\else\def\temp{none}\fi%
           \node[border style=\temp]{\number\numexpr\numexpr\the\mcc-1\relax*\numexpr\the\mcr-1\relax\relax};%
         \fi%
       \fi}
] (a) {
&&&&&&&&&&&\\
&&&&&&&&&&&\\
&&&&&&&&&&&\\
&&&&&&&&&&&\\
&&&&&&&&&&&\\
&&&&&&&&&&&\\
&&&&&&&&&&&\\
&&&&&&&&&&&\\
&&&&&&&&&&&\\
&&&&&&&&&&&\\
&&&&&&&&&&&\\
&&&&&&&&&&&\\
};
\end{tikzpicture}

Die perfekte Beherrschung des \enquote{$1 \times 1$} ist \textbf{notwendige Voraussetzung} für sicheres und schnelles schriftliches Rechnen. Wer Probleme mit dem Lernen hat meldet sich bei den Dozenten, um verschiedene Methoden zu lernen zu besprechen und ggf. Material zum Lernen zu bekommen.


\section{Hilfreiche Rechenregeln}

\subsection{2er-Potenzen}

Wenn ich $444 : 8$ rechnen muss hole ich nicht das Telefon aus der Tasche (wobei man sich damit vielleicht gerade so nicht mehr lächerlich macht), sondern weiß dass $8 = 2^3$, d.h. durch acht zu teilen bedeutet auch drei mal zu halbieren. Also $444 \xrightarrow{:2} 222 \xrightarrow{:2} 111 \xrightarrow{:2} 55,5$. Entsprechend ist es hilfreich die 2er-Potenzen ein paar Stellen weit auswendig zu können: $2^n=\{2, 4, 8, 16, 32, 64, 128, 256, 512, 1024, ...\}$. Wir sollten zumindest bis $2^6=64$ kommen. Kommt Dir 64 irgendwie besonders bekannt vor? Ja sicher, $8^2 = 64$. Also ${2^3}=8$ und $8^2=64$? Dann ist also ${(2^3)}^2 = 64$. Und $2^6 = 64$? ${(2^3)}^2 = 2^6$. In der Tat ist ${(a^m)}^n = a^{m \cdot n}$. Das ist allerdings gerade so nicht mehr im Bereich Hauptschule, kommt aber in technischen Berufen in der Berufsschule - und möglicherweise auch überall sonst wegen ein klein Wenig Finanz-/ Bankmathematik hinsichtlich Steuern, Kalkulation von Preisen, Geschäftsplan, ...

In der anderen Richtung gilt auch dass ich falls ich einen Preis für 16 Personen rechnen muss, also mal 16, ausnutzen kann, dass $16 = 2^4$. Wenn der Preis 13€ ist, dann also $13\text{€} \xrightarrow{\cdot 2} 26\text{€} \xrightarrow{\cdot 2} 52\text{€} \xrightarrow{\cdot 2} 104\text{€} \xrightarrow{\cdot 2} 208\text{€}$. Und falls es 17 Personen sind eben $13\cdot 2^4\text{€}=208\text{€}$, plus 13€ = 221€. Und so weiter...


\subsection{Teilbarkeit durch 3: Quersumme}

Eine Zahl\footnote{im Dezimalsystemnotiert} ist genau dann durch 3 teilbar wenn die Summe ihrer Ziffern durch 3 teilbar ist. Ist 987654321 durch 3 teilbar? Die Quersumme, also die Summe der Ziffern, ist $\Sigma_Q = 9+8+7+6+5+4+3+2+1$. Das ist 45. Weiß ich noch nicht ob 45 durch 3 teilbar ist so ist die Quersumme von 45=9. 9, ist durch drei teilbar, 45 auch und damit auch 987654321. $987654321 : 3 = 3292181107$. Diese Regel ist in Zeiten von Smartphones wirklich eher unwichtig, hilft aber immer noch dabei zu entscheiden ob ich einen Bruch als Ergebnis noch kürzen kann. Auf der ersten Seite in der Prüfung sind Zähler und Nenner dann aber meistens so klein, dass sie noch im 1x1 liegen.



\chapter{Überschlagsrechnen}\index{Überschlagsrechnen}\label{Überschlagsrechnen}

\enquote{Überschlagsrechnen} (\citep[S. 8]{LehrplanMathematikHauptschuleHessen2017}) und systematisches Schätzen\footnote{Am Ende des Projektdurchlaufes wäre die Idealvorstellung einfache sog. \enquote{Fermi-Probleme} selbständig lösen zu können, was der Autor auch für einen guten Test hält ob, abgesehen vom sprachlichen Niveau in Deutsch als Fremdsprache, voll umfänglich die Basis für beliebige Ausbildungsgänge gelegt ist.} ordnet \citep{LehrplanMathematikHauptschuleHessen2017} den Natürlichen Zahlen unter, da mit diesen auch das Dezimalsystem direkt im fünften Jahrgang explizit behandelt werden soll. Um hier einen deutlichen Zeitgewinn zu erreichen gehen wir der in der Erwachsenenbildung üblichen Struktur nach die Zahlen durch ohne sie rigoros einzuführen, sondern verlassen uns insbesondere in $\mathbb{R}$ auf die Intuition anhand des Zahlenstrahls und als kontinuierliches Maß von Strecken.


\chapter{Maßeinheiten}\index{Maßeinheiten}

\section{SI-System}\index{SI-System}

Das \glsdisp{SISystem}{SI-System} wird weltweit für im technischen und wissenschaftlichen Bereich genutzt und fast überall (leider gehören die USA zu den drei Ausnahmen) alltäglich. Für den Hauptschulabschluss brauchen wir \enquote{kms}, das Kilogramm (kg) für Massen und \enquote{Gewichte}, den Meter (m) für Strecken und die Sekunde (s) für Zeitspannen. Wer das Problem der Maßeinheiten versteht findet vielleicht interessant was man bei \youtube unter \enquote{si einheiten} findet und von da aus auch das komplette System erschließen kann. Manches was durch Umrechnungen passiert ist, ist von heute aus zurück schauend auch witzig - wenn z.B. \textit{NASA} 200 Millionen Dollars in den Mars rammt weil ein Lieferant (USA...) in amerikanischen Pfund statt Kilogramm gerechnet hat (\url{https://de.wikipedia.org/wiki/Mars_Climate_Orbiter})...

\section{Präfixe für Größenordnungen}\index{Größenordnungen}\index{Präfixe|see{Größenordnungen}}

\enquote{Kilo} bedeutet 1000, \enquote{Dezi-} bedeutet $\frac{1}{10}$, \enquote{Zenti-} bedeutet $\frac{1}{100}$ (beachte, dass die Abkürzungen \enquote{c} statt \enquote{z} verwenden, da wir die englische Schreibweise übernommen haben), \enquote{Milli} bedeutet $\frac{1}{1000}$. Ein Millimeter ist also ein tausendstel Meter und ein Zentimeter ein hundertstel Meter.

\section{Der Meter, Maßeinheit für Strecken}

Mit dem Meter messen wir Entfernungen, Längen, Strecken. Alle anderen Maßeinheiten für Strecken werden durch den Meter definiert. Ein Millimeter ist ein tausendstel Meter. \enquote{Milli} ist das Präfix für $\frac{1}{1000}$. Prüfungsrelevant sind Meter, Kilometer, Dezimeter, Zentimeter und Millimeter. Mikrometer ($10^{-6}m$) und Nanometer ($10^{-9}m$) kommen gelegentlich in Nachrichten vor, z.B. als Größe von (gesundheitsschädlichen) Partikeln. Für die Prüfung relevant sind:

\begin{eqnarray*}
% \nonumber to remove numbering (before each equation)
  1mm &=& \frac{1}{1000}m \\
  1cm &=& \frac{1}{100}m \\
  1dm &=& \frac{1}{10}m \\
  1km &=& 1000m
\end{eqnarray*}

Für weniger alltäglichen Gebrauch gibt es in beide Richtungen weiter Namen für Maßeinheiten für Strecken, die sich ebenfalls jeweils auf den Meter beziehen, z.B. ist $1m = \frac{1}{9460730472580800}\text{ly}$ (Lichtjahre), d.h. ein Lichtjahr (kommt gelegentlich in den Nachrichten vor) ist somit gleich $9460730472580800\text{m}$. In der anderen Richtung sind die nächsten Präfixe \enquote{Piko} und \enquote{Femto}, so dass $1\text{m}=1000000000000000\text{fm}$.

\begin{uebung}
    Rechne um/ löse für $x$:

    \begin{equation}
        2km + 23m = x m
    \end{equation}
    \begin{equation}
        17km + 791m = x cm
    \end{equation}
    \begin{equation}
        3km + 17m + 5dm + 23cm + 2mm = x mm
    \end{equation}
\end{uebung}

\tikzset{
%Define standard arrow tip
>=stealth',
%Define style for boxes
punkt/.style={
       %rectangle,
       %rounded corners,
       %draw=black, very thick,
       text width=3em,
       minimum height=2em,
       text centered},
% Define arrow style
pil/.style={
       ->,
       thick,
       shorten <=2pt,
       shorten >=2pt,}
       }

\begin{figure}
  \centering
    \begin{tikzpicture}[node distance=1cm, auto,]
        \node[punkt] (meter) {m};
        \node[punkt, above=of meter] (dezimeter) {dm};
        \node[punkt, above=of dezimeter] (zentimeter) {cm};
        \node[punkt, above=of zentimeter] (millimeter) {mm};
        \node[punkt, above=of millimeter] (mikrometer) {$\mu$m};
        \node[punkt, above=of mikrometer] (nanometer) {nm};
        \node[punkt, below=of meter] (kilometer) {km};
        \node[punkt, below=of meter] (kilometerdummy) {}
            edge[pil,bend right=70] node[text width= 15em, anchor=west] {$\cdot 1000$} (meter.east);
        \node[punkt, above=of kilometer] (meterdummy) {}
            edge[pil,bend right=70] node[text width= 11em, anchor=west] {$\cdot 10$} (dezimeter.east)
            edge[pil,bend right=90] node[text width= 15em, anchor=west] {$\cdot 1000$} (millimeter.east)
            edge[pil,bend right=70] node[text width= 3em, anchor=east] {$\div 1000$} (kilometer.west);
        \node[punkt, above=of meter] (dezimeterdummy) {}
            edge[pil,bend right=70] node[text width= 11em, anchor=west] {$\cdot 10$} (zentimeter.east)
            edge[pil,bend right=70] node[text width= 2em, anchor=east] {$\div 10$} (meter.west);
        \node[punkt, above=of dezimeter] (zentimeterdummy) {}
            edge[pil,bend right=70] node[text width= 2em, anchor=east] {$\div 10$} (dezimeter.west)
            edge[pil,bend right=70] node[text width= 11em, anchor=west] {$\cdot 10$} (millimeter.east);
        \node[punkt, above=of zentimeter] (millimeterdummy) {}
            edge[pil,bend right=70] node[text width= 2em, anchor=east] {$\div 10$} (zentimeter.west)
            edge[pil,bend right=90] node[text width= 4em, anchor=east] {$\div 1000$} (meter.west)
            edge[pil,bend right=70] node[text width= 11em, anchor=west] {$\cdot 1000$} (mikrometer.east);
        \node[punkt, above=of millimeter] (mikrometerdummy) {}
            edge[pil,bend right=70] node[text width= 11em, anchor=west] {$\cdot 1000$} (nanometer.east)
            edge[pil,bend right=70] node[text width= 3em, anchor=east] {$\div 1000$} (millimeter.west);
        \node[punkt, above=of mikrometer] (nanometerdummy) {}
            edge[pil,bend right=70] node[text width= 3em, anchor=east] {$\div 1000$} (mikrometer.west);

        %\node[punkt, below=of kilometer] (lowerpaddingdummy) {};
        \node[punkt, left=of kilometer] (leftpaddingdummy) {};
        \node[punkt, left=of leftpaddingdummy] (leftpaddingdummy2) {};
    \end{tikzpicture}
  \caption{Umwandlungen von Strecken (m)}\label{fig:sisystemMUmwandlungenM}
\end{figure}


\section{Das Kilogramm, Maßeinheit für Massen}

\textbf{Das Kilogramm (kg)}\index{Kilogramm}\index{kg|see{Kilogramm}} ist die \textbf{Maßeinheit} des \glsdisp{SISystem}{SI-Systems} für \textbf{Massen}. In der Prüfung werden damit auch Gewichte gemessen.\footnote{Tatsächlich ist das Kilogramm die Einheit für Massen. Da alle Massen im Hauptschulabschluss auf der Erde, mehr oder minder in gleich bleibendem Abstand vom Mittelpunkt der Erde, gemessen werden können wir die Unterscheidung vernachlässigen, auch wenn das dem Physiker weh tut...}

\begin{figure}
  \centering
    \begin{tikzpicture}[node distance=1cm, auto,]
        \node[punkt] (kilogramm) {kg};
        \node[punkt, above=of kilogramm] (gramm) {g};
        \node[punkt, above=of gramm] (milligramm) {mg};
        \node[punkt, above=of milligramm] (mikrogramm) {$\mu$g};
        \node[punkt, below=of kilogramm] (tonne) {t};
        \node[punkt, below=of gramm] (kilogrammdummy) {}
            edge[pil,bend right=70] node[text width= 11em, anchor=west] {$\cdot 1000$} (gramm.east)
            edge[pil,bend right=70] node[text width= 3em, anchor=east] {$\div 1000$} (tonne.west);
        \node[punkt, above=of kilogramm] (grammdummy) {}
            edge[pil,bend right=70] node[text width= 11em, anchor=west] {$\cdot 1000$} (milligramm.east)
            edge[pil,bend right=70] node[text width= 3em, anchor=east] {$\div 1000$} (kilogramm.west);
        \node[punkt, above=of gramm] (milligrammdummy) {}
            edge[pil,bend right=70] node[text width= 3em, anchor=east] {$\div 1000$} (gramm.west)
            edge[pil,bend right=70] node[text width= 11em, anchor=west] {$\cdot 1000$} (mikrogramm.east);
        \node[punkt, above=of milligramm] (mikrogrammdummy) {}
            edge[pil,bend right=70] node[text width= 3em, anchor=east] {$\div 1000$} (milligramm.west);
        \node[punkt, below=of kilogramm] (tonnedummy) {}
            edge[pil,bend right=70] node[text width= 15em, anchor=west] {$\cdot 1000$} (kilogramm.east);

        \node[punkt, left=of kilogramm] (leftpaddingdummy) {};
        \node[punkt, left=of leftpaddingdummy] (leftpaddingdummy2) {};
    \end{tikzpicture}

  \caption{Umwandlungen von Massen (kg)}\label{fig:sisystemMUmwandlungenKG}
\end{figure}


\section{Die Sekunde, Maßeinheit für Zeitspannen}

\textbf{Die Sekunde}\index{Sekunde}\index{s|see{Sekunde}} ist die \textbf{Maßeinheit} des \glsdisp{SISystem}{SI-Systems} für Zeitspannen. Bei den Zeiteinheiten haben sich die Schritte des 60er-Stellensystem der Babylonier gehalten. In der Prüfung müssen wir mit den 60er-Schritten zwischen Sekunden, Minuten und Stunden zurechtkommen. Technisch und Wissenschaftlich rechnet man in Sekunden und den im \gls{SISystem} regelmäßigen Präfixen.

\begin{figure}
  \centering
    \begin{tikzpicture}[node distance=1cm, auto,]
        \node[punkt] (sekunde) {s};
        \node[punkt, above=of sekunde] (millisekunde) {ms};
        \node[punkt, above=of millisekunde] (mikrosekunde) {$\mu$s};
        \node[punkt, above=of mikrosekunde] (nanosekunde) {ns};
        \node[punkt, below=of sekunde] (minute) {min};
        \node[punkt, below=of minute] (stunde) {h};
        \node[punkt, below=of stunde] (tag) {d};
        \node[punkt, below=of tag] (jahr) {y};

        \node[punkt, below=of millisekunde] (sekundedummy) {}
            edge[pil,bend right=70] node[text width= 11em, anchor=west] {$\cdot 1000$} (millisekunde.east)
            edge[pil,bend right=70] node[text width= 3em, anchor=east] {$\div 60$} (minute.west);
        \node[punkt, below=of sekunde] (minutedummy) {}
            edge[pil,bend right=70] node[text width= 11em, anchor=west] {$\cdot 60$} (sekunde.east)
            edge[pil,bend right=70] node[text width= 3em, anchor=east] {$\div 60$} (stunde.west);
        \node[punkt, below=of minute] (stundedummy) {}
            edge[pil,bend right=70] node[text width= 11em, anchor=west] {$\cdot 60$} (minute.east)
            edge[pil,bend right=70] node[text width= 3em, anchor=east] {$\div 24$} (tag.west);
        \node[punkt, below=of stunde] (tagdummy) {}
            edge[pil,bend right=70] node[text width= 11em, anchor=west] {$\cdot 24$} (stunde.east)
            edge[pil,bend right=70] node[text width= 3em, anchor=east] {$\div 365$} (jahr.west);
        \node[punkt, below=of tag] (jahrdummy) {}
            edge[pil,bend right=70] node[text width= 11em, anchor=west] {$\cdot 365$} (tag.east);
        \node[punkt, above=of sekunde] (millisekundedummy) {}
            edge[pil,bend right=70] node[text width= 11em, anchor=west] {$\cdot 1000$} (mikrosekunde.east)
            edge[pil,bend right=70] node[text width= 3em, anchor=east] {$\div 1000$} (sekunde.west);
        \node[punkt, above=of millisekunde] (mikroskundedummy) {}
            edge[pil,bend right=70] node[text width= 11em, anchor=west] {$\cdot 1000$} (nanosekunde.east)
            edge[pil,bend right=70] node[text width= 3em, anchor=east] {$\div 1000$} (millisekunde.west);
        \node[punkt, above=of mikrosekunde] (nanoskundedummy) {}
            edge[pil,bend right=70] node[text width= 3em, anchor=east] {$\div 1000$} (mikrosekunde.west);

        \node[punkt, left=of sekunde] (leftpaddingdummy) {};
        \node[punkt, left=of leftpaddingdummy] (leftpaddingdummy2) {};
    \end{tikzpicture}
  \caption{Umwandlungen von Zeitspannen (s)}\label{fig:sisystemMUmwandlungenS}
\end{figure}

Monate sind nicht eindeutig. Kalendarische Monate haben 28 bis 31 Tage, betriebswirtschaftliche häufig 30. Wichtig für die Prüfung sollte höchstens sein dass ein Jahr zwölf Monate hat (Zinsen, Gehalt, ...).

In Wissenschaft und Technik wird durchgehend mit der Sekunde gerechnet und die Größenordnung wissenschaftlich notiert. So ist ein Tag $1 \text{d} = 60 \cdot 60 \cdot 24 \text{s} = 86.400\text{s} = 8,64 \times 10^4 \text{s}$ und das Universum ist ca. 14 Milliarden Jahre alt, was wissenschaftlich notiert $4,3 \times 10^{17}\text{s}$ sind.

\begin{exkurs}[SI-Einheiten über kms hinaus]
    Der Vollständigkeit halber einmal alle Einheiten des \glsdisp{SISystem}{SI-Systems}: Wir kennen bereits die Maßeinheit für Längen, den Meter\index{Meter} (m), die Einheit für Massen, das Kilogramm\index{Kilogramm} (kg) und die Einheit für Zeit, die Sekunde\index{Sekunde} (s). Die weiteren Einheiten, die nicht relevant sind für die Prüfung, da wir nur diese drei (kms) in der Mathematik brauchen und Erdkunde haben statt Physik, sind das Ampere\index{Ampere} (A) für Stromstärken, Kelvin\index{Kelvin} (K) für Temperaturen, Mol\index{Mol} (mol) für Stoffmengen und Candela\index{Candela} (cd) für Lichtstärke.

    Aus den elementaren Maßeinheiten können wir neue gewinnen durch Kombinationen. So wird Geschwindigkeit in Metern pro Sekunde ($\frac{m}{s}$) gemessen, Beschleunigung in Metern pro Quadratsekunden ($\frac{m}{s^2}$\footnote{Das ist die Veränderung der Geschwindigkeit pro Zeit}), Dichte in Kilogramm pro Kubikmeter ($\frac{kg}{m^3}$) und so weiter...

    Während einige Einheiten ursprünglich recht willkürlich und wenig stabil gewählt wurden (so ist $\frac{1}{1000000}$ des Kreisbogens der Erdoberfläche von Äquator bis Paris sicherlich besser als ein Schritt des Königs), ändert sich aber tatsächlich auch, wenn auch nicht so stark wie beim Tod des Königs, und war tatsächlich auch nur die Basis für den Stab aus einer Legierung, die in Paris im Keller liegt und damit die wirkliche Definition war) bemühen sich Wissenschaftler seit einigen Jahrzehnten die SI-Einheiten auf Naturkonstanten zu definieren und so ist der Meter seit 1983 definiert als die Strecke, die Licht im Vakuum währen der Dauer von $\frac{1}{299792458}$ Sekunden zurücklegt. Nach aktuellem Stand der Wissenschaften ist dies eine in unserem Universum überall und zu allen Zeiten gleiche Größe (und den genauen Nenner hat man natürlich so gewählt dass es dem alten Meter entspricht - nur nun auch so bleibt und nicht von der Temperatur im Pariser Keller abhängig ist).\topicend
\end{exkurs}

\chapter{Schriftlich Rechnen}\index{Schriftlich Rechnen}

Nochmal: Die perfekte Beherrschung des \enquote{$1 \times 1$} ist \textbf{notwendige Voraussetzung} für sicheres und schnelles schriftliches Rechnen! Weitere Beispiele und Erklärung in der Weise wie hier schriftlich festgehalten findet Ihr auch unter \url{https://www.youtube.com/watch?v=SQvYo9ZZ4qg&list=PLVBzQZPmcRQL6vbGv1uDj7styo4I0nzHr} als \textit{Youtube}-Videos.


\section{Schriftliche Addition}\index{Addition!schriftliche}

Wir addieren einen Term von \textbf{Summanden} indem wir alle \textbf{Summanden} korrekt nach 10er-\textbf{Stellenwertsystem} untereinander schreiben und von rechts nach links, also von der 1er-\textbf{Stelle} beginnend, alle gleichen Stellen addieren. Für Ergebnisse ab 10 ist die 1er-\textbf{Stelle} des Ergebnisses die 1er-\textbf{Stelle} der \textbf{Summe} der 1er-\textbf{Stellen} aller \textbf{Summanden}. Weitere \textbf{Stellen} werden an den \textbf{Stellen ihres Stellenwertes} vermerkt und beim nach links gehenden \textbf{Addieren} mit \textbf{addiert} (kleine Beispiele übertragen i.d.R. nur einen Zehner. Das ist die  \enquote{1}, die häufig auf dem Strich über dem Ergebnis erscheint. Bei \textbf{Summen} mit vielen \textbf{Summanden} kommen aber auch größere \textbf{Überträge} zustande.).

\begin{beispiel}[ausführliche Beschreibung schriftlich addieren]
    Wir addieren $12345$ und $6789$, $12345 + 6789$ = x (s. Abb. \ref{fig:schriftlichAddieren}, S. \pageref{fig:schriftlichAddieren}). Wir schreiben die beiden Summanden sauber ausgerichtet (1er unter 1er, 10er unter 10er, ...) untereinander.  Danach addieren wir Stellenweise, zuerst die 1er-Stellen: $9+5=14$. Wir schreiben die $4$ als 1er-Stelle des Ergebnisses auf und notieren die $1$ als Übertrag auf die 10er-Stelle. Für die 10er-Stelle des Ergebnisses addieren wir den Übertrag und die 10er-Stellen der Summanden: $1+8+4=13$. Die letzte Stelle der Summe der 10er-Stellen wird als 10er-Stelle des Ergebnisses notiert.\topicend
\end{beispiel}

\begin{beispiel}[ausführliche Beschreibung schriftlich addieren]
    Wir addieren eine ganze Reihe von Zahlen. Insbesondere beim Addieren (und Subtrahieren) von vielen Zahlen sagen wir auch dass wir \textbf{die Summe -n}\index{Summe}\index{summieren} bilden. Die Zahlen bezeichnen wir dann als \textbf{der Summand -en}\index{Summand}. Wir sagen: \enquote{Wir summieren die Summanden zur Summe.} oder \enquote{Wir bilden die Summe.}

    Wir bilden die Summe von \{83337, 47466, 43895, 18792, 68914, 36092, 644, 17669, 849, 825, 23559, 29410, 423, 29403, 69104, 44024, 220, 82771, 23239\} (s. Abb. \ref{fig:schriftlichAddieren}, S. \pageref{fig:schriftlichAddieren}):

    Wir schreiben alle Summanden sauber der Stellenwertigkeiten nach an den 1er-Stellen ausgerichtet untereinander. Wir summieren die Stellen jeweils einzeln von rechts nach links (und damit von den 1er-Stellen aufwärts). Die 1er-Stelle des Ergebnisses der Summe aller 1er-Stellen schreiben wir als 1er-Stelle des Ergebnisses auf. Die größeren Stellen der Summe der 1er-Stellen schreiben wir (klein als Merkhilfe) unter die entsprechenden Stellen der 10er-Stellen (oder bei sehr langen Listen von Summanden auch darüber hinaus).

    Die Summe der 1er-Stellen ist 85. Wir notieren die 5 als 1er-Stelle des Ergebnisses und die 8 als \textbf{Übertrag} zu den 10er-Stellen. Die Summe der 10er-Stellen und des \textbf{Übertrags} ist 83. Wir notieren die 3 als 10er-Stelle des Ergebnisses und die 8 als \textbf{Übertrag} zu den 100er-Stellen. Die Summe der 100er-Stellen und des Übertrags ist 96. Wir notieren die 6 als 100er-Stelle des Ergebnisses und die 9 als Übertrag zu den 1000er-Stellen. Die Summe der 1000er-Stellen und des Übertrags ist 90. Wir notieren die 0 als 1000er-Stelle des Übertrags und die 9 als Übertrag zu den 10000er-Stellen. Die Summe der 10000er-Stellen und des Übertrags ist 62. Wir notieren die 2 als 10000er-Stellen des Ergebnisses und die 6 als Übertrag zu den 100000er-Stellen. Die Summe der 100000er-Stellen und des Übertrags ist 6. Wir notieren die 6 als 100000er-Stelle des Ergebnisses und lesen das Ergebnis ab: 620635.\topicend
\end{beispiel}

Die Beispiele zeigen, dass schriftlich zu addieren bei Nutzung des karierten Papiers, ordentlicher Notation und etwas Geduld, mit 20 Summanden nicht schwieriger ist als mit zwei, nur längere Konzentration erfordert.

\begin{figure}
  \centering
  \begin{tikzpicture}
    %\draw[step=1mm, line width=0.1mm, black!30!white] (0,0) grid (\width,\hauteur);
    \draw[step=5mm, line width=0.1mm, black!40!white] (0,0) grid (\width,\hauteur);
    %\draw[step=5cm, line width=0.5mm, black!50!white] (0,0) grid (\width,\hauteur);
    %\draw[step=1cm, line width=0.3mm, black!90!white] (0,0) grid (\width,\hauteur);

    \node at (0.75cm, 11.75cm) {1};
    \node at (1.25cm, 11.75cm) {2};
    \node at (1.75cm, 11.75cm) {3};
    \node at (2.25cm, 11.75cm) {4};
    \node at (2.75cm, 11.75cm) {5};
    \node at (0.25cm, 11.25cm) {+};
    \node at (1.25cm, 11.25cm) {6};
    \node at (1.75cm, 11.25cm) {7};
    \node at (2.25cm, 11.25cm) {8};
    \node at (2.75cm, 11.25cm) {9};

    \draw (0.2cm,10.7cm) -- (3cm,10.7cm);

    \node at (2.75cm, 10.25cm) {4};
    \node at (2.35cm, 10.8cm) {\tiny 1};
    \node at (2.25cm, 10.25cm) {3};
    \node at (1.85cm, 10.8cm) {\tiny 1};
    \node at (1.75cm, 10.25cm) {1};
    \node at (1.35cm, 10.8cm) {\tiny 1};
    \node at (1.25cm, 10.25cm) {9};
    \node at (0.75cm, 10.25cm) {1};

    \node at (5.75cm, 11.75cm) {};
    \node at (6.25cm, 11.75cm) {8};
    \node at (6.75cm, 11.75cm) {3};
    \node at (7.25cm, 11.75cm) {3};
    \node at (7.75cm, 11.75cm) {3};
    \node at (8.25cm, 11.75cm) {7};

    \node at (5.25cm, 11.25cm) {+};
    \node at (5.75cm, 11.25cm) {};
    \node at (6.25cm, 11.25cm) {4};
    \node at (6.75cm, 11.25cm) {7};
    \node at (7.25cm, 11.25cm) {4};
    \node at (7.75cm, 11.25cm) {6};
    \node at (8.25cm, 11.25cm) {6};

    \node at (5.25cm, 10.75cm) {+};
    \node at (5.75cm, 10.75cm) {};
    \node at (6.25cm, 10.75cm) {4};
    \node at (6.75cm, 10.75cm) {3};
    \node at (7.25cm, 10.75cm) {8};
    \node at (7.75cm, 10.75cm) {9};
    \node at (8.25cm, 10.75cm) {5};

    \node at (5.25cm, 10.25cm) {+};
    \node at (5.75cm, 10.25cm) {};
    \node at (6.25cm, 10.25cm) {1};
    \node at (6.75cm, 10.25cm) {8};
    \node at (7.25cm, 10.25cm) {7};
    \node at (7.75cm, 10.25cm) {9};
    \node at (8.25cm, 10.25cm) {1};

    \node at (5.25cm, 9.75cm) {+};
    \node at (5.75cm, 9.75cm) {};
    \node at (6.25cm, 9.75cm) {6};
    \node at (6.75cm, 9.75cm) {8};
    \node at (7.25cm, 9.75cm) {9};
    \node at (7.75cm, 9.75cm) {1};
    \node at (8.25cm, 9.75cm) {4};

    \node at (5.25cm, 9.25cm) {+};
    \node at (5.75cm, 9.25cm) {};
    \node at (6.25cm, 9.25cm) {3};
    \node at (6.75cm, 9.25cm) {6};
    \node at (7.25cm, 9.25cm) {0};
    \node at (7.75cm, 9.25cm) {9};
    \node at (8.25cm, 9.25cm) {2};

    \node at (5.25cm, 8.75cm) {+};
    \node at (5.75cm, 8.75cm) {};
    \node at (6.25cm, 8.75cm) {};
    \node at (6.75cm, 8.75cm) {};
    \node at (7.25cm, 8.75cm) {6};
    \node at (7.75cm, 8.75cm) {4};
    \node at (8.25cm, 8.75cm) {4};

    \node at (5.25cm, 8.25cm) {+};
    \node at (5.75cm, 8.25cm) {};
    \node at (6.25cm, 8.25cm) {1};
    \node at (6.75cm, 8.25cm) {7};
    \node at (7.25cm, 8.25cm) {6};
    \node at (7.75cm, 8.25cm) {6};
    \node at (8.25cm, 8.25cm) {9};

    \node at (5.25cm, 7.75cm) {+};
    \node at (5.75cm, 7.75cm) {};
    \node at (6.25cm, 7.75cm) {};
    \node at (6.75cm, 7.75cm) {};
    \node at (7.25cm, 7.75cm) {8};
    \node at (7.75cm, 7.75cm) {4};
    \node at (8.25cm, 7.75cm) {9};

    \node at (5.25cm, 7.25cm) {+};
    \node at (5.75cm, 7.25cm) {};
    \node at (6.25cm, 7.25cm) {};
    \node at (6.75cm, 7.25cm) {};
    \node at (7.25cm, 7.25cm) {8};
    \node at (7.75cm, 7.25cm) {2};
    \node at (8.25cm, 7.25cm) {5};

    \node at (5.25cm, 6.75cm) {+};
    \node at (5.75cm, 6.75cm) {};
    \node at (6.25cm, 6.75cm) {2};
    \node at (6.75cm, 6.75cm) {3};
    \node at (7.25cm, 6.75cm) {5};
    \node at (7.75cm, 6.75cm) {5};
    \node at (8.25cm, 6.75cm) {9};

    \node at (5.25cm, 6.25cm) {+};
    \node at (5.75cm, 6.25cm) {};
    \node at (6.25cm, 6.25cm) {2};
    \node at (6.75cm, 6.25cm) {9};
    \node at (7.25cm, 6.25cm) {4};
    \node at (7.75cm, 6.25cm) {1};
    \node at (8.25cm, 6.25cm) {0};

    \node at (5.25cm, 5.75cm) {+};
    \node at (5.75cm, 5.75cm) {};
    \node at (6.25cm, 5.75cm) {};
    \node at (6.75cm, 5.75cm) {};
    \node at (7.25cm, 5.75cm) {4};
    \node at (7.75cm, 5.75cm) {2};
    \node at (8.25cm, 5.75cm) {3};

    \node at (5.25cm, 5.25cm) {+};
    \node at (5.75cm, 5.25cm) {};
    \node at (6.25cm, 5.25cm) {2};
    \node at (6.75cm, 5.25cm) {9};
    \node at (7.25cm, 5.25cm) {4};
    \node at (7.75cm, 5.25cm) {0};
    \node at (8.25cm, 5.25cm) {3};

    \node at (5.25cm, 4.75cm) {+};
    \node at (5.75cm, 4.75cm) {};
    \node at (6.25cm, 4.75cm) {6};
    \node at (6.75cm, 4.75cm) {9};
    \node at (7.25cm, 4.75cm) {1};
    \node at (7.75cm, 4.75cm) {0};
    \node at (8.25cm, 4.75cm) {4};

    \node at (5.25cm, 4.25cm) {+};
    \node at (5.75cm, 4.25cm) {};
    \node at (6.25cm, 4.25cm) {4};
    \node at (6.75cm, 4.25cm) {4};
    \node at (7.25cm, 4.25cm) {0};
    \node at (7.75cm, 4.25cm) {2};
    \node at (8.25cm, 4.25cm) {4};

    \node at (5.25cm, 3.75cm) {+};
    \node at (5.75cm, 3.75cm) {};
    \node at (6.25cm, 3.75cm) {};
    \node at (6.75cm, 3.75cm) {};
    \node at (7.25cm, 3.75cm) {2};
    \node at (7.75cm, 3.75cm) {2};
    \node at (8.25cm, 3.75cm) {0};

    \node at (5.25cm, 3.25cm) {+};
    \node at (5.75cm, 3.25cm) {};
    \node at (6.25cm, 3.25cm) {8};
    \node at (6.75cm, 3.25cm) {2};
    \node at (7.25cm, 3.25cm) {7};
    \node at (7.75cm, 3.25cm) {7};
    \node at (8.25cm, 3.25cm) {1};

    \node at (5.25cm, 2.75cm) {+};
    \node at (5.75cm, 2.75cm) {};
    \node at (6.25cm, 2.75cm) {2};
    \node at (6.75cm, 2.75cm) {3};
    \node at (7.25cm, 2.75cm) {2};
    \node at (7.75cm, 2.75cm) {3};
    \node at (8.25cm, 2.75cm) {9};

    \draw (5.2cm,2.2cm) -- (9cm,2.2cm);

    \node at (5.75cm, 1.75cm) {6};
    \node at (5.8cm, 2.3cm) {\tiny 6};
    \node at (6.25cm, 1.75cm) {2};
    \node at (6.3cm, 2.3cm) {\tiny 9};
    \node at (6.75cm, 1.75cm) {0};
    \node at (6.8cm, 2.3cm) {\tiny 9};
    \node at (7.25cm, 1.75cm) {6};
    \node at (7.3cm, 2.3cm) {\tiny 8};
    \node at (7.75cm, 1.75cm) {3};
    \node at (7.8cm, 2.3cm) {\tiny 8};
    \node at (8.25cm, 1.75cm) {5};//85

  \end{tikzpicture}
  \caption{schriftlich addieren}\label{fig:schriftlichAddieren}
\end{figure}

\begin{figure}
  \centering
  \begin{tikzpicture}
        \draw[step=5mm, line width=0.1mm, black!40!white] (0,0) grid (\width,\hauteur/2);

        \node at (3.25,5.25) {2};
        \node at (3.5,5.15) {,};
        \node at (3.75,5.25) {3};
        \node at (4.25,5.25) {4};

        \node at (1.25,4.75) {+};
        \node at (3.25,4.75) {0};
        \node at (3.5,4.65) {,};
        \node at (3.75,4.75) {0};
        \node at (4.25,4.75) {0};
        \node at (4.75,4.75) {8};

        \node at (1.25,4.25) {+};
        \node at (2.25,4.25) {1};
        \node at (2.75,4.25) {2};
        \node at (3.25,4.25) {0};
        \node at (3.5,4.15) {};

        \node at (1.25,3.75) {+};
        \node at (3.25,3.75) {7};
        \node at (3.5,3.65) {,};
        \node at (3.75,3.75) {8};
        \node at (4.25,3.75) {8};

        \draw (1.25,3.2) -- (5.25,3.2);

        \node at (4.75,2.75) {8};
        \node at (4.25,2.75) {2};
        \node at (3.8cm, 3.3cm) {\tiny 1};
        \node at (3.75,2.75) {2};
        \node at (3.3cm, 3.3cm) {\tiny 1};
        \node at (3.5,2.65) {,};
        \node at (3.25,2.75) {0};
        \node at (2.8cm, 3.3cm) {\tiny 1};
        \node at (2.75,2.75) {3};
        \node at (2.25,2.75) {1};

        \node at (8.75,5.25) {4};
        \node at (9.25,5.25) {5};
        \node at (9.5,5.15) {,};
        \node at (9.75,5.25) {0};
        \node at (10.25,5.25) {0};
        \node at (10.75,5.25) {2};
        \node at (11.25,5.25) {3};

        \node at (6.75,4.75) {+};
        \node at (7.75,4.75) {1};
        \node at (8.25,4.75) {0};
        \node at (8.75,4.75) {0};
        \node at (9.25,4.75) {7};
        \node at (9.5,4.65) {,};
        \node at (9.75,4.75) {7};
        \node at (10.25,4.75) {8};

        \node at (6.75,4.25) {+};
        \node at (7.75,4.25) {};
        \node at (8.25,4.25) {9};
        \node at (8.75,4.25) {8};
        \node at (9.25,4.25) {7};
        \node at (9.5,4.15) {,};
        \node at (9.75,4.25) {7};
        \node at (10.25,4.25) {6};
        \node at (10.75,4.25) {5};
        \node at (11.25,4.25) {9};

        \node at (6.75,3.75) {+};
        \node at (7.75,3.75) {6};
        \node at (8.25,3.75) {7};
        \node at (8.75,3.75) {8};
        \node at (9.25,3.75) {9};
        \node at (9.5,3.65) {,};
        \node at (9.75,3.75) {9};
        \node at (10.25,3.75) {6};
        \node at (10.75,3.75) {3};
        \node at (11.25,3.75) {};

        \node at (6.75,3.25) {+};
        \node at (7.75,3.25) {};
        \node at (8.25,3.25) {};
        \node at (8.75,3.25) {9};
        \node at (9.25,3.25) {9};
        \node at (9.5,3.15) {,};
        \node at (9.75,3.25) {7};
        \node at (10.25,3.25) {6};
        \node at (10.75,3.25) {5};
        \node at (11.25,3.25) {2};
        \node at (11.75,3.25) {9};

        \node at (6.75,2.75) {+};
        \node at (7.75,2.75) {};
        \node at (8.25,2.75) {5};
        \node at (8.75,2.75) {7};
        \node at (9.25,2.75) {9};
        \node at (9.5,2.65) {,};
        \node at (9.75,2.75) {8};
        \node at (10.25,2.75) {6};
        \node at (10.75,2.75) {4};
        \node at (11.25,2.75) {4};
        \node at (11.75,2.75) {2};

        \node at (6.75,2.25) {+};
        \node at (7.75,2.25) {};
        \node at (8.25,2.25) {};
        \node at (8.75,2.25) {};
        \node at (9.25,2.25) {7};
        \node at (9.5,2.15) {,};
        \node at (9.75,2.25) {8};
        \node at (10.25,2.25) {6};
        \node at (10.75,2.25) {2};
        \node at (11.25,2.25) {3};
        \node at (11.75,2.25) {};

        \draw (6.75,1.7) -- (12,1.7);

        \node at (7.75,1.25) {9};
        \node at (7.8,1.8) {\tiny 2};
        \node at (8.25,1.25) {5};
        \node at (8.3,1.8) {\tiny 4};
        \node at (8.75,1.25) {1};
        \node at (8.8,1.8) {\tiny 5};
        \node at (9.25,1.25) {8};
        \node at (9.3,1.8) {\tiny 5};
        \node at (9.5,1.15) {,};
        \node at (9.75,1.25) {0};
        \node at (9.8,1.8) {\tiny 4};
        \node at (10.25,1.25) {0};
        \node at (10.3,1.8) {\tiny 2};
        \node at (10.75,1.25) {3};
        \node at (10.8,1.8) {\tiny 2};
        \node at (11.25,1.25) {2};
        \node at (11.3,1.8) {\tiny 1};
        \node at (11.75,1.25) {1};
  \end{tikzpicture}
  \caption{schriftlich addieren von Dezimalbrüchen}\label{fig:schriftlichAddierenDezimalbrueche}
\end{figure}


\section{Schriftliche Subtraktion}\index{Subtraktion!schriftliche}

Wir schreiben $3-2=1$ und lesen \enquote{drei minus zwei (ist) gleich 1}. Eine negative Zahl $-x$ zu subtrahieren bedeutet ihren Betrag zu addieren: $3 - (-2) = 5$\footnote{Diese Schreibweise (und noch mehr $3--2$) sollte vermieden werden.} \enquote{Minus minus ist plus.} Wenn man in der Vorstellung des Zahlenstrahls bleiben möchte, dann kann man das Minus als Richtungswechsel sehen und zwei mal 180 Grad ergibt wieder die ursprüngliche Richtung. In \glsdisp{symb:Sum}{Summen} können positive und negative Summanden gemischt vorkommen. Um eine Summe aus Summanden mit unterschiedlichen Vorzeichen\index{Vorzeichen} zu bilden fassen wir zunächst die positiven und negativen Summanden getrennt zusammen. Wollen wir die Summe $\Sigma = -123 +456 -789 +321 -124 +34 -123 +127 -2009$ berechnen bilden wir zunächst die Summen $\Sigma_{\text{+}}$ der positiven Summanden und $\Sigma_{\text{-}}$ der negativen Summanden und bilden anschließend die Differenz der beiden Teilsummen. $\Sigma_{\text{+}} = 456 + 321 + 34 + 123 = 934$. $\Sigma_{\text{-}} = -(123 + 789 + 124 + 127) = -1163$.


\begin{figure}
  \centering
  \begin{tikzpicture}
    %\draw[step=1mm, line width=0.1mm, black!30!white] (0,0) grid (\width,\hauteur);
    \draw[step=5mm, line width=0.1mm, black!40!white] (0,0) grid (\width,\hauteur);
    %\draw[step=5cm, line width=0.5mm, black!50!white] (0,0) grid (\width,\hauteur);
    %\draw[step=1cm, line width=0.3mm, black!90!white] (0,0) grid (\width,\hauteur);

    \node at (0.75cm, 11.75cm) {1};
    \node at (1.25cm, 11.75cm) {2};
    \node at (1.75cm, 11.75cm) {3};
    \node at (2.25cm, 11.75cm) {4};
    \node at (2.75cm, 11.75cm) {5};
    \node at (0.25cm, 11.25cm) {-};
    \node at (1.25cm, 11.25cm) {6};
    \node at (1.75cm, 11.25cm) {7};
    \node at (2.25cm, 11.25cm) {8};
    \node at (2.75cm, 11.25cm) {9};

    \draw (0.2cm,10.7cm) -- (3cm,10.7cm);

    \node at (2.75cm, 10.25cm) {6};
    \node at (2.35cm, 10.8cm) {\tiny 1};
    \node at (2.25cm, 10.25cm) {5};
    \node at (1.85cm, 10.8cm) {\tiny 1};
    \node at (1.75cm, 10.25cm) {5};
    \node at (1.35cm, 10.8cm) {\tiny 1};
    \node at (1.25cm, 10.25cm) {5};
    \node at (0.85cm, 10.8cm) {\tiny 1};
    \node at (0.75cm, 10.25cm) {};

    \node at (4.75cm, 11.75cm) {};
    \node at (5.25cm, 11.75cm) {8};
    \node at (5.75cm, 11.75cm) {3};
    \node at (6.25cm, 11.75cm) {3};
    \node at (6.75cm, 11.75cm) {3};
    \node at (7.25cm, 11.75cm) {7};

    \node at (4.25cm, 11.25cm) {-};
    \node at (4.75cm, 11.25cm) {};
    \node at (5.25cm, 11.25cm) {4};
    \node at (5.75cm, 11.25cm) {7};
    \node at (6.25cm, 11.25cm) {4};
    \node at (6.75cm, 11.25cm) {6};
    \node at (7.25cm, 11.25cm) {6};

    \node at (4.25cm, 10.75cm) {-};
    \node at (4.75cm, 10.75cm) {};
    \node at (5.25cm, 10.75cm) {4};
    \node at (5.75cm, 10.75cm) {3};
    \node at (6.25cm, 10.75cm) {8};
    \node at (6.75cm, 10.75cm) {9};
    \node at (7.25cm, 10.75cm) {5};

    \node at (4.25cm, 10.25cm) {-};
    \node at (4.75cm, 10.25cm) {};
    \node at (5.25cm, 10.25cm) {1};
    \node at (5.75cm, 10.25cm) {8};
    \node at (6.25cm, 10.25cm) {7};
    \node at (6.75cm, 10.25cm) {9};
    \node at (7.25cm, 10.25cm) {1};

    \node at (4.25cm, 9.75cm) {-};
    \node at (4.75cm, 9.75cm) {};
    \node at (5.25cm, 9.75cm) {6};
    \node at (5.75cm, 9.75cm) {8};
    \node at (6.25cm, 9.75cm) {9};
    \node at (6.75cm, 9.75cm) {1};
    \node at (7.25cm, 9.75cm) {4};

    \node at (4.25cm, 9.25cm) {-};
    \node at (4.75cm, 9.25cm) {};
    \node at (5.25cm, 9.25cm) {3};
    \node at (5.75cm, 9.25cm) {6};
    \node at (6.25cm, 9.25cm) {0};
    \node at (6.75cm, 9.25cm) {9};
    \node at (7.25cm, 9.25cm) {2};

    \node at (4.25cm, 8.75cm) {-};
    \node at (4.75cm, 8.75cm) {};
    \node at (5.25cm, 8.75cm) {};
    \node at (5.75cm, 8.75cm) {};
    \node at (6.25cm, 8.75cm) {6};
    \node at (6.75cm, 8.75cm) {4};
    \node at (7.25cm, 8.75cm) {4};

    \node at (4.25cm, 8.25cm) {-};
    \node at (4.75cm, 8.25cm) {};
    \node at (5.25cm, 8.25cm) {1};
    \node at (5.75cm, 8.25cm) {7};
    \node at (6.25cm, 8.25cm) {6};
    \node at (6.75cm, 8.25cm) {6};
    \node at (7.25cm, 8.25cm) {9};

    \node at (4.25cm, 7.75cm) {-};
    \node at (4.75cm, 7.75cm) {};
    \node at (5.25cm, 7.75cm) {};
    \node at (5.75cm, 7.75cm) {};
    \node at (6.25cm, 7.75cm) {8};
    \node at (6.75cm, 7.75cm) {4};
    \node at (7.25cm, 7.75cm) {9};

    \node at (4.25cm, 7.25cm) {-};
    \node at (4.75cm, 7.25cm) {};
    \node at (5.25cm, 7.25cm) {};
    \node at (5.75cm, 7.25cm) {};
    \node at (6.25cm, 7.25cm) {8};
    \node at (6.75cm, 7.25cm) {2};
    \node at (7.25cm, 7.25cm) {5};

    \node at (4.25cm, 6.75cm) {-};
    \node at (4.75cm, 6.75cm) {};
    \node at (5.25cm, 6.75cm) {2};
    \node at (5.75cm, 6.75cm) {3};
    \node at (6.25cm, 6.75cm) {5};
    \node at (6.75cm, 6.75cm) {5};
    \node at (7.25cm, 6.75cm) {9};

    \node at (4.25cm, 6.25cm) {-};
    \node at (4.75cm, 6.25cm) {};
    \node at (5.25cm, 6.25cm) {2};
    \node at (5.75cm, 6.25cm) {9};
    \node at (6.25cm, 6.25cm) {4};
    \node at (6.75cm, 6.25cm) {1};
    \node at (7.25cm, 6.25cm) {0};

    \node at (4.25cm, 5.75cm) {-};
    \node at (4.75cm, 5.75cm) {};
    \node at (5.25cm, 5.75cm) {};
    \node at (5.75cm, 5.75cm) {};
    \node at (6.25cm, 5.75cm) {4};
    \node at (6.75cm, 5.75cm) {2};
    \node at (7.25cm, 5.75cm) {3};

    \node at (4.25cm, 5.25cm) {-};
    \node at (4.75cm, 5.25cm) {};
    \node at (5.25cm, 5.25cm) {2};
    \node at (5.75cm, 5.25cm) {9};
    \node at (6.25cm, 5.25cm) {4};
    \node at (6.75cm, 5.25cm) {0};
    \node at (7.25cm, 5.25cm) {3};

    \node at (4.25cm, 4.75cm) {-};
    \node at (4.75cm, 4.75cm) {};
    \node at (5.25cm, 4.75cm) {6};
    \node at (5.75cm, 4.75cm) {9};
    \node at (6.25cm, 4.75cm) {1};
    \node at (6.75cm, 4.75cm) {0};
    \node at (7.25cm, 4.75cm) {4};

    \node at (4.25cm, 4.25cm) {-};
    \node at (4.75cm, 4.25cm) {};
    \node at (5.25cm, 4.25cm) {4};
    \node at (5.75cm, 4.25cm) {4};
    \node at (6.25cm, 4.25cm) {0};
    \node at (6.75cm, 4.25cm) {2};
    \node at (7.25cm, 4.25cm) {4};

    \node at (4.25cm, 3.75cm) {-};
    \node at (4.75cm, 3.75cm) {};
    \node at (5.25cm, 3.75cm) {};
    \node at (5.75cm, 3.75cm) {};
    \node at (6.25cm, 3.75cm) {2};
    \node at (6.75cm, 3.75cm) {2};
    \node at (7.25cm, 3.75cm) {0};

    \node at (4.25cm, 3.25cm) {-};
    \node at (4.75cm, 3.25cm) {};
    \node at (5.25cm, 3.25cm) {8};
    \node at (5.75cm, 3.25cm) {2};
    \node at (6.25cm, 3.25cm) {7};
    \node at (6.75cm, 3.25cm) {7};
    \node at (7.25cm, 3.25cm) {1};

    \node at (4.25cm, 2.75cm) {-};
    \node at (4.75cm, 2.75cm) {};
    \node at (5.25cm, 2.75cm) {2};
    \node at (5.75cm, 2.75cm) {3};
    \node at (6.25cm, 2.75cm) {2};
    \node at (6.75cm, 2.75cm) {3};
    \node at (7.25cm, 2.75cm) {9};

    \draw (4.2cm,2.2cm) -- (8cm,2.2cm);

    \node at (4.75cm, 1.75cm) {};
    \node at (4.8cm, 2.3cm) {\tiny };
    \node at (5.25cm, 1.75cm) {};
    \node at (5.3cm, 2.3cm) {\tiny };
    \node at (5.75cm, 1.75cm) {};
    \node at (5.8cm, 2.3cm) {\tiny };
    \node at (6.25cm, 1.75cm) {};
    \node at (6.3cm, 2.3cm) {\tiny };
    \node at (6.75cm, 1.75cm) {};
    \node at (6.8cm, 2.3cm) {\tiny };
    \node at (7.25cm, 1.75cm) {?};

    \node at (8.75cm, 11.75cm) {};
    \node at (9.25cm, 11.75cm) {8};
    \node at (9.75cm, 11.75cm) {3};
    \node at (10.25cm, 11.75cm) {3};
    \node at (10.75cm, 11.75cm) {3};
    \node at (11.25cm, 11.75cm) {7};

    \node at (8.25cm, 11.25cm) {-};
    \node at (8.75cm, 11.25cm) {(};
    \node at (8.75cm, 11.25cm) {};
    \node at (9.25cm, 11.25cm) {4};
    \node at (9.75cm, 11.25cm) {7};
    \node at (10.25cm, 11.25cm) {4};
    \node at (10.75cm, 11.25cm) {6};
    \node at (11.25cm, 11.25cm) {6};

    \node at (8.25cm, 10.75cm) {+};
    \node at (8.75cm, 10.75cm) {};
    \node at (9.25cm, 10.75cm) {4};
    \node at (9.75cm, 10.75cm) {3};
    \node at (10.25cm, 10.75cm) {8};
    \node at (10.75cm, 10.75cm) {9};
    \node at (11.25cm, 10.75cm) {5};

    \node at (8.25cm, 10.25cm) {+};
    \node at (8.75cm, 10.25cm) {};
    \node at (9.25cm, 10.25cm) {1};
    \node at (9.75cm, 10.25cm) {8};
    \node at (10.25cm, 10.25cm) {7};
    \node at (10.75cm, 10.25cm) {9};
    \node at (11.25cm, 10.25cm) {1};

    \node at (8.25cm, 9.75cm) {+};
    \node at (8.75cm, 9.75cm) {};
    \node at (9.25cm, 9.75cm) {6};
    \node at (9.75cm, 9.75cm) {8};
    \node at (10.25cm, 9.75cm) {9};
    \node at (10.75cm, 9.75cm) {1};
    \node at (11.25cm, 9.75cm) {4};

    \node at (8.25cm, 9.25cm) {+};
    \node at (8.75cm, 9.25cm) {};
    \node at (9.25cm, 9.25cm) {3};
    \node at (9.75cm, 9.25cm) {6};
    \node at (10.25cm, 9.25cm) {0};
    \node at (10.75cm, 9.25cm) {9};
    \node at (11.25cm, 9.25cm) {2};

    \node at (8.25cm, 8.75cm) {+};
    \node at (8.75cm, 8.75cm) {};
    \node at (9.25cm, 8.75cm) {};
    \node at (9.75cm, 8.75cm) {};
    \node at (10.25cm, 8.75cm) {6};
    \node at (10.75cm, 8.75cm) {4};
    \node at (11.25cm, 8.75cm) {4};

    \node at (8.25cm, 8.25cm) {+};
    \node at (8.75cm, 8.25cm) {};
    \node at (9.25cm, 8.25cm) {1};
    \node at (9.75cm, 8.25cm) {7};
    \node at (10.25cm, 8.25cm) {6};
    \node at (10.75cm, 8.25cm) {6};
    \node at (11.25cm, 8.25cm) {9};

    \node at (8.25cm, 7.75cm) {+};
    \node at (8.75cm, 7.75cm) {};
    \node at (9.25cm, 7.75cm) {};
    \node at (9.75cm, 7.75cm) {};
    \node at (10.25cm, 7.75cm) {8};
    \node at (10.75cm, 7.75cm) {4};
    \node at (11.25cm, 7.75cm) {9};

    \node at (8.25cm, 7.25cm) {+};
    \node at (8.75cm, 7.25cm) {};
    \node at (9.25cm, 7.25cm) {};
    \node at (9.75cm, 7.25cm) {};
    \node at (10.25cm, 7.25cm) {8};
    \node at (10.75cm, 7.25cm) {2};
    \node at (11.25cm, 7.25cm) {5};

    \node at (8.25cm, 6.75cm) {+};
    \node at (8.75cm, 6.75cm) {};
    \node at (9.25cm, 6.75cm) {2};
    \node at (9.75cm, 6.75cm) {3};
    \node at (10.25cm, 6.75cm) {5};
    \node at (10.75cm, 6.75cm) {5};
    \node at (11.25cm, 6.75cm) {9};

    \node at (8.25cm, 6.25cm) {+};
    \node at (8.75cm, 6.25cm) {};
    \node at (9.25cm, 6.25cm) {2};
    \node at (9.75cm, 6.25cm) {9};
    \node at (10.25cm, 6.25cm) {4};
    \node at (10.75cm, 6.25cm) {1};
    \node at (11.25cm, 6.25cm) {0};

    \node at (8.25cm, 5.75cm) {+};
    \node at (8.75cm, 5.75cm) {};
    \node at (9.25cm, 5.75cm) {};
    \node at (9.75cm, 5.75cm) {};
    \node at (10.25cm, 5.75cm) {4};
    \node at (10.75cm, 5.75cm) {2};
    \node at (11.25cm, 5.75cm) {3};

    \node at (8.25cm, 5.25cm) {+};
    \node at (8.75cm, 5.25cm) {};
    \node at (9.25cm, 5.25cm) {2};
    \node at (9.75cm, 5.25cm) {9};
    \node at (10.25cm, 5.25cm) {4};
    \node at (10.75cm, 5.25cm) {0};
    \node at (11.25cm, 5.25cm) {3};

    \node at (8.25cm, 4.75cm) {+};
    \node at (8.75cm, 4.75cm) {};
    \node at (9.25cm, 4.75cm) {6};
    \node at (9.75cm, 4.75cm) {9};
    \node at (10.25cm, 4.75cm) {1};
    \node at (10.75cm, 4.75cm) {0};
    \node at (11.25cm, 4.75cm) {4};

    \node at (8.25cm, 4.25cm) {+};
    \node at (8.75cm, 4.25cm) {};
    \node at (9.25cm, 4.25cm) {4};
    \node at (9.75cm, 4.25cm) {4};
    \node at (10.25cm, 4.25cm) {0};
    \node at (10.75cm, 4.25cm) {2};
    \node at (11.25cm, 4.25cm) {4};

    \node at (8.25cm, 3.75cm) {+};
    \node at (8.75cm, 3.75cm) {};
    \node at (9.25cm, 3.75cm) {};
    \node at (9.75cm, 3.75cm) {};
    \node at (10.25cm, 3.75cm) {2};
    \node at (10.75cm, 3.75cm) {2};
    \node at (11.25cm, 3.75cm) {0};

    \node at (8.25cm, 3.25cm) {+};
    \node at (8.75cm, 3.25cm) {};
    \node at (9.25cm, 3.25cm) {8};
    \node at (9.75cm, 3.25cm) {2};
    \node at (10.25cm, 3.25cm) {7};
    \node at (10.75cm, 3.25cm) {7};
    \node at (11.25cm, 3.25cm) {1};

    \node at (8.25cm, 2.75cm) {+};
    \node at (8.75cm, 2.75cm) {};
    \node at (9.25cm, 2.75cm) {2};
    \node at (9.75cm, 2.75cm) {3};
    \node at (10.25cm, 2.75cm) {2};
    \node at (10.75cm, 2.75cm) {3};
    \node at (11.25cm, 2.75cm) {9};
    \node at (11.75cm, 2.75cm) {)};

    \draw (8.2cm,2.2cm) -- (11.9cm,2.2cm);

    \node at (8.75cm, 1.75cm) {5};
    \node at (8.8cm, 2.3cm) {\tiny 5};
    \node at (9.25cm, 1.75cm) {3};
    \node at (9.3cm, 2.3cm) {\tiny 8};
    \node at (9.75cm, 1.75cm) {7};
    \node at (9.8cm, 2.3cm) {\tiny 9};
    \node at (10.25cm, 1.75cm) {2};
    \node at (10.3cm, 2.3cm) {\tiny 7};
    \node at (10.75cm, 1.75cm) {9};
    \node at (10.8cm, 2.3cm) {\tiny 7};
    \node at (11.25cm, 1.75cm) {8};

    \node at (8.25cm, 1.25cm) {-};
    \node at (8.75cm, 1.25cm) {};
    \node at (9.25cm, 1.25cm) {8};
    \node at (9.75cm, 1.25cm) {3};
    \node at (10.25cm, 1.25cm) {3};
    \node at (10.75cm, 1.25cm) {3};
    \node at (11.25cm, 1.25cm) {7};

    \draw (8.2cm,0.7cm) -- (11.9cm,0.7cm);

    \node at (8.25cm, 0.25cm) {-};
    \node at (8.75cm, 0.25cm) {4};
    \node at (8.8cm, 0.8cm) {\tiny 1};
    \node at (9.25cm, 0.25cm) {5};
    \node at (9.3cm, 0.8cm) {\tiny };
    \node at (9.75cm, 0.25cm) {3};
    \node at (9.8cm, 0.8cm) {\tiny 1};
    \node at (10.25cm, 0.25cm) {9};
    \node at (10.3cm, 0.8cm) {\tiny };
    \node at (10.75cm, 0.25cm) {6};
    \node at (10.8cm, 0.8cm) {\tiny };
    \node at (11.25cm, 0.25cm) {1};

  \end{tikzpicture}
  \caption{schriftlich subtrahieren}\label{fig:schriftlichSubtrahieren}
\end{figure}

\section{Schriftliche Multiplikation}\index{Multiplikation!schriftliche}

\begin{figure}
  \centering
  \begin{tikzpicture}
    %\draw[step=1mm, line width=0.1mm, black!30!white] (0,0) grid (\width,\hauteur);
    \draw[step=5mm, line width=0.1mm, black!40!white] (0,0) grid (\width,\hauteur/2);
    %\draw[step=5cm, line width=0.5mm, black!50!white] (0,0) grid (\width,\hauteur);
    %\draw[step=1cm, line width=0.3mm, black!90!white] (0,0) grid (\width,\hauteur);

    \node at (1.25cm, 5.25cm) {1};
    \node at (1.75cm, 5.25cm) {2};
    \node at (2.25cm, 5.25cm) {3};
    \node at (2.75cm, 5.25cm) {4};
    \node at (3.25cm, 5.25cm) {5};
    \node at (3.75cm, 5.25cm) {$\cdot$};
    \node at (4.25cm, 5.25cm) {6};
    \node at (4.75cm, 5.25cm) {7};
    \node at (5.25cm, 5.25cm) {8};
    \node at (5.75cm, 5.25cm) {9};
    \node at (6.25cm, 5.25cm) {=};
    \node at (6.75cm, 5.25cm) {8};
    \node at (7.25cm, 5.25cm) {3};
    \node at (7.75cm, 5.25cm) {8};
    \node at (8.25cm, 5.25cm) {1};
    \node at (8.75cm, 5.25cm) {0};
    \node at (9.25cm, 5.25cm) {2};
    \node at (9.75cm, 5.25cm) {0};
    \node at (10.25cm, 5.25cm) {5};

    \draw (0.2cm,4.75cm) -- (6.2cm,4.75cm);
    \node at (1.25cm, 3.75cm) {+};
    \node at (1.25cm, 3.25cm) {+};
    \node at (1.25cm, 2.75cm) {+};

    \node at (2.25cm, 4.25cm) {7};
    \node at (2.75cm, 4.25cm) {4};
    \node at (3.25cm, 4.25cm) {0};
    \node at (3.75cm, 4.25cm) {7};
    \node at (4.25cm, 4.25cm) {0};

    \node at (2.75cm, 3.75cm) {8};
    \node at (3.25cm, 3.75cm) {6};
    \node at (3.75cm, 3.75cm) {4};
    \node at (4.25cm, 3.75cm) {1};
    \node at (4.75cm, 3.75cm) {5};

    \node at (3.25cm, 3.25cm) {9};
    \node at (3.75cm, 3.25cm) {8};
    \node at (4.25cm, 3.25cm) {7};
    \node at (4.75cm, 3.25cm) {6};
    \node at (5.25cm, 3.25cm) {0};

    \node at (3.25cm, 2.75cm) {1};
    \node at (3.75cm, 2.75cm) {1};
    \node at (4.25cm, 2.75cm) {1};
    \node at (4.75cm, 2.75cm) {1};
    \node at (5.25cm, 2.75cm) {0};
    \node at (5.75cm, 2.75cm) {5};

    \draw (0.2cm,2.2cm) -- (6.2cm,2.2cm);

    \node at (2.35cm, 2.3cm) {\tiny 1};
    \node at (2.25cm, 1.75cm) {8};
    \node at (2.85cm, 2.3cm) {\tiny 1};
    \node at (2.75cm, 1.75cm) {3};
    \node at (3.35cm, 2.3cm) {\tiny 2};
    \node at (3.25cm, 1.75cm) {8};
    \node at (3.85cm, 2.3cm) {\tiny 1};
    \node at (3.75cm, 1.75cm) {1};
    \node at (4.35cm, 2.3cm) {\tiny 1};
    \node at (4.25cm, 1.75cm) {0};
    \node at (4.75cm, 1.75cm) {2};
    \node at (5.25cm, 1.75cm) {0};
    \node at (5.75cm, 1.75cm) {5};

  \end{tikzpicture}
  \caption{schriftlich multiplizieren}\label{fig:schriftlichMultiplizieren}
\end{figure}

\begin{beispiel}[Multiplikation mit Dezimalbrüchen]
\begin{figure}
  \centering
  \begin{tikzpicture}
        \draw[step=5mm, line width=0.1mm, black!40!white] (0,0) grid (\width,\hauteur/2);
        \node at (0.75,5.75) {1};
        \node at (1.25,5.75) {5};
        \node at (1.5,5.65) {,};
        \node at (1.75,5.75) {7};
        \node at (2.25,5.75) {3};
        \node at (2.75,5.75) {$\times$};
        \node at (3.25,5.75) {2};
        \node at (3.5,5.65) {,};
        \node at (3.75,5.75) {7};
        \node at (4.25,5.75) {1};

        \node at (5.75,5.75) {=};

        \node at (6.75,5.75) {1};
        \node at (7.25,5.75) {5};
        \node at (7.75,5.65) {,};
        \node at (8.25,5.75) {7};
        \node at (8.75,5.75) {3};
        \node at (9.25,5.75) {$\times$};
        \node at (9.75,5.75) {2};
        \node at (6.25,5.25) {+};
        \node at (6.75,5.25) {1};
        \node at (7.25,5.25) {5};
        \node at (7.75,5.15) {,};
        \node at (8.25,5.25) {7};
        \node at (8.75,5.25) {3};
        \node at (9.25,5.25) {$\times$};
        \node at (9.75,5.25) {0};
        \node at (10.25,5.15) {,};
        \node at (10.75,5.25) {7};
        \node at (6.25,4.75) {+};
        \node at (6.75,4.75) {1};
        \node at (7.25,4.75) {5};
        \node at (7.75,4.65) {,};
        \node at (8.25,4.75) {7};
        \node at (8.75,4.75) {3};
        \node at (9.25,4.75) {$\times$};
        \node at (9.75,4.75) {0};
        \node at (10.25,4.65) {,};
        \node at (10.75,4.75) {0};
        \node at (11.25,4.75) {1};

        \node at (1.75,4.75) {3};
        \node at (2.25,4.75) {1};
        \node at (2.5,4.65) {,};
        \node at (2.75,4.75) {4};
        \node at (3.25,4.75) {6};

        \node at (1.75,4.25) {1};
        \node at (2.25,4.25) {1};
        \node at (2.5,4.15) {,};
        \node at (2.75,4.25) {0};
        \node at (3.25,4.25) {1};
        \node at (3.75,4.25) {1};

        \node at (2.25,3.75) {0};
        \node at (2.5,3.65) {,};
        \node at (2.75,3.75) {1};
        \node at (3.25,3.75) {5};
        \node at (3.75,3.75) {7};
        \node at (4.25,3.75) {3};

        \draw (0.25,3.25) -- (7.75,3.25);

        \node at (1.75,2.75) {4};
        \node at (2.25,2.75) {2};
        \node at (2.5,2.65) {,};
        \node at (2.75,2.75) {6};
        \node at (2.85,3.35) {\tiny{1}};
        \node at (3.25,2.75) {2};
        \node at (3.75,2.75) {8};
        \node at (4.25,2.75) {3};

  \end{tikzpicture}
  \caption{Multiplikation mit Dezimalbrüchen}\label{fig:Dezimalsystem02}
\end{figure}

Wir rechnen ein Produkt aus zwei Faktoren mit Kommata aus, \enquote{Dezimalbrüche}. $\prod = a \cdot b$, $a=15,73$, $b=2,71$. Wir schreiben die Faktoren ordentlich auf, so dass wir eine Spalte im karierten Papier pro Stelle haben (ob das Komma [interessant ist für die Nutzung der Spalten nur der zweite Faktor] eine eigene Spalte bekommt ist Geschmackssache. Wir schauen uns die Zahlen genauer an um sowohl etwas über das (10er-) Stellenwertsystem zu lernen, was bei der Prozentrechnung sehr hilfreich sein wird, als auch einen Schritt in Richtung Überschlagsrechnen zu gehen. Mit Prozent oder Hundertstel schätzt sich einfacher als mit Komma-Zahlen. $15,73 \cdot 100 = 1573$. $2,71 \cdot 100 = 271$. Also $15,73=\frac{1573}{100}$ und $2,71=\frac{271}{100}$. In der Bruchrechnung lernen wir dass zwei Brüche $a, b$ miteinander multipliziert werden indem $\frac{p}{q} = \frac{z_a}{n_a} \cdot \frac{z_b}{n_b} = \frac{z_a \cdot z_b}{n_a \cdot n_b}$, also die Zähler miteinander multipliziert werden und die Nenner miteinander multipliziert werden. $\frac{1573}{100} \cdot \frac{271}{100} = \frac{1573 \cdot 271}{100 \cdot 100}$. Also ist $15,73 \cdot 2,71 = \frac{1573 \cdot 271}{100 \cdot 100} = \frac{426.283}{10.000}$. Wir erwarten also dass es als Dezimalbruch $15,73 \cdot 2,71 = 42,6283$ ist.\end{beispiel}



\section{Schriftliche Division}\index{Division!schriftliche}

Wir schreiben die Aufgabe sauber auf kariertes Papier. Jede Ziffer hat ihr eigenes Kästchen.


\begin{beispiel}[Division von Dezimalbrüchen]
    Wir dividieren $0,44 : 0,2$.

    \begin{enumerate}
      \item \enquote{Kommaverschiebung}, wir machen aus der Zahl durch die geteilt werden soll eine natürliche Zahl ohne das Ergebnis zu verändern\footnote{der Ausdruck muss gleich bleiben, Äquivalenzumformung}.
      \item schriftlich dividieren.
    \end{enumerate}

    \begin{align*}
        0,44 & : & 0,2 \\
         & = & 4,4 : 2
    \end{align*}

    $0,44 : 0,2$, gelesen \enquote{Null-Komma-Vier-Vier geteilt durch Null-Komma-Zwei}, ist $\frac{0,44}{0,2}$. Zumindest in der Vorstellung möchten wir in Brüchen nur ganze Zahlen haben. Einen Bruch können wir erweitern: Wir erweitern mit 10, so dass $\frac{0,44}{0,2} = \frac{4,4}{2}$. $4,4 : 2$ ist weder schriftlich noch im Kopf schwierig so dass $4,4 : 2 = 2,2$.
\end{beispiel}

\begin{beispiel}[Division von Dezimalbrüchen]
    Wir dividieren $7,0008 : 0,4$. Durch \enquote{Kommaverschiebung} ergibt das (ohne den Wert zu verändern) $70,008 : 4$. Das dividieren wir wie gehabt schriftlich (s. Abb. \ref{fig:divisionDezimalbrueche}).

    \begin{figure}
        \centering
        \begin{tikzpicture}
            \draw[step=5mm, line width=0.1mm, black!40!white] (0,0) grid (\width,\hauteur);

            \node at (0.75,11.75) {7};
            \node at (1,11.65) {,};
            \node at (1.25,11.75) {0};
            \node at (1.75,11.75) {0};
            \node at (2.25,11.75) {0};
            \node at (2.75,11.75) {8};
            \node at (3.25,11.75) {:};
            \node at (3.75,11.75) {0};
            \node at (4.25,11.65) {,};
            \node at (4.75,11.75) {4};

            \draw (6.25,11.55) -- (6.25,11.95);
            \node at (6.75,11.75) {$\cdot$};
            \node at (7.25,11.75) {1};
            \node at (7.75,11.75) {0};

            \node at (0.25,11.25) {=};
            \node at (0.75,11.25) {7};
            \node at (1.25,11.25) {0};
            \node at (1.5,11.15) {,};
            \node at (1.75,11.25) {0};
            \node at (2.25,11.25) {0};
            \node at (2.75,11.25) {8};
            \node at (3.25,11.25) {:};
            \node at (3.75,11.25) {4};

            \node at (0.75,10.25) {7};
            \node at (1.25,10.25) {0};
            \node at (1.5,10.15) {,};
            \node at (1.75,10.25) {0};
            \node at (2.25,10.25) {0};
            \node at (2.75,10.25) {8};
            \node at (3.25,10.25) {:};
            \node at (3.75,10.25) {4};
            \node at (4.25,10.25) {=};
            \node at (4.75,10.25) {1};
            \node at (5.25,10.25) {7};
            \node at (5.5,10.15) {,};
            \node at (5.75,10.25) {5};
            \node at (6.25,10.25) {0};
            \node at (6.75,10.25) {2};

            \node at (0.25,9.75) {-};
            \node at (0.75,9.75) {4};
            \draw (0.25,9.25) -- (1.25,9.25);
            \node at (0.75,8.75) {3};
            \node at (1.25,8.75) {0};

            \node at (0.25,9.75) {-};
            \node at (0.75,8.25) {2};
            \node at (1.25,8.25) {8};
            \draw (0.25,7.75) -- (1.75,7.75);
            \node at (1.25,7.25) {2};
            \node at (1.75,7.25) {0};

            \node at (0.75,6.75) {-};
            \node at (1.25,6.75) {2};
            \node at (1.75,6.75) {0};
            \draw (0.75,6.25) -- (2.25,6.25);
            \node at (1.75,5.75) {0};
            \node at (2.25,5.75) {0};

            \node at (1.25,5.25) {-};
            \node at (2.25,5.25) {0};
            \draw (1.25,4.75) -- (2.75,4.75);
            \node at (2.25,4.25) {0};
            \node at (2.75,4.25) {8};

            \node at (1.75,3.75) {-};
            \node at (2.75,3.75) {8};
            \draw (1.75,3.25) -- (3.25,3.25);
            \node at (2.75,2.75) {0};
        \end{tikzpicture}
    \caption{Division von Dezimalbrüchen}\label{fig:divisionDezimalbrueche}
    \end{figure}
\end{beispiel}

\begin{figure}
  \centering
  \begin{tikzpicture}
    %\draw[step=1mm, line width=0.1mm, black!30!white] (0,0) grid (\width,\hauteur);
    \draw[step=5mm, line width=0.1mm, black!40!white] (0,0) grid (\width,\hauteur);
    %\draw[step=5cm, line width=0.5mm, black!50!white] (0,0) grid (\width,\hauteur);
    %\draw[step=1cm, line width=0.3mm, black!90!white] (0,0) grid (\width,\hauteur);

    \node at (1.25cm, 11.25cm) {1};
    \node at (1.75cm, 11.25cm) {:};
    \node at (2.25cm, 11.25cm) {2};
    \node at (2.75cm, 11.25cm) {=};
    \node at (3.25cm, 11.25cm) {0};
    \node at (3.75cm, 11.10cm) {,};
    \node at (4.25cm, 11.25cm) {5};

    \node at (0.75cm, 10.75cm) {-};
    \node at (1.25cm, 10.75cm) {0};

    \draw (0.7cm,10.25cm) -- (1.8cm,10.25cm);
    \node at (1.25cm, 9.75cm) {1};
    \node at (1.75cm, 9.75cm) {0};

    \node at (0.75cm, 9.25cm) {-};
    \node at (1.25cm, 9.25cm) {1};
    \node at (1.75cm, 9.25cm) {0};

    \draw (0.7cm,8.75cm) -- (2.3cm,8.75cm);
    \node at (1.75cm, 8.25cm) {0};


    \node at (6.25cm, 11.25cm) {1};
    \node at (6.75cm, 11.25cm) {:};
    \node at (7.25cm, 11.25cm) {1};
    \node at (7.75cm, 11.25cm) {6};
    \node at (8.25cm, 11.25cm) {=};
    \node at (8.75cm, 11.25cm) {0};
    \node at (9.25cm, 11.10cm) {,};
    \node at (9.75cm, 11.25cm) {0};
    \node at (10.25cm, 11.25cm) {6};
    \node at (10.75cm, 11.25cm) {2};
    \node at (11.25cm, 11.25cm) {5};

    \node at (5.75cm, 10.75cm) {-};
    \node at (6.25cm, 10.75cm) {0};
    \draw (5.75cm,10.25cm) -- (6.75cm,10.25cm);

    \node at (6.25cm, 9.75cm) {1};
    \node at (6.75cm, 9.75cm) {0};
    \node at (5.75cm, 9.25cm) {-};
    \node at (6.75cm, 9.25cm) {0};
    \draw (5.75cm,8.75cm) -- (7.25cm,8.75cm);
    \node at (6.25cm, 8.25cm) {1};
    \node at (6.75cm, 8.25cm) {0};
    \node at (7.25cm, 8.25cm) {0};

    \node at (5.75cm, 7.75cm) {-};
    \node at (6.75cm, 7.75cm) {9};
    \node at (7.25cm, 7.75cm) {6};
    \draw (5.75cm,7.25cm) -- (7.75cm,7.25cm);
    \node at (7.25cm, 6.75cm) {4};
    \node at (7.75cm, 6.75cm) {0};

    \node at (6.25cm, 6.25cm) {-};
    \node at (7.25cm, 6.25cm) {3};
    \node at (7.75cm, 6.25cm) {2};
    \draw (6.75cm,5.75cm) -- (8.25cm,5.75cm);
    \node at (7.75cm, 5.25cm) {8};
    \node at (8.25cm, 5.25cm) {0};

    \node at (7.25cm, 4.75cm) {-};
    \node at (7.75cm, 4.75cm) {8};
    \node at (8.25cm, 4.75cm) {0};
    \draw (6.75cm,4.25cm) -- (8.75cm,4.25cm);
    \node at (8.25cm, 3.75cm) {0};

  \end{tikzpicture}
  \caption{schriftlich dividieren}\label{fig:schriftlichDividieren}
\end{figure}



\chapter{Bruchrechnung}\index{Bruchrechnung}

\section{Rechenregeln}

Wir \textbf{addieren} Brüche indem wir sie auf gleiche \textbf{Nenner} bringen und dann die angepassten \textbf{Zähler} \textbf{addieren}. Den kleinsten gemeinsamen \textbf{Nenner} finden wir indem wir das \textbf{kleinste gemeinsame Vielfache (KGV)} der \textbf{Nenner} bestimmen. Das \textbf{KGV} finden wir indem wir die \textbf{Primfaktorzerlegungen} der \textbf{Nenner} \glsdisp{symb:Vereinigung}{vereinigen}\footnote{Wir vereinfachen hier etwas sprachlich, denn die Vereinigung von Mengen ist es nur dann genau wenn wir die Faktoren $m^n$ für verschiedene $n$ aufführen, also z.B. $Pfz(4)=\{2^1, 2^2\}$}.

Wir dürfen nur Brüche mit gleichem \textbf{Nenner} (direkt) \textbf{addieren} oder \textbf{subtrahieren}. Ansonsten müssen wir zuerst alle \textbf{Summanden} auf den gleichen \textbf{Nenner} bringen (\enquote{erweitern}).

\begin{equation}
    \frac{a}{c} + \frac{b}{c} = \frac{a+b}{c}
\end{equation}

\begin{figure}
  \centering
  \begin{tikzpicture}
    %\draw[step=1mm, line width=0.1mm, black!30!white] (0,0) grid (\width,\hauteur);
    \draw[step=5mm, line width=0.1mm, black!40!white] (0,0) grid (\width,\hauteur);
    %\draw[step=5cm, line width=0.5mm, black!50!white] (0,0) grid (\width,\hauteur);
    %\draw[step=1cm, line width=0.3mm, black!90!white] (0,0) grid (\width,\hauteur);

    \node at (1.25cm, 11.25cm) {1};
    \draw (0.75cm, 10.75) -- (1.75cm, 10.75);
    \node at (1.25cm, 10.25cm) {2};

    \node at (2.25cm, 10.75cm) {+};

    \node at (3.25cm, 11.25cm) {1};
    \draw (2.75cm, 10.75) -- (3.75cm, 10.75);
    \node at (3.25cm, 10.25cm) {3};

    \node at (4.25cm, 10.75cm) {=};

    \node at (4.75cm, 10.75cm) {?};


    \node at (5.75cm, 10.75cm) {K};
    \node at (6.25cm, 10.75cm) {G};
    \node at (6.75cm, 10.75cm) {V};
    \node at (7.25cm, 10.75cm) {(};
    \node at (7.75cm, 10.75cm) {2};
    \node at (8.25cm, 10.75cm) {;};
    \node at (8.75cm, 10.75cm) {3};
    \node at (9.25cm, 10.75cm) {)};
    \node at (9.75cm, 10.75cm) {=};
    \node at (10.25cm, 10.75cm) {?};

    \node at (5.75cm, 9.75cm) {P};
    \node at (6.25cm, 9.75cm) {F};
    \node at (6.75cm, 9.75cm) {Z};
    \node at (7.25cm, 9.75cm) {(};
    \node at (7.75cm, 9.75cm) {2};
    \node at (8.25cm, 9.75cm) {)};
    \node at (8.75cm, 9.75cm) {=};
    \node at (9.25cm, 9.75cm) {\{};
    \node at (9.75cm, 9.75cm) {2};
    \node at (10.25cm, 9.75cm) {\}};

    \node at (5.75cm, 9.25cm) {P};
    \node at (6.25cm, 9.25cm) {F};
    \node at (6.75cm, 9.25cm) {Z};
    \node at (7.25cm, 9.25cm) {(};
    \node at (7.75cm, 9.25cm) {3};
    \node at (8.25cm, 9.25cm) {)};
    \node at (8.75cm, 9.25cm) {=};
    \node at (9.25cm, 9.25cm) {\{};
    \node at (9.75cm, 9.25cm) {3};
    \node at (10.25cm, 9.25cm) {\}};

    \node at (1.25cm, 8.25cm) {P};
    \node at (1.75cm, 8.25cm) {F};
    \node at (2.25cm, 8.25cm) {Z};
    \node at (2.75cm, 8.25cm) {(};
    \node at (3.25cm, 8.25cm) {2};
    \node at (3.75cm, 8.25cm) {)};
    \node at (4.25cm, 8.25cm) {$\bigcup$};
    \node at (4.75cm, 8.25cm) {P};
    \node at (5.25cm, 8.25cm) {F};
    \node at (5.75cm, 8.25cm) {Z};
    \node at (6.25cm, 8.25cm) {(};
    \node at (6.75cm, 8.25cm) {3};
    \node at (7.25cm, 8.25cm) {)};
    \node at (7.75cm, 8.25cm) {=};
    \node at (8.25cm, 8.25cm) {\{};
    \node at (8.75cm, 8.25cm) {2};
    \node at (9.25cm, 8.25cm) {;};
    \node at (9.75cm, 8.25cm) {3};
    \node at (10.25cm, 8.25cm) {\}};

    \node at (1.25cm, 7.25cm) {$\Rightarrow$};
    \node at (1.75cm, 7.25cm) {K};
    \node at (2.25cm, 7.25cm) {G};
    \node at (2.75cm, 7.25cm) {V};
    \node at (3.25cm, 7.25cm) {(};
    \node at (3.75cm, 7.25cm) {2};
    \node at (4.25cm, 7.25cm) {;};
    \node at (4.75cm, 7.25cm) {3};
    \node at (5.25cm, 7.25cm) {)};
    \node at (5.75cm, 7.25cm) {=};
    \node at (6.25cm, 7.25cm) {2};
    \node at (6.75cm, 7.25cm) {$\cdot$};
    \node at (7.25cm, 7.25cm) {3};
    \node at (7.75cm, 7.25cm) {=};
    \node at (8.25cm, 7.25cm) {6};

    \node at (1.25cm, 6.25cm) {1};
    \draw (0.75cm, 5.75) -- (1.75cm, 5.75);
    \node at (1.25cm, 5.25cm) {2};

    \node at (2.25cm, 5.75cm) {+};

    \node at (3.25cm, 6.25cm) {1};
    \draw (2.75cm, 5.75) -- (3.75cm, 5.75);
    \node at (3.25cm, 5.25cm) {3};

    \node at (4.25cm, 5.75cm) {=};

    \node at (5.25cm, 6.25cm) {3};
    \draw (4.75cm, 5.75) -- (5.75cm, 5.75);
    \node at (5.25cm, 5.25cm) {6};

    \node at (6.25cm, 5.75cm) {+};

    \node at (7.25cm, 6.25cm) {2};
    \draw (6.75cm, 5.75) -- (7.75cm, 5.75);
    \node at (7.25cm, 5.25cm) {6};

    \node at (8.25cm, 5.75cm) {=};

    \node at (9.25cm, 6.25cm) {5};
    \draw (8.75cm, 5.75) -- (9.75cm, 5.75);
    \node at (9.25cm, 5.25cm) {6};
  \end{tikzpicture}
  \caption{Brüche addieren}\label{fig:BruecheAddieren1}
\end{figure}

\begin{figure}
  \centering
  \begin{tikzpicture}
    %\draw[step=1mm, line width=0.1mm, black!30!white] (0,0) grid (\width,\hauteur);
    \draw[step=5mm, line width=0.1mm, black!40!white] (0,0) grid (\width,\hauteur);
    %\draw[step=5cm, line width=0.5mm, black!50!white] (0,0) grid (\width,\hauteur);
    %\draw[step=1cm, line width=0.3mm, black!90!white] (0,0) grid (\width,\hauteur);

    \node at (0.75cm, 11.25cm) {};
    \node at (1.25cm, 11.25cm) {3};
    \draw (0.25cm, 10.75) -- (1.75cm, 10.75);
    \node at (0.75cm, 10.25cm) {1};
    \node at (1.25cm, 10.25cm) {4};

    \node at (2.25cm, 10.75cm) {+};

    \node at (2.75cm, 11.25cm) {};
    \node at (3.75cm, 11.25cm) {5};
    \draw (2.75cm, 10.75) -- (4.25cm, 10.75);
    \node at (3.25cm, 10.25cm) {1};
    \node at (3.75cm, 10.25cm) {2};

    \node at (4.75cm, 10.75cm) {=};

    \node at (5.25cm, 10.75cm) {?};


    \node at (9.75cm, 10.75cm) {K};
    \node at (10.25cm, 10.75cm) {G};
    \node at (10.75cm, 10.75cm) {V};
    \node at (11.25cm, 10.75cm) {=};
    \node at (11.75cm, 10.75cm) {?};

    \node at (0.25cm, 9.25cm) {P};
    \node at (0.75cm, 9.25cm) {F};
    \node at (1.25cm, 9.25cm) {Z};
    \node at (1.75cm, 9.25cm) {(};
    \node at (2.25cm, 9.25cm) {1};
    \node at (2.75cm, 9.25cm) {4};
    \node at (3.25cm, 9.25cm) {)};
    \node at (3.75cm, 9.25cm) {=};
    \node at (4.25cm, 9.25cm) {\{};
    \node at (4.75cm, 9.25cm) {2};
    \node at (5.25cm, 9.25cm) {;};
    \node at (5.75cm, 9.25cm) {7};
    \node at (6.25cm, 9.25cm) {\}};

    \node at (0.25cm, 8.75cm) {P};
    \node at (0.75cm, 8.75cm) {F};
    \node at (1.25cm, 8.75cm) {Z};
    \node at (1.75cm, 8.75cm) {(};
    \node at (2.25cm, 8.75cm) {1};
    \node at (2.75cm, 8.75cm) {2};
    \node at (3.25cm, 8.75cm) {)};
    \node at (3.75cm, 8.75cm) {=};
    \node at (4.25cm, 8.75cm) {\{};
    \node at (4.75cm, 8.75cm) {2};
    \node at (5.25cm, 8.75cm) {;};
    \node at (5.75cm, 8.75cm) {2};
    \node at (6.25cm, 8.75cm) {;};
    \node at (6.75cm, 8.75cm) {3};
    \node at (7.25cm, 8.75cm) {\}};


    \node at (0.25cm, 8.25cm) {P};
    \node at (0.75cm, 8.25cm) {F};
    \node at (1.25cm, 8.25cm) {Z};
    \node at (1.75cm, 8.25cm) {(};
    \node at (2.25cm, 8.25cm) {1};
    \node at (2.75cm, 8.25cm) {4};
    \node at (3.25cm, 8.25cm) {)};
    \node at (3.75cm, 8.25cm) {$\bigcup$};
    \node at (4.25cm, 8.25cm) {P};
    \node at (4.75cm, 8.25cm) {F};
    \node at (5.25cm, 8.25cm) {Z};
    \node at (5.75cm, 8.25cm) {(};
    \node at (6.25cm, 8.25cm) {1};
    \node at (6.75cm, 8.25cm) {2};
    \node at (7.25cm, 8.25cm) {)};
    \node at (0.75cm, 7.75cm) {=};
    \node at (1.25cm, 7.75cm) {\{};
    \node at (1.75cm, 7.75cm) {2};
    \node at (2.25cm, 7.75cm) {;};
    \node at (2.75cm, 7.75cm) {2};
    \node at (3.25cm, 7.75cm) {;};
    \node at (3.75cm, 7.75cm) {3};
    \node at (4.25cm, 7.75cm) {;};
    \node at (4.75cm, 7.75cm) {7};
    \node at (5.25cm, 7.75cm) {\}};

    \node at (0.25cm, 7.25cm) {K};
    \node at (0.75cm, 7.25cm) {G};
    \node at (1.25cm, 7.25cm) {V};
    \node at (1.75cm, 7.25cm) {=};
    \node at (2.25cm, 7.25cm) {2};
    \node at (2.75cm, 7.25cm) {$\cdot$};
    \node at (3.25cm, 7.25cm) {2};
    \node at (3.75cm, 7.25cm) {$\cdot$};
    \node at (4.25cm, 7.25cm) {3};
    \node at (4.75cm, 7.25cm) {$\cdot$};
    \node at (5.25cm, 7.25cm) {7};
    \node at (5.75cm, 7.25cm) {=};
    \node at (6.25cm, 7.25cm) {8};
    \node at (6.75cm, 7.25cm) {4};

    \node at (0.75cm, 6.25cm) {1};
    \node at (1.25cm, 6.25cm) {8};
    \draw (0.25cm, 5.75) -- (1.75cm, 5.75);
    \node at (0.75cm, 5.25cm) {8};
    \node at (1.25cm, 5.25cm) {4};

    \node at (2.25cm, 5.75cm) {+};

    \node at (3.25cm, 6.25cm) {3};
    \node at (3.75cm, 6.25cm) {5};
    \draw (2.75cm, 5.75) -- (4.25cm, 5.75);
    \node at (3.25cm, 5.25cm) {8};
    \node at (3.75cm, 5.25cm) {4};

    \node at (4.75cm, 5.75cm) {=};

    \node at (5.75cm, 6.25cm) {5};
    \node at (6.25cm, 6.25cm) {3};
    \draw (5.25cm, 5.75) -- (6.75cm, 5.75);
    \node at (5.75cm, 5.25cm) {8};
    \node at (6.25cm, 5.25cm) {4};
  \end{tikzpicture}
  \caption{Brüche addieren}\label{fig:BruecheAddieren2}
\end{figure}


\begin{equation}
    \frac{a}{c} - \frac{b}{c} = \frac{a-b}{c}
\end{equation}

Wir multiplizieren Brüche indem wir die Zähler multiplizieren und das Ergebnis der Zähler des Produktes ist und die Nenner multiplizieren und das Ergebnis der Nenner des Produktes ist.

\begin{beispiel}[Brüche multiplizieren]
    \begin{align}\label{eqn:BruecheMultiplikation01}
      \frac{2}{3} \cdot \frac{5}{7} & = \frac{10}{21}\\
      3 \cdot \frac{5}{7} = \frac{3}{1} \cdot \frac{5}{7} & = \frac{15}{7}\\
      \frac{8}{21} \cdot \frac{7}{16} & = \frac{1}{3} \cdot \frac{1}{2} = \frac{1}{6}\\
      \frac{5}{3} \cdot \frac{3}{5} & = 1
    \end{align}
\end{beispiel}

\begin{equation}
    \frac{a}{c} \cdot \frac{b}{d} = \frac{a b}{c d}
\end{equation}

Wir dividieren durch Brüche wir indem wir mit dem \textbf{Kehrwert} multiplizieren.

\begin{equation}\label{dividieren durch multiplizieren mit Kehrwert}
    \frac{a}{c} \div \frac{b}{d} = \frac{a}{c} \cdot \frac{d}{b} = \frac{a d}{b c}
\end{equation}

Brüche deren \textbf{Zähler} und \textbf{Nenner} einen gemeinsamen \textbf{Teiler} haben kann man \textbf{kürzen}.

\begin{equation}
    \frac{n a}{n b} = \frac{a}{b}
\end{equation}

Wir sagen \enquote{$\frac{n a}{n b}$ lässt sich mit $n$ \textbf{kürzen} und ist als \textbf{gekürzter Bruch} $\frac{a}{b}$}.

\begin{beispiel}
    Der Bruch $r=\frac{2}{6}$ lässt sich \textbf{kürzen}, denn $2$ und $6$ haben den \textbf{gemeinsamen Teiler} $2$. $\frac{2}{6} = \frac{1}{3}$.\topicend
\end{beispiel}

Brüche deren \textbf{Zähler} größer als ihr \textbf{Nenner} ist können als \textbf{gemischter Bruch} geschrieben werden.

\begin{equation}\glsadd{symb:Abrunden}
    \frac{a}{b} = \left\lfloor\frac{a}{b}\right\rfloor \frac{a-b \left\lfloor\frac{a}{b}\right\rfloor}{b}\footnote{$\left\lfloor \frac{a}{b} \right\rfloor$ bedeutet das auf die nächst kleinere ganze Zahl \textbf{abgerundete} Ergebnis von  $a \div b$. $\frac{5}{2} = 2,5$. $\left\lfloor 2,5 \right\rfloor = 2$. Also $\left\lfloor\frac{5}{2} \right\rfloor = 2$.}
\end{equation}

\begin{beispiel}
    $\frac{13}{2} = 6 \frac{1}{2}$
\end{beispiel}


\section{Prozentrechnung}\label{Prozentrechnung}\index{Prozentrechnung}

Wir betrachten \enquote{Prozentrechnung} nicht als eigenes (Haupt-)Thema. \enquote{Prozentrechnung} ist einfach Bruchrechnung mit Hundertsteln.

\begin{eqnarray}
% \nonumber to remove numbering (before each equation)
  \text{\textbf{Grundwert}} (GW) \cdot \text{\textbf{Prozentsatz}} (PS) &=& \text{\textbf{Prozentwert}} (PW)\\
  PS &=& \frac{PW}{GW} \\
  GW &=& \frac{PW}{PS}
\end{eqnarray}

Beim Rechnen mit Geld sind \textbf{Kapital} und \textbf{Guthaben}\footnote{sowie \textbf{Schulden} als negatives \textbf{Guthaben}} Synonyme für \textbf{Grundwert}, \textbf{Zinssatz} für \textbf{Prozentsatz} und \textbf{Zinsen} für \textbf{Prozentwert}. Außerdem ist \textbf{Ist} synonym mit positivem \textbf{Guthaben} und \textbf{Soll} mit negativem \textbf{Guthaben} (\textbf{Schulden}).

\begin{beispiel}
    Zu bestimmen sei wie viel Gramm die 23 Prozent Zucker eines 35 Gramm Schokolandenriegels sind.
    \begin{eqnarray}
    % \nonumber to remove numbering (before each equation)
      Prozentwert &=& Grundwert \cdot Prozentsatz \\
       &=& 35g \cdot \frac{7}{100} \\
       &=& \frac{35}{100}g\\
       &=& 0,35g
    \end{eqnarray}\topicend
\end{beispiel}


\chapter{Geometrie}\index{Geometrie}

In der Mathematik-Prüfung im Hauptschulabschluss \textbf{konstruieren} und \textbf{berechnen} wir Dreiecke, andere elementare 2-di\-men\-sio\-nale Flächen und zwei 3-di\-men\-sio\-nale Körper, die \textbf{Prismen}\index{Prisma} und \textbf{Zylinder}. Praktische Anwendungen sind z.B. wie viel Luft in einem Gebäude zu erwärmen (Anlagenmechaniker SHK) ist oder wie lange es dauert diese durch Lüftung auszutauschen, wie viel Wasser ein Pflanztopf halten kann, ein Schwimmbecken und vieles mehr. Wie in der Hauptschul-Mathematik immer gilt auch hier dass die Inhalte über den Abschluss hinaus für uns alle im Alltag wichtig sind.


\section{Geraden und Strecken}\index{Gerade}\index{Strecke}

\begin{definition}[Gerade]\index{Gerade}\label{def:Gerade}
    Eine gerade, unendlich lange, unendlich dünne und in beide Richtungen unbegrenzte Linie nennen wir \enquote{Gerade}.
\end{definition}

\begin{exkurs}[Abstraktion und Modellbildung]
    Eine Gerade ist damit ein rein mathematisches Objekt. In der Natur kennen wir nichts was unendlich lang ist, und sehen nichts was eine Breite oder Länge von 0 hat. Mathematische Objekte oder Konzepte sind Idealisierungen und Vereinfachungen. Wir nennen das \textbf{Modellbildung}\index{Modell}. Diese Vereinfachung ist eine der mächtigen Methoden, die den (natur-)wissenschaftlichen und technischen Fortschritt mittels Mathematik ermöglichen. Eine für uns besonders nützliche Gerade ist der Zahlenstrahl\index{Zahlenstrahl}.
\end{exkurs}

\begin{definition}[Parallelität]\index{parallel}\label{def:parallel1}
    Wir nennen zwei Geraden in einer Ebene, die sich nicht schneiden, \enquote{Parallelen}. Wir sagen \enquote{a ist \glsdisp{symb:parallel}{parallel} zu b} und schreiben \glsdisp{symb:parallel}{$a \parallel b$}. Die Beziehung ist symmetrisch; d.h. wenn $a \parallel b$, dann ist auch $b \parallel a$.
\end{definition}

\begin{definition}[Parallelität]\index{parallel}\label{def:parallel2}
    Wir nennen zwei Geraden in einer Ebene, die eine (und damit alle) gemeinsame Höhe haben, parallel. Lesen und notieren, s.o.
\end{definition}

\begin{figure}
  \centering
    \begin{tikzpicture}
        \draw (1,0.75) -- (4,0.75);
        \draw (1,2) -- (4,2);
        \draw (2.5,0.25) -- (2.5,2.5);

        \node at (0.75,0.75) {a};
        \node at (0.75,2) {b};
        \node at (2.5,2.75) {c};
        \node at (2.3,0.9) {$\cdot$};
        \node at (2.3,1.8) {$\cdot$};

        \draw (2.0,1.5) ++(0:0.5) arc (-90:-180:0.5);
        \draw (2.0,1.25) ++(0:0.5) arc (90:180:0.5);

    \end{tikzpicture}$\quad$
    \begin{tikzpicture}[rotate=30]
        \draw (1,0.75) -- (4,0.75);
        \draw (1,2) -- (4,2);
        \draw (2.5,0.25) -- (2.5,2.5);

        \node at (0.75,0.75) {a};
        \node at (0.75,2) {b};
        \node at (2.5,2.75) {c};
        \node at (2.3,0.9) {$\cdot$};
        \node at (2.3,1.8) {$\cdot$};

        \draw (2.0,1.5) ++(0:0.5) arc (-90:-180:0.5);
        \draw (2.0,1.25) ++(0:0.5) arc (90:180:0.5);

    \end{tikzpicture}

  \caption{Parallelität: $a \glsdisp{symb:parallel}{\|} b \glsdisp{symb:Und}{\wedge} b \glsdisp{symb:parallel}{\|} a$, sowie $a \glsdisp{symb:senkrecht}{\bot} c \glsdisp{symb:Und}{\wedge} c b \glsdisp{symb:senkrecht}{\bot} c$ und $c \glsdisp{symb:senkrecht}{\bot} a \glsdisp{symb:Und}{\wedge} c \glsdisp{symb:senkrecht}{\bot} b$}\label{fig:Parallelitaet}
\end{figure}

\begin{definition}[Strecke]\index{Strecke}\label{def:Strecke}
    Einen endlich langen Abschnitt (ohne Unterbrechungen) einer Gerade nennen wir \enquote{Strecke}.
\end{definition}

\begin{definition}[Ebene]\index{Ebene}\label{def:Ebene}
    Wir nennen die unendlich weit fortgesetzte \enquote{Zeichenfläche}/ \enquote{Oberfläche der Seite des Mathematik-Heftes} \textbf{Ebene}.
\end{definition}

Alle unsere Flächen (2-dimensionale elementare Flächen und zusammengesetzte Figuren/ ebene Flächen) liegen in der Ebene, die durch die Zahlenstrahlen in x- und y-Richtung aufgespannt wird.


\begin{satz}[Parallelität in 2d]\index{Prallele}\label{def:Parallele}
    Zwei Geraden in einer Ebene\footnote{Insbesondere ist jede 2-dimensionale ebene Fläche (Teil einer) Ebene (der euklidischen Geometrie).} sind entweder parallel oder haben genau einen Schnittpunkt.
\end{satz}


\section{Winkel}\index{Winkel}\label{Winkel}

Wir bezeichnen \textbf{Drehungen} durch \textbf{Winkel}, gemessen in \textbf{Grad ($\varphi$°)}. Eine komplette Drehung, \enquote{sich (genau einmal) im Kreis drehen}, definieren wir als 360°, gelesen \enquote{dreihundertundsechzig Grad}. Hier nutzen wir wieder das (babylonische) 60er-System um besser teilen zu können. Ein Kreis(bogen) hat also 360°.

\begin{definition}[Vollwinkel]
    Wir nennen einen Winkel von 360° ($2 \pi$) \enquote{\textbf{Vollwinkel}}.\index{Vollwinkel}
\end{definition}

\begin{definition}[Gestreckter Winkel]
    Wir nennen einen Winkel von 180° ($\pi$) \enquote{\textbf{Gestreckter Winkel}}.\index{Gestreckter Winkel}
\end{definition}

\begin{definition}[Rechter Winkel]
    Wir nennen einen Winkel von 90° ($\frac{1}{2} \pi$) \enquote{\textbf{Rechter Winkel}}.\index{Rechter Winkel}
\end{definition}

\begin{definition}[Spitzer Winkel]
    Wir nennen einen Winkel $\varphi$ mit $\varphi < 90°$ \enquote{\textbf{Spitzer Winkel}}.\index{Spitzer Winkel}
\end{definition}

\begin{definition}[Stumpfer Winkel]\index{Stumpfer Winkel}
    Wir nennen einen Winkel $\varphi$ mit $90° < \varphi < 180°$ \enquote{\textbf{Stumpfer Winkel}}.\index{Stumpfer Winkel}
\end{definition}

\begin{definition}[Überstumpfer Winkel]\index{Überstumpfer Winkel}
    Wir nennen einen Winkel $\varphi$ mit $180° < \varphi < 360°$ \enquote{\textbf{Überstumpfer Winkel}}.\index{Überstumpfer Winkel}
\end{definition}

Wir benutzen in der Regel für die Winkel an den Punkten $A, B, C, D$ in Vierecken \glsdisp{symb:alpha}{$\alpha$}, \glsdisp{symb:beta}{$\beta$}, \glsdisp{symb:gamma}{$\gamma$} und \glsdisp{symb:delta}{$\delta$}. Entsprechend \glsdisp{symb:alpha}{$\alpha$}, \glsdisp{symb:beta}{$\beta$}, \glsdisp{symb:gamma}{$\gamma$} an $A, B, C$ im Dreieck. Diese Vereinbarung der Symbole solltet Ihr einhalten und wird in den Prüfungen eingehalten, so dass sie mit unseren Planskizzen\index{Planskizze} übereinstimmt.

\begin{satz}[Winkelsumme am Schnittpunkt von Geraden]
    Die Summe der Winkel am Schnittpunkt zweier Geraden beträgt 360° (wir nennen einen Winkel, der genau ein ganzer Kreis ist \textbf{Vollwinkel})\index{Vollwinkel}.
\end{satz}

\begin{satz}[Winkelsumme über einer Geraden]
    Die Summe der Winkel über einer Geraden beträgt 180°\index{Winkelsumme!an Geraden} (wir nennen einen Winkel, der genau ein halber Kreis ist \textbf{Gestreckter Winkel})\index{gestreckter Winkel}.
\end{satz}

\begin{figure}
  \centering
  \begin{tikzpicture}
        \draw (1,1) -- (5,5);
        \draw (1,3) -- (5,3);

        \node at (0.75,0.75) {a};
        \node at (0.75,3) {b};

        \node at (2.75,3.2) {$\delta$};
        \node at (3.5,3.2) {$\gamma$};
        \node at (3.2,2.75) {$\beta$};
        \node at (2.5,2.75) {$\alpha$};

        \draw (3,3) ++(0:1) arc (0:45:1);
        \draw (2.625,3.88) ++(0:1.25) arc (45:180:1.25);
        \draw (1,3) ++(0:1) arc (180:225:1);
        \draw (0.86,2.125) ++(0:1.25) arc (225:360:1.25);
  \end{tikzpicture}
  \caption{Winkelsumme am Schnittpunkt von Geraden: u.a. $\alpha + \beta = 180°$ und $\alpha = \gamma$ sowie $\beta = \delta$}\label{fig:WinkelsummeSchnittpunkt}
\end{figure}

Beachte dass jeweils zwei benachbarte Winkel zusammen 180° ergeben. Da zwei benachbarte Winkel den Halbkreis über einer Gerade schlagen ist ihre Summe 180°. Die gegenüber liegenden Winkel sind gleich.\index{Winkelsumme!an Geraden}


\begin{beispiel}[Winkel]\index{Winkel}
\begin{figure}
  \centering
    \begin{tikzpicture}
        \node (P) at (0,0) {};
        \node (Q) at (7,7) {};
        \node (R) at (0,4) {};
        \node (S) at (7,4) {};
        \node (T) at (0,4) {};
        \node (U) at (4,0) {};

        \draw (P) -- (Q);
        \draw (R) -- (S);
        \draw (T) -- (U);

        \node (SP1) at (4,4) {};
        \centerarc[black](SP1)(180:225:1.5);
        \centerarc[black](SP1)(225:360:1);
        \centerarc[black](SP1)(0:45:1.5);
        \centerarc[black](SP1)(45:180:1);
        \node[label={[label distance=.5cm]202:$\alpha$}] at (SP1) {};
        \node[label={[label distance=.5cm]293:$\beta$}] at (SP1) {};
        \node[label={[label distance=.5cm]23:$\gamma$}] at (SP1) {};
        \node[label={[label distance=.5cm]117:$\delta$}] at (SP1) {};

        \node (SP2) at (2,2) {};
        \centerarc[black](SP2)(-45:45:1.5);
        \centerarc[black](SP2)(45:135:1);
        \centerarc[black](SP2)(135:225:1.5);
        \centerarc[black](SP2)(225:315:1);
        \node[label={[label distance=.5cm]0:$\cdot$}] at (SP2) {};
        \node[label={[label distance=.4cm]90:$\cdot$}] at (SP2) {};
        \node[label={[label distance=.5cm]180:$\varphi$}] at (SP2) {};
        \node[label={[label distance=.4cm]270:$\cdot$}] at (SP2) {};

    \end{tikzpicture}
  \caption{Übliche Beschriftungen von Winkeln}\label{fig:Winkel01}
\end{figure}

\end{beispiel}



\section{Flächen}


\subsection{Polygon}\index{Polygon}\label{Polygon}

\begin{definition}[Polygon]
    Eine ebene Fläche mit Ecken, die von geraden Strecken verbunden werden nennen wird \enquote{Polygon}.
\end{definition}

\begin{figure}
  \centering
    \begin{tikzpicture}
        \draw (1,1) -- (1.5,1.5) -- (1,2) -- (1,1);
        \draw (1,2.5) -- (1.4,2.75) -- (1,2.75) -- (1,2.5);
        \draw (0.5,3.25) -- (1.5,3.35) -- (1.9,3.75) -- (0.5,3.25);
        \draw (0.5,5) -- (1.5,4.8) -- (1.9,5.95) -- (0.5,5);

        \draw (3,1) -- (4,1) -- (4,2) -- (3,2) -- (3,1);
        \draw (3,2.5) -- (4.5,2.5) -- (4.5,3.5) -- (3,3.5) -- (3,2.5);
        \draw (3,4) -- (4.5,4) -- (5,5) -- (3.5,5) -- (3,4);
        \draw (3,5.5) -- (4.5,5.5) -- (4,6.5) -- (3.25,6.5) -- (3,5.5);
        \draw (3,7) -- (4.5,7) -- (3.25,7.15) -- (3.6,7.75) -- (3,7);

        \draw (5.5,1) -- (6,1) -- (6.25,1.5) -- (5.75,2) -- (5.25,1.5) -- (5.5,1);
        \draw (5.5,3) -- (6,3) -- (6.25,3.5) -- (5.75,3.25) -- (5.25,3.5) -- (5.5,3);

        \draw (7.5,1) -- (8,1) -- (8.25,1.5) -- (7.75,2) -- (7.25,1.5) -- (8,1.35) -- (7.5,1);
        \draw (7.5,3) -- (8.5,3) -- (9,3.75)  -- (8.5,4.25)  -- (7.5,4.25)  -- (7,3.75)  -- (7.5,3);
    \end{tikzpicture}
  \caption{Polygone: Dreiecke, Vierecke, Fünfecke und Sechsecke}\label{fig:Polygone}
\end{figure}



\subsection{Flächen (-inhalt)}

Eine rechteckige (ebne/ 2-dimensionale) Fläche beschreiben wir durch Länge und Breite. Wir schreiben für ihre Fläche/ ihren Flächeninhalt $A_R = a b$ und lesen \enquote{Die Fläche eines Rechtecks ist gleich Länge mal Breite.} Gemäß \gls{SISystem} ist unsere Standard-Einheit für Strecken der Meter. Eine quadratische (und damit auch rechteckige) Fläche mit einer Seitenlänge von einem Meter nennen wir einen Quadratmeter und schreiben $A_Q = a^2 = (1\text{m})^2 = 1\text{m}^2$ und lesen \enquote{Die Fläche eines Quadrats mit Seitenlängen von je einem Meter ist gleich ein Quadratmeter} oder \enquote{Der Flächeneinheit eines Quadrates mit Seitenlängen gleich ein Meter ist ein Quadratmeter}. Entsprechend hat ein Quadrat mit Seitenlängen von einem Millimeter eine Fläche oder einen Flächeninhalt von einem Quadratmillimeter. Mit solchen Kacheln können wir offensichtlich die Flächeninhalte aller planen Flächen/ Figuren ausdrücken (auch wenn nicht bei allen klar ist wie wir sie messen können).

\begin{beispiel}
    \begin{figure}
      \centering
        \begin{tikzpicture}
            \draw (1cm, 1cm) -- (2cm, 1cm) -- (2cm, 2cm) -- (1cm, 2cm) -- (1cm, 1cm);%Quadrat 1

            \draw (1cm, 0.9cm) -- (1cm, 0.7cm);%Bemaßung horizontal
            \draw (2cm, 0.9cm) -- (2cm, 0.7cm);
            \draw (1cm, 0.8cm) -- (1.2cm, 0.8cm);
            \draw (1.8cm, 0.8cm) -- (2cm, 0.8cm);
            \node at (1.5cm,0.8cm) {\tiny 1cm};

            \draw (0.6cm, 1cm) -- (0.8cm, 1cm);%Bemaßung vertikal
            \draw (0.6cm, 2cm) -- (0.8cm, 2cm);
            \draw (0.7cm, 1cm) -- (0.7cm, 1.2cm);
            \draw (0.7cm, 2cm) -- (0.7cm, 1.8cm);
            \node[rotate=90] at (0.7cm,1.5cm) {\tiny 1cm};

            \node at (1.5cm, 1.5cm) {$\scriptscriptstyle\text{1cm}^2$};


            \draw (3cm, 1cm) -- (5cm, 1cm) -- (5cm, 3cm) -- (3cm, 3cm) -- (3cm, 1cm);%Quadrat 2

            \draw (3cm, 0.9cm) -- (3cm, 0.7cm);%Bemaßung horizontal
            \draw (5cm, 0.9cm) -- (5cm, 0.7cm);
            \draw (3cm, 0.8cm) -- (3.5cm, 0.8cm);
            \draw (4.5cm, 0.8cm) -- (5cm, 0.8cm);
            \node at (4cm,0.8cm) {\tiny 2cm};

            \draw (2.6cm, 1cm) -- (2.8cm, 1cm);%Bemaßung vertikal
            \draw (2.6cm, 3cm) -- (2.8cm, 3cm);
            \draw (2.7cm, 1cm) -- (2.7cm, 1.5cm);
            \draw (2.7cm, 2.5cm) -- (2.7cm, 3cm);
            \node[rotate=90] at (2.7cm,2cm) {\tiny 2cm};

            \node at (4cm, 2cm) {$\scriptscriptstyle\text{4cm}^2$};


            \draw (6cm, 1cm) -- (9cm, 1cm) -- (9cm, 4cm) -- (6cm, 4cm) -- (6cm, 1cm);%Quadrat 3

            \draw[gray] (7cm, 1cm) -- (7cm, 4cm);
            \draw[gray] (8cm, 1cm) -- (8cm, 4cm);
            \draw[gray] (6cm, 2cm) -- (9cm, 2cm);
            \draw[gray] (6cm, 3cm) -- (9cm, 3cm);

            \draw (6cm, 0.9cm) -- (6cm, 0.7cm);%Bemaßung horizontal
            \draw (9cm, 0.9cm) -- (9cm, 0.7cm);
            \draw (6cm, 0.8cm) -- (7cm, 0.8cm);
            \draw (8cm, 0.8cm) -- (9cm, 0.8cm);
            \node at (7.5cm,0.8cm) {\tiny 3cm};

            \draw (5.6cm, 1cm) -- (5.8cm, 1cm);%Bemaßung vertikal
            \draw (5.6cm, 4cm) -- (5.8cm, 4cm);
            \draw (5.7cm, 1cm) -- (5.7cm, 2cm);
            \draw (5.7cm, 3cm) -- (5.7cm, 4cm);
            \node[rotate=90] at (5.7cm,2.5cm) {\tiny 3cm};

        \end{tikzpicture}
      \caption{Fläche/ Flächeninhalt}\label{fig:geometrieArea1}
    \end{figure}
\end{beispiel}


\subsection{Umfang}\index{Umfang}

\begin{definition}[Umfang]
    Als Umfang einer (ebenen zusammenhängenden, von Löchern freien) Fläche/ Figur bezeichnen wir die Strecke, die ein Faden/ Maßband, ... misst, der/das genau benötigt würde um die Außenlinie einmal einzuschließen. Man kann z.B. auch an die Länge denken, die bei Zeichnen (ohne abzusetzen) die Spitze des Stiftes zurücklegt. Wir schreiben \glsdisp{symb:Umfang}{$U_F$} für den Umfang der Fläche $F$ und ersetzen $F$ durch den ersten Buchstaben der Fläche, z.B. $U_Q$ für den Umfang des Quadrats.
\end{definition}

\begin{beispiel}[Umfang]
    \begin{figure}
      \centering
        \begin{tikzpicture}
            \draw (1cm, 1cm) -- (8cm, 1cm) -- (8cm, 5cm) -- (1cm, 5cm) -- (1cm, 1cm);

            \draw (1cm, 0.7cm) -- (1cm, 0.9cm);
            \draw[thick,->,shorten >=2pt,shorten <=2pt,>=stealth] (1cm, 0.8cm) -- (4cm, 0.8cm) (5cm, 0.8cm) -> (8cm, 0.8cm);
            \node at (4.5cm, 0.8cm) {\tiny 7cm};
            \draw[thick,->,shorten >=2pt,shorten <=2pt,>=stealth] (0.8cm, 5cm) -- (0.8cm, 3.5cm) (0.8cm, 2.5cm) -> (0.8cm, 1cm);
            \node[rotate=90] at (0.8cm, 3cm) {\tiny 4cm};
            \draw[thick,->,shorten >=2pt,shorten <=2pt,>=stealth] (8.2cm, 1cm) -- (8.2cm, 2.5cm) (8.2cm, 3.5cm) -> (8.2cm, 5cm);
            \node[rotate=90] at (8.2cm, 3cm) {\tiny 4cm};
            \draw[thick,->,shorten >=2pt,shorten <=2pt,>=stealth] (8cm, 5.2cm) -- (5cm, 5.2cm) (4cm, 5.2cm) -> (1cm, 5.2cm);
            \node at (4.5cm, 5.2cm) {\tiny 7cm};
        \end{tikzpicture}
      \caption{Umfang: $U_R=7\text{cm} + 4\text{cm} + 7\text{cm} + 4\text{cm} = 22\text{cm}$}\label{fig:geometrieUmfang1}
    \end{figure}

    Der Umfang des Rechtecks mit Länge $a=7\text{cm}$ und Breite $c=4\text{cm}$ (s. Abb. \ref{fig:geometrieUmfang1}) errechnet sich als $U_R=a+b+a+b=2a+2b=14\text{cm}+8\text{cm}=22\text{cm}$.
\end{beispiel}


\subsection{Viereck}\index{Viereck}\label{Viereck}

\begin{definition}[Viereck]
    Ein Polygon mit genau vier Ecken nennen wir \enquote{Viereck}
\end{definition}

\begin{satz}[Winkelsumme Viereck]\label{satz:WinkelsummeViereck}
    Die Summe der Innenwinkel eines Vierecks beträgt 360°.
\end{satz}


\subsection{Rechteck}\label{Rechteck}\index{Rechteck}

\begin{definition}[Rechteck]\label{def:Rechteck01}
    Ein Viereck mit zwei diagonal gegenüber liegenden rechten Winkeln nennen wir \enquote{Rechteck}.
\end{definition}

\begin{definition}[Rechteck]\label{def:Rechteck02}
    Ein Viereck mit drei rechten Winkeln nennen wir \enquote{Rechteck}.\footnote{Ein Viereck mit vier rechten Winkeln hat drei rechte Winkel, und noch einen mehr. In der Mathematik gibt es einen bedeutenden Unterschied zwischen \enquote{das drei rechte Winkel hat} und \enquote{das \textbf{genau} drei rechte Winkel hat}.}
\end{definition}

Beachte dass somit alle Rechtecke Vierecke sind, aber nur einige Vierecke Rechtecke sind. Rechtecke sind spezielle Vierecke. Sie haben alle Eigenschaften von Vierecken, müssen darüber hinaus aber noch die Anforderungen an die rechten Winkel erfüllen.

\begin{figure}
  \centering
  \begin{longtable}{rcl}
      \begin{tikzpicture}
            \draw (1cm,1cm) -- (5cm,1cm) -- (5cm, 3cm) -- (1cm, 3cm) -- (1cm, 1cm);
            \node at (0.75cm, 0.75cm) {A};
            \node at (5.25cm, 0.75cm) {B};
            \node at (5.25cm, 3.25cm) {C};
            \node at (0.75cm, 3.25cm) {D};
            \node at (3cm, 0.75cm) {a};
            \node at (5.25cm, 2cm) {b};
            \node at (3cm, 3.25cm) {c};
            \node at (0.75cm, 2cm) {d};
            \draw (1cm,1cm) ++(0:0.5) arc (0:90:0.5cm);
            \node at (1.2,1.2) {$\scriptstyle \alpha$};
            \draw (4.5cm,1cm) ++(-0.5:0.0) arc (180:90:0.5cm);
            \node at (4.8,1.2) {$\scriptstyle \beta$};
            \draw (4.5cm,3cm) arc (180:270:0.5cm);
            \node at (4.8,2.8) {$\scriptstyle \gamma$};
            \draw (1cm,2.5cm) arc (-90:0:0.5cm);
            \node at (1.2,2.8) {$\scriptstyle \delta$};
      \end{tikzpicture}
      &
        \begin{tikzpicture}
            \node at (0cm, 0cm) {};
            \node at (0.5cm, 1.5cm) {=};
            \node at (1cm,4.5cm) {};
        \end{tikzpicture}
      &
      \begin{tikzpicture}
            \draw (1cm,1cm) -- (5cm,1cm) -- (5cm, 3cm) -- (1cm, 3cm) -- (1cm, 1cm);
            \node at (0.75cm, 0.75cm) {A};
            \node at (5.25cm, 0.75cm) {B};
            \node at (5.25cm, 3.25cm) {C};
            \node at (0.75cm, 3.25cm) {D};
            \node at (3cm, 0.75cm) {a};
            \node at (5.25cm, 2cm) {b};
            \node at (3cm, 3.25cm) {a};
            \node at (0.75cm, 2cm) {b};
            \draw (1cm,1cm) ++(0:0.5) arc (0:90:0.5cm);
            \node at (1.2,1.2) {$\scriptstyle \alpha$};
            \draw (4.5cm,1cm) ++(-0.5:0.0) arc (180:90:0.5cm);
            \node at (4.8,1.2) {$\scriptstyle \alpha$};
            \draw (4.5cm,3cm) arc (180:270:0.5cm);
            \node at (4.8,2.8) {$\scriptstyle \alpha$};
            \draw (1cm,2.5cm) arc (-90:0:0.5cm);
            \node at (1.2,2.8) {$\scriptstyle \alpha$};
      \end{tikzpicture}
  \end{longtable}
  \caption{Planskizze Rechteck}\label{fig:RechteckPlanskizze}
\end{figure}

Wir nennen die Punkte eines Rechtecks (s. Abb. \ref{fig:RechteckPlanskizze}) gegen den \glsdisp{Uhrzeigersinn}{\textbf{Uhrzeigersinn}} $A$, $B$, $C$ und $D$. Die gerade Strecke von $A$ nach $B$, $\overline{AB}$, nennen wir \enquote{die Seite $a$}. Die gerade Strecke von $B$ nach $C$, $\overline{BC}$, nennen wir \enquote{die Seite $b$}. Die gerade Strecke von $C$ nach $D$, $\overline{CD}$, nennen wir \enquote{die Seite $c$}. Die gerade Strecke von $D$ nach $C$, $\overline{DC}$, nennen wir \enquote{die Seite $c$}. Die gerade Strecke von $D$ nach $A$, $\overline{DA}$, nennen wir \enquote{die Seite $d$}. Den Winkel am Punkt $A$ nennen wir \glsdisp{symb:alpha}{$\alpha$}, sprich \enquote{\glsdisp{symb:alpha}{alpha}}. Den Winkel am Punkt $B$ nennen wir \glsdisp{symb:beta}{$\beta$}, sprich \enquote{beta}. Den Winkel am Punkt $C$ nennen wir \glsdisp{symb:gamma}{$\gamma$}, sprich \enquote{gamma}. Den Winkel am Punkt $D$ nennen wir \glsdisp{symb:delta}{$\delta$}, sprich \enquote{delta}.

\begin{eqnarray}\label{eqn:RechteckFormeln}
    A_R &=& a b\\
    U_R &=& 2a + 2b
\end{eqnarray}


\subsection{Quadrat}\index{Quadrat}

\begin{figure}
    \centering
    \begin{longtable}{rcl}
        \begin{tikzpicture}
            \draw (1cm,1cm) -- (4cm,1cm) -- (4cm, 4cm) -- (1cm, 4cm) -- (1cm, 1cm);
            \node at (0.75cm, 0.75cm) {A};
            \node at (4.25cm, 0.75cm) {B};
            \node at (4.25cm, 4.25cm) {C};
            \node at (0.75cm, 4.25cm) {D};
            \node at (2.5cm, 0.75cm) {a};
            \node at (4.25cm, 2.5cm) {b};
            \node at (2.5cm, 4.25cm) {c};
            \node at (0.75cm, 2.5cm) {d};
            \draw (1cm,1cm) ++(0:0.5) arc (0:90:0.5cm);
            \node at (1.2,1.2) {$\scriptstyle \alpha$};
            \draw (3.5cm,1cm) ++(-0.5:0.0) arc (180:90:0.5cm);
            \node at (3.8,1.2) {$\scriptstyle \beta$};
            \draw (3.5cm,4cm) arc (180:270:0.5cm);
            \node at (3.8,3.8) {$\scriptstyle \gamma$};
            \draw (1cm,3.5cm) arc (-90:0:0.5cm);
            \node at (1.2,3.8) {$\scriptstyle \delta$};
        \end{tikzpicture}
      &
        \begin{tikzpicture}
            \node at (0cm, 0cm) {};
            \node at (0.5cm, 1.5cm) {=};
            \node at (1cm,4.5cm) {};
        \end{tikzpicture}
      &
        \begin{tikzpicture}
            \draw (1cm,1cm) -- (4cm,1cm) -- (4cm, 4cm) -- (1cm, 4cm) -- (1cm, 1cm);
            \node at (0.75cm, 0.75cm) {A};
            \node at (4.25cm, 0.75cm) {B};
            \node at (4.25cm, 4.25cm) {C};
            \node at (0.75cm, 4.25cm) {D};
            \node at (2.5cm, 0.75cm) {a};
            \node at (4.25cm, 2.5cm) {a};
            \node at (2.5cm, 4.25cm) {a};
            \node at (0.75cm, 2.5cm) {a};
            \draw (1cm,1cm) ++(0:0.5) arc (0:90:0.5cm);
            \node at (1.2,1.2) {$\scriptstyle \alpha$};
            \draw (3.5cm,1cm) ++(-0.5:0.0) arc (180:90:0.5cm);
            \node at (3.8,1.2) {$\scriptstyle \alpha$};
            \draw (3.5cm,4cm) arc (180:270:0.5cm);
            \node at (3.8,3.8) {$\scriptstyle \alpha$};
            \draw (1cm,3.5cm) arc (-90:0:0.5cm);
            \node at (1.2,3.8) {$\scriptstyle \alpha$};
        \end{tikzpicture}
    \end{longtable}
  \caption{Planskizze Quadrat}\label{fig:QuadratPlanskizze}
\end{figure}

\begin{definition}[Quadrat]
    Wir nennen ein Rechteck mit vier gleich langen Seiten.
\end{definition}

Beachte dass damit alle Quadrate Rechtecke sind, aber nur einige Rechtecke Quadrate. Quadrate sind spezielle Rechtecke. Sie haben alle Eigenschaften von Rechtecken und müssen zusätzlich die Anforderung an die Seitenlängen erfüllen.


\subsection{Dreieck}\index{Dreieck}\label{Dreieck}

\begin{definition}
    Wir nennen ein Polygon mit drei Ecken \enquote{Dreieck}.
\end{definition}

\begin{figure}
  \centering
  \begin{tikzpicture}[scale=1.25]
        \draw (1cm, 1cm) -- (4cm, 1cm) -- (3cm, 4cm) -- (1cm, 1cm);

        \node at (0.75cm, 0.75cm) {A};
        \node at (4.25cm, 0.75cm) {B};
        \node at (3cm, 4.25cm) {C};

        \node at (2.5cm, 0.75cm) {c};
        \node at (3.75cm, 2.75cm) {a};
        \node at (1.75cm, 2.5cm) {b};

        \draw[dashed] (3cm, 1cm) -- (3cm, 4cm);
        \node at (2.75, 2.5) {$h_c$};
        \draw (2.5cm, 1cm) arc (180:90:0.5cm);
        \node at (2.85cm, 1.15cm) {$\cdot$};

        \draw (1cm,1cm) ++(0:0.5) arc (0:57:0.5cm);
        \node at (1.3,1.2) {$\scriptstyle \alpha$};
        \draw (3.5cm,1cm) ++(-0.5:0.0) arc (180:110:0.5cm);
        \node at (3.7,1.2) {$\scriptstyle \beta$};
        \draw (2.65cm,3.5cm) arc (231:295:0.5cm);
        \node at (2.9,3.6) {$\scriptstyle \gamma$};
  \end{tikzpicture}
  \caption{Planskizze Dreieck}\label{fig:DreieckPlanskizze}
\end{figure}

Wir führen über die Inhalte der Nichtschüler-Prüfung auch den \textsc{Satz des Pythagoras} ein, weil dieser erstens sehr hilfreich ist, auch im Alltag, und weil er zweitens in der Prüfung an der Hauptschule vorkommt, und damit die normale Erwartungshaltung, z.B. der ausbildenden Betriebe, ist, dass dieser über 2500 Jahre alte Satz bekannt ist. Darüber hinaus ist es einer der Sätze mit den einfachsten Beweisen. Zumindest können wir dazu ein oder zwei Beweise angeben um zu zeigen was Mathematik eigentlich bedeutet als beweisende Wissenschaft.

\begin{satz}[Winkelsumme Dreieck]
    Die Summe der Innenwinkel im Dreieck beträgt 180°.
    \begin{equation}\label{eqn:DreieckInnenwinkel}
        \alpha + \beta + \gamma = 180°
    \end{equation}
\end{satz}


\subsubsection{weiter Gehendes}

Die Inhalte dieses Abschnittes sind nicht notwendig für die Prüfung. Der \textsc{Satz des Pythagoras} ist jedoch so nützlich im Alltag, dass wir ihn einführen möchten und der \textsc{Kosinussatz} ist nützlich für Proberechnungen mit dem Taschenrechner bei Aufgaben zum Dreieck, die mit Geodreieck und Zirkel gelöst werden sollen. Der nächste Schritt (genauer der zwischen \textsc{Satz des Pythagoras} und \textsc{Kosinussatz}) wäre die \enquote{Trigonometrie}, Sinus, Cosinus, Tangens, ..., die wir aber in der Tat für eher weniger alltäglich relevant halten und nicht vorziehen vor den Mathematikunterricht der Berufs- oder Realschule, wo dies jeweils behandelt werden sollte.

\begin{satz}[Satz des Pythagoras]\index{Satz des Pythagoras}\label{satz:SatzDesPythagoras}
    In einem rechtwinkligen Dreieck ist die Summe der Quadrate der Katheten gleich der Summe der Hypothenuse.
    \begin{equation}\label{eqn:SatzDesPythagoras}
        a^2 + b^2 = c^2
    \end{equation}
\end{satz}

\begin{beispiel}[Satz des Pythagoras]
    Das bekannteste Beispiel für den \textsc{Satz des Pythagoras} dürfte das kleinste rechtwinklige Dreieck mit ganzzahligen Seitenlängen sein: Das Dreieck $a=3$, $b=4$ und $c=5$ Maßeinheiten, z.B. $a=3\text{m}$, $b=4\text{m}$ und $c=5\text{m}$. Mit entsprechenden Schnüren (eine Schnur mit verknoteten Enden und Knoten im Abstand von je einem Meter) wurden u.a. bereits die rechten Winkel der Paläste von Babylon oder die großen Pyramiden vermessen.
\end{beispiel}

\begin{satz}[Kosinussatz]\index{Kosinussatz}\label{satz:Kosinussatz}
    \begin{equation}\label{eqn:Kosinussatz}
        c^2 = a^2 + b^2 - 2ab \cdot \cos \gamma
    \end{equation}
    wobei $\gamma$ der Winkel zwischen den Seiten $a$ und $b$ ist. Der \textsc{Kosinussatz} unterscheidet sich durch den Term $-2ab \cdot \cos \gamma$ vom \textsc{Satz des Pythagoras}. Da der Kosinus von 90° gleich null ist, fällt dieser Term bei einem rechten Winkel weg, und es ergibt sich als Spezialfall der \textsc{Satz des Pythagoras}.
\end{satz}


\subsection{Parallelogramm}\index{Parallelogramm}\label{Parallelogramm}

\begin{definition}[Parallelogramm]
    Wir nennen ein Viereck mit zwei mal zwei \glsdisp{symb:parallel}{\textbf{parallelen}} Seiten ($a || c$ und $b || d$) \enquote{Parallelogramm}.
\end{definition}

Alle Parallelogramme sind Trapeze.

\begin{figure}
  \centering
    \begin{tikzpicture}
        \draw (1,1) -- (6,1) -- (8,4) -- (3,4) -- (1,1);
        \node at (0.75,0.75) {A};
        \node at (6.25,0.75) {B};
        \node at (8.25,4.25) {C};
        \node at (2.75,4.25) {D};

        \node at (3.5,0.75) {a};
        \node at (7.25,2.5) {b};
        \node at (5.5,4.25) {c};
        \node at (1.75,2.5) {d};

        \draw (1cm,1cm) ++(0:0.5) arc (0:57:0.5cm);
        \node at (1.3,1.2) {$\scriptstyle \alpha$};
        \draw (5.75cm,1.375cm) ++(0:0.5) arc (50:180:0.5cm);
        \node at (5.8,1.2) {$\scriptstyle \beta$};
        \draw (7cm,4cm) ++(0:0.5) arc (180:235:0.5cm);
        \node at (7.7,3.8) {$\scriptstyle \gamma$};
        \draw (3cm,4cm) ++(0:0.5) arc (0:-122:0.5cm);
        \node at (3.1,3.8) {$\scriptstyle \delta$};

        \draw[dashed] (5,1) -- (5,4);
        \node at (4.75,2.5) {$h_a$};
        \draw (4.5,1.5) ++(0:0.5) arc (90:180:0.5cm);
        \node at (4.8,1.2) {$\cdot$};
    \end{tikzpicture}
  \caption{Planskizze Parallelogramm}\label{fig:Parallelogramm}
\end{figure}



\subsection{Trapez}\index{Trapez}\label{Trapez}

\begin{definition}[Trapez]\label{def:Trapez}
    Wir nennen ein Viereck mit zwei Parallelen Seiten \enquote{Trapez}.
\end{definition}

\begin{align}\label{eqn:Trapez}
  A_T &= \frac{(a+c) h_a}{2} \\
  U_T &= a+b+c+d
\end{align}

\begin{figure}
  \centering
    \begin{tikzpicture}[scale=0.8]
        \draw (1,1) -- (8,1) -- (7,4) -- (3,4) -- (1,1);
        \node at (0.75,0.75) {A};
        \node at (8.25,0.75) {B};
        \node at (7.25,4.25) {C};
        \node at (2.75,4.25) {D};

        \node at (3.5,0.75) {a};
        \node at (7.25,2.5) {b};
        \node at (5.5,4.25) {c};
        \node at (1.75,2.5) {d};

        \draw (1cm,1cm) ++(0:0.5) arc (0:57:0.5cm);
        \node at (1.3,1.2) {$\scriptstyle \alpha$};
        \draw (7.21cm,1.51cm) ++(0:0.6) arc (120:180:0.6cm);
        \node at (7.75,1.2) {$\scriptstyle \beta$};
        \draw (6cm,4cm) ++(0:0.5) arc (180:287:0.5cm);
        \node at (6.75,3.8) {$\scriptstyle \gamma$};
        \draw (2.225cm,3.595cm) ++(0:0.5) arc (-122:0:0.5cm);
        \node at (3.1,3.8) {$\scriptstyle \delta$};

        \draw[dashed] (5,1) -- (5,4);
        \node at (4.75,2.5) {$h_a$};
        \draw (4.5,1.5) ++(0:0.5) arc (90:180:0.5cm);
        \node at (4.8,1.2) {$\cdot$};
    \end{tikzpicture}
  \caption{Planskizze Trapez}\label{fig:Trapez}
\end{figure}

Beachte dass in der Formel $A_T = \frac{(a+c) h_a}{2}$ nur die Seiten gewählt werden können, die parallel sind, also gelten muss $a \| c$. Der Term $(a+c) h$ ist die Fläche des Parallelogramms, das durch das Verdoppeln des Trapezes entsteht. Entsprechend ist diese Fläche des größeren Parallelogramms durch zwei zu teilen:
\begin{figure}
  \centering
    \begin{tikzpicture}[scale=0.75]
        \draw (1,1) -- (8,1) -- (7,4) -- (3,4) -- (1,1);
        \node at (0.75,0.75) {A};
        \node at (8.25,0.75) {B (C')};
        \node at (7.25,4.25) {C (B')};
        \node at (2.75,4.25) {D};

        \node at (3.5,0.75) {a};
        \node at (7.25,2.5) {b};
        \node at (5.5,4.25) {c};
        \node at (1.75,2.5) {d};

        \draw (1cm,1cm) ++(0:0.5) arc (0:57:0.5cm);
        \node at (1.3,1.2) {$\scriptstyle \alpha$};
        \draw (7.21cm,1.51cm) ++(0:0.6) arc (120:180:0.6cm);
        \node at (7.75,1.2) {$\scriptstyle \beta$};
        \draw (6cm,4cm) ++(0:0.5) arc (180:287:0.5cm);
        \node at (6.75,3.8) {$\scriptstyle \gamma$};
        \draw (2.225cm,3.595cm) ++(0:0.5) arc (-122:0:0.5cm);
        \node at (3.1,3.8) {$\scriptstyle \delta$};

        \draw[dashed] (5,1) -- (5,4);
        \node at (4.75,2.5) {$h_a$};
        \draw (4.5,1.5) ++(0:0.5) arc (90:180:0.5cm);
        \node at (4.8,1.2) {$\cdot$};

        \draw (8,1) -- (12,1) -- (14,4) -- (7,4);
        \node at (10,0.75) {c'};
        \node at (10.5,4.25) {a'};
        \node at (7.75,2.5) {b'};
        \node at (13.25,2.5) {d'};
        \node at (11,2.5) {${A_T}' = \frac{1}{2} A_P$};

        \node at (12.25, 0.75) {D'};
        \node at (14.25, 4.25) {A'};
    \end{tikzpicture}
  \caption{Berechnung von $A_T$ als $\frac{1}{2} A_P$}\label{fig:TrapezATausAP}
\end{figure}




\subsection{Kreis}\index{Kreis}\label{Kreis}

Der \textbf{Kreis} ist die einzige Fläche, die für den Hauptschulabschluss relevant ist, die kein Polygon ist. Sie hat keine Ecke. Sie wird beschrieben durch \textbf{Mittelpunkt} und \textbf{Radius}.

\begin{definition}
    Wir nennen eine Fläche, die von einer kreisförmigen Linie begrenzt ist, d.h. jeder Punkt auf dem Rand der Fläche ist gleich weit vom \textbf{Mittelpunkt} der Fläche entfernt, \enquote{Kreis}.
\end{definition}

\begin{align}\label{eqn:Kreis}
  A_K &= \pi r^2 \\
  U_K &= 2 \pi r
\end{align}


\subsection{(zusammengesetzte) Figuren (in der Ebene)}

Im Berufsleben, oder z.B. auch als Heimwerker, haben wir es selten mit elementaren Flächen zu tun. In der Regel treffen wir auf Flächen, die wir erst aus solchen zusammensetzen oder ausschneiden müssen.

Aufgaben der Art \enquote{zusammengesetzte Figur} lösen wir indem wir sie in elementare Flächen zerlegen (wenn wir die Fläche berechnen sollen):
\begin{enumerate}
  \item \textbf{Zerlegung in elementare Flächen:} Wir \enquote{puzzeln} die Figur aus elementaren Flächen, d.h. wir legen (ohne Überschneidung) aus elementaren Flächen zusammen oder schneiden elementare Flächen aus der Figur aus.
  \item \textbf{Aufstellen der Summenformel:} Wir schreiben die gesamte Fläche $A_\Sigma$ als Summe aus elementaren Flächen. Zusammen gelegte Flächen addieren wird, weggeschittene subtrahieren wir.
  \item \textbf{Berechnen der Unbekannten aus der Summenformel:} Wir schreiben für jede elementare Fläche die Formel, setzen ein und rechnen aus.
  \item \textbf{Berechnen der Summenformel:} Wir rechnen die Summe aus und erhalten die Lösung der Aufgabe.
\end{enumerate}

Tipps:
\begin{itemize}
  \item (In der Hauptschule) berechnet sich jede Fläche, die einen Bogen hat, durch einen Kreis oder Teil eines Kreises, oft, einen Halbkreis.
  \item Mit einem großen Rechteck aus dem man Lücken oder Löcher ausschneidet kann man oft die Rechnung vereinfachen/ verkürzen.
  \item (Scheinbar) schwierige Figuren lassen sich manchmal vereinfachen dadurch dass eine elementare Fläche mehrfach vorkommt.
\end{itemize}


\begin{beispiel}
    Zeichne die abgebildete Figur (s. Abb. \ref{fig:BeispielEbeneFlaeche001}) im Maßstab 1:100 ab und berechne die Fläche.

    \begin{figure}
      \centering
      \begin{tikzpicture}[scale=0.75]
            \draw (1,9) -- (1,4) -- (3.25,1) -- (5.5,4) -- (5.5,9) arc (0:180:2.25);

            \draw[<->] (0.5,4) -- (0.5,9) node[rotate=90, anchor=center,midway,fill=white] {\footnotesize 10m};%rotate=90, anchor=center,
            \draw[<->] (0.5,1) -- (0.5,4) node[rotate=90, anchor=center,midway,fill=white] {\footnotesize 6m};%rotate=90, anchor=center,
            \draw[<->] (1,0.5) -- (5.5,0.5) node[midway,fill=white] {\footnotesize 9m};%rotate=90, anchor=center,
      \end{tikzpicture}
      \hspace{1cm}
      \begin{tikzpicture}[scale=0.75]
            \draw (1,9) -- (1,4) -- (3.25,1) -- (5.5,4) -- (5.5,9) arc (0:180:2.25);
            \draw[dashed] (1,4) -- (5.5,4);
            \draw[dashed] (1,9) -- (5.5,9);

            \draw[<->] (0.5,4) -- (0.5,9) node[rotate=90, anchor=center,midway,fill=white] {\footnotesize 10m};%rotate=90, anchor=center,
            \draw[<->] (0.5,1) -- (0.5,4) node[rotate=90, anchor=center,midway,fill=white] {\footnotesize 6m};%rotate=90, anchor=center,
            \draw[<->] (1,0.5) -- (5.5,0.5) node[midway,fill=white] {\footnotesize 9m};%rotate=90, anchor=center,

            \node at (3.25,2.5) {$A_3$};
            \node at (3.25,6.25) {$A_2$};
            \node at (3.25,10.125) {$A_1$};
      \end{tikzpicture}
      \caption{zeichne im Maßstab und berechne die Fläche}\label{fig:BeispielEbeneFlaeche001}
    \end{figure}

    \paragraph{1.) Zerlegung in elementare Flächen}
    Wir setzen die Figur aus einem Halbkreis $A_1$, einem Rechteck $A_2$ und einem Dreieck $A_3$ zusammen.

    \paragraph{2.) Aufstellen der Summenformel}
    \begin{equation*}
        A_\Sigma = A_1 + A_2 + A_3
    \end{equation*}

    \paragraph{3.) Berechnen der Unbekannten}
    \begin{longtable}{l}
        $A_1 = \frac{1}{2} A_K = \frac{1}{2} \pi r^2 \approx \frac{1}{2} \cdot 3,14 \cdot (4,5\text{m})^2 \approx 32\text{m}^2$\\
        $A_2 = A_R = a \cdot b = 9\text{m} \cdot 10\text{m} = 90\text{m}^2$\\
        $A_3 = A_D = \frac{a \cdot h_a}{2} = \frac{9\text{m} \cdot 6\text{m}}{2} = 27\text{m}^2$
    \end{longtable}

    \paragraph{4.) Berechnen der Summenformel}
    \begin{longtable}{lcl}
        $A_\Sigma$ & = & $A_1 + A_2 + A_3$\\
        & $\approx$ & $149\text{m}^2$
    \end{longtable}

    Antwort: Die Figur hat eine Fläche von rund $149\text{m}^2$.\proofsquare
\end{beispiel}


\begin{beispiel}
    Berechne die Fläche der abgebildeten Figur (s. Abb. \ref{fig:BeispielEbeneFläche02}).

    \begin{figure}
      \centering
      \begin{tikzpicture}[scale=0.5]%rotate=-90, , transform shape
            \draw (1,1) -- (4.5,1) -- (4.5,2.5) -- (6.5,2.5) -- (6.5,1) -- (9.5,1) arc (180:0:1.5) -- (12.5,1) -- (15.5,1) arc (-90:90:5) -- (1,11) -- (1,1);

            \draw[<->] (1,0.5) -- (4.5,0.5) node[midway,fill=white] {\footnotesize 3,5m};%rotate=90, anchor=center,
            \draw[<->] (4.5,0.5) -- (6.5,0.5) node[midway,fill=white] {\footnotesize 2m};%rotate=90, anchor=center,
            \draw[<->] (6.5,0.5) -- (9.5,0.5) node[midway,fill=white] {\footnotesize 3m};%rotate=90, anchor=center,
            \draw[<->] (9.5,0.5) -- (12.5,0.5) node[midway,fill=white] {\footnotesize 3m};%rotate=90, anchor=center,
            \draw[<->] (12.5,0.5) -- (15.5,0.5) node[midway,fill=white] {\footnotesize 3m};%rotate=90, anchor=center,
            \draw[<->] (15.5,0.5) -- (20.5,0.5) node[midway,fill=white] {\footnotesize 5m};%rotate=90, anchor=center,
      \end{tikzpicture}\\
      \vspace{1em}
      \begin{tikzpicture}[scale=0.5]%rotate=-90, , transform shape
            \draw (1,1) -- (4.5,1) -- (4.5,2.5) -- (6.5,2.5) -- (6.5,1) -- (9.5,1) arc (180:0:1.5) -- (12.5,1) -- (15.5,1) arc (-90:90:5) -- (1,11) -- (1,1);

            \draw[<->] (1,0.5) -- (4.5,0.5) node[midway,fill=white] {\footnotesize 3,5m};%rotate=90, anchor=center,
            \draw[<->] (4.5,0.5) -- (6.5,0.5) node[midway,fill=white] {\footnotesize 2m};%rotate=90, anchor=center,
            \draw[<->] (6.5,0.5) -- (9.5,0.5) node[midway,fill=white] {\footnotesize 3m};%rotate=90, anchor=center,
            \draw[<->] (9.5,0.5) -- (12.5,0.5) node[midway,fill=white] {\footnotesize 3m};%rotate=90, anchor=center,
            \draw[<->] (12.5,0.5) -- (15.5,0.5) node[midway,fill=white] {\footnotesize 3m};%rotate=90, anchor=center,
            \draw[<->] (15.5,0.5) -- (20.5,0.5) node[midway,fill=white] {\footnotesize 5m};%rotate=90, anchor=center,

            \draw[dashed] (15.5,1) -- (15.5,11);
            \draw[dashed] (4.5,1) -- (6.5,1);
            \draw[dashed] (9.5,1) -- (12.5,1);

            \node at (8.25,6) {$A_1$};
            \node at (5.5,1.75) {$A_2$};
            \node at (11,1.75) {$A_3$};
            \node at (18,6) {$A_4$};
      \end{tikzpicture}
      \caption{Flächenberechnung einer ebenen Figur}\label{fig:BeispielEbeneFläche02}
    \end{figure}

    \paragraph{1.) Zerlegung in elementare Flächen}

    Wir zerlegen die Figur in das Rechteck $A_1$, die Ausschnitte aus $A_1$, das Rechteck $A_2$ und den Halbkreis $A_3$, und den Halbkreis $A_4$.

    \paragraph{2.) Aufstellen der Summenformel}

    \begin{equation*}
        A_\Sigma = A_1 - A_2 - A_3 + A_4
    \end{equation*}

    \paragraph{3.) Berechnen der Unbekannten in der Summenformel}

    \paragraph{4.) Ausrechnen der Summenformel}

    \begin{longtable}{l}
        $A_1 = A_R = a \cdot b = 14,5\text{m} \cdot 10\text{m} = 145\text{m}^2$\\
        $A_2 = A_R = a \cdot b = 2\text{m} \cdot 1,5\text{m} = 3\text{m}^2$\\
        $A_3 = \frac{1}{2} A_K = \frac{1}{2} \cdot \pi r^2 \approx \frac{1}{2} \cdot 3,14 \cdot (1,5\text{m})^2 \approx 3,5\text{m}^2$\\
        $A_4 = \frac{1}{2} A_K = \frac{1}{2} \cdot \pi r^2 \approx \frac{1}{2} \cdot 3,14 \cdot (5\text{m})^2 \approx 39,25\text{m}^2$
    \end{longtable}

    \paragraph{4.) Ausrechnen der Summenformel}

    \begin{longtable}{lcl}
        $A_\Sigma$ & = & $A_1 - A_2 - A_3 + A_4$\\
        & $\approx$ & $145\text{m}^2 - 3\text{m}^2 - 3,5\text{m}^2 + 39,25\text{m}^2$\\
        & $\approx$ & $178 \text{m}^2$
    \end{longtable}\proofsquare
\end{beispiel}


\begin{beispiel}[zusammengesetzte Figur]
    Dieses Beispiel ist \textbf{viel} komplizierter als die, die Euch in der Prüfung begegnen können. Wer das Beispiel selbst lösen kann, bekommt mit dem Thema \enquote{zusammengesetzte Figuren} in der Prüfung keine Probleme. Jeder sollte verstehen wie die Lösung funktioniert.

    Es gibt unendlich viele korrekte Möglichkeiten eine Figur in kleinere elementare Flächen zu zerlegen. Sinnvoll sind meistens ein paar davon. Die folgende Lösung ist also nur beispielhaft für diese Aufgabe. Herauskommen muss bei allen korrekten Aufteilungen und korrektem Rechnen immer der gleiche Flächeninhalt.

    \begin{figure}[ht]
      \centering
      \begin{tikzpicture}[scale=0.9]
        \draw[fill=gray!15] (1,5) -- (1,9) -- (3,11) -- (4,11) -- (4,10) -- (5,10) -- (5,11) -- (6,11) -- (6,9) -- (7,9) -- (7,11) -- (8,11) -- (8,8) -- (9,8) -- (9,11) -- (10,11)  arc (90:0:3) -- (10,8) -- (10,7) -- (13,7) -- (13,6) -- (10,6) -- (10,5) -- (13,5) -- (13,4)-- (10,4)-- (10,3)-- (13,3)-- (13,2) -- (6,1) -- (6,3) -- (4,3) -- (4,4) arc (-90:180:1.5) -- (2.5,3) -- (2,3) -- (1,5);

        \draw[<->] (1,0.5) -- (10,0.5) node[midway,fill=white] {\footnotesize 9m};%rotate=90, anchor=center,
        \draw[<->] (2.5,5.5) -- (5.5,5.5) node[midway,fill=white] {\footnotesize 3m};%rotate=90, anchor=center,
        \draw[<->] (10,0.5) -- (13,0.5) node[midway,fill=white] {\footnotesize 3m};%rotate=90, anchor=center,
        %\draw[<->] (13.5,1) -- (13.5,11) node[rotate=90, anchor=center, midway,fill=white] {\footnotesize 10m};%
        \draw[<->] (13.25,2) -- (13.25,3) node[rotate=90, anchor=center, midway,fill=white] {\footnotesize 1m};%
        \draw[<->] (13.25,3) -- (13.25,4) node[rotate=90, anchor=center, midway,fill=white] {\footnotesize 1m};%
        \draw[<->] (13.25,4) -- (13.25,5) node[rotate=90, anchor=center, midway,fill=white] {\footnotesize 1m};%
        \draw[<->] (13.25,5) -- (13.25,6) node[rotate=90, anchor=center, midway,fill=white] {\footnotesize 1m};%
        \draw[<->] (13.25,6) -- (13.25,7) node[rotate=90, anchor=center, midway,fill=white] {\footnotesize 1m};%
        \draw[<->] (13.25,7) -- (13.25,8) node[rotate=90, anchor=center, midway,fill=white] {\footnotesize 1m};%
        \draw[<->] (13.25,8) -- (13.25,11) node[rotate=90, anchor=center, midway,fill=white] {\footnotesize 3m};%

        \draw[<->] (1,11.25) -- (3,11.25) node[midway,fill=white] {\footnotesize 2m};%rotate=90, anchor=center,
        \draw[<->] (3,11.25) -- (4,11.25) node[midway,fill=white] {\footnotesize 1m};%rotate=90, anchor=center,
        \draw[<->] (4,11.25) -- (5,11.25) node[midway,fill=white] {\footnotesize 1m};%rotate=90, anchor=center,
        \draw[<->] (5,11.25) -- (6,11.25) node[midway,fill=white] {\footnotesize 1m};%rotate=90, anchor=center,
        \draw[<->] (6,11.25) -- (7,11.25) node[midway,fill=white] {\footnotesize 1m};%rotate=90, anchor=center,
        \draw[<->] (7,11.25) -- (8,11.25) node[midway,fill=white] {\footnotesize 1m};%rotate=90, anchor=center,
        \draw[<->] (8,11.25) -- (9,11.25) node[midway,fill=white] {\footnotesize 1m};%rotate=90, anchor=center,
        \draw[<->] (9,11.25) -- (10,11.25) node[midway,fill=white] {\footnotesize 1m};%rotate=90, anchor=center,
        \draw[<->] (4.5,10) -- (4.5,11) node[rotate=90, anchor=center, midway,fill=white] {\footnotesize 1m};%
        \draw[<->] (6.5,9) -- (6.5,11) node[rotate=90, anchor=center, midway,fill=white] {\footnotesize 2m};%
        \draw[<->] (8.5,8) -- (8.5,11) node[rotate=90, anchor=center, midway,fill=white] {\footnotesize 3m};%

        \draw[<->] (1,2.75) -- (2,2.75) node[midway,fill=white] {\footnotesize 1m};%rotate=90, anchor=center,
        \draw[<->] (1,2.25) -- (2.5,2.25) node[midway,fill=white] {\footnotesize 1,5m};%rotate=90, anchor=center,
        \draw (2.5,2) -- (2.5, 3);
        \draw (2,2.5) -- (2, 3);
        \draw[<->] (2.5,2.75) -- (4,2.75) node[midway,fill=white] {\footnotesize 1,5m};%rotate=90, anchor=center,
        \draw[<->] (4,2.75) -- (6,2.75) node[midway,fill=white] {\footnotesize 2m};%rotate=90, anchor=center,

        \draw[<->] (0.75,1) -- (0.75,3) node[rotate=90, anchor=center, midway,fill=white] {\footnotesize 2m};%
        \draw[<->] (0.75,3) -- (0.75,5) node[rotate=90, anchor=center, midway,fill=white] {\footnotesize 2m};%
        \draw[<->] (0.75,5) -- (0.75,9) node[rotate=90, anchor=center, midway,fill=white] {\footnotesize 4m};%
        \draw[<->] (0.75,9) -- (0.75,11) node[rotate=90, anchor=center, midway,fill=white] {\footnotesize 2m};%
      \end{tikzpicture}
      \caption{Zusammengesetzte Figuren, Aufgabe}\label{fig:ZusammengesetzteFigurenKomplexAufg}
    \end{figure}

    \begin{figure}[ht]
      \centering
      \begin{tikzpicture}[scale=0.9]
        \draw[fill=gray!15] (1,5) -- (1,9) -- (3,11) -- (4,11) -- (4,10) -- (5,10) -- (5,11) -- (6,11) -- (6,9) -- (7,9) -- (7,11) -- (8,11) -- (8,8) -- (9,8) -- (9,11) -- (10,11)  arc (90:0:3) -- (10,8) -- (10,7) -- (13,7) -- (13,6) -- (10,6) -- (10,5) -- (13,5) -- (13,4)-- (10,4)-- (10,3)-- (13,3)-- (13,2) -- (6,1) -- (6,3) -- (4,3) -- (4,4) arc (-90:180:1.5) -- (2.5,3) -- (2,3) -- (1,5);

        \draw[<->] (1,0.5) -- (10,0.5) node[midway,fill=white] {\footnotesize 9m};%rotate=90, anchor=center,
        \draw[<->] (2.5,5.5) -- (5.5,5.5) node[midway,fill=white] {\footnotesize 3m};%rotate=90, anchor=center,
        \draw[<->] (10,0.5) -- (13,0.5) node[midway,fill=white] {\footnotesize 3m};%rotate=90, anchor=center,
        %\draw[<->] (13.5,1) -- (13.5,11) node[rotate=90, anchor=center, midway,fill=white] {\footnotesize 10m};%
        \draw[<->] (13.25,2) -- (13.25,3) node[rotate=90, anchor=center, midway,fill=white] {\footnotesize 1m};%
        \draw[<->] (13.25,3) -- (13.25,4) node[rotate=90, anchor=center, midway,fill=white] {\footnotesize 1m};%
        \draw[<->] (13.25,4) -- (13.25,5) node[rotate=90, anchor=center, midway,fill=white] {\footnotesize 1m};%
        \draw[<->] (13.25,5) -- (13.25,6) node[rotate=90, anchor=center, midway,fill=white] {\footnotesize 1m};%
        \draw[<->] (13.25,6) -- (13.25,7) node[rotate=90, anchor=center, midway,fill=white] {\footnotesize 1m};%
        \draw[<->] (13.25,7) -- (13.25,8) node[rotate=90, anchor=center, midway,fill=white] {\footnotesize 1m};%
        \draw[<->] (13.25,8) -- (13.25,11) node[rotate=90, anchor=center, midway,fill=white] {\footnotesize 3m};%

        \draw[<->] (1,11.25) -- (3,11.25) node[midway,fill=white] {\footnotesize 2m};%rotate=90, anchor=center,
        \draw[<->] (3,11.25) -- (4,11.25) node[midway,fill=white] {\footnotesize 1m};%rotate=90, anchor=center,
        \draw[<->] (4,11.25) -- (5,11.25) node[midway,fill=white] {\footnotesize 1m};%rotate=90, anchor=center,
        \draw[<->] (5,11.25) -- (6,11.25) node[midway,fill=white] {\footnotesize 1m};%rotate=90, anchor=center,
        \draw[<->] (6,11.25) -- (7,11.25) node[midway,fill=white] {\footnotesize 1m};%rotate=90, anchor=center,
        \draw[<->] (7,11.25) -- (8,11.25) node[midway,fill=white] {\footnotesize 1m};%rotate=90, anchor=center,
        \draw[<->] (8,11.25) -- (9,11.25) node[midway,fill=white] {\footnotesize 1m};%rotate=90, anchor=center,
        \draw[<->] (9,11.25) -- (10,11.25) node[midway,fill=white] {\footnotesize 1m};%rotate=90, anchor=center,
        \draw[<->] (4.8,10) -- (4.8,11) node[rotate=90, anchor=center, midway,fill=white] {\footnotesize 1m};%
        \draw[<->] (6.8,9) -- (6.8,11) node[rotate=90, anchor=center, midway,fill=white] {\footnotesize 2m};%
        \draw[<->] (8.8,8) -- (8.8,11) node[rotate=90, anchor=center, midway,fill=white] {\footnotesize 3m};%

        \draw[<->] (1,2.75) -- (2.5,2.75) node[midway,fill=white] {\footnotesize 1,5m};%rotate=90, anchor=center,
        \draw[<->] (2.5,2.75) -- (4,2.75) node[midway,fill=white] {\footnotesize 1,5m};%rotate=90, anchor=center,
        \draw[<->] (4,2.75) -- (6,2.75) node[midway,fill=white] {\footnotesize 2m};%rotate=90, anchor=center,

        \draw[<->] (0.75,1) -- (0.75,3) node[rotate=90, anchor=center, midway,fill=white] {\footnotesize 2m};%
        \draw[<->] (0.75,3) -- (0.75,5) node[rotate=90, anchor=center, midway,fill=white] {\footnotesize 2m};%
        \draw[<->] (0.75,5) -- (0.75,9) node[rotate=90, anchor=center, midway,fill=white] {\footnotesize 4m};%
        \draw[<->] (0.75,9) -- (0.75,11) node[rotate=90, anchor=center, midway,fill=white] {\footnotesize 2m};%

        \draw[dashed] (10,11) -- (10,8);
        \node at (11.5,9.5) {$A_1$};

        \draw[dashed] (6,3) -- (10,3);
        \node at (9,2) {$A_2$};

        \draw[dashed] (1,9) -- (10,9);
        \draw[dashed] (3,11) -- (10,11);
        \node at (3,10) {$A_3$};

        \draw[dashed] (1,3) -- (13,3) -- (13,8) -- (1,8) -- (1,3);
        \node at (8,5.5) {$A_4$};
        \node at (3.5,8.5) {$A_5$};

        \node at (11.5,7.5) {$A_6$};
        \node at (11.5,5.5) {$A_6$};
        \node at (11.5,3.5) {$A_6$};

        \node at (4.4,10.5) {$A_7$};
        \node at (6.4,10.5) {$A_8$};
        \node at (8.4,10.5) {$A_9$};

        \draw[dashed] (2,3) -- (2,8);
        \node at (1.5,6) {$A_{10}$};

        \draw[dashed] (2.5,3) -- (4,3)-- (4,5.5) -- (2.5,5.5) -- (2.5,3);
        \node at (3.25,4) {$A_{11}$};
        \node at (4,6) {$A_{12}$};
      \end{tikzpicture}
      \caption{Aufteilung in elementare Flächen}\label{fig:ZusammengesetzteFigurenKomplexAufgLoes}
    \end{figure}

    \textbf{gs.: Die Fläche $A_{\Sigma}$ der zusammengesetzten Figur.}

    \paragraph{1.) Summenformel für die Figur aufstellen}

    Zunächst teilen wir die Fläche in Teilflächen auf, die elementare Flächen sind (für die wir Formeln haben). Dafür gibt es unendlich viele Möglichkeiten. Das hier ist ein Beispiel für eine gute Lösung (s. Abb. \ref{fig:ZusammengesetzteFigurenKomplexLoes}). In diesem Beispiel kombinieren wir überschneidungsfreies Zusammenlegen mit Ausschneiden. Beachte in diesem Beispiel besonders die Flächen  $A_{11} und A_{12}$, die aus $A_4$ ausgeschnitten werden. Wir müssen sie so schneiden, dass sie keine Überschneidung haben, da wir sonst (aufwändig) korrigieren müssen um die Überschneidung nicht doppelt abzuziehen.
    \begin{align*}
      A_\Sigma & = & A_1 + A_2 + A_3 + A_4 + A_5 + A_{10} \\
      & & - 3 A_6 - A_7 - A_8 - A_9 - A_{11} - A_{12}
    \end{align*}

    \paragraph{2.) Unbekannte der Summenformel berechnen}

    Als nächstes bestimmen wir die einzelnen Teilflächen. Für einen ausführliche Musterlösung (z.B. als Übung für die mündliche Prüfung) geben wir zunächst die Formel der Fläche an, setzen dann ein und rechnen aus. Die fertigen Teilflächen werden dann in die Formel eingesetzt, die wir im ersten Schritt erhalten haben.

    \renewcommand*{\arraystretch}{1.4}%should affect this table only, why..?
    \begin{longtable}{rcl}
        $A_1$ & $=$ & $\frac{1}{4} A_K = \frac{1}{4} \pi r^2 = \pi \cdot (3\text{m})^2 = \frac{1}{4} \pi \cdot 9\text{m}^2 \approx 7,07\text{m}^2$\\
        $A_2$ & $=$ & $A_T = \frac{(a+c)h}{2} = \frac{(1\text{m}+2\text{m}) \cdot 7\text{m}}{2} = 10,5\text{m}^2$\\
        $A_3$ & $=$ & $A_T= \frac{(a+c)h}{2}=\frac{(7\text{m}+9\text{m}) \cdot 3\text{m}}{2}= 24\text{m}^2$\\
        $A_4$ & $=$ & $A_R= a \cdot b = 11\text{m} \cdot 5\text{m}= 55\text{m}^2$\\
        $A_5$ & $=$ & $A_R= a \cdot b = 9\text{m} \cdot 1\text{m}= 9\text{m}^2$\\
        $A_6$ & $=$ & $A_R= a \cdot b = 3\text{m} \cdot 1\text{m}= 3\text{m}^2$\\
        $A_7$ & $=$ & $A_Q= a^2= 1\text{m}^2$\\
        $A_8$ & $=$ & $A_R= a \cdot b= 2\text{m} \cdot 1\text{m}= 2\text{m}^2$\\
        $A_9$ & $=$ & $A_R= a \cdot b= 3\text{m} \cdot 1\text{m}= 3\text{m}^2$\\
        $A_{10}$ & $=$ & $A_T= \frac{(a+c)h}{2}= \frac{(6\text{m}+4\text{m}) \cdot 1,5\text{m}}{2}= 7,5\text{m}^2$\\
        $A_{11}$ & $=$ & $A_R= a \cdot b= 1,5\text{m} \cdot 2,5\text{m}= 3,75\text{m}^2$\\
        $A_{12}$ & $=$ & $\frac{3}{4} A_K= \pi r^2= \pi \cdot (1,5\text{m})^2= \frac{3}{4} \pi \cdot 2,25\text{m}^2\approx 5,3\text{m}^2$
    \end{longtable}

    \paragraph{4.) Summenformel einsetzen und ausrechnen}

    \begin{align}
      A_{\Sigma} & \approx 7,07\text{m}^2 + 10,5\text{m}^2 + 24\text{m}^2\\
       &\quad + 55\text{m}^2 + 9\text{m}^2 - 3(3\text{m}^2) - 1\text{m}^2 - 2\text{m}^2\\
       &\quad - 3\text{m}^2 + 7,5\text{m}^2 - 3,75\text{m}^2 - 5,3\text{m}^2\\
       & \approx 89,02\text{m}^2
    \end{align}

    \paragraph{Plausibilität prüfen}

    Ohne Ausschnitte und großzügig einen Rahmen ziehend wäre es ein Rechteck mit 12m mal 10m, also 120m\textsuperscript{2}. Das klingt plausibel. Durch die ordentliche Notation sollten wir in einer Prüfung auf jeden Fall viele Teilpunkte bekommen und können darauf hoffen auch Fehler zu finden falls wir nochmal über die Rechnung drüber schauen.\proofsquare
\end{beispiel}

Die Geometrie ist für viele von Euch wahrscheinlich auch ein gutes Thema für die mündliche Prüfung, da man sehr deutlich \enquote{sieht} was man tut und anhand der Zeichnung meistens selbst sieht falls man in die falsche Richtung läuft oder abwegige Ergebnisse bekommt. Außerdem dauert das Zeichnen recht lange, so dass man in der Regel nur ein oder zwei Aufgaben in der kurzen mündlichen Prüfung zu zeigen braucht.



\section{Koordinaten}\label{Koordinaten}\index{Koordinaten}\index{Koordinatensystem}

Um über Geometrie sprechen zu können müssen wir Punkte (z.B. Ecken eines Polygons) so beschreiben können, dass wir verstehen \enquote{wo} sie liegen. Wir nutzen dazu wieder den Zahlenstrahl und zwar nehmen wir für jede Dimension einen (senkrecht auf den anderen Achsen stehenden) Zahlenstrahl (und nennen ihn eine \textbf{Achse}). In jeder Dimension müssen wir die Entfernung zum Schnittpunkt der Achsen angeben. In der 2-dimensionalen Fläche schreiben wir für den Punkt A z.B. den \glsdisp{symb:Tupel}{Tupel (das Paar)} $A=(x,y)$.

\begin{beispiel}[Koordinatensystem 2D]
\textbf{Aufgabe:} Zeichne das Polygon $A=(-7|-7)$, $B=(6|-2)$, $C=(7|8)$, $D=(-6|3)$\footnote{Die Koordinaten der Punkte, z.B. $A=(-7|-7)$, können unter anderem auch \enquote{$(-1,-7)$} oder \enquote{$(-7;-7)$} geschrieben werden. In Deutschland sind \enquote{$|$} und \enquote{$;$} gebräuchlich, in den USA vor allem \enquote{$,$}.} und bestimme um was für eine Art von Polygon es sich handelt und dessen Flächeninhalt und Umfang. Einige Maße musst Du in Deiner Zeichnung messen.

\textbf{Lösung:} Zunächst wählen wir ein passendes Koordinatensystem aus. Der kleinste x-Wert der Punkte ist -7, der größte 7, der kleinste y-Wert -7 und der größte y-Wert 8. Wir brauchen also fast exakt ein Quadrat um den \textbf{Ursprung} $(0|0)$ herum  mit einer Seitenlänge von 16. Das ist auch gut lesbar, nehmen wir also. Wir zeichnen das Koordinatensystem, tragen die Punkte ein und verbinden benachbarte Punkte (welche das sind entnehmen wir der Konventionen für Namen in Polygonen [z.B. \enquote{Die Seite \enquote{a} ist die Strecke von der Ecke \enquote{A} zur Ecke \enquote{B}.}]).

\begin{figure}[ht]
  \centering
  \begin{tikzpicture}[x=.5cm, y=.5cm,domain=-9:9,smooth]
       %Raster zeichnen
       \draw [color=gray!50]  [step=5mm] (-11,-10) grid (11,10);
       % Achsen zeichnen
       \draw[->,thick] (-10,0) -- (10,0) node[right] {$x$};
       \draw[->,thick] (0,-9) -- (0,9) node[above] {$y$};
       % Achsen beschriften
       \foreach \c in {-8,-6,...,-2,2,4,...,8}{
         \draw (\c,-.1) -- (\c,.1) node[below=4pt] {$\scriptstyle\c$};
         \draw (-.1,\c) -- (.1,\c) node[left=4pt] {$\scriptstyle\c$};
   }

   \node[label=225:A] (A) at (-7,-7) {$\times$};
   \node[label=315:B] (B) at (6,-2) {$\times$};
   \node[label=45:C] (C) at (7,8) {$\times$};
   \node[label=135:D] (D) at (-6,3) {$\times$};

   \draw (A) -- (B) -- (C) -- (D) -- (A);

  \end{tikzpicture}
  \caption{2-dimensionales Koordinatensystem}\label{fig:koordinatensystem2D}
\end{figure}

Wir sehen direkt (schon in der Aufgabe), dass es sich um ein \textbf{Viereck} handelt, denn jedes Polygon mit genau vier Ecken ist ein Viereck und wir haben in der Aufgabe ein Polygon mit genau vier Ecken gegeben. Wir überprüfen ob es sich auch um ein \textbf{Trapez} handelt. Nach Definition (des Trapez) brauchen wir dafür zusätzlich zum Viereck noch zwei parallele Seiten. In Frage kommen die Seiten, die keine gemeinsamen Ecken haben, also fragen wir ob $a \| c$ und ob $b \| d$.

An der Zeichnung können wir dies messen indem wir versuchen eine gemeinsame Höhe für beide Seiten zu zeichnen, d.h. ist eine Senkrechte auf $a$ auch senkrecht auf $c$, ist $h_a = h_c$?

\begin{figure}[ht]
  \centering
  \begin{tikzpicture}[x=.5cm, y=.5cm,domain=-9:9,smooth]
       %Raster zeichnen
       \draw [color=gray!50]  [step=5mm] (-11,-10) grid (11,10);
       % Achsen zeichnen
       \draw[->,thick] (-10,0) -- (10,0) node[right] {$x$};
       \draw[->,thick] (0,-9) -- (0,9) node[above] {$y$};
       % Achsen beschriften
       \foreach \c in {-8,-6,...,-2,2,4,...,8}{
         \draw (\c,-.1) -- (\c,.1) node[below=4pt] {$\scriptstyle\c$};
         \draw (-.1,\c) -- (.1,\c) node[left=4pt] {$\scriptstyle\c$};
   }

   \node[label=225:A] (A) at (-7,-7) {$\times$};
   \node[label=315:B] (B) at (6,-2) {$\times$};
   \node[label=45:C] (C) at (7,8) {$\times$};
   \node[label=135:D] (D) at (-6,3) {$\times$};

   \draw (A) -- (B) -- (C) -- (D) -- (A);

   \node (Ha)[label=45:$\cdot$] at (2,-3.53846) {};
   \node (Hc)[label=240:$\cdot$] at (-1,4.92308) {};
   \draw (Ha) -- (Hc);
   \node at (2,-2) {\footnotesize $h_a$};
   \node at (-1,3) {\footnotesize $h_c$};
   \draw (Ha) ++(1.0,0.4) arc (20:110:1);
   \draw (Hc) ++(-1.0,-0.4) arc (205:295:1);

  \end{tikzpicture}
  \caption{$h_a = h_c$, daraus folgt: es ist ein Trapez}\label{fig:koordinatensystem2D}
\end{figure}

Anhand der Zeichnung bestätigt sich, dass $h_a=h_c$, d.h. wir haben ein Trapez. Ist es auch ein Parallelogramm? Dafür müssten auch die anderen beiden Seiten parallel sein. Wir prüfen also auch noch ob $h_b = h_d$ ist.

\begin{figure}[ht]
  \centering
  \begin{tikzpicture}[x=.5cm, y=.5cm,domain=-9:9,smooth]
       %Raster zeichnen
       \draw [color=gray!50]  [step=5mm] (-11,-10) grid (11,10);
       % Achsen zeichnen
       \draw[->,thick] (-10,0) -- (10,0) node[right] {$x$};
       \draw[->,thick] (0,-9) -- (0,9) node[above] {$y$};
       % Achsen beschriften
       \foreach \c in {-8,-6,...,-2,2,4,...,8}{
         \draw (\c,-.1) -- (\c,.1) node[below=4pt] {$\scriptstyle\c$};
         \draw (-.1,\c) -- (.1,\c) node[left=4pt] {$\scriptstyle\c$};
   }

   \node[label=225:A] (A) at (-7,-7) {$\times$};
   \node[label=315:B] (B) at (6,-2) {$\times$};
   \node[label=45:C] (C) at (7,8) {$\times$};
   \node[label=135:D] (D) at (-6,3) {$\times$};

   \draw (A) -- (B) -- (C) -- (D) -- (A);

   \node (Ha)[label=45:$\cdot$] at (2,-3.53846) {};
   \node (Hc)[label=240:$\cdot$] at (-1,4.92308) {};
   \draw (Ha) -- (Hc);
   \node at (2,-2) {\footnotesize $h_a$};
   \node at (-1,3) {\footnotesize $h_c$};
   \draw (Ha) ++(1.0,0.4) arc (20:110:1);
   \draw (Hc) ++(-1.0,-0.4) arc (205:295:1);

   \node (Hb)[label=145:$\cdot$] at (6.4,0.15) {};
   \node (Hd)[label=45:$\cdot$] at (-6.2,1.5) {};
   \draw (Hb) -- (Hd);
   \draw (Hb) ++(-0.1,0.9) arc (80:170:1);
   \draw (Hd) ++(1,-0.05) arc (-10:80:1);
   \node at (4.,0.7) {\footnotesize $h_b$};
   \node at (-4,2) {\footnotesize $h_d$};

  \end{tikzpicture}
  \caption{$h_b = h_d$, daraus folgt (mit $h_a = h_c$): es ist ein Parallelogramm}\label{fig:koordinatensystem2D}
\end{figure}
\end{beispiel}





\section{Körper}

\subsection{Volumen}\index{Volumen}\label{Volumen}

Wir nennen den Raum/ Rauminhalt eines Körpers (\enquote{Wie viel Masse hat/ enthält der Körper}) \enquote{\textbf{Volumen}}\index{Volumen}. Zum Beispiel hat ein Würfel mit einer Seitenlänge von einem Dezimeter ($a=1\text{dm}$) ein Volumen/ einen Rauminhalt von einem Kubikdezimeter. Ein anderer Name für Kubikdezimeter ($\text{dm}^3$) ist \textbf{Liter}\index{Liter} (\glsdisp{symb:Liter}{l, $l$, L}). Der Würfel hat ein Volumen von einem Liter.


\subsection{Oberfläche}\index{Oberfläche}\label{Oberfläche}

\begin{definition}[Oberfläche]
    Wir nennen die Summe der Flächeninhalte der Flächen, die einen Körper begrenzen, \enquote{\textbf{Oberfläche}}.
\end{definition}

\begin{beispiel}[Oberfläche Würfel]
    \textbf{Aufgabe:} Berechne die Oberfläche eines Würfels mit einer Seitenlänge von $a=3\text{cm}$.

    Ein Würfel hat sechs Seiten. Alle sechs Seiten sind gleiche Quadrate mit der Fläche $a^2$. Somit ist die Oberfläche des Würfels $A_O = 6a^2$ und bei $a=3\text{cm}$ $A_O=6 (3\text{cm})^2)$, also $6 \cdot 9\text{cm}^2$ gleich $54\text{cm}^2$.
\end{beispiel}


\subsection{Koordinaten in 3d}

Wir haben für die Angabe eines Punktes in der 2-dimensionalen Ebene einen zweistelligen Tupel (ein Paar) benutzt: $(x|y)$. Für die dritte Dimension benutzen wir entsprechend einen dreistelligen Tupel (einen Tripel): $(x|y|z)$. Etwas schwieriger wird die graphische Darstellung. Da wir auf der Oberfläche des Papiers schreiben und zeichnen fehlt uns die dritte Dimension. Wir müssen diese \enquote{auf die Ebene projizieren}.

\begin{beispiel}[Zeichnung von Körpern]
\begin{figure}
  \centering
    \begin{tikzpicture}[x=0.5cm,y=0.5cm,z=0.3cm,>=stealth]
        % The axes
        \draw[->] (xyz cs:x=-13.5) -- (xyz cs:x=13.5) node[above] {$x$};
        \draw[->] (xyz cs:y=-13.5) -- (xyz cs:y=13.5) node[right] {$z$};
        \draw[->] (xyz cs:z=-13.5) -- (xyz cs:z=13.5) node[above] {$y$};
        % The thin ticks
        \foreach \coo in {-13,-12,...,13}
        {
          \draw (\coo,-1.5pt) -- (\coo,1.5pt);
          \draw (-1.5pt,\coo) -- (1.5pt,\coo);
          \draw (xyz cs:y=-0.15pt,z=\coo) -- (xyz cs:y=0.15pt,z=\coo);
        }
        % The thick ticks
        \foreach \coo in {-10,-5,5,10}
        {
          \draw[thick] (\coo,-3pt) -- (\coo,3pt) node[below=6pt] {\coo};
          \draw[thick] (-3pt,\coo) -- (3pt,\coo) node[left=6pt] {\coo};
          \draw[thick] (xyz cs:y=-0.3pt,z=\coo) -- (xyz cs:y=0.3pt,z=\coo) node[below=8pt] {\coo};
        }
        % Dashed lines for the points P, Q
        \draw[dashed]
          (xyz cs:z=-5) --
          +(0,7) coordinate (u) --
          (xyz cs:y=7) --
          +(-5,0) --
          ++(xyz cs:x=-5,z=-5) coordinate (v) --
          +(0,-7) coordinate (w) --
          cycle;
        \draw[dashed] (u) -- (v);
        \draw[dashed] (-5,7) -- (-5,0) -- (w);
        \draw[dashed] (3,0) |- (0,5);

        % Dots and labels for P, Q
        \node[fill,circle,inner sep=1.5pt,label={left:$Q(-5,-5,7)$}] at (v) {};
        \node[fill,circle,inner sep=1.5pt,label={above:$P(3,0,5)$}] at (3,5) {};
        % The origin
        \node[align=center] at (3,-3) (ori) {(0,0,0)};
        \draw[->,help lines,shorten >=3pt] (ori) .. controls (1,-2) and (1.2,-1.5) .. (0,0,0);
    \end{tikzpicture}
  \caption{Koordinaten in 3d}\label{fig:KoordinatenIn3d}
\end{figure}

Selbst \enquote{3-dimensional} zeichnen zu können ist nicht notwendig für die Prüfung. Ihr solltet aber 3-dimensionale Zeichnungen lesen können und Ihr müsst mit den Koordinaten rechnerisch umgehen können. Das prüfungsrelevante Wissen wird somit komplett getestet durch die Bearbeitung einer Aufgabe mit einem Körper, dessen Punkte einer gegebenen Zeichnung zu entnehmen sind. Zu beachten ist dabei, dass wir uns nicht darauf verlassen können welche Achse in welche Richtung zeigt. So ist die Abbildung zwar eine weit verbreitete, in der schulischen Physik in Deutschland wird jedoch eine andere (eine der Drehungen des 3-dimensionalen Kreuzes der Achsen) verwendet. Die Koordinaten müssen also jeweils aufmerksam von den Achsen abgelesen werden.\footnote{Für tieferes Verständnis sei erwähnt, dass es überhaupt nicht möglich ist in der 2-dimensionalen Projektion dreidimensional Punkte sauber zu unterscheiden. Überlege hierzu was \enquote{vor} und \enquote{hinter} dem Punkt liegt (es kann nur eine Gerade abgelesen werden statt einem Punkt, die Gerade \enquote{von unendlich hinter dem Punkt durch den Punkt und weiter durch den Punkt auf dem Bildschirm}). In der Abbildung wird dies dadurch geklärt, dass man annimmt, dass die gestrichelten Linien senkrecht auf den Achsen stehen.}
\end{beispiel}


\subsection{Fläche zu Körper extrudieren}\index{Extrudieren}\label{Extrudieren}

Wir erhalten einen Körper indem wir eine \textbf{Grundfläche} \textbf{extrudieren}. Wir \enquote{ziehen die Fläche senkrecht nach oben} bis wir mit einer zweiten Grundfläche abschließen.

\begin{beispiel}[extrudieren]
    Wir extrudieren die Fläche $A_Q$ (s. \ref{fig:extrudieren}, S. \pageref{fig:extrudieren}). Beachte dass die erste der vier Teil-Abbildungen die \textbf{Fläche} ist, kein \textbf{Körper}. Es macht wenig Sinn dort von einem Volumen zu sprechen. Wenn überhaupt wäre es null. In der zweiten Teil-Abbildung fügen wir die \textbf{dritte Dimension} erst wirklich hinzu. Wir \enquote{ziehen die Grundfläche nach oben} und erhalten so das Volumen , hier von Quadern, $V_Q = a \cdot b \cdot c$. Beachte dass jede Höhe möglich wäre. Nichts verhindert für die letzte Ebene statt Würfel Quader mit einem Bruchteil als Höhe zu nehmen. Durch das Extrudieren entsteht also ein Volumen in $\mathbb{R}^3$.
\end{beispiel}

\begin{figure}
  \centering
  \begin{tikzpicture}[scale=0.35]
    \draw (1,1) -- (1,11);
    \draw (1,11) -- (6,16);
    \draw (6,16) -- (16,16);
    \draw (16,16) -- (11,11);
    \draw (1,11) -- (11,11);
    \draw (6,16) -- (6,6);
    \draw (16,16) -- (16,6);
    \foreach \x in {0,1,...,9}{%top face
        \foreach \y in {0,0.5,...,4.5}{
            \draw[fill=gray!10] (\x + \y + 1, \y + 1) -- (\x + \y + 2, \y + 1) -- (\x + \y + 2.5, \y + 1.5) -- (\x + \y + 1.5, \y + 1.5) -- (\x + \y + 1, \y + 1);
        }
    }
    \draw (11,11) -- (11,1);
  \end{tikzpicture}
  \begin{tikzpicture}[scale=0.35]
    \draw (1,1) -- (1,11);
    \draw (1,11) -- (6,16);
    \draw (6,16) -- (16,16);
    \draw (16,16) -- (11,11);
    \draw (1,11) -- (11,11);
    \draw (6,16) -- (6,6);
    \draw (16,16) -- (16,6);
    \foreach \x in {0,1,...,9}{%front face
        \draw[fill=gray!15] (\x + 1, 1) -- (\x + 2, 1) -- (\x + 2, 2) -- (\x + 1, 2) -- (\x + 1, 1);
    }
    \foreach \x in {0,0.5,...,4.5}{%right face
        \draw[fill=gray!45] (\x + 11, \x + 1) -- (\x + 11.5, \x + 1.5) -- (\x + 11.5, \x + 2.5) -- (\x + 11, \x + 2) -- (\x + 11, \x + 1);
    }
    \foreach \x in {0,1,...,9}{%top face
        \foreach \y in {0,0.5,...,4.5}{
            \draw[fill=gray!10] (\x + \y + 1, \y + 2) -- (\x + \y + 2, \y + 2) -- (\x + \y + 2.5, \y + 2.5) -- (\x + \y + 1.5, \y + 2.5) -- (\x + \y + 1, \y + 2);
        }
    }
    \draw (11,11) -- (11,1);
  \end{tikzpicture}

  \begin{tikzpicture}[scale=0.35]
    \draw (1,1) -- (1,11);
    \draw (1,11) -- (6,16);
    \draw (6,16) -- (16,16);
    \draw (16,16) -- (11,11);
    \draw (1,11) -- (11,11);
    \draw (6,16) -- (6,6);
    \draw (16,16) -- (16,6);
    \foreach \x in {0,1,...,9}{%front face
        \foreach \y in {0,1,...,3}{
            \draw[fill=gray!15] (\x + 1, \y + 1) -- (\x + 2, \y + 1) -- (\x + 2, \y + 2) -- (\x + 1, \y + 2) -- (\x + 1, \y + 1);
        }
    }
    \foreach \x in {0,0.5,...,4.5}{%right face
        \foreach \y in {0,1,...,3}{
            \draw[fill=gray!45] (\x + 11, \x + \y + 1) -- (\x + 11.5, \x + \y + 1.5) -- (\x + 11.5, \x + \y + 2.5) -- (\x + 11, \x + \y + 2) -- (\x + 11, \x + \y + 1);
        }
    }
    \foreach \x in {0,1,...,9}{%top face
        \foreach \y in {0,0.5,...,4.5}{
            \draw[fill=gray!10] (\x + \y + 1, \y + 5) -- (\x + \y + 2, \y + 5) -- (\x + \y + 2.5, \y + 5.5) -- (\x + \y + 1.5, \y + 5.5) -- (\x + \y + 1, \y + 5);
        }
    }
    \draw (11,11) -- (11,1);
  \end{tikzpicture}
  \begin{tikzpicture}[scale=0.35]
    \foreach \x in {0,1,...,9}{%front face
        \foreach \y in {0,1,...,9}{
            \draw[fill=gray!15] (\x + 1, \y + 1) -- (\x + 2, \y + 1) -- (\x + 2, \y + 2) -- (\x + 1, \y + 2) -- (\x + 1, \y + 1);
        }
    }
    \foreach \x in {0,0.5,...,4.5}{%right face
        \foreach \y in {0,1,...,9}{
            \draw[fill=gray!45] (\x + 11, \x + \y + 1) -- (\x + 11.5, \x + \y + 1.5) -- (\x + 11.5, \x + \y + 2.5) -- (\x + 11, \x + \y + 2) -- (\x + 11, \x + \y + 1);
        }
    }
    \foreach \x in {0,1,...,9}{%top face
        \foreach \y in {0,0.5,...,4.5}{
            \draw[fill=gray!10] (\x + \y + 1, \y + 11) -- (\x + \y + 2, \y + 11) -- (\x + \y + 2.5, \y + 11.5) -- (\x + \y + 1.5, \y + 11.5) -- (\x + \y + 1, \y + 11);
        }
    }
  \end{tikzpicture}
  \caption{Extrudieren}\label{fig:extrudieren}
\end{figure}

Wir können uns aus versuchen vorzustellen, dass wir mit Wasser arbeiten (das sicherlich in Eiswürfel aufteilbar ist). Dann können wir z.B. einen letzten Würfel hineinwerfen, der (abgesehen vom Dichte-Unterschied von flüssigem Wasser und Eis) das Volumen um eine hundertstel Schicht vermehrt, indem hundertstel-Scheiben auf hundert Quadrate gelegt werden.

So können wir jeden Körper für den wir eine \textbf{Grundfläche} kennen, die \enquote{senkrecht nach oben gezogen wird (um die \textbf{Körperhöhe})}, das \textbf{Volumen} als
\begin{equation*}
  V_K = A_G \cdot h_K
\end{equation*}
berechnen, sprich \enquote{das Volumen eines Körpers ist seine Grundfläche mal seine Körperhöhe}. Das gilt für Körper mit den Bedingungen wie oben. Das sind für uns hinsichtlich der Prüfung \textbf{Prisma} und \textbf{Zylinder}.

\subsection{Prisma}\index{Prisma}\label{Prisma}

\begin{definition}
    Wir nennen einen Körper dessen Seiten Polygone sind von denen zwei gleich sind und \enquote{parallel gegenüber} liegen \enquote{\textbf{Prisma}}.\footnote{Diese Definition ist nicht rigoros, ergibt sich aber gut verständlich aus der visuellen Präsentation des Extrudierens.}
\end{definition}


\subsection{Quader}\index{Quader}\label{Quader}

\begin{definition}[Quader]
    Wir nennen ein Prisma mit rechteckigen Grundflächen \enquote{\textbf{Quader}}.
\end{definition}

\subsection{Würfel}\index{Würfel}\label{Würfel}

\begin{definition}[Würfel]
    Wir nennen einen Quader mit sechs gleichen Seiten \enquote{\textbf{Würfel}}.
\end{definition}


\subsection{Zylinder}\index{Zylinder}\label{Zylinder}

\begin{definition}[Zylinder]
    Wir nennen einen Körper dessen Grundflächen Kreise sind, die durch einen senkrecht auf ihnen stehenden Mantel miteinander verbunden sind, \enquote{\textbf{Zylinder}}.
\end{definition}

Ein Zylinder ist also eine \enquote{runde Stange} mit glatten Enden. Schneidet/ bohrt man einen gleich langen Zylinder mit kleinerem Durchmesser aus einem größeren, so erhält man ein Rohr mit dem Durchmesser des kleineren Zylinders als Innendurchmesser und dem Durchmesser des größeren Zylinders als Außendurchmesser.


\section{Maßstab}\index{Maßstab}\label{Maßstab}

Zeichnungen geben Objekte der Realität häufig nicht in ihrer natürlichen Größe wieder. Große Objekte werden oft verkleinert dargestellt, sehr kleine vergrößert. Um aus solch einer nicht maßstabsgetreuen Zeichnung die wirklichen Strecken zu ermitteln müssen wir die Messungen an der Zeichnung mit dem Maßstab umrechnen. Eine Entfernung von 7cm auf einer Fahrradkarte im Maßstab 1:100000 sind z.B. in Wirklichkeit 7cm mal 100000, also 700000cm = 7km. Gebräuchlich sind Maßstäbe 1:n, sprich \enquote{eins zu n} für verkleinerte Darstellungen wie auf Landkarten sowie n:1 für vergrößerte Darstellungen, z.B. mikroskopische Lebewesen in einem Biologie-Buch. Häufig wird der Maßstab auch gezeichnet und eine Strecke am Rand der Abbildung, die z.B. 1cm lang ist mit 1mm beschriftet. Das bedeutet, dass die Abbildung im Maßstab 10:1 vergrößert dargestellt ist, z.B. ein Käfer wiederum im Biologie-Buch.


\section{Konstruktion von Dreiecken}

Mit Geodreieck und Zirkel kann man aus genügend Angaben über ein Dreieck dieses konstruieren.\footnote{Die Konstruktion von Dreiecken ist sobald man sie einmal gelernt hat sehr einfach und wird in der Prüfung mit gemessen am Aufwand (Zeit und Schwierigkeit) mit sehr vielen Punkten belohnt.} Wir führen hier ausführlich die (zumindest scheinbar) verschiedenen Probleme auf, die bei der Konstruktion von Dreiecken (in der Prüfung) vorkommen können.

\subsection{Drei Seiten}

Sind drei Seitenlängen eines Dreiecks bekannt:
\begin{enumerate}
  \item Eine beliebige Seite zeichnen, im Beispiel die Seite c. Die beiden Ecken in denen die Seite endet können direkt benannt werden.
  \item Den Zirkel auf die Länge der zweiten Seite, die an einem der beiden bekannten Punkte anfängt, einstellen, im Beispiel b, und einen großzügigen Bogen in die Richtung schlagen in der die Ecke sein muss (zwar gibt es oberhalb und unterhalb von c je einen Punkt, der die Längen der Seiten einhalten würde, würden wir diesen für C wählen würden wir aber die Konventionen für die Namen verletzen, da dann ABC nicht gegen den Uhrzeiger zu einander liegen würden. Im Beispiel zeichnen wir den Bogen also oberhalb von c.
  \item Den Zirkel auf die noch nicht benutzte Seitenlänge einstellen und am entsprechenden noch nicht benutzten Punkt einstechen und einen Bogen schlagen, der den ersten Bogen schneidet. Hier liegt der dritte Eckpunkt des Dreiecks.
  \item Den dritten Punkt, die beiden Seiten zu ihm und falls verlangt andere Beschriftungen hinzufügen (ist für eine Zeichnung nicht angegeben was zu beschriften ist beschriften wir Eckpunkte, Seiten und Winkel und geben die Längen der Seiten und die Gradzahlen der Winkel an.).
\end{enumerate}

\begin{beispiel}[Konstruktion Dreieck bei drei gegebenen Seitenlängen]
\textbf{Aufgabe:} Konstruiere das Dreieck mit Seitenlängen $a=6\text{cm}$, $b=5\text{cm}$ und $c=8\text{cm}$.
\begin{figure}
  \centering
  \begin{tikzpicture}[scale=0.5]
        \node[label=235:A] (A) at (0,0) {$\times$};
        \node[label=325:B] (B) at (8,0) {$\times$};
        \node (c) at (4,-0.25) {c};
        \draw (A) -- (B);
  \end{tikzpicture}
  \begin{tikzpicture}[scale=0.5]
        \node[label=235:A] (A) at (0,0) {$\times$};
        \node[label=325:B] (B) at (8,0) {$\times$};
        \node (c) at (4,-0.25) {c};
        \draw (A) -- (B);
        \centerarc[gray](A)(25:115:5);
  \end{tikzpicture}

  \begin{tikzpicture}[scale=0.5]
        \node[label=235:A] (A) at (0,0) {$\times$};
        \node[label=325:B] (B) at (8,0) {$\times$};
        \node (c) at (4,-0.25) {c};
        \draw (A) -- (B);
        \centerarc[gray!50](A)(25:115:5);
        \centerarc[gray!50](B)(120:155:6);
  \end{tikzpicture}
  \begin{tikzpicture}[scale=0.5]
        \node[label=235:A] (A) at (0,0) {$\times$};
        \node[label=325:B] (B) at (8,0) {$\times$};
        \node (c) at (4,-0.25) {c};
        \draw (A) -- (B);
        \centerarc[gray!50](A)(25:115:5);
        \centerarc[gray!50](B)(120:155:6);
        \node[label=90:C] (C) at (3.31,3.79) {$\times$};
        \draw (B) -- (C) -- (A);
        \node (a) at (6,2) {a};
        \node (b) at (1.45,2.2) {b};
  \end{tikzpicture}

  \caption{Konstruktion Dreieck mit drei Seiten}\label{fig:DreieckKonstruktionDreiSeiten01}
\end{figure}

\textbf{Lösungsschritte} (s. Abb. \ref{fig:DreieckKonstruktionDreiSeiten01})\textbf{:}
\begin{enumerate}
  \item Wir zeichnen eine gerade Strecke von 8cm am unteren Ende des Blattes und nennen sie c. Wir tragen die Punkte A und B ein.
  \item Wir stellen den Zirkel auf die Länge der zweiten Seite an A, der Seite b=5cm ein und schlagen einen großzügigen Bogen nach oben (nach unten würden wir die Namenskonventionen für Dreiecke verletzen).
  \item Wir stellen den Zirkel auf die Länge der zweiten Seite an B, der Seite a=6cm ein und schlagen einen Bogen, der den ersten Bogen schneidet. Dort tragen wir den Punkt C ein und verbinden die Punkte A und C zur Seite b und B und C zur Seite a.
  \item Wir tragen geforderte Beschriftungen ein, falls nichts anders angegeben ist die Namen der Eckpunkte A, B und C, der Seiten a, b und c, jeweils mit Seitenlängen (oder führen die Maße unter oder neben der Zeichnung auf) und die Winkel \glsdisp{symb:alpha}{$\alpha$}, $\beta$ und $\gamma$.
\end{enumerate}

\begin{figure}
  \centering
  \begin{tikzpicture}
        \node[label=235:A] (A) at (0,0) {$\times$};
        \node[label=325:B] (B) at (8,0) {$\times$};
        \node (c) at (4,-0.25) {c};
        \draw (A) -- (B);
        \centerarc[gray!50](A)(25:115:5);
        \centerarc[gray!50](B)(120:155:6);
        \node[label=90:C] (C) at (3.31,3.79) {$\times$};
        \draw (B) -- (C) -- (A);
        \node (a) at (6,2) {a};
        \node (b) at (1.45,2.2) {b};
        \centerarc[black](A)(0:48:1);
        \centerarc[black](B)(140:180:1);
        \centerarc[black](C)(230:322:1);

        \node at (0.5,0.25) {$\alpha$};
        \node at (7.25,0.25) {$\beta$};
        \node at (3.31,3.25) {$\gamma$};

        \node[text width=3cm] at (-1,2.5) {a=6cm\\b=5cm\\c=8cm\\$\alpha \approx 48°$\\$\beta \approx 40°$\\$\gamma \approx 92°$};
  \end{tikzpicture}
  \caption{Lösung}\label{fig:DreieckKonstruktionDreiSeiten02}
\end{figure}

\end{beispiel}


\subsection{eine Seite und zwei Winkel}

Sind zwei Winkel gegeben, so haben wir eigentlich den Fall eine Seite und drei Winkel, denn der fehlende Winkel ist die Winkelsumme minus beide gegebenen Winkel.

\begin{beispiel}[eine Seite und zwei Winkel]
    \textbf{Aufgabe:} Konstruiere das Dreieck mit $c=7cm$, \glsdisp{symb:alpha}{$\alpha=40°$} und $\beta=50°$.

    Dieser Fall ist der einfachere von zwei etwas unterschiedlichen Fällen von einer Seite und zwei Winkeln. Wir haben eine Seite und beide an den Eckpunkten der Seite liegende Winkel. Wir zeichnen die Seite, messen die Winkel und zeichnen in den durch die Winkel vorgegebenen Richtungen Geraden. Der Schnittpunkt der beiden Geraden gibt uns den dritten Punkt.

\begin{figure}
  \centering
  \begin{tikzpicture}
        \node[label=225:A] (A) at (0,0) {$\times$};
        \node[label=315:B] (B) at (7,0) {$\times$};
        \draw (A) -- (B);
        \node (c) at (3.5,-0.25) {c};

        \centerarc[gray!50](A)(0:40:1);
        \draw[gray!50] (A) -- (8,6.7128);
        \node at (.6,0.25) {$\alpha$};
        \centerarc[gray!50](B)(130:180:1);
        \draw[gray!50] (B) -- (3,4.76701);
        \node at (6.4,0.25) {$\beta$};
  \end{tikzpicture}
  \caption{eine Seite und beide an ihren Eckpunkten liegende Winkel}\label{fig:DreieckSeiteUndBeideDaranLiegendenWinkel}
\end{figure}

\begin{figure}
  \centering
  \begin{tikzpicture}
        \node[label=225:A] (A) at (0,0) {$\times$};
        \node[label=315:B] (B) at (7,0) {$\times$};
        \draw (A) -- (B);
        \node (c) at (3.5,-0.25) {c};

        \centerarc[gray!50](A)(0:40:1);
        \draw[gray!50] (A) -- (8,6.7128);
        \node at (.6,0.25) {$\alpha$};
        \centerarc[gray!50](B)(130:180:1);
        \draw[gray!50] (B) -- (3,4.76701);
        \node at (6.4,0.25) {$\beta$};
        \node[label=90:C] (C) at (4.105,3.45) {$\times$};
        \draw (A) -- (C) -- (B);
        \centerarc[gray!50](C)(220:310:1);
        \node (gamma) at (4,2.9) {$\gamma$};
        \node (a) at (5.75,2) {a};
        \node (b) at (1.85,2) {b};

        \node[text width=3cm] at (2,4) {$\alpha=40°$\\$\beta=50°$\\$\gamma=90°$\\$a\approx 4,5\text{cm}$\\$b \approx 5,4\text{cm}$\\$c=7\text{cm}$};
  \end{tikzpicture}
  \caption{Lösung}\label{fig:DreieckSeiteUndBeideDaranLiegendenWinkel2}
\end{figure}

\end{beispiel}


\subsection{Häufige Fehler}
\begin{itemize}
  \item Zu kurze Zirkelbogen die sich nicht treffen. Lösung: Den ersten Bogen durch einen gesuchten Punkt sehr großzügig schlagen, so dass der Punkt ganz sicher irgendwo auf dem Bogen liegt, z.B. hier einen Halbkreis von A aus beginnend auf der Seite c und gegen den Uhrzeigersinn schlagend (C muss über c liegen, da sonst die Namen der Punkte nicht gemäß Konvention wären).
  \item Auf der falschen Skala auf dem Geodreieck Grad messen. Lösung: ein komplett geschlossener Winkel (also kein Winkel) hat null Grad, wird geöffnet und hat im rechten Winkel $90°$ und über der Geraden $180°$. Dementsprechend Plausibilität anhand größer oder kleiner $90°$ prüfen beim Messen.
  \item Ungenauigkeit beim Zeichnen. Lösung: einen großen Maßstab für die Zeichnung wählen (je größer desto genauer) und sorgfältig arbeiten. In der Prüfung sollen schon beim zweiten Millimeter oder Grad Abweichung eines zu bestimmenden Wertes Punkte abgezogen werden. Plausibilität prüfen, z.B. ob die Winkelsumme im Dreieck zwischen 178 und 182 Grad liegt (179 und 181 falls man sicher gehen will und ausschließen dass man seine gesamte Ungenauigkeit an einem Winkel hat und damit mehr als ein Grad Fehler bei diesem Winkel).
\end{itemize}


\chapter{Proportionale und Antiproportionale Zuordnungen}\index{proportionale Zuordnung}\index{antiproportionale Zuordnung|see{proportionale Zuordnung}}

Wir geben eine präzise Definition an, erwarten das aber von Euch nicht unbedingt zu verstehen. Zunächst reicht \enquote{je mehr desto mehr} und \enquote{je weniger desto weniger} für \textbf{proportionale Zuordnungen} und \enquote{je mehr desto weniger} und \enquote{je weniger desto mehr} für \textbf{antiproportionale Zuordnungen}, zusammen mit \enquote{auf einer Seite der Tabelle teilen und das Ergebnis so auf die andere Seite übertragen, dass die Regeln je ... desto ... eingehalten wird}.

\begin{definition}[proportionale Zuordnung]
    Wir nennen eine Funktion $f: \mathbb{R} \rightarrow \mathbb{R}$, $x \mapsto y$ für die $f(x)=y \Leftrightarrow r \cdot f(x)= r\cdot y$ gilt eine \enquote{proportionale Zuordnung}
\end{definition}

\begin{definition}[antiproportionale Zuordnung]
    Wir nennen eine Funktion $f: \mathbb{R} \rightarrow \mathbb{R}$, $x \mapsto y$ für die $f(x)=y \Leftrightarrow \frac{1}{r} f(x)= r\cdot y$ gilt eine \enquote{antiproportionale Zuordnung}
\end{definition}

\begin{figure}
  \centering
  \begin{tikzpicture}
        \draw (2,0.75) -- (2,3);
        \draw (1,2) -- (3,2);
        \node at (1,2.5) {Arbeiter};
        \node at (3,2.5) {Stunden};
        \node (5) at (1.5,1.5) {5};
        \node (10) at (2.5,1.5) {10};
        \node (x) at (1.5,1) {x};
        \node (4 )at (2.5,1) {4};
        \draw[bend left=45] (5) -> (x.west);
  \end{tikzpicture}
  \caption{Tabelle proportionale Zuordnung}\label{fig:ProportionaleZuordnung}
\end{figure}


In $\mathbb{R}$ ist eine proportionale Zuordnung also eine lineare Gleichung und eine antiproportionale Zuordnung eine Funktion der Form $\frac{a}{x} + c$. Beachte aber auch, dass häufig bei diesen Zuordnungen nur ganzzahlige Ergebnisse sinnvoll sind und entsprechen auf- oder abgerundet werden muss.

\begin{beispiel}[Antiproportionale Zuordnung]
    \textbf{Aufgabe:} Fünf Bauarbeiter brauchen 10 Stunden um eine Mauer fertigzustellen. Wie viele Arbeiter braucht man um in 4 Stunden fertig zu werden?

    \textbf{Lösung:} Je mehr Arbeiter arbeiten desto weniger Stunden brauchen sie. Es handelt sich also um eine antiproportionale Zuordnung.

    \begin{center}
        \begin{tabular}{|c|c|}
          \hline
          % after \\: \hline or \cline{col1-col2} \cline{col3-col4} ...
          Anzahl Arbeiter & Stunden bis Fertigstellung\\
          \hline
          5 & 10 \\
          x & 4 \\
          \hline
        \end{tabular}
    \end{center}

    Wie verändert sich die Seite auf der wir beide Daten haben? Wir teilen eine Zahl dieser Seite durch die andere: $10:4=2,5$ oder $4:10=0,4$. Mit einem dieser Faktoren multiplizieren wir die Zahl auf der anderen Seite: $5 \cdot 2,5 = 12,5$. Es werden weniger Stunden, also erwarten wir (weil antiproportional) mehr Arbeiter. Das passt und ist ein (erstmal) korrektes Ergebnis. Die Alternative wäre gewesen $4:10=0,4$, so dass $5 \cdot 0,4 = 0,2$. Das ist je weniger Stunden desto weniger Arbeiter und damit proportional und nicht antiproportional. D.h. wir multiplizieren mit $\frac{1}{0,4}$, was gleich teilen durch 0,4 ist: $5:0,4=12,5$. Die (gänzlich unmathematische) Regel \enquote{teile auf der kompletten Seite eine Zahl durch die andere und multipliziere oder dividiere auf der anderen Seit mit dem Ergebnis so dass das passende je ... desto ... eingehalten wird} passt.

    Das rechnerische Ergebnis ist also 12,5. Nun war die Frage wie viele Arbeiter man braucht. Ein halber Arbeiter macht wenig Sinn und somit sollte die Lösung in etwa \enquote{Man braucht 12,5 Arbeiter je vier Stunden, also 13 Arbeiter}.\footnote{Die Ganzzahligkeit ist eine implizite Anforderung der Aufgabenstellung, die man in der Modellierung des Problems mit Mitteln der Mathematik explizit betonen sollte. Ziel ist neben der Bewältigung der Prüfungsaufgaben Mathematik praktisch anwenden zu können.}\glsdisp{symb:qed}{\proofsquare}
\end{beispiel}

\chapter{Graphische Darstellungen}

Viele Sachverhalte der (schulischen) Mathematik lassen sich sehr \enquote{anschaulich} grafisch darstellen. Wir skizzieren die groben Zusammenhänge in Skizzen, zeichnen Diagramme und Graphen in denen die Lösungen sogar in der grafischen Darstellung (in etwa/ näherungsweise) abgelesen werden können.


\section{Skizze}\index{Skizze}

Eine Skizze nennen wir eine grafische Darstellung, die die Struktur eines Objekts erfasst, die uns jeweils interessierenden Aspekte sichtbar macht, jedoch nicht präzise ist. In der Mathematik im Hauptschulabschluss verstehen wir darunter insbesondere die Planskizzen\index{Planskizze} in der Geometrie mit denen wir uns daran erinnern welche Punkte, Seiten, Strecken, Winkel, ... wo sind relativ zu einander, um die passenden Formeln zu finden und korrekt Werte einzusetzen, die im Text gegeben sind. Beachte dass die Autoren der Prüfungen manchmal die Begriffe Skizze und Zeichnung nicht unterscheiden.


\section{Zeichnung}\index{Zeichnung}

Als \enquote{Zeichnung} bezeichnen wir eine graphische Darstellung, der alle (für das Problem) relevanten Informationen zu entnehmen sind. Insbesondere nennen wir z:B. in einer Planskizze eines Dreiecks die Seiten $a$, $b$ und $c$, in einer Zeichnung haben in der Regel die Seiten Längenangaben wie $3 \text{cm}$, $4 \text{cm}$ und $5 \text{cm}$. Vorsicht: die Begriffe \enquote{Skizze} und \enquote{Zeichnung} werden nicht immer sauber getrennt.


\subsection{Schrägbild}\index{Schrägbild}\label{Schrägbild}

\begin{definition}
    Wir bezeichnen eine (2-dimensionale) Zeichnung eines (3-dimensionalen) Körpers bei der nicht perspektivisch abgebildet wird, sondern eine Fläche frontal zweidimensional gezeichnet wird und die \enquote{Tiefe} in der Weise, dass für eine Einheit, die auf kariertem Papier zwei Kästchen misst (insbesondere 1cm bei maßstabsgetreuer Zeichnung), als eine Diagonale \enquote{nach rechts oben} dargestellt wird mit einer Länge von einem Kästchen diagonal pro Einheit.
\end{definition}

\begin{figure}
  \centering
    \begin{tikzpicture}
        \draw[step=5mm, line width=0.1mm, black!40!white] (0,0) grid (\width,\hauteur);

        \draw[line width=0.3mm] (1,1) -- (7,1) -- (7,7) -- (1,7) -- (1,1);
        \draw[line width=0.3mm] (7,1) -- (10,4);
        \draw[line width=0.3mm] (1,7) -- (4,10);
        \draw[line width=0.3mm] (4,10) --  (10,10);
        \draw[line width=0.3mm] (7,7) -- (10,10);
        \draw[line width=0.3mm] (10,4) -- (10,10);
        \draw[dashed, line width=0.3mm] (4,4) -- (4,10);
        \draw[dashed, line width=0.3mm] (1,1) -- (4,4);
        \draw[dashed, line width=0.3mm] (4,4) -- (10,4);

        \coordinate (pa1) at (1, 0.5);
        \coordinate (pa2) at (7, 0.5);
        \draw[<->] (pa1) -- (pa2) node[midway,fill=white] {6cm};

        \coordinate (pb1) at (0.5, 1);
        \coordinate (pb2) at (0.5, 7);
        \draw[<->] (pb1) -- (pb2) node[rotate=90, anchor=center, midway,fill=white] {6cm};

        \coordinate (pc1) at (7.5, 1);
        \coordinate (pc2) at (10.5, 4);
        \draw[<->] (pc1) -- (pc2) node[rotate=45, anchor=center, midway,fill=white] {6cm};
    \end{tikzpicture}
  \caption{Schrägbild eines Würfels, maßstabsgetreu mit einer Kantenlänge von 6cm}\label{fig:SchraegbildWuerfel}
\end{figure}


\subsection{Netz}\index{Netz}\label{Netz}

\begin{definition}[Netz]
    Wir bezeichnen eine Fläche, die \enquote{durch Falten} die Seiten eines Körpers (und damit den [hohlen] Körper) ergeben würde, als ein \enquote{Netz} des Körpers.
\end{definition}


\section{Diagramm}\index{Diagramm}

Wir bezeichnen eine graphische Darstellung, die einen numerischen Sachverhalt \enquote{auf einen Blick} erkennbar darstellt, andere Aspekte jedoch gar nicht, ein \enquote{Diagramm}.

\subsection{Kreis- oder Tortendiagramm}\index{Kreisdiagramm}\index{Tortendiagramm|see Kreisdiagramm}

Mit einem Kreis- oder Tortendiagramm (synonym) stellen wir die Anteile an einem Ganzen dar, z.B. Wahlergebnisse oder Anteile von Altersgruppen an der Gesamtbevölkerung. Die Anteile müssen ohne Überschneidung sein um im Kreisdiagramm dargestellt werden zu können. Eine Summe von weniger als 100 Prozent ist möglich; in dem Fall markiert man den Rest als \enquote{Rest}, \enquote{Diverse}, \enquote{Sonstige} oder ähnliche Bezeichnungen für den Rest.

\begin{beispiel}[Kreis-/ Tortendiagramm]\label{bsp:Kreisdiagramm01}
    \begin{figure}
      \centering
        \begin{tikzpicture}[scale=3]
            \newcounter{a}
            \newcounter{b}
            \foreach \p/\t in {20/Klasse A, 4/Klasse B, 11/Klasse C,
                               49/Klasse D, 16/Sonstige}
              {
                \setcounter{a}{\value{b}}
                \addtocounter{b}{\p}
                \slice{\thea/100*360}
                      {\theb/100*360}
                      {\p\%}{\t}
              }
        \end{tikzpicture}
      \caption{Kreis-/ Tortendiagramm}\label{fig:Kreisdiagramm01}
    \end{figure}

    Für die Klassen A bis D liegen Daten vor und werden dargestellt. Für andere Klassen liegen keine Daten vor oder es sind so viele sehr kleine Anteile, dass getrennte Darstellung nicht informativ wäre. Diese anderen Klassen sind als \enquote{Sonstige} zusammengefasst.
\end{beispiel}

Ein Kreisdiagramm stellt insgesamt immer hundert Prozent, also eine Gesamtheit dar. Man sollte diesen Diagramm-Typ immer von senkrecht oben (eben als Kreis) darstellen und nicht auf Spielereien wie 3d-Schrägsicht verfallen (wozu das Synonym \enquote{Tortendiagramm} verleitet). Nur bei der Darstellung des Kreises sieht man auf den ersten Blick die wirklichen Verhältnisse. In einer Schrägsicht scheinen die \enquote{vorne} liegenden Anteile größer zu sein als sie sind und die \enquote{hinteren} scheinen verkleinert. Leider kann so eine schlechte Darstellung vorkommen. Deshalb immer nochmal genau auf die Zahlen achten!

\begin{uebung}[Kreisdiagramm]
    Versuche ein Kreisdiagramm für die letzte Bundestagswahl zu zeichnen (nicht nur skizzieren, sondern präzise zeichnen, so dass die Anteile am Kreis tatsächlich den Stimmanteilen in der Wahl entsprechen.
\end{uebung}

\subsection{Balken- oder Säulendiagramm}\index{Balkendiagramm}\index{Säulendiagramm|see Balkendiagramm}

\subsection{Liniendiagramm}\index{Liniendiagramm}

\subsection{Flächendiagramm}\index{Flächendiagramm}


\subsection{Venn-Diagramm}\index{Venn-Diagramm}\label{VennDiagramme}

Mit einem \textbf{Venn-Diagramm} können wir Mengen hinsichtlich ihrer Schnitte darstellen. Venn-Diagramme können für (rigorose) Beweise benutzt werden. Die Mengen\footnote{Es können beliebig viele Mengen sein, wirklich übersichtlich sind i.d.R. nur bis zu drei Mengen (ggf. plus Komplementärmenge in der alle dargestellten Mengen liegen).} Werden als Kreise dargestellt, die sich teilweise überschneiden.

\begin{figure}
  \centering
    % Definition of circles
    \def\firstcircle{(0,0) circle (1.5cm)}
    \def\secondcircle{(0:2cm) circle (1.5cm)}

    \colorlet{circle edge}{blue!50}
    \colorlet{circle area}{blue!20}

    \tikzset{filled/.style={fill=circle area, draw=circle edge, thick},
        outline/.style={draw=circle edge, thick}}

    \setlength{\parskip}{5mm}
    % Set A and B
    \begin{tikzpicture}
        \begin{scope}
            \clip \firstcircle;
            \fill[filled] \secondcircle;
        \end{scope}
        \draw[outline] \firstcircle node {$A$};
        \draw[outline] \secondcircle node {$B$};
        \node[anchor=south] at (current bounding box.north) {$A \cap B$};
    \end{tikzpicture}

    %Set A or B but not (A and B) also known a A xor B
    \begin{tikzpicture}
        \draw[filled, even odd rule] \firstcircle node {$A$}
                                     \secondcircle node{$B$};
        \node[anchor=south] at (current bounding box.north) {$\overline{A \cap B}$};
    \end{tikzpicture}

    % Set A or B
    \begin{tikzpicture}
        \draw[filled] \firstcircle node {$A$}
                      \secondcircle node {$B$};
        \node[anchor=south] at (current bounding box.north) {$A \cup B$};
    \end{tikzpicture}

    % Set A but not B
    \begin{tikzpicture}
        \begin{scope}
            \clip \firstcircle;
            \draw[filled, even odd rule] \firstcircle node {$A$}
                                         \secondcircle;
        \end{scope}
        \draw[outline] \firstcircle
                       \secondcircle node {$B$};
        \node[anchor=south] at (current bounding box.north) {$A - B$};
    \end{tikzpicture}

    % Set B but not A
    \begin{tikzpicture}
        \begin{scope}
            \clip \secondcircle;
            \draw[filled, even odd rule] \firstcircle
                                         \secondcircle node {$B$};
        \end{scope}
        \draw[outline] \firstcircle node {$A$}
                       \secondcircle;
        \node[anchor=south] at (current bounding box.north) {$B - A$};
    \end{tikzpicture}
  \caption{Venn-Diagramm}\label{fig:VennDiagramm}
\end{figure}


\section{Funktionsgraph}\index{Funktionsgraph}\index{Graph|see{Funktionsgraph}}



\chapter{Term Replacement Systems (TRS)}\label{TRS}\index{TRS}

Um effizient Mathematik zu betreiben ist eine nützliche Sichtweise das Rechnen als System zu verstehen in dem Ausdrücke durch anders geformte Ausdrücke ersetzt werden können, ohne den ursprünglichen Ausdruck (wertmäßig) zu verändern. Insbesondere benutzen wir \textbf{Formeln} und ersetzen die in ihnen enthaltenen allgemeinen \textbf{Variablen} durch konkrete Werte (\textbf{einsetzen}), wodurch wir konkrete Ergebnisse \textbf{ausrechnen} können.


\section{Operatorrangfolge}\index{Operatorpriorität}

Die Regel, die in deutschen (Grundschulen) gelehrt wird lautet \enquote{Punkt- vor Strichrechnung}. D.h. \textbf{Multiplikation} und \textbf{Division} werden vor \textbf{Addition} und \textbf{Subtraktion} ausgeführt. Dieser Regel fehlen unter anderem \textbf{Potenz} und \textbf{Wurzel}.

Eine vollständige Regel lautet: \enquote{Arbeite die List von oben nach unten ab und Operationen mit der selben Priorität/ dem selben Rang von links nach rechts in der Reihenfolge ihres Auftretens im Ausdruck (nicht in der Liste):}.

\begin{enumerate}
  \item \enquote{$(t)$}
  \item \enquote{${t_1}^{t_2}$}, \enquote{$\sqrt[t_2]{t_1}$}
  \item \enquote{$\times$}, \enquote{$\div$}
  \item \enquote{$+$}, \enquote{$-$}
\end{enumerate}

\begin{beispiel}
    \begin{equation}
        2 + 3 \cdot 4 = 2 + (3 \cdot 4) = 2 + 12 = 14
    \end{equation}\topicend
\end{beispiel}


\section{Äquivalenzumformungen}\index{Äquivalenzumformungen}


Wir können \enquote{Rechnen} als Vereinfachung von Ausdrücken verstehen. Wir sagen \enquote{Zwei plus drei ist fünf.} $2+3=5$ Ist egal ob in moderner mathematischer Notation oder als Satz in natürlicher deutscher Sprache eine Aussage. Diese konkrete Aussage ist wahr. $2+3=7$ ist eine falsche Aussage. Wahr hingegen ist $2+3 \neq 7$. Eine andere Sicht auf die Mathematik ist, dass es Ziel der Mathematik ist wahre Aussagen zu finden, zu zeigen/ beweisen dass sie wahr sind und für Aussagen, für die man bisher nicht weiß ob sie wahr oder unwahr/ falsch sind, ein gültiges Argument/ Beweis zu finden dass die Aussage wahr ist oder dass sie falsch ist.


\chapter{Lineare Gleichungen einer Variablen}

\enquote{Lineare Gleichung einer Variablen} ist eine Bezeichnung deren Bedeutung man in der Regel erst (lange) nachdem man sie \enquote{zu lösen} gelernt hat etwas abfangen kann. Deshalb beginnen wir mit einem Beispiel.

\begin{beispiel}
    Zwei Anbieter stehen für die Stromversorgung Deiner Wohnung zur Verfügung. Du möchtest überprüfen welcher Anbieter für Dich der günstigere ist. Anbieter $A$ bietet einen Vertrag an mit einem festen Grundbetrag von 20 Euro und einem Preis pro Kilowattstunde von 25 Cents. Anbieter $B$ bietet einen Vertrag an ohne Grundbetrag mit einem Preis von 35 Cents pro Kilowattstunde. Sicher kannst Du das Problem lösen indem Du ausrechnest welcher Vertrag günstiger für Dich gewesen wäre im letzten Monat. Das Verfahren funktioniert sogar ganz ordentlich. Aber was machst Du wenn Dein Verbrauch im letzten Jahr von Monat zu Monat angestiegen ist und die beiden Verträge bei den aktuellen Zahlen nur einen geringen Unterschied ergeben?

    Wir lösen rechnerisch indem wir den \textbf{Schnittpunkt} der Geraden $y_a = 0,25x + 20$ und $y_b = 0,35x$ berechnen. Am Schnittpunkt ist ihr $y$-Wert gleich. Wir setzen also die Formeln der Verträge gleich:
    \begin{align}\label{eqn:lineareGleichungen1}
      0,25x + 20 &=& 0,35x && | -0,25x \\
      20 &=& 0,1x && | \cdot 10 \\
      200 &=& x &&
    \end{align}
    Das bedeutet, dass Vertrag $B$ bei einem Verbrauch von weniger als 200kWh günstiger ist und Vertrag $A$ bei einem größeren Verbrauch. Aus der Vergangenheit extrapolierend ermöglicht dieses Wissen eine wesentlich bessere Planung.\proofsquare
\end{beispiel}

Dieses Beispiel können wir auch grafisch interpretieren (und damit ungefähr lösen) (s. Abb. \ref{fig:lineareGleichungen1}). Der Schnittpunkt der beiden Geraden hat die $x$-Koordinate $x=200$, den Punkt an dem beide Verträge gleich teuer sind. Hier wir im wahren Sinn \enquote{offensichtlich}, dass $B$ links und damit bei kleinerem Verbrauch günstiger ist, und $A$ rechts und damit bei größerem Verbrauch als am Schnittpunkt.
\begin{figure}
  \centering
  \begin{tikzpicture}[domain=0:300, scale=0.04]
        \draw[very thin,color=gray, step=50] (-5,-5) grid (295,145);

        \draw[->] (-5,0) -- (300,0) node[right] {kWh};
        \draw[->] (0,-5) -- (0,150) node[above] {€};

        \draw[color=red] plot (\x,{0.25*\x +20});
        \draw[color=blue] plot (\x,{0.35*\x});

        \node at (50,-10) {\small50};
        \node at (100,-10) {\small100};
        \node at (150,-10) {\small150};
        \node at (200,-10) {\small200};
        \node at (250,-10) {\small250};
        \node at (-10,50) {\small50};
        \node at (-10,100) {\small100};

        \draw[text=red] node at (300,90) {A};
        \draw[text=blue] node at (300,115) {B};
  \end{tikzpicture}
  \caption{Grafische Lösung/Interpretation}\label{fig:lineareGleichungen1}
\end{figure}


\chapter{Stochastik}\index{Stochastik}

Stochastik nennen wir den Bereich der Mathematik der sich mit \textbf{Wahrscheinlichkeiten} befasst. Der Lehrplan \citep{LehrplanMathematikHauptschuleHessen2017} enthält hierzu die verbindlichen Unterrichtsinhalte (s. \citep[S.22]{LehrplanMathematikHauptschuleHessen2017}):
\begin{itemize}
  \item Zufallsversuche, Häufigkeitsverteilungen
    \begin{itemize}
      \item Einschätzen und vergleichen von \enquote{Pech} und \enquote{Glück}, Prognose (Panne, Lottogewinn, Wettervorhersage)
      \item Gewinnchancen
      \item Absolute, relative Häufigkeit
      \item Auswerten von Strichlisten, Tabellen
    \end{itemize}
  \item Berechnen und Schätzen von Wahrscheinlichkeiten
    \begin{itemize}
      \item Häufigkeit durch Versuchsreihen
      \item Ereigniswahrscheinlichkeiten
    \end{itemize}
  \item Mehrstufige Zufallsversuche
    \begin{itemize}
      \item Zufallsversuche mit mehreren Münzen, einfache Urnenexperimente, mehrere Würfel
      \item Baumdiagramm
      \item Pfadregel
    \end{itemize}
\end{itemize}


\chapter{Prüfungstaktik}

\section{Vorbereitung}

Der wichtigste Test ob die Vorbereitung für die schriftliche Prüfung in Mathematik erfolgreich abgeschlossen ist, ist das \textit{Stark-Heft} für Mathematik ohne Lösungsheft unter realistischen Bedingungen zu bearbeiten. Wir werden in den Monaten vor der Prüfung alle darin enthaltenen Prüfungen (die tatsächlichen Prüfungen der letzten fünf Jahre) bearbeiten. Um dadurch eine realistische Einschätzung zu bekommen bearbeitet die alten Prüfungen bitte nur wenn die Dozenten Euch das aufgeben - und dann i.d.R. für realistische Bedingungen sorgen. Braucht Ihr Material zum Üben sprecht die Dozenten an. Wir geben Euch dann passendes Material, das aber nicht in eine komplette Prüfung eingebunden ist, die wir später noch für die Prüfungsvorbereitung brauchen.

Noch vor diesen Tests in der eigentlichen Prüfungsvorbereitung ist die beste Vorbereitung (in Mathematik) jede Stunde mitzuschreiben, die Hausaufgaben zu machen, gemeinsam mit anderen Teilnehmern zu lernen und sich die Inhalte gegenseitig zu erklären, und jedes unbekannte Wort in die Vokabelliste (für die Mathematik) aufzunehmen und zu lernen.


\section{Priorisierung der Aufgaben}

In den Monaten vor der Prüfung besprechen wir welche Aufgaben einfach und welche schwierig sind und welche damit leicht zu bekommende Punkte für die Note sind und an welchen man sich besser nicht aufhält, sondern sie macht wenn man mit den einfachen Aufgaben, auf die es viele Punkte gibt (z.B. ein Dreieck zu konstruieren), fertig ist.


\section{Plausibilitätsprüfung}

Die Aufgaben in den Prüfungen sind mit dem Bemühen gestellt plausible Anwendungen der Realität zu modellieren und ihre Beherrschung zu prüfen. Insbesondere sind damit auch die Zahlen i.d.R. so gewählt, dass eine Plausibilitätsprüfung mit gesundem Menschenverstand möglich ist. Errechnet man als Prüfling z.B. für das Volumen eines Stausees einige Liter (nicht einmal tausende), dann kann das nicht stimmen. Mit griffigen Beispielen für Größenordnungen, u.a. dem Liter als Flasche Cola, Packung Milch o.ä., und dem Kubikmeter als immer wieder benutztes (gerne mit Armen visualisierend) konkretes Beispiel (1.000 Liter) und dem Lehren von fortwährender skeptischer Plausibilitätsprüfung des (rechnenden) Tuns kann ein solcher Fehler sicher aufgedeckt werden und Zeit und Einsicht vorausgesetzt vom Prüfling korrigiert werden. Realistische Beispiele von Größenordnungen sollten möglichst von allen Dozenten häufig angeboten werden, insbesondere bietet sich neben der Mathematik selbst der naturwissenschaftliche Unterricht, hier zur Zeit der Geographie, an.

\begin{uebung}
    Wiederhole falls unsicher die Maßeinheiten und finde zu jeder Maßeinheiten für die Bereiche von $\frac{1}{1000}$, $\frac{1}{100}$, $\frac{1}{10}$, eins, zehn, hundert und tausend mal der Einheit ein für Dich gut bildlich vorstellbares Beispiel. Z.B. ein Liter Cola oder Milch und ein tausendstel Meter als Millimeter auf dem Geodreieck.\topicend
\end{uebung}

\begin{uebung}
    Schätze ab: das Volumen des Raums in dem Du Dich gerade befindest, die Masse/ das Gewicht des Gebäudes in dem Du Dich gerade befindest, das Volumen Deines Körpers, die Entfernungen zum Rathaus, nach Berlin, nach New York, zum Mond, zur Sonne, die Länge Deiner Hand, deines Unterarms, der Spanne Deiner beiden ausgestreckten Arme, die Höhe eines Gebäudes, das Du aus dem Fenster sehen kannst, die Dichte von Wasser, Stahl, Holz, einem Menschen, wie viel Liter Wasser der Eder-Stausee fasst, wie viele Liter Wasser Du durchschnittlich pro Tag verbrauchst, die Masse der Erde, ...\topicend
\end{uebung}


\section{Proben rechnen}

Wenn man (in der Prüfung) Zeit hat kann man seine eigenen Ergebnisse überprüfen und oftmals Fehler selbst finden. Wir nennen eine zweite Rechnung (auf einem anderen Weg), die das selbe Problem löst, also das gleiche Ergebnis\footnote{Oder ein anderes Ergebnis, das sich leicht mit dem ersten vergleichen lässt, z.B. bei einer Rechnung mit verschiedenen Maßeinheiten in eine zweite Einheit ausrechnen und prüfen ob der Faktor zwischen den Ergebnissen passt.} haben muss, eine Proberechnung.

\begin{beispiel}[Proberechnung]
    \begin{align}
      12 \text{mm} + 34 \text{cm} + 56 \text{dm} && &&\\
      + 78 \text{m} + 90 \text{km} &=& x \text{m} && | \text{Aufgabe}\\
      0,012 \text{m} + 0,34 \text{m} + 5,6 \text{m} && &&\\
      + 78 \text{m} + 90.000 \text{m} &=& 90.083,952 \text{m} && | \text{Lösung}\\
      12 \text{mm} + 340 \text{mm} && &&\\
      + 5.600 \text{mm} + 78.000 \text{mm} && &&\\
      + 90.000.000 \text{mm} &=& 90.083.952 \text{mm} && | \text{Probe}
    \end{align}

    Die Proberechnung wird in diesem Fall aus der ersten Zeile errechnet und nicht aus der bereits gefundenen Lösung. Das unabhängig von der Lösung ermittelte zweite Ergebnis (Probe) ist 1.000 mal so hoch bei einer Einheit, die $\frac{1}{1000}$ der Einheit der Lösung ist. Die Probe und Lösung sind also gleich: $90.083,952 \text{m}=90.083.952 \text{mm}$. Wenn man wirklich das zweite Ergebnis ohne die Lösung zu benutzen errechnet hat ist sehr wahrscheinlich dass die Lösung richtig ist. Passt das Ergebnis von Probe und Lösung nicht zusammen ist sicher eine der beiden Rechnungen falsch und man sollte den Fehler suchen.
\end{beispiel}


\chapter{Algorithmen}\index{Algorithmen}

\section{Was sind Algorithmen?}

Wir nennen Vorschriften, die genau abgearbeitet ein Ergebnis erreichen, einen \enquote{Algorithmus}. Ein Rezept in einem Kochbuch können wir als Algorithmus bezeichnen. Uns interessieren hier mathematische Algorithmen, wie das Verfahren einen größten gemeinsamen Teiler (ggT) oder Primzahlen zu finden. Die Werkzeuge Computer und Taschenrechner können z.B. nichts anderes als Algorithmen anzuwenden - das dann allerdings schneller als wir und ohne Fehler.

\section{ggT - Euklidischer Algorithmus}




\part{Material}

\chapter{Vokabeln}

Alle diese Vokabeln müssen gelernt werden! Zusätzlich werden noch viele gebraucht, die im Unterricht auftauchen. Diese hier haben perfekt \textbf{verstanden} zu werden (fehlerfreies Schreiben ist für Mathematik nicht notwendig).
\section{Eingeführte Vokabeln}

Eingeführte Vokabeln werden als bekannt vorausgesetzt. Was hier unbekannt ist unverzüglich lernen!

abrunden,
abschätzen,
die Abschätzung (pl. die Abschätzungen, synonym mit Schätzung),
abziehen,
die Achse (pl. die Achsen),
addieren,
die Addition (pl. die Additionen),
die Antwort (pl. die Antworten),
der Antwortsatz (pl. die Antwortsätze),
antworten,
anwenden,
die Anwendung (pl. die Anwendungen),
der Ausdruck (pl. die Ausdrücke),
ausklammern,
ausmultiplizieren,
die Ausnahme (pl. die Ausnahmen),
ausnehmen,
ausdrücken,
aufrunden,
ausrechnen,
behaupten,
die Behauptung (pl. die Behauptungen),
berechnen,
der Beweis (pl. die Beweise),
beweisen,
die Breite (pl. die Breiten),
der Bruch (pl. die Brüche),
die Bruchrechnung (kein pl.),
dezi- (Präfix für Größenordnungen),
die Differenz (pl. die Differenzen),
die Division (pl. die Divisionen),
dividieren,
das Element (pl. die Elemente),
das Endergebnis (pl. die Endergebnisse),
das Ergebnis (pl. die Ergebnisse),
die Formelsammlung (pl. die Formelsammlungen),
die Geometrie (pl. die Geometrien),
die Höhe (pl. die Höhen),
der Inhalt (pl. die Inhalte),
das Komma (pl. die Kommata),
kubik-,
kürzen,
die Maßeinheit (pl. die Maßeinheiten),
die Menge (pl. die Mengen),
der Meter (pl. die Meter),
die Multiplikation (pl. die Multiplikationen),
multiplizieren,
der Nenner (pl. die Nenner),
prim,
die Primfaktorzerlegung (pl. die Primfaktorzerlegungen),
das Rechteck (pl. die Rechtecke),
rechtwinklig,
parallel,
die Parallele (pl. die Parallelen),
das Parallelogramm (pl. die Parallelogramme),
die Primzahl (pl. die Primzahlen),
quadrat-,
das Quadrat (pl. die Quadrate),
die Raute (pl. die Rauten),
der Raum (pl. die Räume),
senkrecht,
die Strecke (pl. die Strecken),
die Stunde (pl. die Stunden),
subtrahieren,
die Subtraktion (pl. die Subtraktionen),
der Summand (pl. die Summanden),
die Summe (pl. die Summen),
das Trapez (pl. die Trapeze),
der Übertrag (pl. die Überträge),
zählen,
das Volumen (pl. die Volumina),
der Zähler (pl. die Zähler),
die Zahl (pl. die Zahlen),
der Zahlenstrahl (pl. die Zahlenstrahlen),
die Ziffer (pl. die Ziffern),
das Zwischenergebnis (pl. die Zwischenergebnisse),

\section{Noch nicht eingeführte prüfungsrelevante Vokabeln}

abbuchen,
die Abbuchung (pl. die Abbuchungen),
abheben,
der Algorithmus (pl. die Algorithmen),
anheben,
die Anhebung (pl. die Anhebungen),
anpassen,
die Anpassung (pl. die Anpassungen),
antiproportional,
die Annahme (pl. die Annahmen),
annehmen,
auszahlen
die Auszahlung (pl. die Auszahlungen),
berühren,
buchen,
die Buchung (pl. die Buchungen),
deutlich,
das Diagramm (pl. die Diagramme),
drehen,
die Drehung (pl. die Drehungen),
der Durchschnitt (pl. die Durchschnitte),
die Ecke (pl. die Ecken),
die Einheit (pl. die Einheiten),
einzahlen,
die Einzahlung (pl. die Einzahlungen),
das Element (pl. die Elemente),
elementar,
ergeben,
erhöhen,
die Erhöhung (pl. die Erhöhungen),
erweitern,
die Erweiterung (pl. die Erweiterungen),
explizit,
der Faktor (pl. die Faktoren),
faktorisieren,
fehlen,
das Fehlen (kein pl.),
der Fehler (pl. die Fehler),
flach,
die Figur (pl. die Figuren),
die Fläche (pl. die Flächen),
die Folge (pl. die Folgen),
folgen,
folgern,
die Folgerung (pl. die Folgerungen),
die Formel (pl. die Formeln),
der Funktionsgraph (pl. die Funktionsgraphen),
die Ganze Zahl (pl. die Ganzen Zahlen),
das Gefäß (pl. die Gefäße),
das Gehalt (pl. die Gehälter),
das Geodreieck (pl. die Geodreiecke),
gerade,
die Gerade (pl. die Geraden),
das Gewicht (pl. die Gewichte),
gleich,
die Gleichheit (pl. die Gleichheiten),
die Gleichung (pl. die Gleichungen),
die Grafik (pl. die Grafiken),
grafisch,
das Gramm (kein pl.),
die Grundrechenart (pl. die Grundrechenarten),
der Grundsatz (pl. die Grundsätze [ungebräuchlich]),
grundsätzlich,
die Größenordnung (pl. die Größenordnungen),
das Guthaben (pl. die Guthaben),
hinreichend,
identisch,
die Identität (pl. die Identitäten),
das Kapital (kein pl.),
kariert,
der Kasten (pl. die Kästen),
das Kästchen (pl. die Kästchen),
der Kehrwert (pl. die Kehrwerte),
das Kilogramm (kein pl.),
die Klammer (pl. die Klammern),
die Kommunikation (pl. die Kommunikationen),
konstruieren,
die Konstruktion (pl. die Konstruktionen),
kommunizieren,
die Körperhöhe (pl. die Körperhöhen),
der Kreis (pl. die Kreise),
kurz,
die Implikation (pl. die Implikationen),
implizit,
lang,
die Länge (pl. die Längen),
der Lohn (pl. die Löhne),
kürzen,
irren,
der Irrtum (pl. die Irrtümer),
linear,
lösen,
die Lösung (pl. die Lösungen),
die Masse (pl. die Massen),
das Maß (pl. die Maße),
der Maßstab (pl. die Maßstäbe),
maßstabsgetreu,
das Maximum (pl. die Maxima),
messen,
die Messung (pl. die Messungen),
die Methode (pl. die Methoden),
milli- (Präfix für Größenordnungen),
das Minimum (pl. die Minima),
das Modell (pl. die Modelle),
modellieren,
die Nebenrechnung (pl. die Nebenrechnungen),
die Notation (pl. die Notationen),
notfalls,
der Notfall (pl. die Notfälle),
notieren,
notwendig,
ordnen,
die Ordnung\footnote{Eine Ordnung in der Mathematik ist etwas anderes als die Ordnung die man in seiner Wohnung schafft wenn man aufräumt.} (pl. die Ordnungen),
das Paar (pl. die Paare),
parallel,
die Parallele (pl. die Parallelen),
das Parallelogramm (pl. die Parallelogramme),
die Planskizze (pl. die Planskizzen),
plausibel,
die Plausibilität (kein Plural),
das Polygon (pl. die Polygone),
die Potenz (pl. die Potenzen),
potenzieren,
das Prisma (pl. die Prismen),
die Präfix (pl. die Präfixe),
die Probe (pl. die Proben),
die Proberechnung (pl. die Proberechnungen),
das Produkt (pl. die Produkte),
proportional,
das Prozent (pl. die Prozente),
die Prozentrechnung (kein pl.),
der Punkt (pl. die Punkte),
der Quader (pl. die Quader),
der Quotient (pl. die Quotienten),
der Raum (pl. die Räume),
die Raute (pl. die Rauten),
rechnen,
die Rechnung (pl. die Rechnungen),
das Recht (pl. die Rechte),
rechtwinklig,
die Regel (pl. die Regeln),
regeln,
die Relation (pl. die Relationen),
runden,
die Rundung (pl. die Rundungen),
schätzen,
die Schätzung (pl. die Schätzungen),
schneiden,
der Schnitt (pl. die Schnitte),
der Schnittpunkt (pl. die Schnittpunkte),
die Schulden (kein Singular),
schlussfolgern,
die Schlussfolgerung (pl. die Schlussfolgerungen),
die Sekunde (pl. die Sekunden),
senkrecht,
die Senkrechte (pl. die Senkrechten),
die Skizze (pl. die Skizzen),
skizzieren,
sortieren,
spiegeln,
spiegelsymmetrisch,
die Spiegelung (pl. die Spiegelungen),
steigen,
die Steigung (pl. die Steigungen),
die Stelle (pl. die Stellen),
der Stellenwert (pl. die Stellenwerte),
das Stellenwertsystem (pl. die Stellenwertsysteme),
der Stundenlohn (pl. die Stundenlöhne),
das Symbol (pl. die Symbole),
symmetrisch,
die Symmetrie (pl. die Symmetrien),
die Symmetrieachse (pl. die Symmetrieachsen),
der Tag (pl. die Tage),
der Term (pl. die Terme),
die Tonne (pl. die Tonnen),
das Tripel (pl. die Tripel),
das Tupel (pl. die Tupel),
der Uhrzeigersinn (kein pl.),
der Umfang (pl. die Umfänge),
umrechnen,
die Umrechnung (pl. die Umrechnungen),
ungerade,
das Verfahren (pl. die Verfahren),
das Volumen (pl. die Volumina),
die Voraussetzung (pl. die Voraussetzungen),
das Vorzeichen (pl. die Vorzeichen),
die Waage (pl. die Waagen),
der Wert (pl. die Werte),
wiegen,
der Winkel (pl. die Winkel),
der Würfel (pl. die Würfel),
würfeln,
die Wurzel (pl. die Wurzeln),
die Zahlung (pl. die Zahlungen),
die Zählung (pl. die Zählungen),
zeichnen,
die Zeichnung (pl. die Zeichnungen),
zeigen,
die Zeit (pl. die Zeiten),
die Zeitspanne (pl. die Zeitspannen),
zenti- (Präfix für Größenordnungen),
der Zentimeter (pl. die Zentimeter),
zerlegen,
die Zerlegung (pl. die Zerlegungen),
zerschneiden,
der Zins (pl. die Zinsen),
die Zinsrechnung (kein pl.),
der Zirkel (pl. die Zirkel),
zusammensetzen,
das Zwischenergebnis (pl. die Zwischenergebnisse),
der Zylinder (pl. die Zylinder) 


\chapter{Arbeitsblätter}

\newpage
\section{Maßeinheiten: Strecken}

\anweisungArbeitsblatt

\begin{eqnarray}
% \nonumber to remove numbering (before each equation)
  1,23456\text{dm} &=& x \text{cm}\\
  24,86\text{m} &=& x \text{km}\\
  0,0074\text{mm} &=& x \text{dm}\\
  654,456\text{km} &=& x \text{cm}\\
  75,3\text{mm} &=& x \text{km}\\
  54,1745\text{m} &=& x \text{cm}\\
  5678\text{cm} &=& x \text{dm}\\
  0,00054\text{dm} &=& x \text{mm}\\
  2,5468789\text{km} &=& x \text{dm}\\
  32458\text{dm} &=& x \text{km}\\
  12,6385\text{m} &=& x \text{dm}\\
  0,00578\text{km} &=& x \text{mm}\\
  54678534\text{cm} &=& x \text{km}\\
  567878,5\text{mm} &=& x \text{m}\\
  687798,24\text{cm} &=& x \text{m}\\
  534,32\text{dm} &=& x \text{m}\\
  0,005456\text{m} &=& x \text{mm}\\
  6546\text{mm} &=& x \text{cm}\\
  1,6574\text{cm} &=& x \text{mm}\\
  3,6578\text{km} &=& x \text{m}\\
  382\text{cm} &=& x \text{mm}\\
  1\text{km} + 2\text{m} +3\text{dm} +4\text{cm} +5\text{mm} &=& x \text{m} \\
  123.456.789\text{mm} &=& x \text{km} \\
  7 \cdot (13\text{m} + 17\text{cm}) &=& x\text{cm} \\
  27.364.782.634\text{mm} &=& x \text{km}\\
  42,5\text{km} &=& x \text{mm}\\
  12600\text{km} &=& x \text{m}
\end{eqnarray}



\newpage
\section{Maßeinheiten: Massen/ Gewichte}

\anweisungArbeitsblatt

\begin{eqnarray}
% \nonumber % Remove numbering (before each equation)
    758.432 \text{g} &=& x \text{t} \\
    0,0034 \text{t} &=& x \text{mg} \\
    73.534 \text{mg} &=& x \text{kg} \\
    0,378.4 \text{g} &=& x \text{mg} \\
    750 \text{g} &=& x \text{kg} \\
    0,3 \text{kg} &=& x \text{g} \\
    12,5 \text{mg} &=& x \text{g} \\
    123 \text{kg} &=& x \text{t} \\
    1,23 \text{t} &=& x \text{g} \\
    23.464.674 \text{mg} &=& x \text{t} \\
    2,034 \text{t} &=& x \text{kg} \\
    0,0125 \text{kg} &=& x \text{mg} \\
  832.747.982\text{mg} &=& x \text{t} \\
  1,236.78 \text{g} &=& x \text{t} \\
  98764\text{g} &=& x \text{kg} \\
  6,54768578654657\text{t} &=& x \text{mg} \\
  123 \text{mg} + 78 \text{g} + 2,34 \text{kg} + 0,75 \text{t} &=& x \text{kg} \\
  123.245 \text{mg} + 7,8 \text{g} + 12,634 \text{kg} + 0,00075 \text{t} &=& x \text{mg} \\
  6123 \text{mg} + 7,8 \text{g} + 2.123,4 \text{kg} + 0,0075 \text{t} &=& x \text{g}
\end{eqnarray}


\newpage
\section{Geometrie 1 - elementare Flächen}

\anweisungArbeitsblatt

\begin{uebung}[Flächeninhalt]

    Gesucht ist jeweils der Flächeninhalt. Wichtig ist neben dem richtigen Ergebnis auch das Einüben einer ordentlichen Notation. Löse ausführlich mit Lösungsweg.

    \setlength\myLength{1.05cm}
    \begin{figure}%[ht!]
        \centering
        \subfloat{
            \begin{tikzpicture}
                \node[label=225:A] (A) at (0,0) {$\times$};
                \node[label=315:B] (B) at (3,0) {$\times$};
                \node[label=90:C] (C) at (2,3) {$\times$};

                \draw (A) -- (B) -- (C) -- (A);
                \draw[<->] (0,-0.25) -- (3,-0.25) node[midway,fill=white] {\footnotesize 3cm};%rotate=90, anchor=center,
                \draw[<->] (3.25,0) -- (3.25,3) node[rotate=90, anchor=center,midway,fill=white] {\footnotesize 3cm};
            \end{tikzpicture}
        }
        \subfloat{
            \begin{tikzpicture}
                \node[label=225:A] (A) at (0,0) {$\times$};
            \end{tikzpicture}
        }
        \subfloat{
            \begin{tikzpicture}
                \node[label=225:A] (A) at (0,0) {$\times$};
            \end{tikzpicture}
        }\\
        \centering
        \subfloat{
            \begin{tikzpicture}
                \node[label=225:A] (A) at (0,0) {$\times$};
            \end{tikzpicture}
        }
        \subfloat{
            \begin{tikzpicture}
                \node[label=225:A] (A) at (0,0) {$\times$};
            \end{tikzpicture}
        }
        \subfloat{
            \begin{tikzpicture}
                \node[label=225:A] (A) at (0,0) {$\times$};
            \end{tikzpicture}
        }
        \caption{Flächeninhalt elementarer Flächen}\label{fig:2017102501}
    \end{figure}
\end{uebung}



\newpage
\section{Wortschatz 1}

Ordne die Vokabeln den Buchstaben in den Klammern zu. Die Liste enthält die benötigten Vokabeln, die alle in der Liste der eingeführten (und damit zu lernenden) Vokabeln vorkommen. Nicht jede Vokabel, die bereits zu lernen war kommt hingegen im Text vor:
addieren,
der Bruch (pl. die Brüche),
die Bruchrechnung (kein pl.),
dezi- (Präfix für Größenordnungen),
die Differenz (pl. die Differenzen),
die Division (pl. die Divisionen),
dividieren,
das Element (pl. die Elemente),
das Ergebnis (pl. die Ergebnisse),
die Formelsammlung (pl. die Formelsammlungen),
die Maßeinheit (pl. die Maßeinheiten),
die Menge (pl. die Mengen),
die Multiplikation (pl. die Multiplikationen),
multiplizieren,
der Nenner (pl. die Nenner),
prim,
die Primfaktorzerlegung (pl. die Primfaktorzerlegungen),
die Primzahl (pl. die Primzahlen),
zählen,
subtrahieren,
die Subtraktion (pl. die Subtraktionen),
der Summand (pl. die Summanden),
die Summe (pl. die Summen),
der Übertrag (pl. die Überträge),
der Zähler (pl. die Zähler),
die Zahl (pl. die Zahlen),
der Zahlenstrahl (pl. die Zahlenstrahlen)

\vspace{1em}

Um die \textbf{\texttt{[-a-]}} $\Sigma = -3 +2 -1 -4 -5 +7$ zu berechnen hilft es sich die \textbf{\texttt{[-b-]}} als Strecken/ Längen am \textbf{\texttt{[-c-]}} vorzustellen. So ist es klar ob das \textbf{\texttt{[-d-]}} eine positive oder negative \textbf{\texttt{[-e-]}} sein muss. Eine besonders gut nachvollziehbare (und damit einfache) Methode ist es zunächst die negativen und positiven Zahlen getrennt von einander zu \textbf{\texttt{[-f-]}}. Dann \textbf{\texttt{[-g-]}} wir die ohne Beachtung des Vorzeichens kleinere Zahl von der Größeren. So bilden wir die \textbf{\texttt{[-h-]}} zwischen den Beträgen der Zahlen und haben damit von der längeren Strecke von 0 bis zur Zahl die kürzere abgezogen.

Den \textbf{\texttt{[-i-]}} $q_1 = \frac{1}{2}$ können wir durch den \textbf{\texttt{[-i-]}} $q_2 = \frac{1}{2}$ \textbf{\texttt{[-j-]}} indem wir ihn mit dem Kehrwert ${(q_2)}^{-1}$ \textbf{\texttt{[-k-]}}. Die \textbf{\texttt{[-k-]}} der \textbf{\texttt{[-l-]}} führen wir durch indem wir die \textbf{\texttt{[-m-]}} miteinander \textbf{\texttt{[-k-]}} und die \textbf{\texttt{[-n-]}} miteinander \textbf{\texttt{[-k-]}} und das \textbf{\texttt{[-d-]}} kürzen.

\printglossary
\printglossary[type=symbols,style=long]
%\printglossary[type=symb, style=long]

\chapter{Formelsammlung}

\section{Flächen}

\subsection{Viereck}\label{fs:Viereck}

Wir nennen ein Polygon mit vier Ecken ein \enquote{Viereck}.

Die Summe der Innenwinkel im Viereck beträgt 360°.


\subsection{Rechteck}\label{fs:Rechteck}

Wir nennen ein Viereck mit rechten Winkeln \enquote{Rechteck} (s. \ref{Rechteck}, S. \pageref{Rechteck}).
\begin{align}\label{eqn:fsRechteck}
    A_R &= a \cdot b\\
    U_R &= 2a + 2b
\end{align}


\subsection{Quadrat}\label{fs:Quadrat}

Wir nennen ein Rechteck mit gleichen Seitenlängen \enquote{Quadrat} ().
\begin{align}\label{eqn:fsQuadrat}
  A_Q &= 4a \\
  U_Q &= a^2
\end{align}


\subsection{Dreieck}\label{fs:Dreieck}

Wir nennen ein Polygon mit drei Ecken \enquote{Dreieck} ().
\begin{align}\label{eqn:fsDreieck}
  A_D &= \frac{a \cdot h_a}{2} \\
  U_D &= a+b+c
\end{align}


\subsection{Parallelogramm}

Wir nennen ein Viereck dessen gegenüber liegende Seiten zu einander parallel sind \enquote{Parallelogramm}.
\begin{align}\label{eqn:fsPrallelogramm}
  A_P &= a \cdot h_a \\
  U_P &= 2a + 2b
\end{align}

\subsection{Trapez}

Wir nennen ein Viereck mit zwei parallelen Seiten \enquote{Trapez}.
\begin{align}\label{eqn:fsTrapez}
  A_T &= \frac{a+c}{2} h_a \\
  U_T &= a+b+c+d
\end{align}


\subsection{Kreis}

Wir nennen die Fläche, deren Rand ein Zirkel um einen Mittelpunkt zieht, deren Rand damit einen festen Radius vom Mittelpunkt entfernt ist, \enquote{Kreis}.
\begin{align}\label{eqn:fsKreis}
  A_K &= \pi r^2 \\
  U_K &= 2 \pi r
\end{align}



\section{Körper}


\subsection{Würfel}

Wir nennen einen Körper dessen sechs Seiten (Seitenflächen) Quadrate sind \enquote{Würfel}.
\begin{align}\label{eqn:fsWuerfel}
  V &= a^3 \\
  O &= 6a^2
\end{align}


\subsection{Quader}

Wir nennen einen Körper dessen sechs Seiten (Seitenflächen) Rechtecke sind \enquote{Quader}.
\begin{align}\label{eqn:fsQuader}
  V &= a \cdot b \cdot c \\
  O &= 2ab + 2ac + 2bc
\end{align}


\subsection{Prisma}

Wir nennen einen Körper mit zwei gleichen Seiten (Seitenflächen) ein Polygon sind dessen gleiche Ecken mit geraden Strecken (senkrecht zum Polygon) verbunden sind \enquote{Prisma}.
\begin{align}\label{eqn:fsPrisma}
  V &= A_G \cdot h_K \\
  O &= 2 A_G + A_M
\end{align}


\subsection{Zylinder}

Wir nennen den Körper aus zwei Kreisen als Grundflächen, die durch einen (senkrechten, geraden) Mantel verbunden sind, \enquote{Zylinder}.
\begin{align}\label{eqn:fsZylinder}
  V &= A_G \cdot h_K \\
  O &= 2 A_G + A_M\\
  A_G (A_{\text{Kreis}}) &= \pi r^2\\
  A_M &= U_K \cdot h_K\\
    &= 2 \pi r h_K
\end{align}

\printindex
\printbibliography

\end{document}


