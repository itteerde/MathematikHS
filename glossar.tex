\newglossaryentry{CAS}{
    name={Computeralgebrasystem},
    description={\url{https://de.wikipedia.org/wiki/Computeralgebrasystem}: ''Ein Computeralgebrasystem (CAS) ist ein Computerprogramm, das der Bearbeitung algebraischer Ausdrücke dient. Es löst nicht nur mathematische Aufgaben mit Zahlen (wie ein einfacher Taschenrechner), sondern auch solche mit symbolischen Ausdrücken (wie Variablen, Funktionen, Polynomen und Matrizen).''
    }
}

\newglossaryentry{GGT}{
    name={größter gemeinsamer Teiler},
    description={Größter gemeinsamer Teiler.}
}

\newglossaryentry{KGV}{
    name={kleinstes gemeinsames Vielfache},
    description={Das kleinste Gemeinsame Vielfache einer Zahl.}
}

\newglossaryentry{SISystem}{
    name={SI-System},
    description={\enquote{Das Internationale Einheitensystem oder SI (frz. Système international d’unités) ist das am weitesten verbreitete Einheitensystem für physikalische Größen.} (\url{https://de.wikipedia.org/wiki/Internationales_Einheitensystem}, abgerufen 2017-09-13 15:27)}
}

\newglossaryentry{TRS}{
    name={Term Replacement System (TRS)},
    description={\url{https://en.wikipedia.org/wiki/Rewriting}: ''In mathematics, computer science, and logic, rewriting covers a wide range of (potentially non-\-de\-term\-inistic) methods of replacing subterms of a formula with other terms. What is considered are rewriting systems (also known as rewrite systems, rewrite engines or reduction systems). In their most basic form, they consist of a set of objects, plus relations on how to transform those objects.''}
} 