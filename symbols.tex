\newglossaryentry{symb:Abrunden}{
    type=symbols,
    name={$\left\lfloor x \right\rfloor$},
    description={Abrunden auf die nächste kleinere ganze Zahl. $\left\lfloor 1,99 \right\rfloor = 1$.},
    sort={abrunden}
}

\newglossaryentry{symb:alpha}{
    type=symbols,
    name={$\alpha$},
    description={Der griechische Buchstabe $\alpha$, sprich \enquote{alpha} bezeichnet insbesondere den Winkel am Punkt $A$ in Polygonen.},
    sort={alpha}
}

\newglossaryentry{symb:Area}{
    type=symbols,
    name={$A_F$},
    description={Flächeninhalt (engl. area) einer Fläche $F$. Wir ersetzen $F$ für elementare Flächen durch den ersten Buchstaben des Namens der Fläche.},
    sort={A}
}

\newglossaryentry{symb:Aufrunden}{
    type=symbols,
    name={$\left\lceil x \rceil$},
    description={Aufrunden auf die nächste größere ganze Zahl. $\left\lceil 1,001 \right\rceil = 2$.},
    sort={abrunden}
}

\newglossaryentry{symb:beta}{
    type=symbols,
    name={$\beta$},
    description={Der griechische Buchstabe $\beta$, sprich \enquote{beta} bezeichnet insbesondere den Winkel am Punkt $B$ in Polygonen.},
    sort={beta}
}

\newglossaryentry{symb:delta}{
    type=symbols,
    name={$\delta$},
    description={Der griechische Buchstabe $\delta$, sprich \enquote{delta} bezeichnet insbesondere den Winkel am Punkt $D$ in Polygonen.},
    sort={delta}
}

\newglossaryentry{symb:Div}{
    type=symbols,
    name={$\div$},
    description={Symbol für die Division. Gelegentilich wird \enquote{/} benutzt, insbesondere in Texten in denen keine Brüche gesetzt werden können sowieo auf vielen Taschenrechnern.
    },
    sort={Division}
}

\newglossaryentry{symb:gamma}{
    type=symbols,
    name={$\gamma$},
    description={Der griechische Buchstabe $\gamma$, sprich \enquote{gamma} bezeichnet insbesondere den Winkel am Punkt $C$ in Polygonen.},
    sort={gamma}
}

\newglossaryentry{symb:Gleich}{
    type=symbols,
    name={$=$},
    description={
        Wir sagen \enquote{gleich} oder \enquote{ist gleich} und bezeichnen damit die Gleichheit der seiten einer Gleichung, insbesondere die Gleichheit eines zu berechnenden Ausdrucks und des Ergebnisses der Rechnung. $2-1=1$, \enquote{zwei minus eins (ist) gleich eins.}
        },
    sort={gleich}
}

\newglossaryentry{symb:Groeszer}{
    type=symbols,
    name={$>$},
    description={\enquote{Die Zahl $a$ ist größer als die Zahl $b$}: $a > b$},
    sort={größer}
}

\newglossaryentry{symb:Kleiner}{
    type=symbols,
    name={$<$},
    description={\enquote{Die Zahl $a$ ist kleiner als die Zahl $b$}: $a < b$},
    sort={kleiner}
}

\newglossaryentry{symb:Lambda}{
    name={$\lambda$},
    description={Eine beliebige Zahl, mit der der nachfolgende Ausdruck multipliziert wird.},
    type=symbols,
    sort={lambda}
}

\newglossaryentry{symb:Mal}{
    type=symbols,
    name={$\cdot$},
    description={Symbol für die Multiplikation. Gelegentlich wird, insbesondere für die Lesbarkeit $\times$ benutzt},
    sort={mal}
}

\newglossaryentry{symb:Minus}{
    type = symbols,
    name={$-$},
    description={Wir sagen \enquote{minus} und bezeichnen damit die Operation der Subtraktion. $2-1=1$, \enquote{zwei minus eins (ist) gleich eins.}},
    sort={minus}
}

\newglossaryentry{symb:N}{
    type=symbols,
    name={$\mathbb{N}$},
    description={Die Menge der Natürlichen Zahlen $\{0, 1, 2, 3, 4, ...\}$},
    sort={N}
}

\newglossaryentry{symb:Phi}{
    name={$\varphi$},
    description={Ein beliebiger Winkel.},
    type=symbols,
    sort={phi}
}

\newglossaryentry{symb:Pi}{
    name={$\pi$},
    description={Die Kreiszahl.},
    type=symbols,
    sort={pi}
}

\newglossaryentry{symb:Plus}{
    type=symbols,
    name={$+$},
    description={Wir sagen \enquote{plus} und bezeichnen damit die Operation der Addition. $1+2=3$, \enquote{eins plus zwei (ist) gleich drei}.},
    sort={plus}
}

\newglossaryentry{symb:Sum}{
    type=symbols,
    name={$\Sigma$},
    description={Summe, z.B. $\sum\limits_{n=1}^{100}(n)=5050$},
    sort=Sigma
}

\newglossaryentry{symb:teilt}{
    type=symbols,
    name={$|$},
    description={\enquote{$a|b$} bedeutet dass $b$ durch $a$ teilbar ist.},
    sort=teilt
}

\newglossaryentry{symb:teiltnicht}{
    type=symbols,
    name={$\nmid$},
    description={\enquote{$a\nmid b$} bedeutet dass $b$ nicht durch $a$ teilbar ist.},
    sort={teilt nicht}
}

\newglossaryentry{symb:Ungefaehr}{
    type=symbols,
    name={$\approx$},
    description={ungefähr/ rund},
    sort=ungefaehr
}

\newglossaryentry{symb:Vereinigung}{
    type=symbols,
    name={$\bigcup$},
    description={Vereinigung (von Mengen)},
    sort=Vereinigung
} 