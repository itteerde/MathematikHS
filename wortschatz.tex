die Achse (pl. die Achsen),
abbuchen,
die Abbuchung (pl. die Abbuchungen),
abheben,
abrunden,
abschätzen,
die Abschätzung (pl. die Abschätzungen, synonym mit Schätzung),
abziehen,
addieren,
die Addition (pl. die Additionen),
der Algorithmus (pl. die Algorithmen),
antiproportional,
aufrunden,
anwenden,
die Anwendung (pl. die Anwendungen),
der Ausdruck (pl. die Ausdrücke),
ausklammern,
ausmultiplizieren,
ausrechnen,
ausdrücken,
auszahlen
die Auszahlung (pl. die Auszahlungen),
berechnen,
berühren,
der Beweis (pl. die Beweise),
beweisen,
die Breite (pl. die Breiten),
der Bruch (pl. die Brüche),
die Bruchrechnung (kein pl.),
buchen,
die Buchung (pl. die Buchungen),
das Diagramm (pl. die Diagramme),
die Differenz (pl. die Differenzen),
die Division (pl. die Divisionen),
dividieren,
drehen,
der Durchschnitt (pl. die Durchschnitte),
die Ecke (pl. die Ecken),
die Einheit (pl. die Einheiten),
die Einzahlung (pl. die Einzahlungen),
ergeben,
das Ergebnis (pl. die Ergebnisse),
erweitern,
die Erweiterung (pl. die Erweiterungen),
der Faktor (pl. die Faktoren),
faktorisieren,
fehlen,
der Fehler (pl. die Fehler),
flach,
die Fläche (pl. die Flächen),
die Folge (pl. die Folgen),
folgen,
folgern,
die Folgerung (pl. die Folgerungen),
der Funktionsgraph (pl. die Funktionsgraphen),
die Ganze Zahl (pl. die Ganzen Zahlen),
das Geodreieck (pl. die Geodreiecke),
gerade,
die Gerade (pl. die Geraden),
das Gewicht (pl. die Gewichte),
gleich,
die Gleichheit (pl. die Gleichheiten),
die Gleichung (pl. die Gleichungen),
die Grafik (pl. die Grafiken),
grafisch,
das Gramm (kein pl.),
die Grundrechenart (pl. die Grundrechenarten),
die Größenordnung (pl. die Größenordnungen),
das Guthaben (pl. die Guthaben),
hinreichend,
die Höhe (pl. die Höhen),
identisch,
die Identität (pl. die Identitäten),
das Kapital (kein pl.),
das Kilogramm (kein pl.),
die Klammer (pl. die Klammern),
die Kommunikation (pl. die Kommunikationen),
konstruieren,
die Konstruktion (pl. die Konstruktionen),
kommunizieren,
die Körperhöhe (pl. die Körperhöhen),
der Kreis (pl. die Kreise),
kurz,
kürzen,
lang,
die Länge (pl. die Längen),
kürzen,
irren,
der Irrtum (pl. die Irrtümer),
linear,
lösen,
die Lösung (pl. die Lösungen),
die Masse (pl. die Massen),
das Maß (pl. die Maße),
die Maßeinheit (pl. die Maßeinheiten),
der Maßstab (pl. die Maßstäbe),
maßstabsgetreu,
das Maximum (pl. die Maxima),
die Menge (pl. die Mengen),
messen,
die Messung (pl. die Messungen),
das Minimum (pl. die Minima),
das Modell (pl. die Modelle),
modellieren,
die Multiplikation (pl. die Multiplikationen),
der Nenner (pl. die Nenner),
die Notation (pl. die Notationen),
notieren,
notwendig,
ordnen,
die Ordnung\footnote{Eine Ordnung in der Mathematik ist etwas anderes als die Ordnung die man in seiner Wohnung schafft wenn man aufräumt.} (pl. die Ordnungen),
das Paar (pl. die Paare),
parallel,
die Parallele (pl. die Parallelen),
das Parallelogramm (pl. die Parallelogramme),
die Planskizze (pl. die Planskizzen),
plausibel,
die Plausibilität (kein Plural),
das Polygon (pl. die Polygone),
prim,
die Primfaktorzerlegung (pl. die Primfaktorzerlegungen),
die Primzahl (pl. die Primzahlen),
das Prisma (pl. die Prismen),
die Präfix (pl. die Präfixe),
die Probe (pl. die Proben),
die Proberechnung (pl. die Proberechnungen),
das Produkt (pl. die Produkte),
proportional,
das Prozent (pl. die Prozente),
die Prozentrechnung (kein pl.),
der Punkt (pl. die Punkte),
rechnen,
der Quader (pl. die Quader),
der Quotient (pl. die Quotienten),
die Raute (pl. die Rauten),
rechnen,
die Rechnung (pl. die Rechnungen),
das Recht (pl. die Rechte),
die Relation (pl. die Relationen),
runden,
die Rundung (pl. die Rundungen),
schätzen,
die Schätzung (pl. die Schätzungen),
schneiden,
der Schnitt (pl. die Schnitte),
der Schnittpunkt (pl. die Schnittpunkte),
die Schulden (kein Singular),
schlussfolgern,
die Schlussfolgerung (pl. die Schlussfolgerungen),
die Sekunde (pl. die Sekunden),
senkrecht,
die Senkrechte (pl. die Senkrechten),
die Skizze (pl. die Skizzen),
skizzieren,
sortieren,
spiegeln,
die Spiegelung (pl. die Spiegelungen),
steigen,
die Steigung (pl. die Steigungen),
die Stelle (pl. die Stellen),
der Stellenwert (pl. die Stellenwerte),
das Stellenwertsystem (pl. die Stellenwertsysteme),
die Strecke (pl. die Strecken),
die Stunde (pl. die Stunden),
subtrahieren,
die Subtraktion (pl. die Subtraktionen),
der Summand (pl. die Summanden),
die Summe (pl. die Summen),
symmetrisch,
die Symmetrie (pl. die Symmetrien),
die Symmetrieachse (pl. die Symmetrieachsen),
der Tag (pl. die Tage),
der Term (pl. die Terme),
das Tripe (pl. die Tripel),
das Tupel (pl. die Tupel),
der Übertrag (pl. die Überträge),
der Uhrzeigersinn (kein pl.),
der Umfang (pl. die Umfänge),
ungerade,
die Voraussetzung (pl. die Voraussetzungen),
das Volumen (pl. die Volumina),
die Waage (pl. die Waagen),
der Wert (pl. die Werte),
wiegen,
der Winkel (pl. die Winkel),
der Würfel (pl. die Würfel),
zählen,
der Zähler (pl. die Zähler),
die Zahl (pl. die Zahlen),
der Zahlenstrahl (pl. die Zahlenstrahlen),
die Zahlung (pl. die Zahlungen),
die Zählung (pl. die Zählungen),
zeichnen,
die Zeichnung (pl. die Zeichnungen),
zeigen,
die Zeit (pl. die Zeiten),
die Zeitspanne (pl. die Zeitspannen),
zerlegen,
die Zerlegung (pl. die Zerlegungen),
die Ziffer (pl. die Ziffern),
der Zins (pl. die Zinsen),
die Zinsrechnung (kein pl.),
der Zirkel (pl. die Zirkel),
das Zwischenergebnis (pl. die Zwischenergebnisse),
der Zylinder (pl. die Zylinder)
