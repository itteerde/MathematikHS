\section{Eingeführte Vokabeln}

Eingeführte Vokabeln werden als bekannt vorausgesetzt. Was hier unbekannt ist unverzüglich lernen!

abrunden,
abschätzen,
die Abschätzung (pl. die Abschätzungen, synonym mit Schätzung),
abziehen,
die Achse (pl. die Achsen),
addieren,
die Addition (pl. die Additionen),
die Antwort (pl. die Antworten),
der Antwortsatz (pl. die Antwortsätze),
antworten,
anwenden,
die Anwendung (pl. die Anwendungen),
der Ausdruck (pl. die Ausdrücke),
ausklammern,
ausmultiplizieren,
die Ausnahme (pl. die Ausnahmen),
ausnehmen,
ausdrücken,
aufrunden,
ausrechnen,
behaupten,
die Behauptung (pl. die Behauptungen),
berechnen,
der Beweis (pl. die Beweise),
beweisen,
die Breite (pl. die Breiten),
der Bruch (pl. die Brüche),
die Bruchrechnung (kein pl.),
dezi- (Präfix für Größenordnungen),
die Differenz (pl. die Differenzen),
die Division (pl. die Divisionen),
dividieren,
das Element (pl. die Elemente),
das Endergebnis (pl. die Endergebnisse),
das Ergebnis (pl. die Ergebnisse),
die Formelsammlung (pl. die Formelsammlungen),
die Geometrie (pl. die Geometrien),
die Höhe (pl. die Höhen),
der Inhalt (pl. die Inhalte),
das Komma (pl. die Kommata),
kubik-,
kürzen,
die Maßeinheit (pl. die Maßeinheiten),
die Menge (pl. die Mengen),
der Meter (pl. die Meter),
die Multiplikation (pl. die Multiplikationen),
multiplizieren,
der Nenner (pl. die Nenner),
prim,
die Primfaktorzerlegung (pl. die Primfaktorzerlegungen),
das Rechteck (pl. die Rechtecke),
rechtwinklig,
parallel,
die Parallele (pl. die Parallelen),
das Parallelogramm (pl. die Parallelogramme),
die Primzahl (pl. die Primzahlen),
quadrat-,
das Quadrat (pl. die Quadrate),
die Raute (pl. die Rauten),
der Raum (pl. die Räume),
senkrecht,
die Strecke (pl. die Strecken),
die Stunde (pl. die Stunden),
subtrahieren,
die Subtraktion (pl. die Subtraktionen),
der Summand (pl. die Summanden),
die Summe (pl. die Summen),
das Trapez (pl. die Trapeze),
der Übertrag (pl. die Überträge),
zählen,
das Volumen (pl. die Volumina),
der Zähler (pl. die Zähler),
die Zahl (pl. die Zahlen),
der Zahlenstrahl (pl. die Zahlenstrahlen),
die Ziffer (pl. die Ziffern),
das Zwischenergebnis (pl. die Zwischenergebnisse),

\section{Noch nicht eingeführte prüfungsrelevante Vokabeln}

abbuchen,
die Abbuchung (pl. die Abbuchungen),
abheben,
der Algorithmus (pl. die Algorithmen),
anheben,
die Anhebung (pl. die Anhebungen),
anpassen,
die Anpassung (pl. die Anpassungen),
antiproportional,
die Annahme (pl. die Annahmen),
annehmen,
auszahlen
die Auszahlung (pl. die Auszahlungen),
berühren,
buchen,
die Buchung (pl. die Buchungen),
deutlich,
das Diagramm (pl. die Diagramme),
drehen,
die Drehung (pl. die Drehungen),
der Durchschnitt (pl. die Durchschnitte),
die Ecke (pl. die Ecken),
die Einheit (pl. die Einheiten),
einzahlen,
die Einzahlung (pl. die Einzahlungen),
das Element (pl. die Elemente),
elementar,
ergeben,
erhöhen,
die Erhöhung (pl. die Erhöhungen),
erweitern,
die Erweiterung (pl. die Erweiterungen),
explizit,
der Faktor (pl. die Faktoren),
faktorisieren,
fehlen,
das Fehlen (kein pl.),
der Fehler (pl. die Fehler),
flach,
die Figur (pl. die Figuren),
die Fläche (pl. die Flächen),
die Folge (pl. die Folgen),
folgen,
folgern,
die Folgerung (pl. die Folgerungen),
die Formel (pl. die Formeln),
der Funktionsgraph (pl. die Funktionsgraphen),
die Ganze Zahl (pl. die Ganzen Zahlen),
das Gefäß (pl. die Gefäße),
das Gehalt (pl. die Gehälter),
das Geodreieck (pl. die Geodreiecke),
gerade,
die Gerade (pl. die Geraden),
das Gewicht (pl. die Gewichte),
gleich,
die Gleichheit (pl. die Gleichheiten),
die Gleichung (pl. die Gleichungen),
die Grafik (pl. die Grafiken),
grafisch,
das Gramm (kein pl.),
die Grundrechenart (pl. die Grundrechenarten),
der Grundsatz (pl. die Grundsätze [ungebräuchlich]),
grundsätzlich,
die Größenordnung (pl. die Größenordnungen),
das Guthaben (pl. die Guthaben),
hinreichend,
identisch,
die Identität (pl. die Identitäten),
das Kapital (kein pl.),
kariert,
der Kasten (pl. die Kästen),
das Kästchen (pl. die Kästchen),
der Kehrwert (pl. die Kehrwerte),
das Kilogramm (kein pl.),
die Klammer (pl. die Klammern),
die Kommunikation (pl. die Kommunikationen),
konstruieren,
die Konstruktion (pl. die Konstruktionen),
kommunizieren,
die Körperhöhe (pl. die Körperhöhen),
der Kreis (pl. die Kreise),
kurz,
die Implikation (pl. die Implikationen),
implizit,
lang,
die Länge (pl. die Längen),
der Lohn (pl. die Löhne),
kürzen,
irren,
der Irrtum (pl. die Irrtümer),
linear,
lösen,
die Lösung (pl. die Lösungen),
die Masse (pl. die Massen),
das Maß (pl. die Maße),
der Maßstab (pl. die Maßstäbe),
maßstabsgetreu,
das Maximum (pl. die Maxima),
messen,
die Messung (pl. die Messungen),
die Methode (pl. die Methoden),
milli- (Präfix für Größenordnungen),
das Minimum (pl. die Minima),
das Modell (pl. die Modelle),
modellieren,
die Nebenrechnung (pl. die Nebenrechnungen),
die Notation (pl. die Notationen),
notfalls,
der Notfall (pl. die Notfälle),
notieren,
notwendig,
ordnen,
die Ordnung\footnote{Eine Ordnung in der Mathematik ist etwas anderes als die Ordnung die man in seiner Wohnung schafft wenn man aufräumt.} (pl. die Ordnungen),
das Paar (pl. die Paare),
parallel,
die Parallele (pl. die Parallelen),
das Parallelogramm (pl. die Parallelogramme),
die Planskizze (pl. die Planskizzen),
plausibel,
die Plausibilität (kein Plural),
das Polygon (pl. die Polygone),
die Potenz (pl. die Potenzen),
potenzieren,
das Prisma (pl. die Prismen),
die Präfix (pl. die Präfixe),
die Probe (pl. die Proben),
die Proberechnung (pl. die Proberechnungen),
das Produkt (pl. die Produkte),
proportional,
das Prozent (pl. die Prozente),
die Prozentrechnung (kein pl.),
der Punkt (pl. die Punkte),
der Quader (pl. die Quader),
der Quotient (pl. die Quotienten),
der Raum (pl. die Räume),
die Raute (pl. die Rauten),
rechnen,
die Rechnung (pl. die Rechnungen),
das Recht (pl. die Rechte),
rechtwinklig,
die Regel (pl. die Regeln),
regeln,
die Relation (pl. die Relationen),
runden,
die Rundung (pl. die Rundungen),
schätzen,
die Schätzung (pl. die Schätzungen),
schneiden,
der Schnitt (pl. die Schnitte),
der Schnittpunkt (pl. die Schnittpunkte),
die Schulden (kein Singular),
schlussfolgern,
die Schlussfolgerung (pl. die Schlussfolgerungen),
die Sekunde (pl. die Sekunden),
senkrecht,
die Senkrechte (pl. die Senkrechten),
die Skizze (pl. die Skizzen),
skizzieren,
sortieren,
spiegeln,
spiegelsymmetrisch,
die Spiegelung (pl. die Spiegelungen),
steigen,
die Steigung (pl. die Steigungen),
die Stelle (pl. die Stellen),
der Stellenwert (pl. die Stellenwerte),
das Stellenwertsystem (pl. die Stellenwertsysteme),
der Stundenlohn (pl. die Stundenlöhne),
das Symbol (pl. die Symbole),
symmetrisch,
die Symmetrie (pl. die Symmetrien),
die Symmetrieachse (pl. die Symmetrieachsen),
der Tag (pl. die Tage),
der Term (pl. die Terme),
die Tonne (pl. die Tonnen),
das Tripel (pl. die Tripel),
das Tupel (pl. die Tupel),
der Uhrzeigersinn (kein pl.),
der Umfang (pl. die Umfänge),
umrechnen,
die Umrechnung (pl. die Umrechnungen),
ungerade,
das Verfahren (pl. die Verfahren),
das Volumen (pl. die Volumina),
die Voraussetzung (pl. die Voraussetzungen),
das Vorzeichen (pl. die Vorzeichen),
die Waage (pl. die Waagen),
der Wert (pl. die Werte),
wiegen,
der Winkel (pl. die Winkel),
der Würfel (pl. die Würfel),
würfeln,
die Wurzel (pl. die Wurzeln),
die Zahlung (pl. die Zahlungen),
die Zählung (pl. die Zählungen),
zeichnen,
die Zeichnung (pl. die Zeichnungen),
zeigen,
die Zeit (pl. die Zeiten),
die Zeitspanne (pl. die Zeitspannen),
zenti- (Präfix für Größenordnungen),
der Zentimeter (pl. die Zentimeter),
zerlegen,
die Zerlegung (pl. die Zerlegungen),
zerschneiden,
der Zins (pl. die Zinsen),
die Zinsrechnung (kein pl.),
der Zirkel (pl. die Zirkel),
zusammensetzen,
das Zwischenergebnis (pl. die Zwischenergebnisse),
der Zylinder (pl. die Zylinder) 