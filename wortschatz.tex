\section{Vokabeln}

abbuchen,
Abbuchung (f., pl. die Abbuchungen),
abheben,
abrunden,
abschätzen,
Abschätzung (f., pl. die Abschätzungen, synonym mit Schätzung),
abziehen,
addieren,
Algorithmus (m., pl. die Algorithmen),
anheben,
Anhebung (f., pl. die Anhebungen),
Annahme (f., pl. die Annahmen),
annehmen,
anpassen,
Anpassung (f., pl. die Anpassungen),
antiproportional,
antworten,
Anwendbarkeit (f., pl. die Anwendbarkeit),
anwenden,
Art (f., pl. die Arten),
aufrunden,
ausdrücken,
ausklammern,
ausmultiplizieren,
ausnehmen,
ausrechnen,
auszahlen
Auszahlung (f., pl. die Auszahlungen),
behaupten,
berechnen,
berühren,
beschriften,
Beschriftung (f., pl. die Beschriftungen),
beweisen,
Bogen (m., pl. die Bögen),
buchen,
Buchung (f., pl. die Buchungen),
der Antwortsatz (pl. die Antwortsätze),
der Ausdruck (pl. die Ausdrücke),
der Beweis (pl. die Beweise),
der Bruch (pl. die Brüche),
deutlich,
dezi- (Präfix für Größenordnungen),
Diagramm (n., pl. die Diagramme),
die Achse (pl. die Achsen),
die Addition (pl. die Additionen),
die Antwort (pl. die Antworten),
die Anwendung (pl. die Anwendungen),
die Ausnahme (pl. die Ausnahmen),
die Behauptung (pl. die Behauptungen),
die Breite (pl. die Breiten),
die Bruchrechnung (kein pl.),
Differenz (f., pl. die Differenzen),
dividieren,
Division (f., pl. die Divisionen),
drehen,
Drehung (f., pl. die Drehungen),
Durchschnitt (m., pl. die Durchschnitte),
Ecke (f., pl. die Ecken),
Einheit (f., pl. die Einheiten),
einzahlen,
Einzahlung (f., pl. die Einzahlungen),
Element (n., pl. die Elemente),
Element (n., pl. die Elemente),
elementar,
Endergebnis (n., pl. die Endergebnisse),
ergeben,
Ergebnis (n., pl. die Ergebnisse),
erhöhen,
Erhöhung (f., pl. die Erhöhungen),
erweitern,
Erweiterung (f., pl. die Erweiterungen),
explizit,
Faktor (m., pl. die Faktoren),
faktorisieren,
Fehlen (n., kein pl.),
fehlen,
Fehler (m., pl. die Fehler),
Figur (f., pl. die Figuren),
flach,
Fläche (f., pl. die Flächen),
Folge (f., pl. die Folgen),
folgen,
folgern,
Folgerung (f., pl. die Folgerungen),
Formel (f., pl. die Formeln),
Formelsammlung (f., pl. die Formelsammlungen),
Funktionsgraph (m., pl. die Funktionsgraphen),
Ganze Zahl (f., pl. die Ganzen Zahlen),
Gefäß (n., pl. die Gefäße),
Gehalt (n., pl. die Gehälter),
Geodreieck (n., pl. die Geodreiecke),
Geometrie (f., pl. die Geometrien),
Gerade (f., pl. die Geraden),
gerade,
Gewicht (n., pl. die Gewichte),
gleich,
Gleichheit (f., pl. die Gleichheiten),
Gleichung (f., pl. die Gleichungen),
Grafik (f., pl. die Grafiken),
grafisch,
Gramm (n., kein pl.),
Größenordnung (f., pl. die Größenordnungen),
Grundrechenart (f., pl. die Grundrechenarten),
Grundsatz (m., pl. die Grundsätze [ungebräuchlich]),
grundsätzlich,
Guthaben (n., pl. die Guthaben),
Halbkreis (m., pl. die Halbkreise),
heiß,
hinreichend,
Höhe (f., pl. die Höhen),
identisch,
Identität (f., pl. die Identitäten),
Implikation (f., pl. die Implikationen),
implizit,
Inhalt (m., pl. die Inhalte),
irren,
Irrtum (m., pl. die Irrtümer),
kalt,
Kaltmiete (f., pl. die Kaltmieten),
Kapital (n., kein pl.),
kariert,
Kästchen (n., pl. die Kästchen),
Kasten (m., pl. die Kästen),
Kehrwert (m., pl. die Kehrwerte),
Kilogramm (n., kein pl.),
Klammer (f., pl. die Klammern),
Komma (n., pl. die Kommata),
Kommunikation (f., pl. die Kommunikationen),
kommunizieren,
konstruieren,
Konstruktion (f., pl. die Konstruktionen),
Körperhöhe (f., pl. die Körperhöhen),
Korrektur (f., pl. die Korrekturen),
korrigieren,
Kreis (m., pl. die Kreise),
kubik-,
kurz,
kürzen,
kürzen,
lang,
Länge (f., pl. die Längen),
linear,
Lohn (m., pl. die Löhne),
lösen,
Lösung (f., pl. die Lösungen),
Maß (n., pl. die Maße),
Masse (f., pl. die Massen),
Maßeinheit (f., pl. die Maßeinheiten),
Maßstab (m., pl. die Maßstäbe),
maßstabsgetreu,
Maximum (n., pl. die Maxima),
Menge (f., pl. die Mengen),
messen,
Messung (f., pl. die Messungen),
Meter (m., pl. die Meter),
Methode (f., pl. die Methoden),
Miete (f., pl. die Mieten),
mieten,
milli- (Präfix für Größenordnungen),
Minimum (n., pl. die Minima),
Modell (n., pl. die Modelle),
modellieren,
Multiplikation (f., pl. die Multiplikationen),
multiplizieren,
Nebenkosten (kein Singular),
Nebenrechnung (f., pl. die Nebenrechnungen),
Nenner (m., pl. die Nenner),
Notation (f., pl. die Notationen),
Notfall (m., pl. die Notfälle),
notfalls,
notieren,
notwendig,
ordnen,
Ordnung\footnote{Eine Ordnung in der Mathematik ist etwas anderes als die Ordnung die man in seiner Wohnung schafft wenn man aufräumt.} (f., pl. die Ordnungen),
Paar (n., pl. die Paare),
parallel,
parallel,
Parallele (f., pl. die Parallelen),
Parallele (f., pl. die Parallelen),
Parallelogramm (n., pl. die Parallelogramme),
Parallelogramm (n., pl. die Parallelogramme),
Planskizze (f., pl. die Planskizzen),
plausibel,
Plausibilität (f., kein Plural),
Polygon (n., pl. die Polygone),
Potenz (f., pl. die Potenzen),
potenzieren,
Präfix (f., pl. die Präfixe),
prim,
Primfaktorzerlegung (f., pl. die Primfaktorzerlegungen),
Primzahl (f., pl. die Primzahlen),
Prisma (n., pl. die Prismen),
Probe (f., pl. die Proben),
Proberechnung (f., pl. die Proberechnungen),
Produkt (n., pl. die Produkte),
proportional,
Prozent (n., pl. die Prozente),
Prozentrechnung (f., kein pl.),
Punkt (m., pl. die Punkte),
Quader (m., pl. die Quader),
Quadrat (n., pl. die Quadrate),
quadrat-,
Quotient (m., pl. die Quotienten),
Raum (m., pl. die Räume),
Raum (m., pl. die Räume),
Raute (f., pl. die Rauten),
Raute (f., pl. die Rauten),
rechnen,
Rechnung (f., pl. die Rechnungen),
Recht (n., pl. die Rechte),
Rechteck (n., pl. die Rechtecke),
rechtwinklig,
rechtwinklig,
Regel (f., pl. die Regeln),
regeln,
Relation (f., pl. die Relationen),
runden,
Rundung (f., pl. die Rundungen),
schätzen,
Schätzung (f., pl. die Schätzungen),
schlussfolgern,
Schlussfolgerung (f., pl. die Schlussfolgerungen),
schneiden,
Schnitt (m., pl. die Schnitte),
Schnittpunkt (m., pl. die Schnittpunkte),
Schulden (kein Singular),
Schwerpunkt (m., pl. die Schwerpunkte),
Sekunde (f., pl. die Sekunden),
senkrecht,
senkrecht,
Senkrechte (f., pl. die Senkrechten),
Skizze (f., pl. die Skizzen),
skizzieren,
sortieren,
spiegeln,
spiegelsymmetrisch,
Spiegelung (m., pl. die Spiegelungen),
steigen,
Steigung (f., pl. die Steigungen),
Stelle (f., pl. die Stellen),
Stellenwert (m., pl. die Stellenwerte),
Stellenwertsystem (n., pl. die Stellenwertsysteme),
Strecke (f., pl. die Strecken),
Stunde (f., pl. die Stunden),
Stundenlohn (m., pl. die Stundenlöhne),
subtrahieren,
Subtraktion (f., pl. die Subtraktionen),
Summand (m., pl. die Summanden),
Summe (f., pl. die Summen),
Symbol (n., pl. die Symbole),
Symmetrie (f., pl. die Symmetrien),
Symmetrieachse (f., pl. die Symmetrieachsen),
symmetrisch,
Tag (m., pl. die Tage),
Term (m., pl. die Terme),
Tonne (f., pl. die Tonnen),
Trapez (n., pl. die Trapeze),
Tripel (n., pl. die Tripel),
Tupel (n., pl. die Tupel),
Übertrag (m., pl. die Überträge),
Uhrzeigersinn (m., kein pl.),
umbuchen,
Umbuchung (f., pl. die Umbuchungen),
Umfang (m., pl. die Umfänge),
umrechnen,
Umrechnung (f., pl. die Umrechnungen),
ungerade,
Verfahren (n., pl. die Verfahren),
Volumen (n., pl. die Volumina),
Volumen (n., pl. die Volumina),
Voraussetzung (f., pl. die Voraussetzungen),
Vorzeichen (n., pl. die Vorzeichen),
Waage (f., pl. die Waagen),
warm,
Warmmiete (f., pl. die Warmmieten),
Wert (m., pl. die Werte),
wiegen,
Winkel (m., pl. die Winkel),
Würfel (m., pl. die Würfel),
würfeln,
Wurzel (f., pl. die Wurzeln),
Zahl (f., pl. die Zahlen),
zählen,
Zahlenstrahl (m., pl. die Zahlenstrahlen),
Zähler (m., pl. die Zähler),
Zahlung (f., pl. die Zahlungen),
Zählung (f., pl. die Zählungen),
zeichnen,
Zeichnung (f., pl. die Zeichnungen),
zeigen,
Zeit (f., pl. die Zeiten),
Zeitspanne (f., pl. die Zeitspannen),
zenti- (Präfix für Größenordnungen),
Zentimeter (m., pl. die Zentimeter),
zerlegen,
Zerlegung (f., pl. die Zerlegungen),
zerschneiden,
Ziffer (f., pl. die Ziffern),
Zins (m., pl. die Zinsen),
Zinsrechnung (f., kein pl.),
Zirkel (m., pl. die Zirkel),
zusammensetzen,
Zwischenergebnis (n., pl. die Zwischenergebnisse),
Zwischenergebnis (n., pl. die Zwischenergebnisse),
Zylinder (m., pl. die Zylinder) 